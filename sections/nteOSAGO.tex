\setcounter{page}{1}
\clubpenalty=100000  % Недопуск Висячей строки в начале страницы
\widowpenalty=100000 %Недопуск висячей строки в конце абзаца
%%%%%%%%%%%%%%%%%%%%%%%%%%%%%%%%%%%%%%%%
%      Шапка экспертной организации  
%%%%%%%%%%%%%%%%%%%%%%%%%%%%%%%%%%%%%%%%
%%%%%%%%%%%%%%%%%%%%%%%%%%%%%%%%%%%%%%%%%
%
%   Экспертная организация ООО Южнорегиональная экспертная группа
%
%%%%%%%%%%%%%%%%%%%%%%%%%%%%%%%%%%%%%%%%%
\noindent %\qrcode[height=21mm]{\NomerDoc от \окончено }  %%% Добавлен QR-Code
\begin{pspicture}(21mm,21mm)
\obeylines
\psbarcode{%
%	\NomerDoc от \окончено
	BEGIN:VCARD^^J
	VERSION:4.0^^J
	%N:Мраморнов; Александр; Вчеславович^^J
	FN:Александр Мраморнов^^J
%	ORG:IP Alexandr Mramornov^^J
	TITLE: эксперт
	ORG: ИП
	URL:http://www.yourexp.ru^^J
	EMAIL:4516611@gmail.com^^J
	TEL:+7-918-451-6611^^J
	ADR:г. Краснодар, с/т № 2 А/О «Югтекс», ул. Зеленая, 472^^J
	END:VCARD
}{width=1.0 height=1.0}{qrcode}%
\end{pspicture}%%% Добавлен QR-Code
\vspace{-4mm}
\begin{center}
	\large\textbf{ИНДИВИДУАЛЬНЫЙ\quad ПРЕДПРИНИМАТЕЛЬ  \\[-1.5mm] МРАМОРНОВ  АЛЕКСАНДР ВЯЧЕСЛАВОВИЧ \\[-5.5mm]}
	%  
	\noindent\rule{\textwidth}{2pt}\\[-6mm]  % Горизонтальная линия
	% \line(1,0){460}% (1,0) -горизонтальная линия, и (0,1) - вертикальная 
\end{center}

\begin{center}
	\begin{footnotesize}\setstretch{0.3}
		%	\small\textbf\setlength   	%\raisebox{5mm}
		\vspace{-3.5mm}г. Краснодар, с/т № 2 А/О «Югтекс», ул. Зеленая, 472, 
		Телефон: 8-918-451-66-11, e-mail: 4516611@gmail.com\\ [-2mm]{ИНН\quad 231200665168\quad ОГРНИП \quad 310231220400043}
	\end{footnotesize}	\\[10mm]
\end{center}


\begin{flushright}
% 
	 \hfill	Краснодар, 2020    \\[8mm]
\end{flushright}
\begin{center}
	\LARGE\textbf{ЭКСПЕРТНОЕ ЗАКЛЮЧЕНИЕ}
	\bigskip\\[0mm]
	%	{\normnumxtbf{\NomerDoc}}	}{den}
\end{center}
\par
\vspace{-6mm}
\noindent независимой технической экспертизы по определению размера расходов на восстановительный ремонт транспортного средства   \тс  \\[2mm]

%\raggedright 
%\def\hrf#1{\hbox to#1{\hrulefill}}
\noindent \textbf{№ \NomerDoc}\hfill           \textbf{\окончено}\\%[2mm]
%Приостановлено\hfill      \datastop\\
%Возобновлено\hfill          \datarestart\\
%Окончено\hfill                \dataend\\%[4mm]

\noindent\parbox[l][16mm]{16.5cm}
{\def\hrf#1{\hbox to#1{\hrulefill}}
	\noindent Начато\hfill            \datastart\\%[2mm]
	%	Приостановлено\hfill      \datastop\\
	%	Возобновлено\hfill          \datarestart\\
	Окончено\hfill                \окончено
}
\relax

%%%%%%%%%%%% Если судебка
%
%\datastart г. ~в {\small ООО~ "ЮЖНО-РЕГИОНАЛЬНАЯ ЭКСПЕРТНАЯ ГРУППА"} \,  при определении  \, \sud  \,  от \, \dataopr \, о назначении \opr \, по гражданскому делу \delonum \, поступили:
%
%\begin{enumerate}\setlist{nolistsep}\item  Материалы гражданского дела \delonum \, в двух томах, том 1 на 276 листах, том 2  на 143 листах.\\[-2mm]
%	%	\item  
%\end{enumerate}
%
%%%%%%%%%%%%  Если независимая
\vspace{4mm}
Составлено на основании	договора № \NomerDoc\,  возмездного оказания услуг по проведению независимой технической экспертизы (далее экспертиза)  транспортного средства и письменного заявления заказчика о проведении экспертизы.

Заказчик  экспертизы: \заказчик, \адресзаказчика. 

Полис ОСАГО: \polis.

% Документ, удостоверяющий личность заказчика: водительское удостоверение    03\ 16\ 422344\ выдан 09.06.2011

%Транспортное средство виновника ДТП:  не предоставлялось.

\paragraph*{}
Экспертиза произведена  экспертом--техником
%{\small ООО "ЮЖНО-РЕГИОНАЛЬНАЯ ЭКСПЕРТНАЯ ГРУППА"}
\,  Мраморновым Александром Вячеславовичем, имеющим высшее техническое образование по специальности «техническая физика», диплом РВ № 311964 от 28.02.1989, квалификация -- инженер-физик, специальное образование в области оценки: Диплом ПП-1 № 037211 Российской экономической академии им. Г.В. Плеханова, квалификация -- оценка и экспертиза объектов и прав собственности, специальное образование в области независимой технической экспертизы транспортных средств: Диплом ПП-I № 424167, квалификация: эксперт-техник (специализация 150210 специальности 190601.65 – Автомобили и автомобильное хозяйство), состоящий в Государственном реестре экспертов-техников (№ в реестре 256, https://data.gov.ru/opendata/7707211418-experts,  общий трудовой  стаж 30 лет, стаж  экспертной работы  12 лет. \par Заключение подготовлено по месту фактического расположения ИП по адресу: г. Краснодар, с/т № 2 А/О «Югтекс», ул. Зелёная, 472.
  % Шапка организации ИП
%%   вопросы экспертизы
\subsection{Вопросы экспертизы}
\begin{enumerate}
\item  <<Установить наличие, характер и объем (степень) технических повреждений транспортного средства  \tc \,>>?\\[-2mm]
%\item  <<Установить причины возникновения технических повреждений транспортного средства \tc \,и возможность их отнесения к рассматриваемому дорожно-транспортному происшествию (далее ДТП)>>?\\[-2mm]
\item <<Установить технологию, объем восстановительного  ремонта транспортного средства \tc \,>>?\\[-2mm]
\item <<Установить размер затрат на восстановительный ремонт (с учётом износа) транспортного средства \tc \,>>?\\[-2mm]
%	
\end{enumerate}
\subsection{Исходные данные} %Название по шаблону 
Исходные  данные,  необходимые  для   исследования,  изложены   в  заявлении о проведения исследования  (оценки)  колесного  транспортного  средства (далее —  KTC):

\begin{enumerate}\item Цифровые фотоснимки
транспортного средства \тс \, в поврежденном состоянии.\\[-2mm]
	%\item Свидетельство о регистрации \свид транспортного средства \тс \\ [-2mm]
\item Копия экспертного заключения № 006/06/19 <<Об определении рыночной стоимости затрат на восстановление а/м МАЗДА 6 г/н О552 СА 123 после ДТП иутраты товарной стоимости (УТС) >>\\ [-2mm]
%	\item Постановление об административном правонарушении дорожно-транспортном %происшествии, имевшем место   \датадтп  с участием  ТС \тс,\, согласно которой %на ТС \тс \, в результате ДТП повреждено:\, "\повреждения".  \\[-2mm]
%	\item Полис страхования  ОСАГО \polis.
	\end{enumerate}
%
\addcontentsline{toc}{section}{Использованные нормативы и источники информации}
%\subsubsection*{}
%\left( \addcontentsline{toc}{section}{Использованные нормативы и источники информации}

\subsection{Использованные нормативы и источники информации}
%
\begin{enumerate}
\item 
Махнин\,Е.\,Л., Новоселецкий\, И.\,Н., Федотов\, С.\,В. \emph{Методические рекомендации по проведению судебных автотехнических экспертиз и исследований колёсных транспортных средств в целях определения размера ущерба, стоимости восстановительного ремонта и оценки} // -- М.: ФБУ РФЦСЭ при Минюсте России, 2018.-326 с.  ISBN 978-5-91133-185-6.
%
%
%
%
\item ТУ 017207-255-00232934-2014 \emph{Кузова автомобилей LADA. Технические требования при приёмке в ремонт, ремонте и выпуске из ремонта предприятиями дилерской сети ОАО "АВТОВАЗ"}//  ОАО НВП "ИТЦ АВТО", 2014
%
\item Смирнов  В.Л., Прохоров  Ю.С., Боюр В.С.  и др. \emph{Автомобили ВАЗ. Кузова. Технология ремонта, окраски и  антикоррозионной защиты. Часть II}// - Н.Новгород: АТИС, 2001.- 241с.
%
\item 
Савич Е.Л. \emph{Техническое  обслуживание  и  ремонт  легковых  автомобилей} : учеб. пособие / Е.Л. Савич, М.М. Болбас, В.К. Ярошевич ; под общ. ред. Е.Л. Савича. -Мн. : Вышэйшая школа,  2001. - 479 с. - ISBN985-06-0502-2.
%
\item 
Автомобили ВАЗ-2121, 21213, 21214, 2131 и их модификации: <<Трудоемкости работ (услуг) по техническому обслуживанию и ремонту>> /Куликов А.В., Христов П.Н., Климов В.Е.,  Боюр В.С., Рева В.В., Зимин В.А., Завьялова Н.Н., Хлыненкова Г.А. -- ИТЦТ "АвтоВАЗтехобслуживание", Тольяти -- 2005. 
%
\item
Автомобили LADA SAMARA и их модификации: <<Трудоемкости работ (услуг) по техническому обслуживанию и ремонту>> /Куликов А.В., Христов П.Н., Климов В.Е., Рева В.В., Боюр В.С., Васильев М.В., Фахрутдинов Р.В.,  Прудских Д.А., Гирко В.Б., Шмелева В.А., Зимин В.А. --  ОАО НВП "ИТЦ АВТО",  -- 2006. - 252 стр.
%
\item 
Автомобили LADA PRIORA. Трудоемкости работ (услуг) по техническому обслуживанию и ремонту /Куликов А.В., Христов П.Н., Климов В.Е., Рева В.В., Козлов П.Л., Боюр В.С., Прудских Д.А., Шмелева В.А., Зимин В.А. -- ООО "ИТЦТ АВОСФЕРА", Тольяти -- 2009. -- 344 с.
%
\item 
{Трудоемкости работ по техническому обслуживанию и ремонту автомобилей автомобилей Lada  Granta}/   \url{https://docplayer.ru/30250248-Trudoemkosti-rabot-po-teh\-nicheskomu-obsluzhivaniyu-i-remontu-avtomobiley-lada- granta.html}.
%
%
\item
{Специализированное программное обеспечение для расчёта стоимости  восстановительного ремонта, содержащее нормативы трудоёмкости работ, регламентируемые изготовителями транспортного средства}//   AudaPadWeb, лицензионное соглашение № AS/APW-658  RU-P-409-409435.
%
%
%
\item
{Специализированное программное обеспечение для расчёта стоимости  восстановительного ремонта, содержащее нормативы трудоёмкости работ, регламентируемые изготовителями транспортного средства ОАО «АвтоВАЗ», ЗАО «Джи-Эм-АвтоВАЗ», ОАО «СеАЗ» и ОАО «ЗМА»}//   Автосфера АС:Смета, v.3.9.11// ООО "ИТЦ «ИнтегроМаш», \url{https://autosmeta.pro}.
%
%
%
\item Информационный портал по техническому обслуживанию и ремонту автомобилей	 ВАЗ:\\ \url{www.autosphere.ru}.

%%
\end{enumerate}

%\bibliographystyle{utf8gost705u}  %% стилевой файл для оформления по ГОСТу
%\bibliography{biblio}     %% имя библиографической базы (bib-файла)
%%%%%%%%%%%%%%%%%%%%%%%%%%%%%%%%%%%%%%%%%%%%%%%%%%%%%%%%%%%%%%%%%%%%%%%%%%%%%%%%%
\subsection{Технические средства}  %% Список не удалять!!!
\begin{itemize}
%
%\item Диагностический сканер SDconnect   с программным обеспечением Xentry Diagnostics v19.11.3.1
%\item 	Линейка масштабная магнитная с цветографической шкалой, 100мм
%\item  Рулетка измерительная металлическая, 5м
%%\item Универсальный стенд для измерения углов установки колес Hunter Engineering %ProAlign с программным инструментом регулировки схождения колес без блокировки руля %автомобиля WinToe
%%\item Цифровой фотоаппарат Canon 760D s/n 143032001327 с объективом Canon EF-S %18-135, тип используемой памяти: Transcend,  32Gb
\item  Специализированное программное обеспечение для расчёта стоимости  восстановительного ремонта, содержащее нормативы трудоёмкости работ, регламентируемые изготовителями транспортного средства     AudaPadWeb, лицензионное соглашение № AS/\- APW-658  RU-P-409-409435
\item  Программа обработки фото-видео изображений ImageJ, разработчик  Wayne Rasband (wa-yne@codon.nih.gov),
свободная лицензия GPL.
\item  ПЭВМ под управлением операционной системы Windows 10 с установленным набором макрорасширений LaTeX системы компьютерной вёрстки TeX, cвободная лицензия LaTeX Project Public License (LPPL). 
%	
	\end{itemize}
%%%%%%%%%%%%%%%%%%%%%%%%%%%%%%%%%%%%%%%%%%%%%%%%%%%%%%%%%%%%%%%%%%%%%%%%%%%%%%%%%%%%%%%%%%%%%%%
\subsection{Условные обозначения}
\begin{description}
%	 
%%\item[АВС] --Антиблокировочная система
\item[АМТС] --автомототранспортное средство
\item[ДВС] --двигатель внутреннего сгорания
\item[ДТП] --дорожно--транспортное происшествие
\item[гос.\,рег.\,знак] -государственный регистрационный знак
\item[КТС] --колесно транспортное средство 
\item[ЛКП] --лакокрасочное покрытие
%\item[л.д.] --Лист дела
%%\item[Колесо турбины]  -- крыльчатка турбины
\item[ТС] --транспортное средство
%\item[ТK, ТКР] -- Турбокомпрессор. Состоит из двух частей: турбины и компрессора, объединенных общим валом. Вал вращается в подшипниках, размещенных в центральном корпусе ТК
%\item[ЭБУ] --Электронный блок управления
%\item[DTC] --Diagnostic Trouble Codes, диагностические коды неисправностей
%\item[FRAME] --номер кузова транспортного средства, выпущенного для продажи на внутреннем рынке Японии и содержащий информацию производителя о транспортном средстве
\item[OBDII] -- On-board diagnostics, протокол бортовой диагностики автомобиля
\item[SRS] -- система пассивной защиты водителя и пассажиров
\item[VIN] --vehicle identification number, 17--значный идентификационный номер транспортного средства, соответствующий стандарту ISO 3779--2012.
%
\end{description}

\subsection{Методы исследования}

\begin{itemize}
\item  Органолептический метод – исследование и оценка качества объектов с помощью органов чувств
%\item 	Прямой измерительный метод – путем измерения размеров деталей специальными %измерительными приборами
\item Расчётный метод (косвенный измерительный метод) – путём расчётов различных параметров на основе результатов измерений и других данных
\item Экспертный метод (метод экспертной оценки) — совокупности операций по выбору комплекса или единичных характеристик объекта, определению их действительных значений и оценкой экспертом соответствия их установленным требованиям и/или технической информации
%%	\item Метод натурной реконструкции??
\end{itemize}

\subsection{Ограничения и пределы применения полученных результатов}

Следующие допущения и условия, ограничивающие пределы применения полученных результатов, являются неотъемлемой частью данного экспертного заключения.
\begin{itemize}
\item  {Результаты, полученные экспертом-техником, носят рекомендательный консультационный характер и не являются обязательными. Исполнитель высказывает своё субъективное суждение о наиболее вероятных будущих (абстрактных) расходах, их предполагаемом размере и дает заключение в пределах своей компетенции.}
\item { Под компетенцией эксперта-техника понимают его знания и опыт в области теории и методов экспертных исследований ТС, а также круг полномочий, представленных ему законом, и вопросов, которые он может решать на основе своих специальных познаний.
В компетенцию эксперта-техника входит исследование технического состояния поврежденного ТС в целях установления характера повреждений ТС, установления причины возникновения технических повреждений технологии, методов, стоимости его ремонта.}
\item  {Исполнитель в рамках своих обязательств по заключенному договору об экспертном обслуживании признает свою ответственность перед заказчиком и настоящим утверждает, что экспертное заключение выполнено профессионально, тщательно и с должной заботливостью и внимаем, как это обычно принято для компетентного специалиста в области технической экспертизы ТС при ОСАГО, а полученная величина восстановительных расходов, разумна и реальна.}
\item  {Исполнитель считает, что поскольку, по общему правилу, оценка доказательств является прерогативой и компетенцией органа дознания, следствия или суда, а в досудебном порядке - страховщика, постольку после проверки результатов экспертизы последним, их признания и принятия решения о выплате страхового возмещения этап возможного оспаривания достоверности исследований между заказчиком и исполнителем завершен.
Соответственно, обязанности Исполнителя по договору являются надлежаще исполненными в полном объеме и от исполнителя не требуется свидетельствовать по поводу произведённого исследования перед третьими лицами.}
\item  {Отдельные части настоящего экспертного исследования не могут трактоваться раздельно, а только в связи с полным текстом о проведенных расчетах.}
\item  {Исходные данные, использованные исполнителем при подготовке экспертного заключения, получены из надежных источников и считаются достоверными. Тем не менее, исполнитель не может гарантировать абсолютную точность, поэтому там, где это, возможно, делаются ссылки на источники информации.}
\item  {В процессе экспертного исследования специальная юридическая экспертиза документов, касающихся прав собственности на ТС, не проводилась.}
\item  {Суждения, содержащиеся в экспертном заключении, основываются на текущей ситуации на дату аварии и, в будущем, могут быть подвержены изменениям.
Исполнитель не принимает на себя никакой ответственности за изменение экономических, юридических и иных факторов, которые могут возникнуть после даты исследования и повлиять на результаты технической экспертизы.
Данное заключение составлено на основании Правил Независимой Технической Экспертизы и может применяться только при решении вопроса о выплате страхового возмещения по ОСАГО.
}\end{itemize}
% 
\section{Исследование}
%
%%\subsection{Исследование предоставленных на экспертизу документов}
%
Настоящее исследование проводится на основании материалов, предоставленных Заказчиком, а также на основании данных, самостоятельно полученных экспертом-техником. Выводы, содержащиеся в настоящем Заключении, могут расцениваться как достоверные только в контексте того количества информации, на основании которого они были сделаны. При поступлении дополнительной или измененной информации данные выводы могут быть
скорректированы. %При анализе скрытых повреждений экспертом-техником не принимается во внимание наличие или отсутствие записей о них в документах компетентных органов, в связи с отсутствием у сотрудников компетентных органов объективной возможности  идентификации таких повреждений на месте происшествия.
 \par Из предоставленных материалов   экспертом-техником установлена следующая общая информация об автомобиле, имеющая значение для дачи заключения:\\
 \parbox[]{10cm}{}
	\begin{itemize}
		\item[ ] 
			\begin{description}
			\item[Марка, модель] -- \тс
		%	\item[VIN] -- \vin
		%	\item[Год выпуска] -- \год
			\item[Шасси] -- Отсутствует
			\item[Цвет ЛКП] -- Синий
			\item[Пробег] --  \пробег\,км%, считан с одометра
		%	\item[Двигатель] --
		%	\item[ПТС] --\птс
        %						
		\end{description}
		\end{itemize}
	\subparagraph*{} Идентификационный код автомобиля (VIN)  \vin \, содержит следующую информацию о транспортном средстве, имеющую значение для 	дачи заключения:\\[3mm]
        %
\noindent\parbox[]{10cm}{
\begin{itemize}
	\item[ ] 
	    \begin{description}
	\item[Дата изготовления] \hfill \началоэкспл
	\item[Расположенние руля] \hfill Left
	\item[Двигатель] \hfill бензиновый, 149,6 л.с.
	\item[Объем двигатель] \hfill 1998 $ \text{см}^3 $
	\item[КПП] \hfill АКПП
	\item[Тип кузова] \hfill  Седан
	\item[Количество дверей] \hfill 4 
	%	\item[VDS] --
    %	
		\end{description}
\end{itemize}}\\%[3mm]
    %
\vspace{5mm}
\textit{Источник: https://emex.ru/catalogs/original/?screen=units\&vin=/\вин}\\
    %Пробег автомобиля  расчетный, согласно [1]  составляет 
  \begin{figure}[!h]
	\centering
	\includegraphics[width=0.98\linewidth]{images/1}
	\caption{{\footnotesize {Общие виды поврежденного ТС \тс. Фототаблица предоставлена заказчиком}}}
	\label{ris:images/1}
\end{figure}
 
Согласно сведениям, содержащимся в предоставленных материалах, \osm \, экс\-пертом-техником Яковлевым С.В. проведен осмотр поврежденного транспортного средства \tc.\,
% Осмотр проводился в сухую, ясную погоду с 14 час. 00 мин. до 14 час. 30 мин. на открытой площадке по адресу:  г. Краснодар, ул. Кореновская, 24. При осмотре присутствовали: владелец ТС \tc \, \владелец. 
Соответствие маркировочных обозначений на кузове представленного ТС записям в регистрационных документах ТС экс\-пертом-техником установлено. Видимые изменения конструкции ТС отсутствуют.  Представленный на исследование автомобиль \tc\, имеет кузов типа "\типкузова". Кузов автомобиля окрашен рефлексной %(лессирующей) 
с "металлическим" эффектом эмалью %(краской) 
синего цвета.

\renewcommand\baselinestretch{0.86}\small\normalsize 
\subsection{\underline{По  вопросу}\, \, \,	\textbf{\small{<<Установить наличие, характер и объем (степень) технических повреждений транспортного средства  \tc \,>>?}}}
\renewcommand\baselinestretch{1.2}\small\normalsize
%
% 1.\ Наличие, характер и объем технических повреждений, а так же  планируемые (предполагаемые) ремонтные воздействия для восстановления поврежденного автомобиля  исследованы в присутствии заинтересованных лиц, зафиксированы в акте осмотра № 006/06/19  от \osm% \NomerDoc \, (Приложение №1)
% ,  и фотоматериалах по принадлежности (Приложение №2). 
%
Первичное установление наличия и характера повреждений транспортного средства, в отношении которых определяются расходы на восстановительный ремонт, в соответствии с  Единой методикой определения размера расходов на восстановительный ремонт 
%в отношении поврежденного транспортного средства  
утверждены Положением Банка России от «19» сентября 2014 года № 432-П (Единой методикой), должно производится во время осмотра транспортного средства и фиксироваться актом осмотра, в который  должны включаться сведения о повреждениях транспортного средства, с обязательной  характеристикой поврежденных элементов с указанием расположения, вида и объема повреждения.   
%
В акте № 006/06/19  от \osm\, повреждений транспортного средства (ТС)  не содержатся установленные Единой методикой обязательные количественные показатели (характеристики) повреждений транспортного средства.  Отсутствие количественных характеристик повреждений в представленном акте осмотра предполагает его применение в настоящем исследовании исключительно в виде списка повреждений и перечня ремонтных воздействий, определенных экспертом-техником Яковлевым С.В. для восстановительного ремонта транспортного средства. %  не позволяет эксперту использовать указанный акт осмотра для объективного выбора необходимого и достаточного комплекса работ по восстановительному ремонту транспортного средства.
%
%На момент определения о проведении повторной судебной экспертизы, объект исследования неоднократно подвергался видоизменениям как при исследовании специалистом ООО "ЭКСПЕРТ", так и при производстве первичной судебной экспертизы экспертами ИП Куприянова Виктора Александровича. 
%
\par
Согласно данному акту осмотра № 006/06/19  от \osm \, для устранения повреждений ТС необходимо было произвести следующие ремонтные работы:
%\vspace{\baselineskip}  % вставка пустой строки
% 
\begin{longtable}{|p{1cm}|p{11cm}|p{3cm}|}
\caption[]{\footnotesize {Ремонтные воздействия по акту осмотра № 006/06/19 от \, \osm \, ИП Яковлева С.В.}} \label{tab:4}\\ 
	 \hline
		\rowcolor[HTML]{C0C0C0} 
	%	\multicolumn{1}{|c|}
	%	{\cellcolor[HTML]{C0C0C0}N/N} & Наименование запчасти (материала) & Ремонтное воздействие 
    %	
 \text{N/N} & Наименование запчасти (материала) & Ремонтное воздействие  \\ \hline \endhead
		\Rownum  & Панель задка  & Замена, окраска \\ \hline
		\rowcolor[HTML]{EFEFEF} 
		\Rownum  & Боковина задняя левая   & Замена, окраска \\ \hline
	    \Rownum  & Боковина задняя правая  & Замена, окраска  \\ \hline
		
\end{longtable}
%
\par Таким образом, размер восстановительных расходов (затрат) \тс\,  может быть произведен в объеме сведений, содержащихся в представленной Таблице \ref{tab:4}.
%\vspace{\baselineskip}  % вставка пустой строки
%
%\renewcommand\baselinestretch{0.86}\small\normalsize 
%\subsection{\underline{По  вопросу}\, 	\textbf{\small{2. <<Установить причины возникновения технических повреждений транспортного средства \tc \, и возможность их отнесения к рассматриваемому дорожно-транспортному происшествию (ДТП)>>?}}}
%\renewcommand\baselinestretch{1.2}\small\normalsize
%%
%%2.\ Причины возникновения технических повреждений и возможность их отнесения к
%%рассматриваемому ДТП исследованы при осмотре ТС. 
%Для определения причины возникновения повреждений, указанных в Акте осмотра ТС
%% №   (Приложение № 1) 
%экспертом-техником изучены документы, представленные Заказчиком и сведения о ДТП, с участием ТС \вин% \, по данным открытых источников https://гибдд.рф. 
%По предоставленным документам экспертом установлена причина ДТП, установлены обстоятельства ДТП, выявлены повреждения ТС и установлены причины их образования. Проведено исследование характера выявленных повреждений, сопоставление повреждений ТС потерпевшего с повреждениями ТС иных участников ДТП в соответствии со сведениями, зафиксированными в документах о ДТП. Проведена проверка взаимосвязанности повреждений на ТС с заявленными обстоятельствами ДТП, определен объем восстановительных работ (Приложение №1 Акт осмотра ТС). 
%Из открытых банков данных полиции следует, что автомобиль с VIN  \вин\,  как минимум дважды становился участником ДТП.
%Первый раз 29.06.2018  06:40, извещение о ДТП № 030046913, в котором автомобиль получил повреждения задней правой двери, заднего правого порога, заднего правого колеса, подушки SRS справа, Рис. \ref{ris:images/d1} и второй раз 22.05.2019 06:50, извещение о ДТП № 030034947, в котором автомобиль получил повреждения деталей передней левой и задней частей кузова, Рис. \ref{ris:images/d2}.
%
%
%\begin{figure}[H]\centering
%	\parbox[t]{0.49\textwidth}
%	{\centering
%		\includegraphics[width=.49\textwidth]{images/d1}
%		\caption{\footnotesize {Повреждения в ДТП 29.06.2018 }}
%		\label{ris:images/d1}}
%	\hfil \hfil%раздвигаем боксы по горизонтали 
%	\parbox[t]{0.49\textwidth}
%	{\centering
%		\includegraphics[width=.49\textwidth]{images/d2}
%		\caption{\footnotesize {Повреждения в ДТП 22.05.2019}}
%		\label{ris:images/d2}}
%\end{figure}
%%
%{\noindent  \footnotesize \tikz \fill [red] (1,0.5) rectangle (0.1,0.1); --{\footnotesize  Вмятины, вырывы, заломы, перекосы, разрывы и другие повреждения с изменением геометрии элементов (деталей) кузова и эксплуатационных характеристик ТС.}\\
%\tikz \fill [yellow] (1,0.5) rectangle (0.1,0.1); --  {\footnotesize Повреждения колёс (шин), элементов ходовой части, стекол, фар, указателей поворота, стоп-сигналов и других стеклянных элементов (в т.ч. зеркал), а также царапины, сколы, потертости лакокрасочного покрытия или пластиковых конструктивных деталей и другие повреждения без изменения геометрии элементов (деталей) кузова и эксплуатационных характеристик ТС.}\\[1mm]
%%%%%%%%%%%%%%%%%%%%%%%%%%%%%%%%%%%%%%%%%%%%%%%%%%%%%%%%%%%%%%%%%%%%%%%%%%%}%%%%%%%%%%%%%%%%%%%%%%%%%%%%%%%%%%%%%%%%%%%%%%%%%%
%
%\renewcommand\baselinestretch{1.2}\small\normalsize
%\begin{spacing}{1.25}Таким образом, перечень повреждений, указанный в акте осмотра № 006/06/19 от \osm \, ИП Яковлева С.В. может соответствовать повреждениям автомобиля \тс\,, полученным в результате ДТП \датадтп. 
%\end{spacing}
%\renewcommand\baselinestretch{0.86}\small\normalsize 
%\subsection{\underline{По  вопросу}\, \, \,	\textbf{\small{3. <<Установить технологию, объем восстановительного  ремонта транспортного средства \tc \,>>?}}}
%\renewcommand\baselinestretch{1.2}\small\normalsize
%%%%%%%%%%%%%%%%%%%%%%%%% Для автомобилей, старше семи лет:
%Колесное транспортное средство сроком эксплуатации более 7 лет относится к категории транспортных средств с граничным сроком эксплуатации [1], для которой возможно применение ремонтных операций при условии экономической целесообразности и  технической возможности.  
%
\renewcommand\baselinestretch{0.86}\small\normalsize 
\subsection{\underline{По  вопросу}\, \, \,	\textbf{\small{<<Установить размер затрат на восстановительный ремонт (с учётом износа) транспортного средства \tc \,>>?}}}
\renewcommand\baselinestretch{1.2}\small\normalsize
%
В соответствии с существующей экспертной методикой размер расходов на восстановительный ремонт определяется исходя из стоимости ремонтных работ (работ по восстановлению, в том числе окраске, контролю, диагностике и регулировке, сопутствующих работ), стоимости используемых в процессе восстановления транспортного средства деталей (узлов, агрегатов) и материалов взамен поврежденных. 
%                                         
%В соответствии с принятой экспертной методикой [1], стоимость восстановительного ремонта АМТС  $ C_p $ определяется по формуле:
%%
%\begin{equation}\label{eq:r}
%C_\text{вp} =C_p + C_\text{м} + C_\text{зч}\cdot\left( 1-\frac{\text{И}}{100}\right)  \,\,\,\, \text{где:}
%\end{equation}
%%
%%
%\begin{itemize}
%%	
%\item[ ]$C_\text {р} $ --  стоимость ремонтных работ по восстановлению КТС, руб.;
%\item[ ]$ C_\text{м} $ --  стоимость необходимых ремонтных материалов, руб.;
%\item[ ]$ C_\text{зч} $ --  стоимость новых запасных частей, руб;
%\item[ ] $ \text{И} $ -- коэффициент износа составной части, подлежащей замене, \%.
%\end{itemize}
%%
%%
%Коэффициент износа составных частей (И) КТС (кроме автобусов и грузовых автомобилей) при определении стоимости восстановительного ремонта расчитывается по формуле:
%
%\begin{equation}\label{eqsnos}
%\text{И} =\text{И1}\cdot\text{П}+\text{И2}\cdot \text{Д}, \%  \,\,\,\, \text{где:}
%\end{equation}
%
%\begin{itemize}
%	\item [] $ \text{И1} $ --усредненный показатель износа на 1000 км пробега, \%; 
%	\item [] $ \text{П} $ -- общий пробег (фактический или расчетный) за срок эксплуатации КТС, тыс.км;
%	\item [] $ \text{И2} $ -- усредненный показатель старения за 1 год эксплуатации, \%;
%	\item [] $ \text{Д} $ -- срок эксплуатации КТС (от даты изготовления КТС до момента, на который определяется износ), лет. 
%\end{itemize}
%
%Для исследуемого автомобиля \тс, согласно справочным таблицам [1]:
%\begin{equation}\label{eqsnosr}
%\text{И} =\text{И1}\cdot\text{П}+\text{И2}\cdot \text{Д} = 0.23\cdot 214  + 0.85\cdot 9 = 49 \, \%
%\end{equation}
%%%%%%%%%%%%%%%%%%%%%%%%%%%%%%%%%%%%%%%%%%%%%%%%%%%%%%%%%%%%%%%%%%%%%%%%%%%%%%%%%%%%%%%%%%%%%%%%%%%%%%%%%%%%%%%%
Стоимость восстановительных работ $ C_{\text{вр}} $ определяется на основании норм трудоёмкостей $ T_i $, \, предусмотренных заводом-изготовителем, и стоимостных параметров $ C_{i\text{нч}} $ (стоимости нормо-часа) работ по техническому обслуживанию и ремонту АМТС.  Расчет размера расходов (в рублях) на восстановительный ремонт производится по формуле:
\begin{equation}\label{eq:cr}
C_{\text{вр}}  =\sum{C_{ip}}= \sum\left({T_{ij}}\cdot {C_{i\text{нч}}}\right) + \sum{C_{ip^{\text{\,\,\,руб}}}} , \,\,\,\text{где:} 
\end{equation}
%\vspace{2mm}
\begin{itemize}
	\item[ ]$ C_{ip} $ -- стоимость работ i-го вида: $C_\text {зам} $, $ C_\text{восст} $, $ C_\text{рег} $, $C_\text{контр} $, $ C_\text{антикор} $, $ C_\text{зч} $, $ C_\text{ом} $,$ C_\text{соп} $, $ C_\text{вм} $, руб;
	\item[ ]$ T_{ij} $ -- трудоёмкость j-й операции(комплекса) по i-му виду работ, руб;
	\item[ ]$ C_{i\text{нч}} $ -- стоимость нормо-часа по i-му виду работ, руб;
	\item[ ]$ C_{ip^\text{\,\,руб}} $ -- стоимость работ $ C_{ip} $, принятая непосредственно в денежном выражении, руб.
\end{itemize}
%
\par При определении стоимости восстановительного ремонта АМТС с учётом износа под износом следует понимать количественную меру физического старения АМТС и его элементов, достигнутого в результате эксплуатации, т.е. эксплуатационный износ.
%
Расчёт износа производится в  соответствии с Положением Банка России от «19» сентября 2014 года № 432-П «О единой методике определения размера расходов на восстановительный ремонт в отношении повреждённого транспортного средства» [2].
Износ комплектующих изделий (деталей, узлов, агрегатов) рассчитывается по следующей формуле:
%
\begin{equation}\label{eq:I}
\text{И}_{\text{ки}} 
= 100\cdot\left( 1-e^ {-\left( \Delta_{T} \cdot T_{\text{КИ}} + \Delta_{L} \cdot L_{\text{КИ}} \right)}\right), \,\,\,\,\text{где:}   
\end{equation}
%
\begin{itemize}
	\item[ ]$ \text{И}_{\text{ки}} $ -- износ комплектующего изделия (детали, узла, агрегата) (процентов); 
	\item[ ]$ e $ -- основание натуральных логарифмов (e =  2,72);
	\item[ ]$ \Delta_{T}$ --  срок эксплуатации комплектующего изделия (детали, узла, агрегата) (лет);
	\item[ ]$ T_{\text{КИ}} $ -- стоимость работ $ C_{ip} $, принятая непосредственно в денежном выражении, руб
	\item[ ]$ \Delta_{L} $ --коэффициент, учитывающий влияние на износ комплектующего (детали, узла, агрегата) величины пробега транспортного средства с этим комплектующим изделием;
	\item[ ]$ L_{\text{КИ}} $ --пробег транспортного средства на дату дорожно-транспортного происшествия (тысяч километров).  
		\end{itemize}
\vspace{5mm}
\par Значения коэффициентов $ \Delta_{T}$  и $ \Delta_{L} $  для различных категорий и марок транспортных средств приведены в п.5. сп. лит~[2]. При этом, на комплектующие изделия (детали, узлы, агрегаты), которые находятся в заведомо худшем состоянии, чем общее состояние транспортного средства в целом, и его основные части, вследствие влияния факторов, не учтённых при расчете износа (например, проведение ремонта с нарушением технологии, не устранение значительных повреждений лакокрасочного покрытия), может быть начислен дополнительный индивидуальный износ. 
Износ шины транспортного средства рассчитывается по следующей формуле:
\begin{equation}\label{eq:sh}
\text{И}_{\text{ш}} = \frac{\text{Н}_{\text{н}}-\text{Н}_{\text{ф}}}{\text{Н}_{\text{н}}-\text{Н}_{\text{доп}}} \cdot{100}\%  \,\,\,\,\text{где:} 
\end{equation}
%
\begin{itemize}
	\item[ ] $ \text{И}_{\text{ш}} $ -- износ шины, \%;
	\item[ ] $ \text{Н}_{\text{н}} $ -- высота рисунка протектора новой шины, мм;
	\item[ ] $\text{Н}_{\text{ф}} $ -- фактическая высота рисунка протектора шины, мм;
	\item[ ] $ \text{Н}_{\text{доп}} $ --минимально допустимая высота рисунка протектора шины в соответствии с требованиями законодательства Российской Федерации, мм.
\end{itemize}
%
\vspace{5mm}
\renewcommand\baselinestretch{1}\small\normalsize
%
%Износ шины дополнительно увеличивается для шин с возрастом от 3 до 5 лет - на 15 процентов, свыше 5 лет - на 25 процентов.
%%%%%%%%%%%%%%%%%%%%%%%%%%%%%%%%%%%%%%%%%%%%%%%%%%%%%%%%%%%%%%%%%%%%%%%%%%%%%%%%%%%5
%В результате исследования   экспертом-техником установлено, что для устранения повреждений \тс \, необходимо  выполнить следующие  работы:
%\begin{center}
%	\begin{tabulary}{\textwidth}{LCL}
%\hline 
%\textbf{Наименование детали}      &   & \textbf{Ремонтное воздействие}\\
%\hline Турбина левая              &   &    Заменить\\
%Блок ДВС                          &   &    Отремонтировать гильзованием, заменой колец и прокладок \\
%	\end{tabulary}  
%\end{center}
%%%%%%%%%%%%%%%%%%%%%%%%%%%%%%%%%%%%%%%%%%%%%%%%%%%%%%%%%%%%%%%%%%%%%%%%%   CSV    %%%%%%%%%%%%%
%\csvautobooklongtable[separator=comma]{test.csv} % semicolon- ;  comma- , pipe- | tab- 	
%%%%%%%%%%%%%%%%%%%%%%%%%%%%%%%%%%%%%%%%%%%%%%%%%%%
%%%%%%%%%%%%%%%%%%%%%%%%%%%%%%%%%%%%%%%%%%%%%%%%%%%%%%%%%%%%%%%%%%%%%%%%%%%%%%%%%
%
%%%%%%%%%%%%%%%%%%%%%%%%%%%%%%%%%%%%%%%%%%%%%%%%%%%%%%%%%%%%%%%%%%%%%%%%%%%%%%%%%
%
%%%%%%%%%%%%%%%%%%%%%%%%%%%%%%%%%%%%%%%%%%%%%%%%%%%%%%%%%%%%%%%%%%%%%%%%%%%%%%%%5
\renewcommand\baselinestretch{1.2}\small\normalsize
%%
%\textbf{Произвести  необходимые для выполнения  ремонта разборочно-сборочные, подготовительные и вспомогательные работы в соответствии с требованиями завода–изготовителя транспортного средства.}\\
%%
Расчет стоимости восстановительных расходов выполнен в программе \auda\, в модуле ОСАГО ПРО. Процент износа ТС \тс\, так же произведен в указанной программе, на момент  исследуемого страхового события составлял 22.96\%. \par Ниже представлены результаты расчета, полная калькуляция стоимости ремонта включена в раздел <<Приложение>> настоящего заключения.
%%\smallskip
\begin{figure}[H]
	\centering
	\includegraphics[width=0.8\linewidth]{images/Screenshot_2}
%%	\caption{}
%%	\label{fig:screenshot001}
\end{figure}
%\begin{figure}[H]
%	\centering
%	\includegraphics[width=0.9\linewidth]{images/screenshot002}
%%%	\caption{}
%	\label{aud}
%\end{figure}
\medskip
\renewcommand\baselinestretch{1.2}\small\normalsize
%%%%%%%%%%%%%%%%  Если не ОСАГО
%Стоимость коммерческого нормо-часа работ применена  с учетом условий регионального рынка услуг и сложившихся средних расценок по видам работ, типу ТС, а также по маркам и моделям ТС  и   составляет  1300 р/ч для данного транспортного средства. \\
%Трудоёмкость работ по разборке/сборке/замене  соответствует трудоемкости работ, рекомендованной заводом изготовителем ТС. 
%Расчет стоимости ремонта, согласно положениям [1] производится с учетом  применения оригинальных запасных частей, которые поставляются изготовителем КТС авторизованным ремонтным организациям. Техническое состояние запасных частей учитывается коэффициентом износа, что в совокупности с установкой оригинальных запасных частей в максимальной степени отвечает понятию «восстановительный ремонт», то есть восстановления состояния КТС, при котором используются установленные изготовителем составные части, но с использованным частично ресурсом. 
%
%%%%%%%%%%%%%%%%%% Если ОСАГО
Стоимость одного нормо-часа работ определена в соответствии с пунктом 3.8.1 Единой методики [2] путем применения электронных баз данных стоимостной информации.
Трудоемкость работ по разборке/сборке/замене  соответствует трудоемкостям работ, рекомендованным заводом изготовителем ТС. Трудоемкости окрасочных работ приняты в соответствии с технологией  AZT (http://www.schwacke.ru/down/azt \_reparaturlackierung\_ru.pdf). Расчет размера расходов на материалы произведен  согласно пункту 3.7.2 Приложения к Единой методике [2]. 
Стоимость запасных частей определена в соответствии с пунктом 3.6.3 Единой методики путем применения электронных баз данных стоимостной информации.
% (по утвержденному справочнику: http://prices.autoins.ru/priceAutoParts/repair\_parts.html.
%
\par Таким образом, в результате проведенных расчетов (см. Приложение, калькуляция \NomerDoc) определена стоимость восстановительного ремонта транспортного средства  \тс, которая составляет 1\,085\,696 (Один миллион восемьдесят пять тысяч шестьсот девяносто шесть) рублей без учета износа,
стоимость восстановительных расходов, с учетом уменьшения стоимости запасных частей вследствие их износа,  составляет 871\,841 (Восемьсот семьдесят одна тысяча восемьсот сорок один) рубль, что с учетом округления составляет 871\,800 (Восемьсот семьдесят одна тысяча восемьсот) рублей.
\nopagebreak
\include{rinok}
\include{go}
%
\section{Выводы}
%
%
\begin{enumerate}
 \item  Наличие, характер и объем (степень) технических повреждений, причиненных ТС, определены при осмотре и зафиксированы в Акте осмотра \NomerDoc.
 % \, и фототаблице повреждений, являющимися неотъемлемой частью настоящего экспертного заключения.
 \\[-2mm]

%\item  Направление, расположение и характер повреждений определены путем сопоставления полученных повреждений, изучения административных материалов по рассматриваемому событию, и  являются  следствиями рассматриваемого ДТП (события).\\[-2mm]

\item  Технология и объем необходимых ремонтных воздействий зафиксированы в калькуляции \NomerDoc \, по определению стоимости восстановительного ремонта транспортного средства \тс. Расчетная стоимость восстановительного ремонта составляет 1\,085\,696 (Один миллион восемьдесят пять тысяч шестьсот девяносто шесть) рублей.
\\[-2mm]
\item  Размер затрат на проведение восстановительного ремонта с учётом износа (восстановительные расходы) транспортного средства \тс \, составляет 871\,800 (Восемьсот семьдесят одна тысяча восемьсот) рублей  рублей.\\[-2mm]   
\item Рыночная стоимость транспортного средства ТС \тс\, на момент повреждения составляла  980000 (Девятьсот восемьдесят тысяч рублей). \\[-2mm]
\item Стоимость годных остатков ТС \тс \, \, составляет 370 000 (Триста семьдесят тысяч) рублей.
\end{enumerate}
\vspace{15mm}
\relax
Приложение к заключению:\\
\textit{
	Приложение № 1. Расшифровка модельных опций ТС \тс \\
	Приложение № 2. Калькуляция стоимости восстановительных расходов ТС \тс\\
%	Приложение № 3. Цифровые копии регистрационных документов ТС\\
%	Приложение № 4. Цифровая копия постановление по делу об административном правонарушении дорожно-транспортном происшествии\\
	Приложение № 3. Правоустанавливающие документы\\
	   }

\vspace{20mm}
%\newlength{\ML}
{Эксперт-техник}\hfill           {Мраморнов А.В.}


%\includepdf[pages=-]{myfile.pdf}

%\includepdf[pages=-]{calc.pdf}