\subsection{Расчёт утраты товарной стоимости ТС}

\par В целях обеспечения единства практики применения судами законодательства, регулирующего отношения в области обязательного страхования гражданской ответственности владельцев транспортных средств, Пленум Верховного Суда Российской Федерации, руководствуясь статьей 126 Конституции Российской Федерации, статьями 2 и 5 Федерального конституционного закона от 5 февраля 2014 года N 3-ФКЗ "О Верховном Суде Российской Федерации", постановляет дать следующие разъяснениz
Общие положения Постановление Пленума Верховного Суда Российской Федерации от 26 декабря 2017 г. N 58 г. Москва "О применении судами законодательства об обязательном страховании гражданской ответственности владельцев транспортных средств" 

п. 37. К реальному ущербу, возникшему в результате дорожно-транспортного происшествия, наряду со стоимостью ремонта и запасных частей относится также утрата товарной стоимости, которая представляет собой уменьшение стоимости транспортного средства, вызванное преждевременным ухудшением товарного (внешнего) вида транспортного средства и его эксплуатационных качеств в результате снижения прочности и долговечности отдельных деталей, узлов и агрегатов, соединений и защитных покрытий вследствие дорожно-транспортного происшествия и последующего ремонта.

 Утрата товарной стоимости подлежит возмещению и в случае, если страховое возмещение осуществляется в рамках договора обязательного страхования в форме организации и (или) оплаты восстановительного ремонта повреждённого транспортного средства на станции технического обслуживания, с которой у страховщика заключён договор о ремонте транспортного средства, в установленном законом пределе страховой суммы.

\par Расчёт утраты товарной стоимости в настоящем исследовании производится согласно \emph {Методическим рекомендациям по проведению судебных автотехнических экспертиз и исследований колёсных транспортных средств в целях определения размера ущерба, стоимости восстановительного ремонта и оценки}  [6].

\par Утрата товарной стоимости (УТС) обусловлена снижением товарной стоимости из-за ухудшения потребительских свойств вследствие наличия дефектов (повреждений), или следов их устранения либо наличия достоверной информации, что дефекты (повреждения) устранялись [6, п. 8].

	УТС может быть рассчитана для ТС, находящихся как в повреждённом, так и в отремонтированном состоянии (при возможности установить степень повреждения).

УТС может определяться при необходимости выполнения одного из нижеперечисленных видов ремонтных воздействий или если установлено их выполнение:

-	устранение перекоса кузова или рамы ТС;

-	замена несъемных элементов кузова ТС (полная или частичная); ремонт съёмных или несъемных элементов кузова (включая оперение) ТС (в том числе пластиковых капота, крыльев, дверей, крышки багажника);

-	полная или частичная окраска наружных (лицевых) поверхностей кузова (включая оперение) ТС, бамперов;

-	полная или частичная разборка салона ТС, вызывающая нарушение качества заводской сборки.

УТС не рассчитывается:

а)	если срок эксплуатации легковых автомобилей превышает 5 лет;

б)	если легковые автомобили эксплуатируются в интенсивном режиме, а срок эксплуатации превышает 2,5 года;


в)	в случае замены кузова до оцениваемых повреждений (за исключением кузова грузового КТС, установленного на раме за кабиной);

%)	если КТС ранее подвергалось восстановительному ремонту (в том числе окраске - полной, наружной, частичной; «пятном с переходом») или имело аварийные повреждения, кроме повреждений, указанных в [6, п. 8.4];

д)	если КТС имело коррозионные повреждения кузова или кабины на момент происшествия.



Нижеприведенные повреждения не требуют расчёта УТС вследствие исследуемого происшествия, а их наличие до исследуемого происшествия не обуславливает отказ от расчёта УТС при таких повреждениях:

а)	эксплуатационных повреждениях ЛКП в виде меления, трещин, а также повреждений, вызванных механическими воздействиями - незначительных по площади сколов, рисок, не нарушающих защитных функций ЛКП составных частей оперения;

б)	одиночного эксплуатационного повреждения оперения кузова (кабины) в виде простой деформации, не требующего окраски, площадью не более 0,25 дм2;

в)	повреждения, которые приводят к замене отдельных составных частей, которые не нуждаются в окрашивании и не ухудшают внешний вид КТС (стекло, фары, бампера неокрашиваемые, пневматические шины, колёсные диски, внешняя и внутренняя фурнитура и т. п.). Если, кроме указанных составных частей, повреждены составные части кузова, рамы, кабины или детали оперения - крылья съёмные, капот, двери, крышка багажника, - то расчёт величины УТС должен учитывать все повреждения составных частей в комплексе;

г)	в случае окраски молдингов, облицовок, накладок, ручек, корпусов зеркал и других мелких наружных элементов, колёсных дисков.

В случае исследуемого события для автомобиля \тс\, VIN \vin\,  условия,  при которых производится расчёт УТС выполняются.\\


\par Величина УТС зависит от вида, характера и объёма повреждений и ремонтных воздействий по их устранению.
\par Величина УТС ($ C_\text{YTC} $) определяется на дату оценки (исследования) по формуле: 

\begin{equation}\label{uts}
C_{YTC} = C_{TC} \cdot \dfrac{\sum\limits_{i=1}^n K_{YTCi}}{100\%},\hspace{5mm} \text{руб.},
\end{equation}

\noindent где:\\
\noindent $ C_{TC} $ -- стоимость ТС на дату оценки (исследования), руб;\\
$ K_{YTCi} $ -- коэффициент УТС по i-му элементу КТС, ремонтному воздействию, \%.
 
\par Рыночная стоимость транспортного средства ( $ C_{TC} $ ), согласно п.6.1. Единой методики [3], принимается равной средней стоимости аналога на указанную
дату по данным имеющихся инфор\-мационно-справочных материалов,
содержащих сведения о средней стоимости транспортного средства.

\par  При ремонте съёмной составной части сумма стоимости ремонта (включая стоимость разборки для ремонта и при необходимости снятия детали для ремонта) и величины УТС (без учёта УТС вследствие окраски) не должна превышать суммы стоимости этой составной части (с учётом коэффициента износа) и стоимости работ по ее замене.

\par   Значение коэффициента УТС $ K_{\text{утсокр}} $ при подетальной окраске наружных поверхностей кузова ТС рассчитывается с учётом количества окрашиваемых кузовных составных частей и бамперов по формуле:

\begin{equation}\label{f:yc}
K_{\text{утсокр}}=K_{\text{утсокр(1)}}+K_{\text{утсокр(N-1)}}\cdot(N-1), \hspace{5mm} \% 
\end{equation}
        
\noindent где:\\
\noindent $ \text{К}_{\text{утсокр(1)}} $ - коэффициент УТС по окраске первой кузовной составной части или бампера, \%;\\
$ \text{К}_{\text{утсокр(N-1)}} $ - коэффициент УТС по окраске второй и каждой следующей кузовной составной части или бампера, \%;\\
N - количество окрашиваемых составных частей, по которым рассчитывается УТС.\\
Значения коэффициентов УТС ($ K_{YTC} $) определены по результатам экспертной практики и приведены в приложении [6, Приложение 2.9].

\par Для исследуемого автомобиля \тс \, соответствующие ремонтным воздействиям  коэффициенты УТС приведены ниже в таблице:

\begin{table}[H]
		%\caption{}
	\begin{tabular}{|p{5mm}|p{80mm}|c|c|c|}
	\hline 
	\textbf{п/п} & \textbf{Наименование детали} &\textbf{ К-замена }& \textbf{К-ремонт }&\textbf{ К-окраска} \\ 
	\hline 
	1 & Дверь задняя левая & -- & -- & 0,5 \\ 
	\hline 
%	2 & Бампер задний & -- & -- & 0,35 \\ 
%	\hline 
	2 & Боковина левая & -- & 0,2 & 0,35 \\ 
	\hline 
		3 & Порог левой боковины  & -- & 0,2 & 0,35 \\ 
	\hline 

	
\end{tabular} 

\end{table}

\vspace{7mm}

$  \sum\limits_{i=1}^n K_{YTCi} = 0.5+0.2+0.35+0.2+0.35 = 1.6$\\
  
  
Рыночная стоимость транспортного средства \тс\, на момент повреждения по данным справочника \url { https://automama.ru/krasnodar/cars/ } составляет 660 000 (Шестьсот шестьдесят тысяч) рублей.
  
$   C_{TC} = C_{TC} \cdot \dfrac{\sum\limits_{i=1}^n K_{YTCi}}{100} = 660000 \cdot 1.6/100 = 10560 $%, или с учетом округления 372000 (Триста семьдесят две тысячи) рублей.\\
(Десять тысяч пятьсот шестьдесят) рублей.

\par Таким образом, величина УТС автомобиля \тс\, составляет (Десять тысяч пятьсот шестьдесят) рублей.