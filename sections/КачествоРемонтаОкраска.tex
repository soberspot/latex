\setcounter{page}{1}
\clubpenalty=10000 
\widowpenalty=10000
%%%%%%%%%%%%%%%%%%%%%%%%%%%%%%%%%%%%%%%%
%      Шапка экспертной организации  
%%%%%%%%%%%%%%%%%%%%%%%%%%%%%%%%%%%%%%%%
%
%%%%%%%%%%%%%%%%%%%%%%%%%%%%%%%%%%%%%%%%%%
%
%   Экспертная организация ООО Южнорегиональная экспертная группа
%
%%%%%%%%%%%%%%%%%%%%%%%%%%%%%%%%%%%%%%%%%
\noindent %\qrcode[height=21mm]{\NomerDoc от \окончено }  %%% Добавлен QR-Code
\begin{pspicture}(21mm,21mm)
\obeylines
\psbarcode{%
	%\NomerDoc от \окончено
	BEGIN:VCARD^^J
	VERSION:4.0^^J
	%N:Мраморнов; Александр; Вчеславович^^J
	FN:Александр Мраморнов^^J
%	ORG:IP Alexandr Mramornov^^J
	TITLE: эксперт
	ORG: ИП
	URL:http://www.yourexp.ru^^J
	EMAIL:4516611@gmail.com^^J
	TEL:+7-918-451-6611^^J
	ADR:г. Краснодар, с/т № 2 А/О «Югтекс», ул. Зеленая, 472^^J
	END:VCARD
}{width=1.0 height=1.0}{qrcode}%
\end{pspicture}
\begin{center}
	\normalsize\textbf{$\cdots$\\[-1.5mm] <<$\cdots$>> \\[-5mm]}
	%  
	\noindent\rule{\textwidth}{1pt}\\[-6mm]  % Горизонтальная линия
	% \line(1,0){460}% (1,0) -горизонтальная линия, и (0,1) - вертикальная 
\end{center}

\begin{center}
	\begin{footnotesize}\setstretch{0.3}
		%	\small\textbf\setlength   	%\raisebox{5mm}
		\vspace{-3.5mm}$\cdots$\\[0mm]
		Телефон: \quad $\cdots$, e-mail:\quad $\cdots$\\ [-2mm]{$\cdots$\quad$\cdots$}
	\end{footnotesize}	\\[10mm]
\end{center}


\begin{flushright}
	%Краснодар,
	$\cdots$, 2020    \\[8mm]
\end{flushright}
\begin{center}
	\LARGE\textbf{ ЗАКЛЮЧЕНИЕ ЭКСПЕРТА}
	\bigskip\\[0mm]
	%	{\normnumxtbf{\NomerDoc}}	}{den}
\end{center}
\par
\vspace{-3mm}\noindent по гражданскому делу \delonum \, по иску \isk \\[0mm]

%\raggedright 
%\def\hrf#1{\hbox to#1{\hrulefill}}
\noindent \textbf{№ $\cdots$}\hfill           \textbf{\окончено}\\%[2mm]
%Приостановлено\hfill      \datastop\\
%Возобновлено\hfill          \datarestart\\
%Окончено\hfill                \dataend\\%[4mm]

\noindent\parbox[l][16mm]{16.5cm}
{\def\hrf#1{\hbox to#1{\hrulefill}}
	\noindent Начато\hfill            \datastart\\%[2mm]
	%	Приостановлено\hfill      \datastop\\
	%	Возобновлено\hfill          \datarestart\\
	Окончено\hfill                \окончено\\%[4mm]
}
\relax

\datastart г. ~в {\small $\cdots$} \,  при определении  \, \sud  \,  от \, \dataopr \, о назначении \opr \, по гражданскому делу \delonum \, поступили:

\begin{enumerate}\setlist{nolistsep}\item  Материалы гражданского дела \delonum \\[-2mm]
	%	\item  
\end{enumerate}Экспертиза произведена экспертом  $\cdots$  
%%%%%%%%%%%%%%%%%%%%%%%%%%%%%%%%%%%%%%%%%%
%
%   Экспертная организация ООО Южнорегиональная экспертная группа
%
%%%%%%%%%%%%%%%%%%%%%%%%%%%%%%%%%%%%%%%%%
\noindent %\qrcode[height=21mm]{\NomerDoc от \окончено }  %%% Добавлен QR-Code
\begin{pspicture}(21mm,21mm)
\obeylines
\psbarcode{%
%	\NomerDoc от \окончено
	BEGIN:VCARD^^J
	VERSION:4.0^^J
	%N:Мраморнов; Александр; Вчеславович^^J
	FN:Александр Мраморнов^^J
%	ORG:IP Alexandr Mramornov^^J
	TITLE: эксперт
	ORG: ИП
	URL:http://www.yourexp.ru^^J
	EMAIL:4516611@gmail.com^^J
	TEL:+7-918-451-6611^^J
	ADR:г. Краснодар, с/т № 2 А/О «Югтекс», ул. Зеленая, 472^^J
	END:VCARD
}{width=1.0 height=1.0}{qrcode}%
\end{pspicture}%%% Добавлен QR-Code
\vspace{-4mm}
\begin{center}
	\large\textbf{ИНДИВИДУАЛЬНЫЙ\quad ПРЕДПРИНИМАТЕЛЬ  \\[-1.5mm] МРАМОРНОВ  АЛЕКСАНДР ВЯЧЕСЛАВОВИЧ \\[-5.5mm]}
	%  
	\noindent\rule{\textwidth}{2pt}\\[-6mm]  % Горизонтальная линия
	% \line(1,0){460}% (1,0) -горизонтальная линия, и (0,1) - вертикальная 
\end{center}

\begin{center}
	\begin{footnotesize}\setstretch{0.3}
		%	\small\textbf\setlength   	%\raisebox{5mm}
		\vspace{-3.5mm}г. Краснодар, с/т № 2 А/О «Югтекс», ул. Зеленая, 472, 
		Телефон: 8-918-451-66-11, e-mail: 4516611@gmail.com\\ [-2mm]{ИНН\quad 231200665168\quad ОГРНИП \quad 310231220400043}
	\end{footnotesize}	\\[10mm]
\end{center}


\begin{flushright}
% 
	 \hfill	Краснодар, 2020    \\[8mm]
\end{flushright}
\begin{center}
	\LARGE\textbf{ЭКСПЕРТНОЕ ЗАКЛЮЧЕНИЕ}
	\bigskip\\[0mm]
	%	{\normnumxtbf{\NomerDoc}}	}{den}
\end{center}
\par
\vspace{-6mm}
\noindent независимой технической экспертизы по определению размера расходов на восстановительный ремонт транспортного средства   \тс  \\[2mm]

%\raggedright 
%\def\hrf#1{\hbox to#1{\hrulefill}}
\noindent \textbf{№ \NomerDoc}\hfill           \textbf{\окончено}\\%[2mm]
%Приостановлено\hfill      \datastop\\
%Возобновлено\hfill          \datarestart\\
%Окончено\hfill                \dataend\\%[4mm]

\noindent\parbox[l][16mm]{16.5cm}
{\def\hrf#1{\hbox to#1{\hrulefill}}
	\noindent Начато\hfill            \datastart\\%[2mm]
	%	Приостановлено\hfill      \datastop\\
	%	Возобновлено\hfill          \datarestart\\
	Окончено\hfill                \окончено
}
\relax

%%%%%%%%%%%% Если судебка
%
%\datastart г. ~в {\small ООО~ "ЮЖНО-РЕГИОНАЛЬНАЯ ЭКСПЕРТНАЯ ГРУППА"} \,  при определении  \, \sud  \,  от \, \dataopr \, о назначении \opr \, по гражданскому делу \delonum \, поступили:
%
%\begin{enumerate}\setlist{nolistsep}\item  Материалы гражданского дела \delonum \, в двух томах, том 1 на 276 листах, том 2  на 143 листах.\\[-2mm]
%	%	\item  
%\end{enumerate}
%
%%%%%%%%%%%%  Если независимая
\vspace{4mm}
Составлено на основании	договора № \NomerDoc\,  возмездного оказания услуг по проведению независимой технической экспертизы (далее экспертиза)  транспортного средства и письменного заявления заказчика о проведении экспертизы.

Заказчик  экспертизы: \заказчик, \адресзаказчика. 

Полис ОСАГО: \polis.

% Документ, удостоверяющий личность заказчика: водительское удостоверение    03\ 16\ 422344\ выдан 09.06.2011

%Транспортное средство виновника ДТП:  не предоставлялось.

\paragraph*{}
Экспертиза произведена  экспертом--техником
%{\small ООО "ЮЖНО-РЕГИОНАЛЬНАЯ ЭКСПЕРТНАЯ ГРУППА"}
\,  Мраморновым Александром Вячеславовичем, имеющим высшее техническое образование по специальности «техническая физика», диплом РВ № 311964 от 28.02.1989, квалификация -- инженер-физик, специальное образование в области оценки: Диплом ПП-1 № 037211 Российской экономической академии им. Г.В. Плеханова, квалификация -- оценка и экспертиза объектов и прав собственности, специальное образование в области независимой технической экспертизы транспортных средств: Диплом ПП-I № 424167, квалификация: эксперт-техник (специализация 150210 специальности 190601.65 – Автомобили и автомобильное хозяйство), состоящий в Государственном реестре экспертов-техников (№ в реестре 256, https://data.gov.ru/opendata/7707211418-experts,  общий трудовой  стаж 30 лет, стаж  экспертной работы  12 лет. \par Заключение подготовлено по месту фактического расположения ИП по адресу: г. Краснодар, с/т № 2 А/О «Югтекс», ул. Зелёная, 472.

\include{titul/шапкаИПзаключениеспециалиста}
%
%%   вопросы экспертизы
\section{Вопросы экспертизы}
%Заказчик поручает, а Исполнитель принимает на себя обязательство выполнить Заказчику  комплекс работ в виде автотехнических исследований автомобиля Mazda 6, VIN RUMGJ52\-6802007133 (дата начала гарантии 07.05.2018 г.), по следующим вопросам:
\begin{enumerate}


\item Соответствуют ли проведённые ремонтные работы перечню работ, указанных в заказ-наряде ВВ00014034 от 28.10.2019?

	\item  Имеются ли недостатки выполненных  индивидуальным предпринимателем Касирум Виктором Валентиновичем работ согласно заказ-наряда ВВ00014034 от 28.10.2019  в рамках договора  № 77 от 18 октября 2019, заключённого между индивидуальным предпринимателем Касирум Виктором Валентиновичем и ООО "АвтоСтар-Юг" на   оказания услуг по техническому обслуживанию и ремонту автотранспорта?	

%	\item Если 	недостатки имеются, то какие?

	\item Какова стоимость их устранения?
\end{enumerate}

\addcontentsline{toc}{section}{Использованн=е нормативы и источники информации}
%
%\left( \addcontentsline{toc}{section}{Использованные нормативы и источники информации}

\subsection{Использованные нормативы и источники информации}
%
\begin{enumerate}
\item 
Махнин\,Е.\,Л., Новоселецкий\, И.\,Н., Федотов\, С.\,В. \emph{Методические рекомендации по проведению судебных автотехнических экспертиз и исследований колёсных транспортных средств в целях определения размера ущерба, стоимости восстановительного ремонта и оценки} // -- М.: ФБУ РФЦСЭ при Минюсте России, 2018.-326 с.  ISBN 978-5-91133-185-6.
%
%
%
%
\item ТУ 017207-255-00232934-2014 \emph{Кузова автомобилей LADA. Технические требования при приёмке в ремонт, ремонте и выпуске из ремонта предприятиями дилерской сети ОАО "АВТОВАЗ"}//  ОАО НВП "ИТЦ АВТО", 2014
%
\item Смирнов  В.Л., Прохоров  Ю.С., Боюр В.С.  и др. \emph{Автомобили ВАЗ. Кузова. Технология ремонта, окраски и  антикоррозионной защиты. Часть II}// - Н.Новгород: АТИС, 2001.- 241с.
%
\item 
Савич Е.Л. \emph{Техническое  обслуживание  и  ремонт  легковых  автомобилей} : учеб. пособие / Е.Л. Савич, М.М. Болбас, В.К. Ярошевич ; под общ. ред. Е.Л. Савича. -Мн. : Вышэйшая школа,  2001. - 479 с. - ISBN985-06-0502-2.
%
\item 
Автомобили ВАЗ-2121, 21213, 21214, 2131 и их модификации: <<Трудоемкости работ (услуг) по техническому обслуживанию и ремонту>> /Куликов А.В., Христов П.Н., Климов В.Е.,  Боюр В.С., Рева В.В., Зимин В.А., Завьялова Н.Н., Хлыненкова Г.А. -- ИТЦТ "АвтоВАЗтехобслуживание", Тольяти -- 2005. 
%
\item
Автомобили LADA SAMARA и их модификации: <<Трудоемкости работ (услуг) по техническому обслуживанию и ремонту>> /Куликов А.В., Христов П.Н., Климов В.Е., Рева В.В., Боюр В.С., Васильев М.В., Фахрутдинов Р.В.,  Прудских Д.А., Гирко В.Б., Шмелева В.А., Зимин В.А. --  ОАО НВП "ИТЦ АВТО",  -- 2006. - 252 стр.
%
\item 
Автомобили LADA PRIORA. Трудоемкости работ (услуг) по техническому обслуживанию и ремонту /Куликов А.В., Христов П.Н., Климов В.Е., Рева В.В., Козлов П.Л., Боюр В.С., Прудских Д.А., Шмелева В.А., Зимин В.А. -- ООО "ИТЦТ АВОСФЕРА", Тольяти -- 2009. -- 344 с.
%
\item 
{Трудоемкости работ по техническому обслуживанию и ремонту автомобилей автомобилей Lada  Granta}/   \url{https://docplayer.ru/30250248-Trudoemkosti-rabot-po-teh\-nicheskomu-obsluzhivaniyu-i-remontu-avtomobiley-lada- granta.html}.
%
%
\item
{Специализированное программное обеспечение для расчёта стоимости  восстановительного ремонта, содержащее нормативы трудоёмкости работ, регламентируемые изготовителями транспортного средства}//   AudaPadWeb, лицензионное соглашение № AS/APW-658  RU-P-409-409435.
%
%
%
\item
{Специализированное программное обеспечение для расчёта стоимости  восстановительного ремонта, содержащее нормативы трудоёмкости работ, регламентируемые изготовителями транспортного средства ОАО «АвтоВАЗ», ЗАО «Джи-Эм-АвтоВАЗ», ОАО «СеАЗ» и ОАО «ЗМА»}//   Автосфера АС:Смета, v.3.9.11// ООО "ИТЦ «ИнтегроМаш», \url{https://autosmeta.pro}.
%
%
%
\item Информационный портал по техническому обслуживанию и ремонту автомобилей	 ВАЗ:\\ \url{www.autosphere.ru}.

%%
\end{enumerate}

%%%%%%%%%%%%%%%%%%%%%%%%%%%%%%%%%%%%%%%%%%%%%%%%%%%%%%%%%%%%%%%%%%%%%%%%%%%%%%%%%
\subsection{Технические средства}  %% Список не удалять!!!
\begin{itemize}
%
%%
%\item Диагностический сканер BOSH VCM II S/N 1324-88682639 c програмным обеспечением Mazda IDS - 115.02
%%\item   Диагностический сканер SDconnect   с программным обеспечением Xentry Diagnostics v19.11.3.1
\item Толщиномер лакокрасочных покрытий Etari ET-111 s/n 1301876 для измерения толщины лакокрасочного покрытия неразрушающим методом с диапазоном измерений от 0 до 2000 мкм с набором калибровочных мер
\item   Линейка масштабная магнитная с цветографической шкалой, 100мм
\item   Линейка измерительная металлическая 100мм, ГОСТ 427-75, цена деления мм, пг ± 0,15мм
\item  Лупа просмотровая ЛП-$ 5^х $, ЛП-$ 10^х $ 
\item Набор деревянных чертёжных карандашей компании KOH-I-NOOR фирмы Hardmuth AG с твёрдостью 3B-2B-B-HB-F-H-2H-3H
%%\item   Рулетка измерительная металлическая, 5м
%%\item  Универсальный стенд для измерения углов установки колес Hunter Engineering %ProAlign с программным инструментом регулировки схождения колес без блокировки руля %автомобиля WinToe
\item 	Цифровой фотоаппарат Canon 760D s/n 143032001327 с объективом Canon EF-S 18-135, тип используемой памяти: Transcend,  32Gb
\item  Специализированное программное обеспечение для расчёта стоимости  восстановительного ремонта, содержащее нормативы трудоёмкости работ, регламентируемые изготовителями транспортного средства     AudaPadWeb, лицензионное соглашение № AS/\- APW-658  RU-P-409-409435
%\item Он-лайн программа моделирования кинематики подвески автомобиля // \url{http://www.vsusp.com/}
%\item Он-лайн ресурс проверки метаданных цифровых изображений \url{http://exif.regex.info/exif.cgi}
\item  Программа обработки изображений с открытым исходным кодом ImageJ, разработанная для научных многомерных изображений.  Wayne Rasband (wa-yne@codon.nih.gov),
свободная лицензия GPL.
\item  ПЭВМ под управлением операционной системы Windows 10 с установленным набором макрорасширений LaTeX системы компьютерной вёрстки TeX, свободная лицензия LaTeX Project Public License (LPPL). 
%	
\end{itemize}
%%%%%%%%%%%%%%%%%%%%%%%%%%%%%%%%%%%%%%%%%%%%%%%%%%%%%%%%%%%%%%%%%%%%%%%%%%%%%%%%%%%%%%%%%%%%%%%%%%%%%%
\subsection{Условные обозначения}

\begin{description}
%	 
%%\item[АВС] --антиблокировочная система
\item[АМТС] --автомототранспортное средство
%\item[ГРМ] -- газораспределительный механизм
%\item[ДВС] --двигатель внутреннего сгорания
\item[ДТП] --дорожно--транспортное происшествие
\item[гос.\,рег.\,знак] --государственный регистрационный знак
\item[КТС] --колёсное транспортное средство 
\item[ЛКП] --лакокрасочное покрытие
\item[л.д.] --лист дела
%%\item[Колесо турбины]  -- крыльчатка турбины
\item[ТС] --транспортное средство
%\item[ТK, ТКР] -- турбокомпрессор. Состоит из двух частей: турбины и компрессора, объединенных общим валом. Вал вращается в подшипниках, размещенных в центральном корпусе ТК
%\item[ЦПГ] -- цилиндро-поршневая группа
%\item[ЭБУ] --электронный блок управления
%%\item[FRAME] -- номер кузова транспортного средства, выпущенного для продажи на внутреннем рынке Японии и содержащий информацию производителя о транспортном средстве
%%\item[OBDII] -- On-board diagnostics. Протокол бортовой диагностики автомобиля
%%\item[SRS] -- Cистема пассивной защиты водителя и пассажиров
\item[VIN] --vehicle identification number, 17--значный идентификационный номер транспортного средства, соответствующий стандарту ISO 3779--2012.
%
\end{description}
%%%%%%%%%%%%%%%%%%%%%%%%%%%%%%%%%%%%%%%%%%%%%

\subsection{Методы исследования}

\begin{itemize}
\item  Органолептический метод – исследование и оценка качества объектов с помощью органов чувств
\item 	Прямой измерительный метод – путём измерения размеров деталей специальными %измерительными приборами
\item Расчётный метод (косвенный измерительный метод) – путём расчётов различных параметров на основе результатов измерений и других данных
\item Экспертный метод (метод экспертной оценки) — совокупности операций по выбору комплекса или единичных характеристик объекта, определению их действительных значений и оценкой экспертом соответствия их установленным требованиям и/или технической информации

\end{itemize}
%%%%%%%%%%%%%%%%
%
\subsection{Термины и определения}
\begin{description}
	\item[Аварийные повреждения] -- повреждения, механизм образования которых определяется контактом с посторонними объектами, что привело к деформации или разрушению и к необходимости ремонта или замены составной части, или контактам с агрессивной средой, которая привела к необходимости ремонта (замены) составной части [1, часть II, п. 1.5].
	%	\item[Восстановительный ремонт]-- один из способов возмещения ущерба, состоящий в выполнении технологических операций ремонта КТС, действующий в сети торгово-сервисного обслуживания, созданной изготовителем этого КТС [1, часть II, п. 1.4].
%	\item[Годные остатки] -- работоспособные, имеющие остаточную стоимость детали (агрегаты, узлы) поврежденного автотранспортного средства, годные к дальнейшей эксплуатации, которые можно демонтировать с поврежденного автотранспортного средства и реализовать.
	\item[Дата исследования]-- дата, на которую проводятся расчёты и используются стоимостные данные КТС, запасных частей, материалов, нормо-часа ремонтных работ [1, часть II, п. 1.5].
	\item [Декоративные свойства лакокрасочного покрытия] -- способность лакокрасочного покрытия придавать окрашенной
	поверхности заданный цвет и блеск
	\item[Дефект] -- неисправное состояние детали, узла, агрегата, характеризуемое выходом  параметров за допустимые пределы, но не делающее  неработоспособным. 
	\item [Защитные свойства лакокрасочного покрытия] --
	Способность лакокрасочного покрытия предотвращать или замедлять
	коррозию металлических или разрушение неметаллических поверхностей в
	условиях агрессивного воздействия внешних факторов.
	\item[Лак] -- продукт, который после нанесения на поверхность образует твёрдую прозрачную 	плёнку, обладающую защитными, декоративными или специальными техническими свойствами.
	\item[Лакокрасочное покрытие (ЛКП)] -- сплошное покрытие, полученное в результате нанесения 	одного или нескольких слоёв лакокрасочного материала на окрашиваемую поверхность
%	\item[Линия удара]-- линия, определяемая направлением вектора равнодействующего импульса сил, возникающих при контакте ТС при столкновении до прекращения взаимного внедрения деформирующихся при ударе частей. Положением линии удара на ТС определяются направление и величина момента импульса сил, возникающих при ударе, и, следовательно, направлением и интенсивность разворота ТС относительно центра масс после столкновения.  
%	\item[Моделирование]-- исследование каких-либо явлений, процессов или систем объектов путем построения и изучения их моделей.
\item[Малозначительный дефект] -- дефект, который существенно не влияет на использование продукции по назначению и ее долговечность
	\item[Морфологические признаки]-- признаки, отображающие внешнее и внутреннее строение объекта
	\item [Недостаток лакокрасочного покрытия] -- отклонение лакокрасочного покрытия от 	требований нормативно-технической документации, образовавшееся в процессе нанесения и
	формирования лакокрасочного покрытия (производственный недостаток)
	\item[Неустранимый дефект] -- дефект, устранение которого технически невозможно или экономически нецелесообразно.
	\item[Производственный (технологический) дефект] -- дефект, вызванный нарушением установленной технологии изготовления детали, узла, агрегата
	\item[Срок эксплуатации КТС]-- период времени от даты изготовления (даты выпуска) КТС, до даты оценки (исследования), определяемой условиями задачи исследования (независимо от даты его регистрации и начала использования по назначению (эксплуатации))
	\item[Устранимый дефект] -- дефект, устранение которого возможно путем технического
	обслуживания или ремонта.
	\item[Эмаль] -- жидкий или порошкообразный продукт, содержащий пигменты, который после
	нанесения на поверхность образует непрозрачную плёнку, обладающую защитными,
	декоративными или специальными техническими свойствами.
\end{description}
%%%%%%%%%%%%%%%%%%%%%%%
\subsection{Исходные данные и объекты исследования}

Для производства исследования предоставлено:
\begin{itemize}
	\item Копия договора № 77 от 18 октября 2019, заключённого между индивидуальным предпринимателем Касирум Виктором Валентиновичем и ООО "АвтоСтар-Юг" в лице директора Кравченко Владимира Ильича на предмет оказания услуг по техническому обслуживанию и ремонту автотранспорта 
	\item Копия акта об оказании услуг ВВ00014034 от 28.10.2019. Сварочные работы на сумму 25 000 руб.  на автомобиле ГАЗ 330210, регистрационный знак К040ЕТ23, VIN XTH330210T1576416
	\item Копия товарной накладной ВВ00079229 от 28.10.2019
%	\item Копия акта об оказании услуг ВВ00014034 от 28.10.2019
	\item Транспортное средство ГАЗ 330210, регистрационный знак К040ЕТ23, VIN XTH330210T1576416.
	\end{itemize}

%\subsection{Ранее по материалам дела выполнено}
%
%\noindent Судебная автотехническая экспертиза, выполненная  экспертом Дереберя Н.В.\\
%Повторная судебная автотехническая экспертиза, выполненная экспертом Алифиренко В.В.

%\subsection*{Обстоятельства }
%30.10.2018 г.  автомобиль ДЭУ  г/н В324ХТ93 под управлением водителя Щербакова А. А. выехал на левую сторону дороги, где совершил наезд на стоящий а/м nissan г/н С827СС, который при столкновении был отброшен на движущийся сзади а/м мерседес К934СЕ123 под управлением водителя Тахшазян Г.М.


\section{Исследование}
%
Экспертиза качества автомототранспортного средства проводится в процессе досудебного урегулирования спора о причинах возникновения недостатков автомототранспортного средства между заказчиком и  исполнителем работ в соответствии  с заключённым между ними гражданско-правовым договором.
Задачами экспертизы качества автомототранспортных средств является разрешение вопросов, требующих специальных знаний в области конструирования, производства и эксплуатации объекта экспертизы: определение качества исследуемого объекта, включающее выбор номенклатуры показателей, характеризующих его техническое состояние; определение текущих (фактических) значений этих показателей и их сопоставление с базовыми, установленными нормативными актами Российской Федерации и технической документацией изготовителя; идентификация дефекта и событий, предшествовавших и сопровождавших его проявление; вещной обстановки, сопутствующей выявленному дефекту, причин его возникновения, характера дефекта, влияния на возможность использования объекта по назначению.

\subsection{Исследование предоставленных на экспертизу документов}

\begin{itemize}
	\item Из копии заказ-наряда  ВВ00079229 от 28.10.2019 известно, что на автомобиле ГАЗ 330210, регистрационный знак К040ЕТ23, VIN XTH330210T1576416 в рамках договора № 77 от 18 октября 2019 индивидуальным предпринимателем Касирум Виктором Валентиновичем должны были  проводиться кузовные и малярные работы.
	\item Из копии акта об оказании услуг ВВ00014034 от 28.10.2019 известно, что на автомобиле ГАЗ 330210, регистрационный знак К040ЕТ23, VIN XTH330210T1576416 индивидуальным предпринимателем Касирум В.В. выполнены сварочные работы на сумму 25 000 (Двадцать пять тысяч) рублей. Состав работ не раскрыт. 
	\item Согласно товарной накладной ВВ00079229, для ремонта автомобиля были приобретены на сумму 23 235 (двадцать три тысячи двести тридцать пять) рублей	запасные части, указанные в талице [\ref{tab:4}].
\end{itemize}


 \begin{table}[H]
	\centering
	\caption{{\footnotesize Запчасти  к заказ-наряду № ВВ00079229 от 28.10.2019}}
	\label{tab:4}
	\begin{tabular}{|M{34mm}|M{82mm}|M{15mm}|M{15mm}|}
		\hline
		\rowcolor[HTML]{C0C0C0} 
		\multicolumn{1}{|c|}{\cellcolor[HTML]{C0C0C0}N кат} & Наименование запчасти &  Цена за шт. & Всего цена, руб \\ \hline
		33028401020    & Решетка радиатора  &  -     & -     \\ \hline
		\rowcolor[HTML]{EFEFEF} 
		3302-2803015 & Бампер передний Газель  &  -     & -    \\ \hline
		%	\rowcolor[HTML]{EFEFEF} 
		ЦБ095353    & Ремонтная вставка двери передней левой внутрення  & -  & -      \\ \hline
		\rowcolor[HTML]{EFEFEF}
		ЦБ113596  & Ремонтная вставка двери передней левой    &  -    & -      \\ \hline
	%	\rowcolor[HTML]{EFEFEF} 
	364531   & Ремонтная вставка двери передней правой Газель  &  -    & -     \\ \hline
			\rowcolor[HTML]{EFEFEF}
		ЦБ091509    & Ремонтная вставка двери правой внутренняя  &  -  & -      \\ \hline
		%%%  ........................
	%	\rowcolor[HTML]{C0C0C0}  
	ЦБ094500 & Рем.проем дери левый  & -   &  - \\ \hline
			\rowcolor[HTML]{EFEFEF} 
		ЦБ113668 & Рем.проем дери правый  &  -   & - \\ \hline 
	%		\rowcolor[HTML]{C0C0C0}  
		1762464 & Подножка двери ГАЗель левая  &  -   &  - \\ \hline
		\rowcolor[HTML]{EFEFEF} 
	ЦБ093109 & Подножка двери ГАЗель правая  &   -  &   \\ \hline
	%	\rowcolor[HTML]{C0C0C0}  
	ЦБ113387 & Усилитель проёма левого  &  -  & - \\ \hline
		\rowcolor[HTML]{EFEFEF} 
	ЦБ096536 & Усилитель проёма правого  &  -   & -  \\ \hline
%		\rowcolor[HTML]{C0C0C0}  
	0101 & Усилитель проёма двери  &  -   & -  \\ \hline
		\rowcolor[HTML]{EFEFEF} 
	ЦБ092814 & Усилитель проёма правый  &  -   & - \\ \hline
	%	\rowcolor[HTML]{C0C0C0}  
	6002371101 & Фара левая ГАЗ-31029, 3302 (Освар)  &  -   & -  \\ \hline
	\rowcolor[HTML]{EFEFEF} 
	6002371100 & Фара правая ГАЗ-31029, 3302 (Освар)  &  -   & -  \\ \hline
%	\rowcolor[HTML]{C0C0C0} 
	33028403027/2610 & Арка переднего крыла левая,правая  &  -   & -  \\ \hline
	\rowcolor[HTML]{EFEFEF} 
	33025401623 & Накладка под дверь пластмассовая левая ГАЗ 3302(серый)  &  -   & - \\ \hline
		А-14ВМ   & Сиденье ГАЗель   &  -    & -      \\ \hline
\end{tabular}
\end{table}


\subsection{Исследование транспортного средства}
%

\subsubsection{Осмотр транспортного средства }

Первоначально, осмотр транспортного средства \тс\, регистрационный знак \грз\, должен был состояться 17 декабря 2019 года на СТОА по адресу: г. Краснодар, ул. Московская, 2А в 10-00.  В назначенное время специалист и представитель заказчика Кравченко В. И. прибыли на место осмотра, однако сотрудниками сервисного центра ИП Касирум И.П. в предоставлении транспортного средства для проведения осмотра было отказано. На просьбу заказчика работ Кравченко В.И. о предоставлении автомобиля  от сотрудников сервисного центра был получен ответ о невозможности предоставления автомобиля ввиду отсутствия автомобиля в сервисном центре.

 {Состоявшийся осмотр транспортного средства \тс\, \грз\, проводился \osm\, с 15 час. 00 мин. до 16 час. 00 мин. в сухую, ясную погоду  на открытой площадке по адресу: \местоосмотра. На осмотре присутствовал представитель по доверенности  \заказчик.  %Виновник ДТП %уведомлен надлежащим образом, на осмотр не явился.
		Соответствие маркировочных обозначений на кузове представленного ТС записям в представленных документах ТС  специалистом установлено. Видимые изменения конструкции ТС отсутствуют.  Представленный на исследование автомобиль \tc\, имеет кузов типа "\типкузова". Кабина автомобиля окрашена 		эмалью (краской) \colr\, цвета. Кабина автомобиля свежеокрашенна. Капот не установлен, находится внутри бортовой платформы вместе с б/у резиновыми уплотнителями и б/у пластиковыми облицовками кабины.   
		
		
		
		\begin{multicols}{3}
			[
		%	\section{}
		На рис. \ref{ris:images/ов} приведены общие виды  исследуемого автомобиля \тс:
			]
			\noindent	\includegraphics[width=52mm,  height =37mm]{jpg/t1}
			\columnbreak
			\includegraphics[width=52mm,  height =37mm]{jpg/t2}
			\columnbreak
			\includegraphics[width=52mm,  height =37mm]{jpg/t3}
			\end{multicols}			
			\captionof{figure}{\footnotesize{Вид спереди справа, вид спереди, вид спереди слева кабины исследуемого ТС}}	\label{ris:images/ов}
		
		\vspace{3mm}
		
		На рис. \ref{ris:images/вин} приведено изображение VIN \vin\, нанесённого на дополнительной идентификационную  табличку, установленную на вертикальной поверхности правого дверного проёма кабины.  На снимке отчётливо видны закрашенные, вероятно ремонтной эмалью, углы.
			
		
		\begin{figure}[H]
			\centering
			\includegraphics[width=0.7\linewidth]{jpg/вин}
			\caption{{\footnotesize {Идентификационный номер \вин, нанесённый на маркировочной табличке автомобиля \тс\, \грз}}}
			\label{ris:images/вин}
		\end{figure}
		
		
		
		
		
		
\begin{figure}[H]
	\centering
	\includegraphics[width=0.9\linewidth]{jpg/д1}
	\caption{{\footnotesize {Кабина автомобиля. Вид сбоку справа}}}
	\label{справа}
\end{figure}






   \begin{figure}[H]
   	\centering
   	\includegraphics[width=0.9\linewidth]{jpg/t4}
   	\caption{{\footnotesize {Кабина. Вид сбоку слева}}}
   	\label{слева}
   \end{figure}
   
   
   
    
   
      \begin{figure}[H]
   	\centering
   	\includegraphics[width=0.9\linewidth]{jpg/капот}
   	\caption{{\footnotesize {Капот в грузовом кузове. Вид сбоку справа}}}
   	\label{капот1}
   \end{figure}
   
   
   
   
   
   
       \begin{figure}[H]
   	\centering
   	\includegraphics[width=0.9\linewidth]{jpg/капот1}
   	\caption{{\footnotesize {Капот в грузовом отсеке. Вид спереди}}}
   	\label{капот2}
   \end{figure}
   
   
   
   
   
   \begin{figure}[H]\centering
   	\parbox[t]{0.49\textwidth}
   	{\centering
   		\includegraphics[width=.49\textwidth, height =65mm ]{jpg/бамперсправа}
   		\caption{\footnotesize {Передний бампер \тс,\, вид справа }}
   		\label{бампер}}
   	\hfil \hfil
   	\parbox[t]{0.49\textwidth}
   	{\centering
   		\includegraphics[width=.49\textwidth,  height =65mm]{jpg/бамперд}
   		\caption{\footnotesize {Крупный план правой боковой части переднего бампера. На снимке хорошо видно не штатное крепление бампера и механические повреждения в виде сквозных отверстий, счёсов материала детали и множественных царапин}}
   		\label{бампер1}}
   \end{figure}





\begin{figure}[H]\centering
	\parbox[t]{0.49\textwidth}
	{\centering
		\includegraphics[width=.49\textwidth, height =65mm ]{jpg/решетка}
		\caption{\footnotesize {Панель рамки радиатора \тс,\, вид спереди}}
		\label{рамкарадиатора}}
	\hfil \hfil
	\parbox[t]{0.49\textwidth}
	{\centering
		\includegraphics[width=.49\textwidth,  height =65mm]{jpg/решеткапп}
		\caption{\footnotesize {Правый верхний угол рамки радиатора \tc. Штатное крепёжное отверстие разорван, фиксация детали выполнена строительным саморезом}}
		\label{рамкарадиатора2}}
\end{figure}
   
   
   
   
   
   
   \begin{figure}[H]\centering
   	\parbox[t]{0.49\textwidth}
   	{\centering
   		\includegraphics[width=.49\textwidth, height =65mm ]{jpg/сиденье}
   		\caption{\footnotesize {Вид справа в кабину \тс}}
   		\label{кабинасправа}}
   	\hfil \hfil
   	\parbox[t]{0.49\textwidth}
   	{\centering
   		\includegraphics[width=.49\textwidth,  height =65mm]{jpg/сиденьев}
   		\caption{\footnotesize {Сиденье водителя \tc. Обшивка сиденья разорвана, эластичный элемент разорван}}
   		\label{сиденьеводителя}}
   \end{figure}
   
   
   
   
   
   
\subsubsection{Соответствие установленных на ТС деталей предоставленным документам}

Согласно предоставленных документов, на автомобиле \тс\, регистрационный знак \грз\, должны были установлены детали в соответствии с перечнем, приведённым выше  в талице [\ref{tab:4}]. В процессе исследования, по прямым и косвенным признакам установлено не полное соответствие установленных деталей указанному перечню. А именно, на \тс\,  должна была произведена замена следующих деталей:
\begin{enumerate}
	\item Сиденье переднее левое;
	\item Бампер передний;
	\item Решётка радиатора;
%	\item Капот?
\end{enumerate}
  В процессе проведенного исследования установлено,   замена вышеуказанных деталей не произведена. Панель капота демонтирована с автомобиля,  ремонтные работы с указанной деталью не произведены. 
   
   Таким образом,   имеет место несоответствие фактических выполненных работ работам, указанным исполнителем в представленных документах.
     
      
   \subsubsection{Исследование  ТС \тс }
   
   \indent Осмотр лакокрасочного покрытия автомобиля проводился в условиях рассеянного естественного освещения с расстояния  0,3 -- 0,5 м от осматриваемой поверхности. Осматриваемая поверхность чистая  и сухая.
   Наличие дефектов лакокрасочного покрытия специалистом определялись органолептическим методом. Исследование толщины лакокрасочного покрытия - инструментальным методом. 
   Согласно требованиям ГОСТ 7593-80* Покрытия лакокрасочные грузовых автомобилей. Технические требования (п.п. 4.1.)  при окрашивании автомобиля контролю подлежат:\\
   
   -- применяемые лакокрасочные материалы;
   
   -- подготовка поверхностей под окраску в соответствии с пп. 1.3 и 1.4;
   
   -- толщина покрытий на кабине, деталях оперения, металлических платформах.
   
   
   
%    Контроль качества лакокрасочных материалов проводится в соответствии с нормативно-технической документацией на эти материалы.
   
   -- Поверхности металлических деталей и сборочных единиц, подлежащие окрашиванию, должны соответствовать требованиям: ГОСТ 9.402-80 и другой технической документации, утверждённой в установленном порядке.
   
   
   Поверхность лакокрасочных покрытий по внешнему виду должна соответствовать требованиям ГОСТ 9.032-74.
   
   --Для сборочных единиц и деталей грузовых автомобилей устанавливаются следующие классы покрытий:
   III - для кабины;
   
   --V - для сборочных единиц и деталей платформы;
   
   --VI - для рамы и других деталей шасси, для двигателя автомобиля и его сборочных единиц и деталей;
   
   --VII - для рессор.
   
   
      
   \par 
   Качество  оказываемых  услуг  (выполняемых  работ)  должно
   соответствовать  условиям  договора,  а  при  отсутствии в договоре  требований  к  качеству  или  при их недостаточности - требованиям,
   обычно предъявляемым к качеству услуг (работ) такого рода.
   
   \begin{figure}[H]\centering
   	\parbox[t]{0.49\textwidth}
   	{\centering
   		\includegraphics[width=.49\textwidth, height =65mm ]{jpg/арка}
   		\caption{\footnotesize {Вид снизу слева на наружную арку \тс,\, деталь заменена, покрыта слоем защитного грунта серого цвета. Сварочный шов защищен шовным герметиком. Антигравийное покрытие отсутствует.}}
   		\label{аркалевая}}
   	\hfil \hfil
   	\parbox[t]{0.49\textwidth}
   	{\centering
   		\includegraphics[width=.49\textwidth,  height =65mm]{jpg/аркап}
   		\caption{\footnotesize {Вид снизу справа на наружную арку \tc\, деталь заменена, покрыта слоем защитного грунта серого цвета. Сварочный шов защищен шовным герметиком. Антигравийное покрытие отсутствует.}}
   		\label{аркаправая}}
   \end{figure}

   \begin{figure}[H]\centering
	\parbox[t]{0.49\textwidth}
	{\centering
		\includegraphics[width=.49\textwidth, height =65mm ]{jpg/д2}
		\caption{\footnotesize {Нижняя правая сторона кабины \тс,\, поверхность имеет вмятину диаметром 3 см глубиной 0,5 см}}
		\label{кабинаснизусправа}}
	\hfil \hfil
	\parbox[t]{0.49\textwidth}
	{\centering
		\includegraphics[width=.49\textwidth,  height =65mm]{jpg/дк3}
		\caption{\footnotesize {Нижняя правая сторона кабины \tc\, участок деформированной поверхности площадью 2 дм2. Волнистость поверхности 5 мм, толщина шпатлевочного слоя более 2мм}}
		\label{кабинаволнистость}}
\end{figure}


\begin{figure}[H]\centering
	\parbox[t]{0.49\textwidth}
	{\centering
		\includegraphics[width=.49\textwidth, height =65mm ]{jpg/панелвлп}
		\caption{\footnotesize {Нижняя правая сторона кабины \тс,\, поверхность имеет вмятину диаметром 3 см глубиной 0,5 см}}
		\label{панелвлп}}
	\hfil \hfil
	\parbox[t]{0.49\textwidth}
	{\centering
		\includegraphics[width=.49\textwidth,  height =65mm]{jpg/панелвл}
		\caption{\footnotesize {Нижняя левая сторона кабины \tc\, участок деформированной поверхности площадью 2,5 дм2. Волнистость поверхности 5 мм, толщина шпатлевочного слоя более 2мм}}
		\label{панельвл}}
\end{figure}



   \begin{figure}[H]\centering
	\parbox[t]{0.49\textwidth}
	{\centering
		\includegraphics[width=.49\textwidth, height =65mm ]{jpg/дк25}
		\caption{\footnotesize {Заменный фрагмент левой панели боковины (арка) \тс,\, вид сбоку слева}}
		\label{аркасправа}}
	\hfil \hfil
	\parbox[t]{0.49\textwidth}
	{\centering
		\includegraphics[width=.49\textwidth,  height =65mm]{jpg/дс4}
		\caption{\footnotesize {Замененный фрагмент правой панели боковины (арка) \tc\, вид сбоку справа}}
		\label{аркаслева}}
\end{figure}
   
%   
   
    Измерение волнистости поверхности проводилось с помощью линейки измерительной    металлической 100мм, линейки поверочной Force 9G0610 600х16х36мм, линейки    масштабной магнитной с цветографической шкалой 100мм.
   
   
   В среднем, волнистость левых и правых панелей составила 5 мм на1 метр, волнистость
   левой задней панели более 10 мм на 1 метр.
   В совокупности, лакокрасочное покрытие исследуемого автомобиля имеет большое
   число дефектов в виде неровностей более 5 мм, хорошо различимых с большого расстояния без оптических
   приборов, значительно меняющих внешний вид автомобиля.
   Кроме  перечисленных выше, на исследуемом автомобиле установлены
   следующие дефекты: наружные панели имеют значительную, неравномерную шагрень;
   имеются участки наплывов лакокрасочного покрытия; штрихи, риски – остаточные следы от
   обработки поверхности под окраску; большое количество посторонних включений в
   лакокрасочном покрытии, превышающее размер 0,5 мм; имеются участки неполного
   заполнения герметиком межпанельных швов. 
   
   
   
   
   
   \begin{figure}[H]\centering
   	\parbox[t]{0.49\textwidth}
   	{\centering
   		\includegraphics[width=.49\textwidth, height =65mm ]{jpg/дк27}
   		\caption{\footnotesize {Область сопряжения нижнего переднего края левой двери и панели боковины \тс. При закрытой двери зазор принимает отрицательное значение, вдавливая поверхность и разрушая лакокрасочное покрытие. Причина - нарушение монтажа ремонтной вставки панели боковины.}}
   		\label{ris:images/b3}}
   	\hfil \hfil
   	\parbox[t]{0.49\textwidth}
   	{\centering
   		\includegraphics[width=.49\textwidth,  height =65mm]{jpg/дс2}
   		\caption{\footnotesize {Область сопряжения нижнего переднего края правой двери и панели боковины \тс. При открывании двери зазор принимает отрицательное значение, вдавливая поверхность и разрушая лакокрасочное покрытие. Причина - нарушение монтажа ремонтной вставки панели боковины.}}
   		\label{ris:images/b4}}
   \end{figure}

%
\begin{figure}[H]\centering
	\parbox[t]{0.49\textwidth}
	{\centering
		\includegraphics[width=.49\textwidth, height =65mm ]{jpg/дл}
		\caption{\footnotesize {Увеличенный фрагмент области сопряжения нижнего переднего края левой двери и панели боковины \тс}}
		\label{крайдвери}}
	\hfil \hfil
	\parbox[t]{0.49\textwidth}
	{\centering
		\includegraphics[width=.49\textwidth,  height =65mm]{jpg/дп}
		\caption{\footnotesize {Увеличенный фрагмент области сопряжения нижнего переднего края правой двери и панели боковины \тс}}
		\label{дп}}
\end{figure}
%%
%
   
   \begin{figure}[H]\centering
   	\parbox[t]{0.49\textwidth}
   	{\centering
   		\includegraphics[width=.49\textwidth, height =65mm ]{jpg/дк16}
   		\caption{\footnotesize {Корпус правой фары окрашен напылом эмали}}
   		\label{фараправая}}
   	\hfil \hfil
   	\parbox[t]{0.49\textwidth}
   	{\centering
   		\includegraphics[width=.49\textwidth,  height =65mm]{jpg/дк17}
   		\caption{\footnotesize {Корпус левой фары окрашен напылом  эмали}}
   		\label{фаралевая}}
   \end{figure}



%   
   \begin{figure}[H]\centering
   	\parbox[t]{0.49\textwidth}
   	{\centering
   		\includegraphics[width=.49\textwidth, height =65mm ]{jpg/дк22}
   		\caption{\footnotesize {Частично выравненная слоем шпатлевки вмятина на левой двери размером 1.5Х1.5 см, глубиной 2 мм.}}
   		\label{дверьшпатлевка}}
   	\hfil \hfil
   	\parbox[t]{0.49\textwidth}
   	{\centering
   		\includegraphics[width=.49\textwidth,  height =65mm]{jpg/дк24}
   		\caption{\footnotesize {Фрагмент нижнего  внутреннего угла  левой двери. Лакокрасочное покрытие нанесено на плохо подготовленную или загрязненную поверхность}}
   		\label{дверьлеваявнутри}}
   \end{figure}
%   
%   
%   
   \begin{figure}[H]\centering
   	\parbox[t]{0.49\textwidth}
   	{\centering
   		\includegraphics[width=.49\textwidth, height =65mm ]{jpg/дк15}
   		\caption{\footnotesize {Корпус замка левой двери закрашен}}
   		\label{замоклевый}}
   	\hfil \hfil
   	\parbox[t]{0.49\textwidth}
   	{\centering
   		\includegraphics[width=.49\textwidth,  height =65mm]{jpg/дк23}
   		\caption{\footnotesize {Корпус замка правой двери закрашен}}
   		\label{замокправый}}
   \end{figure}
%   
   
    \begin{figure}[H]\centering
   	\parbox[t]{0.49\textwidth}
   	{\centering
   		\includegraphics[width=.49\textwidth, height =65mm ]{jpg/дк101}
   		\caption{\footnotesize {Передняя стойка правой двери. Под слоем ЛКП необработанный шпатлевочный слой}}
   		\label{шпатлевкасправа}}
   	\hfil \hfil
   	\parbox[t]{0.49\textwidth}
   	{\centering
   		\includegraphics[width=.49\textwidth,  height =65mm]{jpg/дк11}
   		\caption{\footnotesize {Передняя стойка правой двери. Под слоем ЛКП необработанная поверхность. Наплыв ЛКП на внешнем слое}}
   		\label{стойканаплыв}}
   \end{figure}





 \begin{figure}[H]\centering
	\parbox[t]{0.49\textwidth}
	{\centering
		\includegraphics[width=.49\textwidth, height =65mm ]{jpg/дс1}
		\caption{\footnotesize {Облицовка правого порога.  Сопрягаемые детали деформированы в процессе монтирования}}
		\label{облицовкасправа}}
	\hfil \hfil
	\parbox[t]{0.49\textwidth}
	{\centering
		\includegraphics[width=.49\textwidth,  height =65mm]{jpg/дк28}
		\caption{\footnotesize {Облицовка левого порога. Крепление отсутствует. Наружная кромка окрашена эмалью цвета кабины}}
		\label{облицовкаслева}}
\end{figure}
%   
   

    
           
    \begin{figure}[H]\centering
   	\parbox[t]{0.49\textwidth}
   	{\centering
   		\includegraphics[width=.49\textwidth, height =65mm ]{jpg/дк20}
   		\caption{\footnotesize {Облицовка левой стойки со следами эмали цвета кузова}}
   		\label{облицовкастойки}}
   	\hfil \hfil
   	\parbox[t]{0.49\textwidth}
   	{\centering
   		\includegraphics[width=.49\textwidth,  height =65mm]{jpg/дк21}
   		\caption{\footnotesize {Волнистость передней левой двери, образованная в наплывом лакокрасочного покрытия}}
   		\label{наплывслева}}
   \end{figure}

    \begin{figure}[H]\centering
   	\parbox[t]{0.49\textwidth}
   	{\centering
   		\includegraphics[width=.49\textwidth, height =65mm ]{jpg/дк1}
   		\caption{\footnotesize {Облицовка кабины справа со следами эмали цвета кабины}}
   		\label{облицовкаправо}}
   	\hfil \hfil
   	\parbox[t]{0.49\textwidth}
   	{\centering
   		\includegraphics[width=.49\textwidth,  height =65mm]{jpg/дк5}
   		\caption{\footnotesize {Волнистость поверхности}}
   		\label{волнистостьсправа}}
   \end{figure}

 \begin{figure}[H]\centering
	\parbox[t]{0.49\textwidth}
	{\centering
		\includegraphics[width=.49\textwidth, height =65mm ]{jpg/дк2}
		\caption{\footnotesize {Увеличенный фрагмент пластиковой облицовки со следами эмали цвета кузова}}
		\label{облицовкаэмаль}}
	\hfil \hfil
	\parbox[t]{0.49\textwidth}
	{\centering
		\includegraphics[width=.49\textwidth,  height =65mm]{jpg/дк4}
		\caption{\footnotesize {Отслоение ЛКП в торцевой части правой двери}}
		\label{отслоениелкп}}
\end{figure}
   
    \begin{figure}[H]\centering
   	\parbox[t]{0.49\textwidth}
   	{\centering
   		\includegraphics[width=.49\textwidth, height =65mm ]{jpg/дк13}
   		\caption{\footnotesize {Обтекатель передний справа со следами эмали цвета кабины}}
   		\label{обтекательп}}
   	\hfil \hfil
   	\parbox[t]{0.49\textwidth}
   	{\centering
   		\includegraphics[width=.49\textwidth,  height =65mm]{jpg/дк20}
   		\caption{\footnotesize {Обтекатель передний и облицовка левой передней стойки со следами эмали цвета кабины}}
   		\label{обтекательл}}
   \end{figure}
   
   
    \begin{figure}[H]\centering
   	\parbox[t]{0.49\textwidth}
   	{\centering
   		\includegraphics[width=.49\textwidth, height =65mm ]{jpg/дк8}
   		\caption{\footnotesize {Уплотнитель стекла правой двери со следами эмали цвета кабины}}
   		\label{уплотнительп}}
   	\hfil \hfil
   	\parbox[t]{0.49\textwidth}
   	{\centering
   		\includegraphics[width=.49\textwidth,  height =65mm]{jpg/дк9}
   		\caption{\footnotesize {Облицовка правого зеркала наружного со следами эмали цвета кабины}}
   		\label{облицовкаправогозеркала}}
   \end{figure}


 \begin{figure}[H]\centering
	\parbox[t]{0.49\textwidth}
	{\centering
		\includegraphics[width=.49\textwidth, height =65mm ]{jpg/дк6}
		\caption{\footnotesize {Дверь правая}}
		\label{дверьправая}}
	\hfil \hfil
	\parbox[t]{0.49\textwidth}
	{\centering
		\includegraphics[width=.49\textwidth,  height =65mm]{jpg/дк7}
		\caption{\footnotesize {Уплотнитель стекла правой двери со следами эмали цвета кабины}}
		\label{уплотнительсправа}}
\end{figure}

    \begin{figure}[H]\centering
   	\parbox[t]{0.49\textwidth}
   	{\centering
   		\includegraphics[width=.49\textwidth, height =65mm ]{jpg/дк30}
   		\caption{\footnotesize {Фрагмент с неравномерностью плоскостей сопрягаемых поверхностей кабины справа более 10 мм}}
   		\label{неравномерностьплоскости}}
   	\hfil \hfil
   	\parbox[t]{0.49\textwidth}
   	{\centering
   		\includegraphics[width=.49\textwidth,  height =65mm]{jpg/дк31}
   		\caption{\footnotesize {Фрагмент волнистости поверхности двери правой более 3.5 мм }}
   		\label{волнистость}}
   \end{figure}
   
   
   

   
Технические требования, предъявляемые к лакокрасочным покрытиям грузовых
автомобилей, установлены ГОСТ 7593-80. Поверхности металлических деталей и
сборочных единиц, подлежащие окрашиванию, должны отвечать требованиям ГОСТ 9.402-
80 и требованиям технической документации, предъявляемым при применении наносных
материалов для выравнивания и моделирования поверхности. По внешнему виду
поверхность лакокрасочных покрытий должна отвечать требованиям ГОСТ 9.032-74 и
соответствовать III классу покрытий, таблица рис. \ref{нормап}, \ref{нормап2}:

\begin{figure}[H]
	\centering
	\includegraphics[width=0.99\linewidth]{jpg/нормадляпокрытий}
	\caption[]{{\footnotesize Требования ГОСТ 9.032-74 к III классу лакокрасочных покрытий}}
	\label{нормап}
\end{figure}
\begin{figure}[H]
	\centering
	\includegraphics[width=0.99\linewidth]{jpg/нормадляпокрытий2}
	\caption[]{{\footnotesize Требования ГОСТ 9.032-74 к III классу лакокрасочных покрытий. Продолжение}.}
	\label{нормап2}
\end{figure}

Срок службы лакокрасочного покрытия должен соответствовать сроку службы до
капитального ремонта грузового автомобиля при условии эксплуатации автомобиля не более
5лет. Декоративные свойства покрытий должны быть не ниже балла 3, защитные свойства -
не ниже балла 1 по ГОСТ 9.407-84. Для выравнивания наружных поверхностей
металлических деталей и узлов допускается местное нанесение шпатлевок и пластмасс. Не
допускаются дефекты покрытия, влияющие на защитные свойства покрытия (проколы,
кратеры, сморщивание и другие).

%
%\begin{figure}[H]
%	\centering
%	\includegraphics[width=0.99\linewidth]{jpg/гост2}
%	\caption[]{по ГОСТ 9.407-84}
%	\label{кратеры}
%\end{figure}

По ГОСТ7593-80 п. 1.7: Металлические платформы,
сборочные единицы и детали платформ окрашивают в два слоя по грунтовке. Допускается
окрашивать металлические платформы, сборочные единицы м детали платформ в два слоя
эмалями, нормативно-техническая документация на которые предусматривает применение
их без грунтовки. Внутренние поверхности металлических платформ должны быть
окрашены в один слой без грунтовки. 

Форма лицевой поверхности отремонтированной
детали при визуальном сравнении должна быть аналогична форме и геометрии новой
детали.


Специалист допускает, что при ремонте автомобиля \тс\, использовались лакокрасочные материалы среднего класса, аналогичные универсальным шпатлевкам типа NOVOL UNI.  Согласно технической карте NOVOL LT-01-01, шпатлевка наносится на стальные лакокрасочные поверхности, обезжиренные и ошлифованные сухим способом абразивом P80-P120 или на старые лакокрасочные покрытия, обезжиренные и ошлифованные абразивом P220-P280, толщина наносимого слоя  должна составлять не более 3мм. 




  \noindent В процессе исследования специалистом установлено:
  
  1.  Твёрдость лакокрасочного покрытия соответствует нормальной твердости лакокрасочного покрытия автомобиля. Твёрдость кузовного
  лакокрасочного покрытия определялась по ГОСТ 54586-2011 (ИСО 15184:1998) методом
  воздействия карандашей с профилем установленных размеров, формы и твёрдости.
  Использовался набор деревянных чертёжных карандашей компании KOH-I-NOOR фирмы
  Hardmuth AG. Пластическая деформация лакокрасочного покрытия, когезионные
  разрушения отсутствуют уже при воздействии карандаша твёрдостью «H». Таким образом,
  нанесённое лакокрасочное покрытие соответствует твёрдости карандаша «Н», что
  соответствует нормальной твёрдости лакокрасочного покрытия кузовов автомобиля.
  
  
  2. Произведено измерение толщины комплексного лакокрасочного покрытия.
  Измерение толщины слоя ЛКП проводилось по ГОСТ Р 51694-200 универсальным
  толщиномером ETARI 111 электромагнитным (магнитоиндукционным) методом измерений.
  В процессе измерений периодически проводилась проверка калибровки прибора по тестовым
  калибровочным пластинам 100 и 150 мкм. Проверка нулевого значения прибора проводилась
  по полированной мере. За весь период измерения, при измерении толщины калибровочной
  пластины отклонение от калибра не превысило ±5 мкм. 
  Анализ статистических характеристик толщины лакокрасочного покрытия показывает
  разброс толщин лакокрасочного покрытия от 88 мкм до значений, превышающих 2000 мкм.
  Большой разброс толщины лакокрасочного покрытия автомобиля в данном случае
  объясняется не ошибками, или погрешностями измерений, а низким качеством подготовки
  поверхности под окраску, толщиной и площадью нанесённого выравнивающего материала
  (шпатлевки).
  
 \subsection{Анализ результатов исследования}   
 
  
 
  Результаты исследования позволяют сделать выводы о нижеследующем:
  
  \indent \textbf{Объем выполненных работ} не соответствует договорному объёму работ:
  
  1. Бампер передний - указана замена, фактически замена не произведена;
  
  2. Капот- деталь  не ремонтировалась;
  
  3. Сиденье водителя - указана замена; фактически замена не произведена.
  
  4. Решётка радиатора - указана замена, фактически замена не произведена.
  
  
  
  \indent В   процессе ремонта установка ремонтных вставок произведена без надлежащей подготовки сопрягаемых поверхностей, следствием чего является отрицательный зазор примыкания нижней кромки правой двери к поверхности арки колеса, влекущее в месте сопряжения деталей фрагментарную деформацию деталей и повреждения лакокрасочного покрытия,  несовпадение крепёжных отверстий в металлических деталях отверстиям в пластиковой облицовке порогов и арок колес. Отсутствует антикоррозийная защита скрытых полостей и антигравийная защита колёсных арок. Не установлены и отсутствуют  перечне деталей щитки арок передних колес. Установка облицовок порогов и боковин кабины произведена на основание с нарушенной геометрией. Форма поверхности арок колес выполнена моделирование поверхности шпатлевочной массой, в отдельных участках не допустимой толщиной 4-5 мм. Правка и рихтовка лицевых поверхностей  деталей произведена не в должном объёме. Лицевые поверхности деталей  имеют волнистости поверхности, превышающие допустимые значения, плоскости не   выправлены и не отрихтованы в полном объёме, имеются видимые участки с нанесённой но не отшлифованной шпатлевочной массой, при этом толщина слоя шпатлевки  значительно превышает допустимую (3 мм), отделочный слой  лакокрасочного покрытия нанесён с многочисленными видимыми дефектами в виде наплывов, посторонних включений, шагрени. Пластиковые облицовочные панели и фары окрашены напылом эмали цвета кабины,  что, в данном случае, указывает на повторное окрашивание кабины, вероятно, с целью устранения допущенных дефектов, но без необходимого демонтажа деталей пластиковой облицовки и с низким качеством выполнения укрывочных работ.
  
Отслоение ЛКП с поверхности панели рамки радиатора и  с поверхности  задней  торцовой части правой двери, определённая  по ГОСТ 9.407-2015  составляет 1-2 балла.
  	

  
  Таким образом, установленные технические характеристики лакокрасочного
  покрытия исследуемого автомобиля при сопоставлении с требованиями государственных
  стандартов, предъявляемым к лакокрасочным покрытиям грузовых автомобилей и
  технической документации производителей лакокрасочных материалов, а также к качеству, обычно предъявляемых к подобным работам,  однозначно
  указывают на несоответствие лакокрасочного покрытия автомобиля \tc\, VIN
  \вин\, регистрационный знак \грз\, установленным нормам. Основными
  причинами несоответствия  являются несоблюдение требований, предъявляемых при выполнении жестяницких работ,    нарушение технологии   окраски и не соответствие документам  установленных в процессе ремонта комплектующих. 
  
   
%%%%%%%%%%%%%%%%%%%%%%%%%%%%%%%%%%%%%%%%%%
%%%%  Таблица повреждений автомобиля
%%%%%%%%%%%%%%%%%%%%%%%%%%%%%%%%%%%%%%%%%%
%\begin{longtable}{|M{100mm}|G{56mm}|}
%\caption[]{\footnotesize {Повреждения автомобиля, установленные при его осмотре}} \label{tab:5}\\ \hline
%\centering {\bf  \small Описание повреждения} & {\bf \small  Изображение}\\  \hline  \endhead
%%------------------------------------------------------------
%	{\small Дверь передняя левая -  деформация в труднодоступном месте на площади 6дм2}&  \imt{images/1}\\  \hline 
 
	% {\small    } & \imt{fp1} \\ \hline  % Фото повреждений
	% {\small    } & \imt{fp1} \\ \hline  % Фото повреждений
	% {\small    } & \imt{fp1} \\ \hline  % Фото повреждений
	% {\small    } & \imt{fp1} \\ \hline  % Фото повреждений
	% {\small    } & \imt{fp1} \\ \hline  % Фото повреждений
	% {\small    } & \imt{fp1} \\ \hline  % Фото повреждений
	% {\small    } & \imt{fp1} \\ \hline  % Фото повреждений
	
%\end{longtable}

   

\subsubsection*{Расчёт стоимости устранения выявленных недостатков}

\par Размер расходов на восстановительный ремонт определяется исходя из стоимости ремонтных работ (работ по восстановлению, в том числе окраске, контролю, диагностике и регулировке, сопутствующих работ), стоимости используемых в процессе восстановления транспортного средства деталей (узлов, агрегатов) и материалов взамен повреждённых [1].\\
%                                         
Стоимость восстановительного ремонта АМТС ( $ C_\text{вp} $) определяется по формуле:
%
\begin{equation}\label{eq:r}
C_\text{вp} =C_p + C_\text{м} + C_\text{зч} 
\end{equation}
%
\noindent где:
%
\begin{itemize}
	%	
	\item[ ]$C_\text {р} $ --  стоимость ремонтных работ по восстановлению КТС, руб.;
	\item[ ]$ C_\text{м} $ --  стоимость необходимых ремонтных материалов, руб.;
	\item[ ]$ C_\text{зч} $ --  стоимость новых запасных частей, руб;
	%\item[ ] $ \text{И} $ -- коэффициент износа составной части, подлежащей замене, \%.
\end{itemize}

Согласно результатов исследования, для устранения выявленных недостатков автомобиля \тс\, \грз \, необходимо произвести следующие работы:

\begin{enumerate}
	\item Бампер передний - заменить
	\item Капот - заменить, окрасить;
	\item Сиденье водителя - заменить;
	\item Кабина - произвести наружную окраску;
	\item Фары - удалить напыл лакокорасочного покрытия;
	\item Детали облицовки кабины - удалить напыл лакокрасочного покрытия;
	\item Арка наружная правая - выправить, окрасить.
	\item Кабина - выправить имеющиеся неровности, устранить дефекты подготовки поверхности под окраску и дефекты окраски.
\end{enumerate}


\par Полный расчёт стоимости восстановительного ремонта выполнен в программе \auda.\\
Полный текст калькуляции представлен в Приложении  <<Калькуляция стоимости восстановительного ремонта>>.\\ 
\indent Результаты расчёта представлены ниже:
%
\begin{figure}[H]
	\centering
	\includegraphics[width=0.95\linewidth]{jpg/Screenshot_1}
	%%	\caption{}
	%%	\label{fig:screenshot001}
\end{figure}
\begin{figure}[H]
	\centering
	\includegraphics[width=0.95\linewidth]{jpg/Screenshot_2}
	%%	\caption{}
	%%	\label{fig:screenshot001}
\end{figure}
\begin{figure}[H]
	\centering
	\includegraphics[width=0.95\linewidth]{jpg/Screenshot_3}
	%%	\caption{}
	%%	\label{fig:screenshot001}
\end{figure}
\begin{figure}[H]
	\centering
	\includegraphics[width=0.95\linewidth]{jpg/Screenshot_4}
	%%	\caption{}
	%%	\label{fig:screenshot001}
\end{figure}
\begin{figure}[H]
	\centering
	\includegraphics[width=0.95\linewidth]{jpg/Screenshot_5}
	%%	\caption{}
	%%	\label{fig:screenshot001}
\end{figure}
%\begin{figure}[H]
%	\centering
%	\includegraphics[width=0.9\linewidth]{images/screenshot002}
%%%	\caption{}
%	\label{aud}
%\end{figure}
\medskip
\renewcommand\baselinestretch{1.2}\small\normalsize
%%%%%%%%%%%%%%%%  Не ОСАГО
Стоимость коммерческого нормо-часа работ применена  с учётом условий регионального рынка услуг и сложившихся средних расценок по видам работ, типу ТС, а также по маркам и моделям ТС  и   составляет 500 р/ч для данного транспортного средства. Трудоёмкость работ по разборке/сборке/замене  соответствует трудоёмкости работ, рекомендованной заводом изготовителем ТС [1, часть II, п. 7.32], а так же рекомендованные значения оценочной трудоёмкости ремонта кузовных составных частей [1, часть II, п. 7.33]. Расчёт стоимости ремонта, согласно положениям Методики [1] производится с учётом  применения оригинальных запасных частей. %, которые поставляются изготовителем КТС авторизованным ремонтным организациям. %Техническое состояние запасных частей учитывается коэффициентом износа, что в совокупности с установкой оригинальных запасных частей в максимальной степени отвечает понятию «восстановительный ремонт», то есть восстановления состояния КТС, при котором используются установленные изготовителем составные части, но с использованным частично ресурсом.  
%
\par Таким образом, в результате проведённых расчётов (<<Ремонт-калькуляция № \NomerDoc>>) определена стоимость устранения выявленных недостатков ремонта на транспортном средстве  \тс, регистрационный  знак \грз,\, составляющая  \textit{41 211 (Сорок одна тысяча двести одиннадцать) рублей.}
%%%%%%%%%%%%%%% УТС, годные,  рыночная
%\nopagebreak
 

\section{Вывод} 

\begin{enumerate}
	
\item Проведённые ремонтные работы не соответствуют перечню работ, указанных в заказ-наряде ВВ00014034 от 28.10.2019.

	\item  Работы, выполненные  индивидуальным предпринимателем Касирум Виктором Валентиновичем  согласно заказ-наряда ВВ00014034 от 28.10.2019  в рамках договора  № 77 от 18 октября 2019, заключённого между индивидуальным предпринимателем Касирум Виктором Валентиновичем и ООО "АвтоСтар-Юг" на   оказания услуг по техническому обслуживанию и ремонту автотранспорта имеют недостатки.
	
	
%\item Имеются недостатки в составе:
%
%-
%
%-
%
%-
%
%-


\item Стоимость устранения недостатков составляет 41 211 (Сорок одна тысяча двести одиннадцать) рублей.
\end{enumerate}
  
  \vspace{20mm}
{Специалист}\hfill           {Мраморнов А.В.}


