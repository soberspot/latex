\setcounter{page}{1}
\clubpenalty=10000 
\widowpenalty=10000

%%%%%%%%%%%%%%%%%%%%%%%%%%%%%%%%%%%%%%%%
%      Шапка экспертной организации  
%%%%%%%%%%%%%%%%%%%%%%%%%%%%%%%%%%%%%%%%
%
\input   % Шапка 
{titul/оборудованиеАЭИип}  
%
%%   вопросы экспертизы
\subsection{Вопросы исследования}
%Заказчик поручает, а Исполнитель принимает на себя обязательство выполнить Заказчику  комплекс работ в виде автотехнических исследований автомобиля Mazda 6, VIN RUMGJ52\-6802007133 (дата начала гарантии 07.05.2018 г.), по следующим вопросам:
\begin{enumerate}
   
   \item
    <<Какие неисправности имеет коробка перемены передач \кпп\, транспортного средства \тс\, регистрационный знак \грз?>>
   
  \item
    <<Какова причина их возникновения?>>
%   
%  \item
%    Причина выхода из строя  двигателя \двигатель\,  имеет производственный или эксплуатационный характер?
%   
%  \item
%    Могли ли имеющиеся у  двигателя \двигатель\, неисправности возникнуть вследствие капитального ремонта?
%	\item  <<Связано ли повреждение панели рамки радиатора слева и брызговика с лонжероном переднего левого автомобиля ВАЗ 21099 с указанным ДТП?>>	
% \item  <<Что послужило причиной выхода двигателя автомобиля из строя?>>	   
    \item  <<Является ли данная причина:
\begin{itemize}
        \item производственной, т.е. недостатком сборки и/или материала;
        \item связанной с некачественным/несвоевременным обслуживанием автомобиля;
        \item связанной с неразрешенными/недопустимыми переделками агрегата и/или его систем;
        \item связанной с предыдущим ремонтом (если применимо);
        \item эксплуатационной, т.е. возникшей по причине неправильной/ненормальной эксплуатации;
        \item  естественным износом в соответствии с пробегом автомобиля?>> 
 	\end{itemize}
%	
%     \item
%    <<Какова стоимость их устранения?>>
    
\end{enumerate}


\subsection{Термины и определения}
\begin{description}
    \item
    [Аварийные повреждения] -- повреждения, механизм образования которых определяется контактом с посторонними объектами, что привело к деформации или разрушению и к необходимости ремонта или замены составной части, или контактам с агрессивной средой, которая привела к необходимости ремонта (замены) составной части; %[1, часть II, п. 1.5].
    %	\item[Восстановительный ремонт]-- один из способов возмещения ущерба, состоящий в выполнении технологических операций ремонта КТС, действующий в сети торгово-сервисного обслуживания, созданной изготовителем этого КТС [1, часть II, п. 1.4].
    %	\item[Годные остатки] -- работоспособные, имеющие остаточную стоимость детали (агрегаты, узлы) поврежденного автотранспортного средства, годные к дальнейшей эксплуатации, которые можно демонтировать с поврежденного автотранспортного средства и реализовать.
\item
[Дата исследования]-- дата, на которую проводятся расчёты и используются стоимостные данные КТС, запасных частей, материалов, нормо-часа ремонтных работ;% [1, часть II, п. 1.5].
%    \item [Декоративные свойства лакокрасочного покрытия] -- способность лакокрасочного покрытия придавать окрашенной
%    поверхности заданный цвет и блеск
\item
[Дефект] -- это каждое отдельное несоответствие объекта требованиям, установленным документацией;
\item
[Исправное состояние (исправность)] -- состояние объекта, при котором он соответствует всем требованиям, установленным в документации на него;
\item
[Конструктивный отказ] -- отказ, возникший по причине, связанной с несовершенством или нарушением установленных правил и (или) норм проектирования и конструирования;
%\item
%[Защитные свойства лакокрасочного покрытия] --
%    Способность лакокрасочного покрытия предотвращать или замедлять
%    коррозию металлических или разрушение неметаллических поверхностей в
%    условиях агрессивного воздействия внешних факторов.
%\item
%[Лак] -- продукт, который после нанесения на поверхность образует твёрдую прозрачную 	плёнку, обладающую защитными, декоративными или специальными техническими свойствами.
%\item
%[Лакокрасочное покрытие (ЛКП)] -- сплошное покрытие, полученное в результате нанесения 	одного или нескольких слоёв лакокрасочного материала на окрашиваемую поверхность
%\item
%[Линия удара]-- линия, определяемая направлением вектора равнодействующего импульса сил, возникающих при контакте ТС при столкновении до прекращения взаимного внедрения деформирующихся при ударе частей. Положением линии удара на ТС определяются направление и величина момента импульса сил, возникающих при ударе, и, следовательно, направлением и интенсивность разворота ТС относительно центра масс после столкновения.  
\item
[Моделирование]-- исследование каких-либо явлений, процессов или систем объектов путём построения и изучения их моделей;
\item
[Малозначительный дефект] -- дефект, который существенно не влияет на использование продукции по назначению и ее долговечность;
\item
[Механизм отказа] -- процесс, который приводит к отказу
\item
[Морфологические признаки]-- признаки, отображающие внешнее и внутреннее строение объекта;
%\item
%[Недостаток лакокрасочного покрытия] -- отклонение лакокрасочного покрытия от 	требований нормативно-технической документации, образовавшееся в процессе нанесения и
%    формирования лакокрасочного покрытия (производственный недостаток)
\item
[Неработоспособное состояние (неработоспособность)] -- состояние объекта, в котором он не способен выполнять хотя бы одну требуемую функцию по причинам, зависящим от него или из-за профилактического технического обслуживания;
\item
[Неисправное состояние (неисправность) ] -- это состояние объекта, при котором он не соответствует хотя бы одному из требований, установленных в документации на него;
\item
[Неустранимый дефект] -- дефект, устранение которого технически невозможно или экономически нецелесообразно;
\item
[Отказ]  -–  событие, заключающееся в нарушении работоспособного состояния объекта;

\item
[Пластичность] --  способность  материала
приобретать  необратимые  изменения  формы  под действием нагрузки;
\item
[Производственный отказ] -- отказ, возникший по причине, связанной с несовершенством или нарушением установленного процесса изготовления или ремонта, выполняемого на ремонтном предприятии;
\item
[Производственный (технологический) дефект] -- дефект, вызванный нарушением установленной технологии изготовления детали, узла, агрегата;
\item
[Работоспособное состояние] -- состояние объекта, в котором он способен выполнять требуемые функции;
%\item[Предел упругости ] -- свойство вещества, максимальная нагрузка, после снятия которой не возникает остаточных (пластических) деформаций;
\item
[Прочность] --свойство материала сопротивляться разрушению под действием внешних сил;
\item
[Скрытый отказ] -- отказ, не обнаруживаемый визуально или штатными методами и средствами контроля и диагностирования, но выявляемый при проведении технического обслуживания или специальными методами диагностирования;
%    \item[Срок эксплуатации КТС]-- период времени от даты изготовления (даты выпуска) КТС, до даты оценки (исследования), определяемой условиями задачи исследования (независимо от даты его регистрации и начала использования по назначению (эксплуатации))
%\item
%[Упругость] --  способность  материалов  изменять форму  под  действием  нагрузки  и  возвращаться  в исходное состояние после снятия нагрузки
\item
[Устранимый дефект] -- дефект, устранение которого возможно путем технического  обслуживания или ремонта;
%    \item[Эмаль] -- жидкий или порошкообразный продукт, содержащий пигменты, который после
%    нанесения на поверхность образует непрозрачную плёнку, обладающую защитными,
%    декоративными или специальными техническими свойствами.
\item
[Эксплуатационный отказ] -- отказ, возникший по причине, связанной с нарушением установленных правил и (или) условий эксплуатации;
\item
[Явный отказ] -- отказ, обнаруживаемый визуально или штатными методами и средствами контроля и диагностирования при подготовке объекта к применению или в процессе его применения;
\end{description}

\subsection{Задачи, поставленные перед специалистами}

Провести необходимые исследования и ответить на поставленные вопросы.

\subsection{Для производства исследования представлено}

\begin{enumerate}
 \item Транспортное средство \тс, VIN \vin, регистрационный знак \грз
  \item Свидетельство о регистрации ТС \тс\, 99 16 912301
 \item Заказ-наряд № 000015516 от 01.03.2020
 \item Сопроводительный лист к заказ-наряду № 0000155164 от 01.03.2020
 \item Договор купли-продажи ТС от 23.02.2020
\end{enumerate}
%
\addcontentsline{toc}{section}{Использованные нормативы и источники информации}
%
\input{bib/bibliographyАКПП}
%
%\vspace{20mm}
%\pagebreak




%%%%%%%%%%%%%%%%%%%%%%%%%%%%%%%%%%%%%%%%%%%%%%%%%%%%%%%%%%%%%%%%%%%%%%%%%%%%%%%%%
\subsection{Технические средства}  %% Список не удалять!!!
\begin{itemize}
%    
%\item Диагностический сканер BOSH VCM II S/N 1324-88682639 c програмным обеспечением Mazda IDS - 115.02
%%\item   Диагностический сканер SDconnect   с программным обеспечением Xentry Diagnostics v19.11.3.1
\item 
Диагностический сканер мультимарочный Launch X431 Master, S/N: 980648311300V0;
\item 
Диагностическая программа MotorData, версия 5.12 ЗАО «Легион-Автодата»;
\item 	
Линейка измерительная металлическая, ГОСТ 427-75, заводской номер № 51118, 0-500мм, цена деления 1 мм, погрешность ± 0,15 мм, поверка ГМС 092400442;
\item 	
Линейка масштабная магнитная с цветографической шкалой, 100 мм;
%\item   
%Линейка поверочная ШД-1000;
%\item 	
%Микрометр МК-75, заводской № 0291, 50-75 мм, 0.01 мм, 2кт, поверка ГМС 096909895, Гос-реестр СИ 32779-12;
\item 	
Микрометр МК-50, заводской № 6306, 25-50 мм, цена деления 0.01 мм, 2кт, поверка ГМС 096909896, Госреестр СИ 32779-12;
%\item 	
%Нутромер индикаторный НИ 100-160, цена деления 0,01 мм
\item 	
Индикатор часового типа ИЧ 10 кл. 1, ГОСТ 577-68, цена деления 0,01 мм заводской номер 33797, свидетельство о поверке № 5/461, ООО "Кировский завод "Красный инструмен-тальщик" 26.02.2016г.;
\item   
Программа обработки фото-видео изображений ImageJ, разработчик  Wayne Rasband (wayne@codon.nih.gov),
свободная лицензия GPL;
\item   
ПЭВМ под управлением операционной системы Windows 10 с установленным набором макрорасширений LaTeX системы компьютерной вёрстки TeX, cвободная лицензия LaTeX Project Public License (LPPL);
%\item 	Рулетка измерительная металлическая, 0-5000 мм, «HORTZ» №451, отклонение от действи-тельной длины ± 1,20 мм, сертификат о калибровке ФБУ «Краснодарский ЦСМ» № 39; 
\item 	
Цифровой фотоаппарат Canon 760D s/n 143032001327 с объективом Canon EF-S 18-135, тип используемой памяти: Transcend,  32Gb;
\item 	
Штангенциркуль, ШЦ-1-0,02, ГОСТ 166-89, заводской номер HS 101210714, 0-150 мм, цена деления 0,02 мм, класс точности 2кт, поверка ГМС 092400441;
%\item 	Штангенциркуль ШЦЦ-1-150, ГОСТ 8.113-85, 0-150 мм, цена деления 0,01 мм, погрешность ± 0,03 мм, заводской номер 51117, сертификат о калибровке ФБУ «Краснодарский ЦСМ» № 38, Госреестр СИ 32779-12;
\item 	
Щуп (набор № 2) 0,02-0,5 мм, 100 мм, кл 2.
%
%%\item  Универсальный стенд для измерения углов установки колес Hunter Engineering %ProAlign с программным инструментом регулировки схождения колес без блокировки руля %автомобиля WinToe
%\item  Специализированное программное обеспечение для расчёта стоимости  восстановительного ремонта, содержащее нормативы трудоёмкости работ, регламентируемые изготовителями транспортного средства     AudaPadWeb, лицензионное соглашение № AS/\- APW-658  RU-P-409-409435
%\item Он-лайн программа моделирования кинематики подвески автомобиля // \url{http://www.vsusp.com/}
%	
\end{itemize}
%%%%%%%%%%%%%%%%%%%%%%%%%%%%%%%%%%%%%%%%%%%%%%%%%%%%%%%%%%%%%%%%%%%%%%%%%%%%%%%%%%%%%%%%%%%%%%%%%%%%%%
\subsection{Условные обозначения}
%
\begin{description}
%	 
%%\item[АВС] --антиблокировочная система
%\item[АМТС] --автомототранспортное средство
\item
[АКПП] -- автоматическая коробка перемены передач;
%\item
%[БЦ] -- блок цилиндров
%\item[ВГШ] -- верхняя головка шатуна
\item
[ДВС] --двигатель внутреннего сгорания
%\item
%[ГБЦ] --головка блока цилиндров
%\item
%[ГРМ] -- газораспределительный механизм, включает распределительный вал,
%клапаны, толкатели, пружины и др.
%\item[ДТП] --дорожно--транспортное происшествие
\item
[гос.\,рег.\,знак] --государственный регистрационный знак
%\item[КТС] --колёсное транспортное средство 
%\item
%[КШМ] -- кривошипно-шатунный механизм, состоящий из коленчатого вала,
%вкладышей подшипников коленвала и шатунов
%\item[ЛКП] --лакокрасочное покрытие
%\item[л.д.] --лист дела
%%\item[Колесо турбины]  -- крыльчатка турбины
%\item[НГШ] -- нижняя головка шатуна
%\item
%[ОЖ] --охлаждающая жидкость 
\item
[ТО] --техническое обслуживание
\item
[ТС] --транспортное средство
%\item[ТK, ТКР] -- турбокомпрессор. Состоит из двух частей: турбины и компрессора, объединенных общим валом. Вал вращается в подшипниках, размещенных в центральном корпусе ТК
%\item
%[ЦПГ] -- цилиндропоршневая группа, состоящая из поршня, поршневых колец и
%цилиндра
%\item
%[ШПГ] -- шатунно-поршневая группа, состоящая из шатуна, поршня и поршневого
%пальца
\item
[ЭБУ] --электронный блок управления
%%\item[FRAME] -- номер кузова транспортного средства, выпущенного для продажи на внутреннем рынке Японии и содержащий информацию производителя о транспортном средстве
\item
[OBDII] -- On-board diagnostics. Протокол бортовой диагностики автомобиля
%%\item[SRS] -- Cистема пассивной защиты водителя и пассажиров
\item[TCU] --блок управления коробки передач
\item
[VIN] --vehicle identification number, 17--значный идентификационный номер транспортного средства, соответствующий стандарту ISO 3779--2012.
%
\end{description}
%%%%%%%%%%%%%%%%%%%%%%%%%%%%%%%%%%%%%%%%%%%%%

\subsection{Методы исследования}

\begin{itemize}
\item 
Измерительный метод – путём измерения размеров деталей специальными измерительными приборами в соответствии с правилами ГОСТ 26433.1-89 «Правила выполнения измерений. Элементы заводского исполнения»;
\item  
Органолептический метод – исследование и оценка качества объектов с помощью органов чувств;
\item 
Расчётный метод (косвенный измерительный метод) – путём расчётов различных параметров на основе результатов измерений и других данных;
\item 
Экспертный метод (метод экспертной оценки) — совокупности операций по выбору комплекса или единичных характеристик объекта, определению их действительных значений и оценкой экспертом соответствия их установленным требованиям и/или технической информации;
\item 
Графоаналитический метод;
\item 
Метод масштабной  реконструкции.
\end{itemize}
%%%%%%%%%%%%%%%%%%%%%%%%%%%%%%%%%%%%%%%
%
%
%\subsection{Ранее по материалам дела выполнено}
%\noindent Судебная автотехническая экспертиза, выполненная  экспертом Дереберя Н.В.\\
%Повторная судебная автотехническая экспертиза, выполненная экспертом Алифриенко В.В.
%\subparagraph*{} Определением $\cdots$
%\subsection{Обстоятельства дела}
%\begin{itemize}
%\item $\cdots$
%
%\end{itemize}
%
\subsection{Обстоятельства дела}

Из заявления  собственника известно, что автомобиль был приобретен на вторичном рынке. Примерно через три дня после приобретения,  при троганье с места и в движении начали появляться симптомы неисправности АКПП, на панели приборов зажглась пиктограмма "Check". Двигатель и остальные узлы и агрегаты нареканий не вызывали.  При обращении в сервисный центр, в результате компьютерной диагностики   было установлено наличие ошибок в блоке управления АКПП, органолептическим методом было зафиксировано  неудовлетворительное состояние гидравлической жидкости. Замена гидравлической жидкости не привела к устранению неисправностей.    

\section{Исследование}
%
%\subsection{История ремонта и сервисного обслуживания}
%
%На основании предоставленных материалов составлена история ремонта и сервисного обслуживания транспортного средства \тс \, по датам и пробегу, Таблица \ref*{tab:hist}:
%
%{\small 
%\begin{longtable}{|p{16mm}|p{12mm}|p{29mm}|p{50mm}|p{35mm}|}
%\caption[]{\footnotesize {\textbf{История ремонта и сервисного обслуживания по дате и пробегу}}} \label{tab:hist}\\\hline\hline
%%%------------------------------------
%\toprule\textbf{Дата} &\textbf{Пробег, км} &\textbf{№\,Акта,Заказ-наряда, накладной}& \textbf{Вид работы}& \textbf{Примечание}\\\hline \toprule \endhead 
%%%-----------------------------------
%%%%%	% Строки
%	%
%%27.04.2018 & 5  & Заказ-наряд № 480261860-1 & Предпродажная подготовка  &  прим. \\ \hline
%%
%\hs{27.04.2018}{5000}{№ 7643}{расточка блока цилиндров}{просто так}
%\hs{27.04.2018}{5000}{№ 7643}{расточка блока цилиндров}{просто так}
%%\hs{arg1}{arg2}{arg3}{arg4}{arg5}
%%\hs{arg1}{arg2}{arg3}{arg4}{arg5}
%%\hs{arg1}{arg2}{arg3}{arg4}{arg5}
%%\hs{arg1}{arg2}{arg3}{arg4}{arg5}
%%\hs{arg1}{arg2}{arg3}{arg4}{arg5}
%%\hs{arg1}{arg2}{arg3}{arg4}{arg5}
%%\hs{arg1}{arg2}{arg3}{arg4}{arg5}
%%\hs{arg1}{arg2}{arg3}{arg4}{arg5}
%%\hs{arg1}{arg2}{arg3}{arg4}{arg5}
%%\hs{arg1}{arg2}{arg3}{arg4}{arg5}
%%
%%
%\end{longtable}}%\setcounter{rownum}{0} % Обнуляем счетчик строк для следующей таблицы
%

%
%\par 03.09.2019 автомобиль с посторонним стуком в ДВС на эвакуаторе доставлен  в сервисный центр ООО "Формула-МК" по адресу: г. Краснодар, ул. Аэропортовская, 4/1.  Первичная диагностика показала, что при увеличении оборотов до 2000 об/мин слышен стук в ДВС. При приеме ТС выявлено, что уровень масла ниже минимальной отметки, уровень охлаждающей жидкости на минимальном уровне, сигнализаторы или контрольные лампы на панели приборов не горят. Специалистами сервисного центра произведена замена масла, слито 3л масла, цвет масла темный, по субъективной оценке специалиста, выполнявшего замену масла, в слитом масле присутствовал запах бензина. Залито новое масло  до максимального уровня. После замены масла стук в ДВС не прошел. При считывании ошибок зафиксирована ошибка Р0524 (слишком низкое давление масла) на пробеге 32 674 км. Выполнена проверка согласно MESI по симптому № 21 <<Шум в двигателе>>. По итогам проверки, так как источник звука находится внутри ДВС, принято решение произвести частичную разборку для определения источника звука. Дополнительно выполнили проверку давления масла: нижний предел при 1500 об/мин - 2.4 бар; при 4500 об/мин - 4.4 бар. %\rem{ Какое давление масла должно быть по техдоку?} 
%Проверили компрессию для данного двигателя (степень сжатия 14) 1ц -6.5 кг/см2; 2ц-6.5 кг/см2; 3ц-6.5 кг/см2; 4ц-6.0 кг/см2. Выполнили снятие поддона ДВС и нижних головок шатуна. Вкладыш 4-го цилиндра имеет задиры, шатунная шейка коленвала 4го цилиндра имеет задиры, вкладыши 2 и 3 цилиндров имеют задиры. В маслозаборнике присутствуют металлические частицы.  На основании вышеизложенного, специалистами сервисного центра причиной возникновения неисправности названа эксплуатация автомобиля  с уровнем масла ниже рекомендованного заводом изготовителем.

\subsection{Исследование предоставленных на экспертизу документов}

%%%%%%%%%%%%%%%%%%%%%%%% ДЛЯ АВТОМОБИЛЯ
% \subparagraph*{}Из Электронной сервисной книжки  известна следующая информация об автомобиле, имеющая значение для дачи заключения:

Из регистрационных документов известна следующая информация об автомобиле, имеющая значение для дачи заключения:

\begin{description}
   \item[Марка, модель] --\тс
   \item[VIN] -- \vin
   \item[Год выпуска] --\год
   \item[Шасси] --отсутствует
   \item[Цвет ЛКП] --\цвет
   \item[Двигатель] --\двигатель
   \item[Тип КПП] --\кпп
   \item[Привод] --передний
   %\item[Трансмиссия] -- 
   \item[ПТС] --\птс
   \item[Свидетельство о регистрации] --\свид
   \item[Расположенние руля] --левое
\end{description}

\subparagraph*{} Идентификационный код автомобиля (VIN) \vin\, содержит следующую информацию о транспортном средстве, имеющую значение для 	дачи заключения:

\фото{модель}{Расшифровка комплектации по VIN \vin}


\subsection{Исследование транспортного средства}

Исследование автомобиля \тс\, VIN  \vin\,  проводилось  специалистом \датаосмотра\, с использованием производственных мощностей автотехобслуживающего предприятия мультимарочного сервисного центра, расположенного по адресу: \местоосмотра\,  с 10-00 до 15-00 ,  в светлое время суток при естественном и искусственном освещении. При проведении осмотра присутствовали: \присутствовали.   Внешним осмотром установлено:\\
\begin{itemize}
\item 
Автомототранспортное средство \тс\, VIN  \vin\, соответствует товарным образцам автомобилей, собранным из сборочных комплектов  по технологии крупноузловой сборки южнокорейской компании  SsangYong  российской автомобилестроительной фирмой ООО «СОЛЛЕРС — Дальний Восток».  Имеет кузов типа «пятидверный универсал».  Кузов автомобиля окрашен рефлексной ("лессирующей", с "металлическим" эффектом) эмалью (краской) \colr\, цвета. Общий вид автотранспортного средства представлен на Рис. \ссылка{рис:спереди},  \ссылка{рис:видсзадисправа}. 
\item 
Маркировочные обозначения, нанесённые на кузове представленного ТС, цвет кузова, тип кузова, модель ТС, государственные регистрационные номера соответствуют записям  регистрационных документов ТС, Рис. \ссылка{рис:vin}, \ссылка{рис:vin2}, \ссылка{рис:vin4}, \ссылка{рис:св1}, \ссылка{рис:св2};
\item 
Внешние признаки внесения изменений в конструкцию ТС отсутствуют;
\item 
Тягово-сцепное устройство и признаки его установки на автомобиле отсутствуют;
%\item 
%Автомобиль предоставлен частично разобранным: ...
\item  
Показания одометра на момент осмотра составляют \пробег;
\item 
Автомобиль не имеет   повреждений аварийного характера;
\item 
Внешний вид ТС удовлетворительный, по совокупности характерных признаков, пробег автомобиля составляет не более 100 000 км;
\item 
Узлы и агрегаты, размещённые в моторном отсеке видимых повреждений, или признаков, указывающих на возможные повреждения, не имеют.
\end{itemize}

На момент обращения в сервисный центр  при начале движения автомобиль периодически трогается с незначительным рывком. Наиболее сильно рывки ощущаются при трогании на разогретом двигателе.Так же рывки (толчки) происходят в движении на прогретой до рабочей температуры АКПП. Наблюдаются толчки при переключении с 1-й на 2-ую передачу при резком нажатии педали акселератора, толчки при переключении на 3-ю передачу. Владелец автомобиля сообщает о периодическом "зависании" второй передачи. На панели приборов загорается индикатор ошибки, АКПП автомобиля \тс\,  переключается в аварийный режим. 
%Как правило, переключение в аварийный режим вызвано внутренними нарушениями  АКПП.


%
%Автомобиль предоставлен частично разобранным: демонтирован поддон двигателя, вкладыши коленчатого вала, шатунные катушки зажигания, свечи зажигания.  Отдельно представлена пластиковая емкость, содержащая 3 литра масла из двигателя ТС \тс.
%На момент осмотра на автомобиле имеются повреждения переднего бампера снизу слева в виде задиров, крыло заднее правое  имеет царапины ЛКП, бампер задний справа имеет царапины ЛКП, имеется повреждение лобового стекла. Давление в шинах колес передней и задней оси 2.3 бар, шины BRIDGESTONE TURA NZA 225/55R17 97V

Компьютерное диагностирование ЭБУ автомобиля показало неисправности трансмиссии.
С помощью диагностического оборудования получены  коды неисправностей (DTC) (Таблица \ref{table:ошибки}). 
\vspace{3mm}

\begin{table}[h]
    \caption{Таблица зарегистрированных ошибок.}
    \label{table:ошибки}
    \begin{tabular}{c|m{45mm}|m{35mm}|m{63mm}}\hline
      \textbf{  n/n} & \textbf{Код ошибки} & \textbf{Повторяемость} & \textbf{Описание} \\
        \hline 
        1 & P0700 & Фиксируется постоянно & TCU Signal Fault. Система управления трансмиссией (запрос MIL), подсистема электронный блок управления трансмиссией (TCU). Ошибка системы управления коробкой передач \\
        \hline
        2 & P1124 & Фиксируется спорадически & Accelerator Pedal Sensor MalfunctionStuck . Ошибка сенсора педали акселератора.\\ \hline
    \end{tabular}
\end{table}

Первоначально, с целью устранения неисправностей, специалистами сервисного центра  в АКПП автомобиля была произведена замена гидравлической жидкости. При замене использовалась гидравлическая жидкость Fuchs TITAN ATF 3292.    Слитая из АКПП гидравлическая жидкость густая,  тёмного цвета,  с включениями большого количества мелкодисперсных металлических частиц.  Вероятно, жидкость в АКПП не менялась весь период эксплуатации автомобиля (\пробег).  В результате замены жидкости %, после  адаптации АКПП, 
неисправности АКПП автомобиля сохранились, но стали менее выражены.  Ошибка (DTC) P0700, после cброса диагностическим сканером, в движении ТС вновь появляется стабильно. Ошибка P1124 - ошибка датчика педали акселератора, на исследуемом автомобиле имеет случайный характер, (алгоритмически формируется при  одновременном сигнале на ЭБУ от педалей газа и тормоза  при скорости более 25 км/ч), как правило,  связана с  работой реостата электронной педали акселератора. После сброса ошибка  не возникала.   Фактически, ошибка P0700 является информационной, указывает на имеющиеся неисправности в АКПП.    Неисправности АКПП, ощущаемые как толчки, рывки, вызваны гидроударом в связи с падением давления масла в коробке, что может быть обусловлено неисправностями как электрогидравлических компонентов, так и неисправностями механической части агрегата. В таком случае, установить причину неисправности КПП возможно только проведя исследование и диагностирование  компонентов АКПП, с необходимыми для этого  демонтажем и полной разборкой агрегата.

Исследуемый автомобиль оснащён автоматической коробкой перемены передач  первой модификации: 36100-34110 (Рис. \ссылка{рис:коробкамодель2}), агрегат изготавливался в период 15.10.2010-29.06.2011 и предназначался для установки на перенеприводную версию автомобиля \тс.

В процессе исследования произведена проверка электрических жгутов, разъёмов АКПП. Повреждения электрических разъёмов, жгута проводов АКПП отсутствуют.

Видимые возможные повреждения (механические повреждения, признаки воздействие воды, влаги)  электронного блока управления АКПП (TCU) отсутствуют.

Произведён демонтаж, разборка и поэлементный осмотр   компонентов АКПП.

Демонтирован гидроблок. Поверхности всех соленоидов электрорегуляторов VBS Norm-Low  и VBS Norm-High покрыты маслянистым осадком вещества тёмного цвета (Рис. \ссылка{рис:блоксоленоидов2}, \ссылка{рис:соленоид}); 

На корпусе оборотного датчика присутствуют  металлические  частицы (Рис. \ссылка{рис:Screenshot_1});

Дифференциал в удовлетворительном состоянии (Рис. \ссылка{рис:диференциал}), пригоден для повторного использования;

Роликовый подшипник дифференциала имеет признаки начального износа. Дорожки качения, тела качения и сепараторы повреждены малыми инородными абразивными частицами. Повреждения, препятствующие дальнейшей эксплуатации подшипника отсутствуют, (Рис. \ссылка{рис:подшипникрол3});

Обойма подшипника имеет признаки начального износа,  (Рис. \ссылка{рис:подшипникоб2});

Приводной вал в удовлетворительном состоянии (Рис. \ссылка{рис:вал}), пригоден для дальнейшего использования;

Роликовые конические подшипники приводного вала с признаками начального эксплуатационного износа, тела качения имеют незначительные царапины,  подшипники пригодны для дальнейшего использования, (Рис. \ссылка{рис:подшипникрол2});

Обойма подшипника с признаками незначительного эксплуатационного износа. Повреждения,препятствующие дальнейшей эксплуатации подшипника отсутствуют,   (Рис. \ссылка{рис:подшипникоб});

На магните-улавливателе  присутствует пастообразная масса частиц  темного цвета, обладающими магнитными свойствами (Рис. \ссылка{рис:магнит});

Масляный насос (Рис. \ссылка{рис:масляныйнасос}) со следами износа, внутренняя шестерня имеет  выработку рабочих поверхностей,  узел подлежит замене,  (Рис. \ссылка{рис:шестерня}, \ссылка{рис:шестерня2});

Фетровый фильтр тёмного, серо-жёлтого цвета, на материале фильтра  хорошо различимо присутствие большого количества мелкодисперсных частиц цвета металла  (Рис. \ссылка{рис:фильтр}), фильтр расходный компонент, повторное использование недопустимо;

Суппорт в удовлетворительном состоянии (Рис. \ссылка{рис:суппорт}), узел пригоден для дальнейшего использования;

Корпус тормозного барабана имеет поверхностные повреждения  тормозной лентой (Рис. \ссылка{рис:барабан}), деталь пригодна к повторному использованию после шлифовки корпуса;

Тормозная лента, средняя часть, темно-коричневого цвета с явными признаками перегрева, (Рис. \ссылка{рис:тормознаялента}, \ссылка{рис:тормознаялента2}), деталь подлежит замене; 

Шток сервопривода имеет характерную выработку, обрезиненная часть затвердевшая, эластичность низкая (Рис. \ссылка{рис:клапан2}, деталь подлежит замене);

Отверстие в корпусе АКПП под шток сервопривода имеет выработку (возможна реставрация корпуса втулкой или развёртка  отверстия при использовании  штока  ремонтного размера (Рис. \ссылка{рис:клапан3});

Пакеты фрикционов коричневого и темно-коричневого цвета с накладками из целлюлозы. Толщина всех пакетов фрикционов уменьшена на \approx 2-3 mm.  Фрикционные диски  с естественным эксплуатационным износом, поверхности стальных колец  третьей передачи   содержат характерные признаки, указывающие на проскальзывание, недостаточное сцепление фрикционов при передаче крутящего момента, дальнейшее использование пакетов фрикционов не допустимо. (Рис. \ссылка{рис:пакееще}, \ссылка{рис:пакееще2}, \ссылка{рис:пакееще3}, \ссылка{рис:пакет}, \ссылка{рис:пакет2});

Шестерни заднего планетарного механизма в удовлетворительном состоянии, узел пригоден к дальнейшей эксплуатации, (Рис. \ссылка{рис:планетарка}); 

Солнечная шестерня в удовлетворительном состоянии, пригодна для эксплуатации, (Рис. \ссылка{рис:валеще},  \ссылка{рис:вал2})

Посадочное место подшипника заднего планетарного механизма в корпусе АКПП имеет  выработку, диаметр увеличен на \approx  2,5 мм (Рис. \ссылка{рис:подшипниквыроботка}, \ссылка{рис:подшипниквыроботка2}, для повторного использования необходимо восстановить посадочное место подшипника), в случае невозможности - заменить корпус АКПП;

Подшипник имеет признаки износа, к дальнейшей эксплуатации не пригоден;

Планетарный механизм АКПП в удовлетворительном состоянии, пригоден к дальнейшей эксплуатации;

Покрытия поршней утратили эластичность, повторное использование поршней не допустимо;


%Гидротрансформатор - 

 
\subsection{Анализ результатов исследования}

  В результате исследования деталей и узлов автоматической коробки перемены передач  автомобиля  \тс\, \вин\, установлено, что АКПП находится в неисправном состоянии вследствие физического износа компонентов агрегата, к эксплуатации непригодна. По совокупности установленных повреждений, причиной неисправного состояния АКПП является некачественное/несвоевременное обслуживание автомобиля.  Изготовителем рекомендована проверка состояния гидравлической жидкости через каждые 15 000 км или через 12 месяцев, замена гидравлической жидкости через каждые 60 000 км при эксплуатации в тяжелых условиях. На момент исследования пробег автомобиля составлял \пробег\, при этом уровень гидравлической жидкости соответствовал норме, жидкость имела тёмный, почти чёрный цвет, с включениями мелкодисперсных частиц вещества цвета металла. Физическое состояние гидравлической жидкости позволяет специалисту полагать, что в период эксплуатации автомобиля отсутствовал своевременный контроль состояния жидкости, а необходимая периодическая замена не производилась.
  В процессе эксплуатации гидравлическая жидкость окисляется, насыщается  продуктами износа, подвергается многократным циклам нагрева/охлаждения, что с течением времени приводит к утрате её физико-химических свойств, необходимых для нормальной эксплуатации агрегата.  
   Замена жидкости, выполненная специалистами СТОА, с целью устранения имеющихся неисправностей АКПП существенно не улучшила  работу агрегата, так как в результате  детального исследования компонентов АКПП установлено, что на момент замены гидравлической жидкости  в АКПП исследуемого автомобиля  уже присутствовали  механические повреждения части деталей, делающие невозможной нормальную эксплуатацию агрегата.
  
  С учётом вышеизложенного, исследованием установлено, что нормальная эксплуатация автомобиля \тс\, \вин\, возможна только после ремонта или замены АКПП.


\повопросу{1. Какие неисправности имеет коробка перемены передач \кпп\, транспортного средства \тс\, регистрационный знак \грз?}

На момент настоящего исследования  коробка перемены передач \кпп\, транспортного средства \тс\, регистрационный знак \грз\,
имеет повреждения  масляного насоса, тормозного барабана, тормозной ленты, подшипника и корпуса АКПП в виде износа посадочного места подшипника заднего планетарного механизма,  штока сервопривода и его посадочного отверстия в корпусе АКПП, уменьшение вследствие истирания толщин пакетов фрикционов, утрата эластичности поршней, имеющих эластомерные покрытия; загрязнены гидроблок, соленоиды электрорегуляторы VBS Norm-Low  и VBS Norm-Hig, масляный фильтр.

\повопросу{2. Какова причина их возникновения?}

Причиной возникновения неисправностей АКПП является  повреждения электрогидравлического блока и механических компонентов АКПП.

\повопросу{3. Является ли данная причина:
   производственной, т.е. недостатком сборки и/или материала;
      связанной с некачественным/несвоевременным обслуживанием автомобиля;
      связанной с неразрешенными/недопустимыми переделками агрегата и/или его систем;
      связанной с предыдущим ремонтом (если применимо);
      эксплуатационной, т.е. возникшей по причине неправильной/ненормальной эксплуатации;
       естественным износом в соответствии с пробегом автомобиля?>> 
}


В результате исследования не установлены какие-либо недостатки, отвечающие критериям "производственный недостаток" согласно ГОСТ 27.002-2015.

Изменения конструкции, признаки изменения конструкции исследуемого автомобиля отсутствуют.

Признаки, следы предыдущего ремонта АКПП отсутствуют.

Пробег автомобиля на момент настоящего исследования составляет \пробег. Общее состояние автомобиля соответствует состоянию автомобиля, с пробегом менее 100 000 км.

Состояние  деталей и узлов АКПП указывает на некачественное/несвоевременное техническое обслуживание агрегата, отсутствие контроля за состоянием гидравлической жидкости, что привело к выходу его из строя вследствие  преждевременного износа элементов конструкции АКПП.

 Имеющиеся повреждения, по совокупности морфологических признаков, имеют накопительный характер, образованы в процессе длительной эксплуатации АКПП при значительном  (несколько десятков тысяч км) перепробеге сервисного интервала замены гидравлической жидкости.
 
 

Таким образом, причина возникновения неисправностей АКПП автомобиля \тс\, VIN \тс\, имеет накопительный характер, обусловлена значительным (несколько десятков тысяч км) перепробегом сервисного интервала замены гидравлической жидкости в АКПП. Является эксплуатационной и возникла по причине неправильной/ненормальной эксплуатации ТС. 

%\повопросу{4. Какова стоимость их устранения?}


%
%\subparagraph{Первая возможная причина - }
%
%\subparagraph{Вторая возможная причина - }
%




\vspace{5mm}
\section{Выводы}

\begin{enumerate}
	\item \textbf{ На момент настоящего исследования  коробка перемены передач \кпп\, транспортного средства \тс\, регистрационный знак \грз\,
        имеет повреждения  масляного насоса, тормозного барабана, тормозной ленты, подшипника и посадочного места подшипника заднего планетарного механизма,  штока сервопривода и его посадочного отверстия в корпусе АКПП, уменьшение вследствие истирания толщин пакетов фрикционов, утрата эластичности поршней, имеющих эластомерные покрытия; загрязнены гидроблок, соленоиды электрорегуляторы VBS Norm-Low  и VBS Norm-Hig, масляный фильтр.  }
    \item \textbf{Причиной возникновения неисправностей является нарушение работы АКПП вследствие повреждения электрогидравлического блока и механических повреждений деталей АКПП.}
        \item \textbf{Причина возникновения неисправностей АКПП автомобиля \тс\, VIN \тс\,  имеет накопительный характер, обусловлена значительным (несколько десятков тысяч км) перепробегом сервисного интервала замены гидравлической жидкости в АКПП. Является  эксплуатационной и возникла по причине неправильной/ненормальной эксплуатации транспортного средства \тс VIN \vin.}
\end{enumerate}
\vspace{15mm}
\relax


%\vspace{20mm}

%\noindent {Специалист, инженер-механик}\hfill    {   Фефелов С. Л.}\\
{Специалист}\hfill           { Мраморнов А.В.}

\vspace{15mm}
%\noindet{\footnotesize  Страниц документа \pageref{LastPage} }
%\pagebreak
\begin{center}
 \textbf{{\Large    ФОТОТАБЛИЦА ПОВРЕЖДЕНИЙ}}
\end{center}

\фото{спереди}{Автомоиль \тс, вид спереди}
\фото{видсзадисправа}{Автомобиль \тс, вид сзади справа}
\дварядом{vin2}{Идентификационная табличка автомобиля \тс, установленная в подкапотном пространстве}{vin}{Контрольная маркировочная табличка автомобиля \тс, установлена на левой средней стойке }
\дварядом{пробег}{Пробег автомобиля \тс\, на момент исследования 89 773 км}{коробкамодель2}{Идентификационная  табличка АКПП автомобиля \тс}
\фото{vin4}{Идентификационный номер (VIN), выбит на кузове на перегородке моторного отсека \тс}

\фото{коробкаразобрана2}{Демонтированная и разобранная АКПП автомобиля \тс}
\фото{коробкаразобрана}{Демонтированная и разобранная АКПП автомобиля \тс}

\фото{коробкамодель}{АКПП BTR M11}
\фото{коробкаразобрана3}{АКПП BTR M11}

\фотомасштаб{блоксоленоидов2}{Блок соленоидов}{145mm}
\фотомасштаб{блоксоленоидов}{Блок соленоидов}{145mm}

\фотомасштаб{блоксоленоидов3}{Блок соленоидов}{145mm}
\фотомасштаб{соленоид}{Соленоиды АКПП.}{145mm}

\фотомасштаб{Screenshot_1}{Оборотный датчик с признаками механического износа АКПП}{145mm}
\фотомасштаб{магнит}{Магнит с частицами металла}{145mm}


\фотомасштаб{фильтр}{Фильтр АКПП}{145mm}
\фотомасштаб{барабан}{Тормозной барабан}{145mm}


\фотомасштаб{тормознаялента}{Тормозная лента с признаками}{145mm}
\фотомасштаб{тормознаялента2}{Перегретая тормозная лента}{145mm}

\фотомасштаб{пакет}{Пакеты фрикционов}{145mm}
\фотомасштаб{пакет2}{Пакеты фрикционов}{145mm}

\фотомасштаб{пакееще}{Пакеты фрикционов}{145mm}
\фотомасштаб{пакееще2}{Пакеты фрикционов}{145mm}

\фотомасштаб{пакееще3}{Пакеты фрикционов}{145mm}

\фотомасштаб{диференциал}{Состояние дифференциала удовлетворительное}{145mm}
\фотомасштаб{подшипникрол3}{Роликовый конический подшипник дифференциала с начальными признаками износа}{145mm}

\фотомасштаб{суппорт}{}{145mm}
\фотомасштаб{планетарка}{Планетарный механизм}{145mm}

\фотомасштаб{вал}{Приводной вал}{145mm}

\фотомасштаб{валеще}{Солнечная шестерня}{145mm}
\фотомасштаб{валпланетарки}{Верхняя часть вала}{145mm}

\фотомасштаб{вал2}{Фрагмент солнечной шестерни}{145mm}


\фотомасштаб{клапан3}{}{145mm}
\фотомасштаб{клапан2}{}{145mm}

\фотомасштаб{подшипникоб}{}{145mm}
\фотомасштаб{подшипникоб2}{}{145mm}

%\фотомасштаб{уплотнитель}{}{145mm}
\фотомасштаб{подшипникпланетарки}{}{145mm}

\фотомасштаб{подшипникрол}{}{145mm}
\фотомасштаб{подшипникрол3}{}{145mm}

\фотомасштаб{подшипникрол2}{}{145mm}

\фотомасштаб{масляныйнасос}{Масляный насос}{145mm}
\фотомасштаб{шестерня}{Шестерни масляного насоса}{145mm}
\фотомасштаб{шестерня2}{Выработка  шестерни масляного насоса}{145mm}

\фотомасштаб{подшипниквыроботка2}{}{135mm}
\фотомасштаб{подшипниквыроботка}{Разбито посадочное место подшипника заднего планетарного механизма}{135mm}

\фото{схема}{Схема АКПП BTR M11}

\стс{св1}{Свидетельство о регистрации 25 HE 954028 исследуемого автомобиля \тс}{св2}{Свидетельство о регистрации 25 HE 954028 исследуемого автомобиля \тс}


%\фото{}{}
%\фото{}{}
%\imgh{160mm}{1}{2}
%\фото{}{}
%\дварядом{}{}{}{}
%\дварядом{}{}{}{}
%\imgh{160mm}{1}{2}