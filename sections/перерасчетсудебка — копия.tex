\setcounter{page}{1}
\clubpenalty=10000 
\widowpenalty=10000

%%%%%%%%%%%%%%%%%%%%%%%%%%%%%%%%%%%%%%%%
%      Шапка экспертной организации  
%%%%%%%%%%%%%%%%%%%%%%%%%%%%%%%%%%%%%%%%
%
%%%%%%%%%%%%%%%%%%%%%%%%%%%%%%%%%%%%%%%%%%
%
%   Экспертная организация ООО Южнорегиональная экспертная группа
%
%%%%%%%%%%%%%%%%%%%%%%%%%%%%%%%%%%%%%%%%%
\noindent %\qrcode[height=21mm]{\NomerDoc от \окончено }  %%% Добавлен QR-Code
\begin{pspicture}(21mm,21mm)
\obeylines
\psbarcode{%
	%\NomerDoc от \окончено
	BEGIN:VCARD^^J
	VERSION:4.0^^J
	%N:Мраморнов; Александр; Вчеславович^^J
	FN:Александр Мраморнов^^J
%	ORG:IP Alexandr Mramornov^^J
	TITLE: эксперт
	ORG: ИП
	URL:http://www.yourexp.ru^^J
	EMAIL:4516611@gmail.com^^J
	TEL:+7-918-451-6611^^J
	ADR:г. Краснодар, с/т № 2 А/О «Югтекс», ул. Зеленая, 472^^J
	END:VCARD
}{width=1.0 height=1.0}{qrcode}%
\end{pspicture}
\begin{center}
	\normalsize\textbf{ОБЩЕСТВО С ОГРАНИЧЕННОЙ ОТВЕТСТВЕННОСТЬЮ \\[-1.5mm] <<ЮЖНО-РЕГИОНАЛЬНАЯ\quad ЭКСПЕРТНАЯ\quad ГРУППА>> \\[-5mm]}
	%  
	\noindent\rule{\textwidth}{1pt}\\[-6mm]  % Горизонтальная линия
	% \line(1,0){460}% (1,0) -горизонтальная линия, и (0,1) - вертикальная 
\end{center}

\begin{center}
	\begin{footnotesize}\setstretch{0.3}
		%	\small\textbf\setlength   	%\raisebox{5mm}
		\vspace{-3.5mm}350072, Россия, Краснодарский край, г. Краснодар, Ростовское шоссе, 14/2, оф. 67\\[0mm]
		Телефон: \quad 8-918-451-66-11, e-mail:\quad 4516611@gmail.com\\ [-2mm]{ИНН 2311213020\quad КПП 231101001 ОГРН 1162375014560}
	\end{footnotesize}	\\[10mm]
\end{center}


\begin{flushright}
	Краснодар, 2020    \\[8mm]
\end{flushright}
\begin{center}
	\LARGE\textbf{ ЗАКЛЮЧЕНИЕ ЭКСПЕРТА}
	\bigskip\\[0mm]
	%	{\normnumxtbf{\NomerDoc}}	}{den}
\end{center}
\par
\vspace{-3mm}\noindent по гражданскому делу \delonum \, \isk \\[0mm]

%\raggedright 
%\def\hrf#1{\hbox to#1{\hrulefill}}
\noindent \textbf{№ 22-2019}\hfill           \textbf{\dataend}\\%[2mm]
%Приостановлено\hfill      \datastop\\
%Возобновлено\hfill          \datarestart\\
%Окончено\hfill                \dataend\\%[4mm]

\noindent\parbox[l][16mm]{16.5cm}
{\def\hrf#1{\hbox to#1{\hrulefill}}
	\noindent Начато\hfill            \datastart\\%[2mm]
	%	Приостановлено\hfill      \datastop\\
	%	Возобновлено\hfill          \datarestart\\
	Окончено\hfill                \dataend\\%[4mm]
}
\relax

\datastart г. ~в {\small ООО~ "ЮЖНО-РЕГИОНАЛЬНАЯ ЭКСПЕРТНАЯ ГРУППА"} \,  при определении  \, \sud  \,  от \, \dataopr \, о назначении \opr \, по гражданскому делу \delonum \, поступили:

\begin{enumerate}\setlist{nolistsep}\item  Материалы гражданского дела \delonum \, в двух томах, том 1 на 276 листах, том 2  на 143 листах.\\[-2mm]
	%	\item  
\end{enumerate}Экспертиза произведена экспертом {\small ООО "ЮЖНО-РЕГИОНАЛЬНАЯ ЭКСПЕРТНАЯ ГРУППА"} \,  Мраморновым Александром Вячеславовичем, имеющим высшее техническое образование по специальности «техническая физика», диплом РВ №311964 от 28.02.1989, квалификация -- инженер-физик, специальное образование в области оценки: Диплом ПП-1 № 037211 Российской экономической академии им. Г.В. Плеханова, квалификация -- оценка и экспертиза объектов и прав собственности, специальное образование в области независимой технической экспертизы транспортных средств: Диплом ПП-I № 424167, квалификация: эксперт-техник (специализация 150210 специальности 190601.65 – Автомобили и автомобильное хозяйство), состоящий в Государственном реестре экспертов-техников (№ в реестре 256, https://data.gov.ru/opendata/7707211418-experts,  общий трудовой  стаж 30 лет, стаж  экспертной работы  12 лет.  % Шапка организации ООО ЮРЕКСГРУП
%%%%%%%%%%%%%%%%%%%%%%%%%%%%%%%%%%%%%%%%%
%
%   Экспертная организация ИП
%
%%%%%%%%%%%%%%%%%%%%%%%%%%%%%%%%%%%%%%%%%
%\noindent\qrcode[height=20mm]{\NomerDoc от \dataend } 
\noindent %\qrcode[height=21mm]{\NomerDoc от \окончено }  %%% Добавлен QR-Code
\begin{pspicture}(21mm,21mm)
\obeylines
\psbarcode{%
	%\NomerDoc от \окончено
	BEGIN:VCARD^^J
	VERSION:4.0^^J
	%N:Мраморнов; Александр; Вчеславович^^J
	FN:Александр Мраморнов^^J
%	ORG:IP Alexandr Mramornov^^J
	TITLE: эксперт
	ORG: ИП
	URL:http://www.yourexp.ru^^J
	EMAIL:4516611@gmail.com^^J
	TEL:+7-918-451-6611^^J
	ADR:г. Краснодар, с/т № 2 А/О «Югтекс», ул. Зеленая, 472^^J
	END:VCARD
}{width=1.0 height=1.0}{qrcode}%
\end{pspicture}

 %%% Добавлен QR-Code
\vspace{-4mm}
\begin{center}
	\large\textbf{ИНДИВИДУАЛЬНЫЙ\quad ПРЕДПРИНИМАТЕЛЬ  \\[-1.5mm] МРАМОРНОВ  АЛЕКСАНДР ВЯЧЕСЛАВОВИЧ \\[-5.5mm]}
	%  
	\noindent\rule{\textwidth}{2pt}\\[-6mm]  % Горизонтальная линия
	% \line(1,0){460}% (1,0) -горизонтальная линия, и (0,1) - вертикальная 
\end{center}

\begin{center}
	\begin{footnotesize}\setstretch{0.3}
		%	\small\textbf\setlength   	%\raisebox{5mm}
		\vspace{-2.5mm}г. Краснодар, с/т № 2 А/О «Югтекс», ул. Зеленая, 472, 
		Телефон: 8-918-451-66-11, e-mail: 4516611@gmail.com\\ [-2mm]{ИНН\quad 231200665168\quad ОГРНИП \quad 310231220400043}
	\end{footnotesize}	\\[10mm]
\end{center}


\begin{flushright}
% 
	 \hfill	Краснодар, 2020    \\[8mm]
\end{flushright}  
\begin{center}
	\LARGE\textbf{ЗАКЛЮЧЕНИЕ ЭКСПЕРТА}
\end{center}
\par
\vspace{4mm}


\par
\vspace{-3mm}\noindent по гражданскому делу \delonum \, \isk \\[0mm]

%\raggedright 
%\def\hrf#1{\hbox to#1{\hrulefill}}
\noindent \textbf{№ 22-2019}\hfill           \textbf{\dataend}\\%[2mm]
%Приостановлено\hfill      \datastop\\
%Возобновлено\hfill          \datarestart\\
%Окончено\hfill                \dataend\\%[4mm]

\noindent\parbox[l][16mm]{16.5cm}
{\def\hrf#1{\hbox to#1{\hrulefill}}
	\noindent Начато\hfill            \datastart\\%[2mm]
	%	Приостановлено\hfill      \datastop\\
	%	Возобновлено\hfill          \datarestart\\
	Окончено\hfill                \dataend\\%[4mm]
}
\relax

\datastart г. ~в {\small ООО~ "ЮЖНО-РЕГИОНАЛЬНАЯ ЭКСПЕРТНАЯ ГРУППА"} \,  при определении  \, \sud  \,  от \, \dataopr \, о назначении \opr \, по гражданскому делу \delonum \, поступили:

\begin{enumerate}\setlist{nolistsep}\item  Материалы гражданского дела \delonum \, в двух томах, том 1 на 276 листах, том 2  на 143 листах.\\[-2mm]
	%	\item  
	\end{enumerate}

\paragraph*{}
Экспертиза произведена  экспертом
\,  Мраморновым Александром Вячеславовичем, имеющим высшее  образование по специальности «техническая физика», диплом РВ №311964 от 28.02.1989, квалификация -- инженер-физик, специальное образование в области оценки: Диплом ПП-1 № 037211 Российской экономической академии им. Г.В. Плеханова, квалификация -- оценка и экспертиза объектов и прав собственности, специальное образование в области независимой технической экспертизы транспортных средств: Диплом ПП-I № 424167, квалификация: эксперт-техник (специализация 150210 специальности 190601.65 – Автомобили и автомобильное хозяйство), состоящий в Государственном реестре экспертов-техников (№ в реестре 256, https://data.gov.ru/opendata/7707211418-experts,  общий трудовой  стаж 30 лет, стаж  экспертной работы  12 лет. 
\par Заключение подготовлено по месту фактического расположения ИП по адресу: г. Краснодар, с/т № 2 А/О «Югтекс», ул. Зеленая, 472.
\vspace{4mm}
%
%%   вопросы экспертизы
\subsection{Вопросы экспертизы}

\begin{enumerate}
	\item Соответствуют ли нормо-часы, указанные в Актах об оказании услуг и заказ-нарядах к государственному контракту № 1818188100962002312194450/965/18 от 08.10.2018, нормо-часам, с учётом года выпуска автотранспортных средств, установленных заводом-изготовителем? Если - нет, то рассчитать стоимость работ, указанных в прилагаемых Актах об оказании услуг и заказ-нарядах к государственному контракту  № 1818188100962002312194450/965/18 от 08.10.2018., по нормам завода-изготовителя, действовавшим на дату проведения ремонта автотранспортных средств.
\end{enumerate}


\subsection{Для производства исследования представлено} %Название по шаблону минюста
\begin{enumerate}
	\item Копия материалов гражданского дела, в том числе акты об оказании услуг и заказ-наряды к государственному контракту  № 1818188100962002312194450/965/18 от 08.10.2018  в количестве 55 шт. 
	\end{enumerate}
%
%
%\vspace{-275mm}
\addcontentsline{toc}{section}{Использованные нормативы и источники информации}
%
%\left( \addcontentsline{toc}{section}{Использованные нормативы и источники информации}

\subsection{Использованные нормативы и источники информации}
%
\begin{enumerate}
\item 
Махнин\,Е.\,Л., Новоселецкий\, И.\,Н., Федотов\, С.\,В. \emph{Методические рекомендации по проведению судебных автотехнических экспертиз и исследований колёсных транспортных средств в целях определения размера ущерба, стоимости восстановительного ремонта и оценки} // -- М.: ФБУ РФЦСЭ при Минюсте России, 2018.-326 с.  ISBN 978-5-91133-185-6.
%
%
%
%
\item ТУ 017207-255-00232934-2014 \emph{Кузова автомобилей LADA. Технические требования при приёмке в ремонт, ремонте и выпуске из ремонта предприятиями дилерской сети ОАО "АВТОВАЗ"}//  ОАО НВП "ИТЦ АВТО", 2014
%
\item Смирнов  В.Л., Прохоров  Ю.С., Боюр В.С.  и др. \emph{Автомобили ВАЗ. Кузова. Технология ремонта, окраски и  антикоррозионной защиты. Часть II}// - Н.Новгород: АТИС, 2001.- 241с.
%
\item 
Савич Е.Л. \emph{Техническое  обслуживание  и  ремонт  легковых  автомобилей} : учеб. пособие / Е.Л. Савич, М.М. Болбас, В.К. Ярошевич ; под общ. ред. Е.Л. Савича. -Мн. : Вышэйшая школа,  2001. - 479 с. - ISBN985-06-0502-2.
%
\item 
Автомобили ВАЗ-2121, 21213, 21214, 2131 и их модификации: <<Трудоемкости работ (услуг) по техническому обслуживанию и ремонту>> /Куликов А.В., Христов П.Н., Климов В.Е.,  Боюр В.С., Рева В.В., Зимин В.А., Завьялова Н.Н., Хлыненкова Г.А. -- ИТЦТ "АвтоВАЗтехобслуживание", Тольяти -- 2005. 
%
\item
Автомобили LADA SAMARA и их модификации: <<Трудоемкости работ (услуг) по техническому обслуживанию и ремонту>> /Куликов А.В., Христов П.Н., Климов В.Е., Рева В.В., Боюр В.С., Васильев М.В., Фахрутдинов Р.В.,  Прудских Д.А., Гирко В.Б., Шмелева В.А., Зимин В.А. --  ОАО НВП "ИТЦ АВТО",  -- 2006. - 252 стр.
%
\item 
Автомобили LADA PRIORA. Трудоемкости работ (услуг) по техническому обслуживанию и ремонту /Куликов А.В., Христов П.Н., Климов В.Е., Рева В.В., Козлов П.Л., Боюр В.С., Прудских Д.А., Шмелева В.А., Зимин В.А. -- ООО "ИТЦТ АВОСФЕРА", Тольяти -- 2009. -- 344 с.
%
\item 
{Трудоемкости работ по техническому обслуживанию и ремонту автомобилей автомобилей Lada  Granta}/   \url{https://docplayer.ru/30250248-Trudoemkosti-rabot-po-teh\-nicheskomu-obsluzhivaniyu-i-remontu-avtomobiley-lada- granta.html}.
%
%
\item
{Специализированное программное обеспечение для расчёта стоимости  восстановительного ремонта, содержащее нормативы трудоёмкости работ, регламентируемые изготовителями транспортного средства}//   AudaPadWeb, лицензионное соглашение № AS/APW-658  RU-P-409-409435.
%
%
%
\item
{Специализированное программное обеспечение для расчёта стоимости  восстановительного ремонта, содержащее нормативы трудоёмкости работ, регламентируемые изготовителями транспортного средства ОАО «АвтоВАЗ», ЗАО «Джи-Эм-АвтоВАЗ», ОАО «СеАЗ» и ОАО «ЗМА»}//   Автосфера АС:Смета, v.3.9.11// ООО "ИТЦ «ИнтегроМаш», \url{https://autosmeta.pro}.
%
%
%
\item Информационный портал по техническому обслуживанию и ремонту автомобилей	 ВАЗ:\\ \url{www.autosphere.ru}.

%%
\end{enumerate}


%%%%%%%%%%%%%%%%%%%%%%%%%%%%%%%%%%%%%%%%%%%%%%%%%%%%%%%%%%%%%%%%%%%%%%%%%%%%%%%%%
\subsection{Технические средства}  %% Список не удалять!!!
\begin{itemize}

%
%\item  Специализированное программное обеспечение для расчёта стоимости  восстановительного ремонта, содержащее нормативы трудоёмкости работ, регламентируемые изготовителями транспортного средства     AudaPadWeb, лицензионное соглашение № AS/\- APW-658  RU-P-409-409435

\item  ПЭВМ под управлением операционной системы Windows 10 с установленным набором макрорасширений LaTeX системы компьютерной вёрстки TeX, cвободная лицензия LaTeX Project Public License (LPPL)
%	
\end{itemize}
%%%%%%%%%%%%%%%%%%%%%%%%%%%%%%%%%%%%%%%%%%%%%%%%%%%%%%%%%%%%%%%%%%%%%%%%%%%%%%%%%%%%%%%%%%%%%%%%%%%%%%
\subsection{Условные обозначения и принятые сокращения}
\begin{description}
%	 
%%\item[АВС] --Антиблокировочная система
\item[АМТС] -- автомототранспортное средство
\item[ДВС] -- двигатель внутреннего сгорания
\item[ГБЦ] -- головка блока цилиндров
%\item[ДТП] -- дорожно--транспортное происшествие
\item[гос.\,рег.\,знак] -- государственный регистрационный знак
\item[КТС] -- колесо-транспортное средство 
%\item[ЛКП] -- лакокрасочное покрытие
\item[МКПП] -- механическая коробка перемены передач
%\item[л.д.] --Лист дела
%%\item[Колесо турбины]  -- крыльчатка турбины
\item[ТО] -- техническое обслуживание транспортного средства
\item[ТС] -- транспортное средство
%\item[ТK, ТКР] -- Турбокомпрессор. Состоит из двух частей: турбины и компрессора, объединенных общим валом. Вал вращается в подшипниках, размещенных в центральном корпусе ТК
\item[ЭБУ] -- электронный блок управления
%%\item[FRAME] "--*Номер кузова транспортного средства, выпущенного для продажи на внутреннем рынке Японии и содержащий информацию производителя о транспортном средстве
%\item[DTC] --Diagnostic Trouble Codes, диагностические коды неисправностей
\item[VIN] -- vehicle identification number, 17--значный идентификационный номер транспортного средства, соответствующий стандарту ISO 3779--2012
\item[н/н мехобработка] -- не нормированная изготовителем транспортного средства трудоёмкость механической обработки детали
\item[н/у] -- не установлена изготовителем  
%
\end{description}
\subsection{Методы исследования}
\begin{itemize}
\item Экспертный метод (метод экспертной оценки) -- совокупность операций по выбору комплекса или единичных характеристик объекта, определение их действительных значений и оценка экспертом их соответствия установленным требованиям и/или технической информации;
\item Метод сравнения -- сопоставление фактических данных и данных источников.
\end{itemize}

\subsection{Исходные данные}

1. Акты об оказании услуг и заказ-наряды к государственному контракту № 1818188100962002312194450/965/18 от 08.10.2018г. в количестве 55 шт. (л.д. 24-104):


\begin{longtable}{|p{10mm}|P{80mm}|P{70mm}|}
	\caption[]{\footnotesize {\textbf{Перечень транспортных средств и соответствующих им актов об оказании услуг и заказ-нарядов к государственному контракту № 1818188100962002312194450/965/18 от 08.10.2018. }}} \label{tab:hist}\\
	\hline%
	п/п	& Транспортное средство & Номер заказ-наряд\\\hline\hline\endhead
	%
	\csvreader[separator=semicolon, late after line=\\\hline]%
	%
	{csv/Автомобили_перечень.csv}{1=\имя,3=\firstname,4=\matnumber}%
	{\thecsvrow & \firstname~\имя & \matnumber}%
	
\end{longtable}




2. Таблица учёта завышения норма-часов при выполнении ремонта служебного автотранспорта марки Lada Priora  в рамках государственных контрактов по гособоронзаказу, заключённых в 2018 году (л.д. 105-107). 

3. Определённая условиями государственного контракта  № 1818188100962002312194450/965/18 от 08.10.2018 стоимость  нормо-часа работ в размере {945.25 р/час}.



\section{Исследование}

Согласно поставленных судом вопросов экспертизы, исследование должно быть проведено по действующим на момент ремонта  нормам завода-изготовителя с учётом года выпуска автотранспортных средств.

Исследование проводится на основании действующей нормативно-технической и технологической документации на ТО и ремонт автомобилей LADA, [5-11].


Сборники Трудоёмкости работ (услуг) по техническому обслуживанию и ремонту, [5], [6], [7], [8] содержат данные по трудоёмкостям операций технического обслуживания, диагностики и ремонта автомобилей семейства LADA.  

В данных трудоемкостях указано время (в часах и десятых долях часа) для выполнения операций мойки, смазочно-заправочных, ремонта, замены, диагностики, регулировки, окраски, антикоррозионной обработки.

Указанные трудоёмкости  распространяются на работы (услуги) по ТО и ремонту автомобилей LADA, выполняемые
предприятиями технического обслуживания автомобилей (ПТОА). 

В настоящем заключении экспертом применены термины,  определения и правила учёта трудоёмкости, принятые в  нормативно-технической и технологической документации на техническое обслуживание и ремонт автомобилей LADA. В частности, в соответствии с разделом 
"Общие положения", имеющемся в каждом из приведённых Сборников Трудоёмкости работ (услуг) [5, стр. 5-8], [6, стр. 5-8], [7, стр. 5-8], [8, стр. 2-5].

\subparagraph{Позиции трудоёмкостей.} Каждой позиции трудоёмкости присвоен неповторяющийся пятизначный номер. Первые две цифры
номера позиции указывают на раздел, к которому она относится:\\
00 - техническое обслуживание\\
10 - 84 - ремонт\\
85 - окраска\\
86- антикоррозионная обработка\\
87– поиск не явно выраженных неисправностей\\
Три последние цифры в номерах позиций представляют собой собственно порядковый номер позиции
внутри разделов и подразделов.

\subparagraph{Применение надбавок.}  Предприятие может применять надбавки к настоящим трудоемкостям при ремонте автомобилей старше
5 лет до 10 \%, старше 8 лет до 20 \% (кроме работ разделов: техническое обслуживание, окраска,
антикоррозионная обработка, поиск не явно выраженных неисправностей).

 Каждой позиции трудоемкости соответствует:\\
	• код детали/узла (семь цифр) и код работы (две цифры);\\
	• наименование детали, краткая характеристика выполняемой работы и при необходимости полная характеристика;\\
	• норма времени на выполнение работ и максимально допустимое количество повторов выполнения этой работы на одном автомобиле.\\% Если звездочка отсутствует, то количество равно "1".
%	Прочерк вместо нормы времени указывает на то, что данная работа не выполняется на соответствующей модели автомобиля.
    • Трудоемкость работ (услуг) по ТО и ремонту автомобилей, не предусмотренных изготовителем, отсутствующим в справочниках трудоёмкостей завода-изготовителя, \textbf{определяется по согласованию с заказчиком}.
       
    \фото{коды}{Таблица кодов работ семейства автомобилей LADA. Фрагмент.} 
    
    \фото{надбавка}{Фрагмент  <<{\footnotesize Автомобили LADA PRIORA. Трудоемкости работ (услуг) по техническому обслуживанию и ремонту /Куликов А.В., Христов П.Н. и др. -- ООО "ИТЦТ АВОСФЕРА", Тольяти}>>}
    
    
	 \par Стоимость выполнения работ рассчитывается на основании приведённых норм времени и действующей на предприятии на дату ремонта стоимости нормо-часа и представляет собой только стоимость услуги,  стоимость заменяемых агрегатов, узлов и деталей, а также основных материалов, используемых при ТО и ремонте автомобилей, рассчитывается отдельно.
	 \par Под нормо-часом  понимается показатель, характеризующий количество времени, необходимое для выполнения какой-либо работы, операции, оказания услуги или выпуска единицы продукции.
	 \par В общем случае, упрощенно расчет стоимости  ремонтных работ $ C_p $, можно записать  формулой \ref{key}:
	 
	 
{\large \begin{equation}\label{key}
	  C_p = \sum_{i=1}^{n}(C_i) = \sum_{j=1}^{k}(T_{ji})\cdot {C_{nch}} + \sum(C_{man}),  \hspace{10mm} \text{где:}
\end{equation}}

\noindent$T_{ji} $ - трудоемкость  i-ой операции;\\
	 $ C_{nch} $ - стоимость нормо-часа;
	 $ C_i $ - стоимость работы;\\
	 $ C_{man} $ - стоимость ненормированных работ, принятая в денежном выражении.\\
		 
	 
%	  ТРУДОЕМКОСТИ РАБОТ ПО ТЕХНИЧЕСКОМУ ОБСЛУЖИВАНИЮ И РЕМОНТУ АВТОМОБИЛЕЙ LADA содержат следующие ОБЩИЕ ПОЛОЖЕНИЯ, идентичные для исследуемых моделей транспортных средств.   В данных трудоемкостях указано время (в часах и десятых долях часа) для выполнения операций мойки, сма­зочно-заправочных, ремонта, замены, диагностики, регулировки, окраски, антикоррозионной обработки.2  Трудоемкости  распространяются  на  работы  (услуги)  по  ТО  и  ремонту  автомобилей  LADA,  выполняемые предприятиями технического обслуживания автомобилей (ПТОА).Трудоемкости работ на  предпродажную  подготовку  автомобилей  LADA  приведены в  технологической инст­рукции "Автомобили LADA  -  предпродажная подготовка".3  Каждой позиции трудоемкости присвоен неповторяющийся пятизначный номер.  Первые две цифры номера позиции указывают на раздел, к которому она относится:00 - техническое обслуживание10 - 84 - ремонт85 - окраска86 - антикоррозионная обработка87 - поиск не явно выраженных неисправностейТри последние  цифры в номерах позиций представляют собой собственно порядковый номер позиции внутри разделов и подразделов.В разделе "Техническое обслуживание"  позиции сгруппированы по подразделам в зависимости от вида выпол­няемой работы (по кодам работ).В разделе "Ремонт"  первые две цифры дополнительно указывают на систему автомобиля, к которой относится объект работы, и позиции сгруппированы по подразделам в зависимости от принадлежности к соответствующей системе/подсистеме автомобиля.Внутри разделов и подразделов позиции  трудоемкостей расположены в порядке возрастания номеров деталей (узлов) по каталогу LADA.4 Каждой позиции трудоемкости соответствует:- код детали/узла (семь цифр) и код работы (две цифры), см.таблицу 1;- наименование детали, краткая характеристика выполняемой работы и при необходимости полная характери­стика;- норма времени на выполнение работ и максимально допустимое количество повторов выполнения этой рабо­ты на одном автомобиле (указывается через звездочку *).
%}

%Обозначения столбцов в расчётных таблицах:
%\begin{description}
%	\item[N] -- номер позиции
%	\item[Наименование] -- наименование ремонтной операции 
%	\item[Кол.оп.] -- количество операций
%	\item[Фактическая трудоемкость, н/ч] -- норма времени по заказ-наряду
%	\item[Всего, руб] -- цена операции, произведение норматива и единицы нормо-часа (945,25 р.)
%	\item[Код детали, код работы] -- соответствующий  позиции трудоёмкости  семизначный код детали и двухзначный код  работы по каталогу LADA
%	\item[№ позиции] -- уникальный пятизначный номер позиции	
%	\item[Нормативная трудоемкость, н/ч] -- трудоёмкость операции, рассчитанная по справочнику
%	\item[Допустимая надбавка] -- допустимое изготовителем увеличение времени ремонта  в  \% в зависимости от возраста автомобиля
%%	\item[label] description 
%\end{description} 
%
%\vspace{3mm}

В процессе экспертизы   произведено детальное исследование представленных актов и заказ-нарядов. Эксперт отмечает, что ввиду отсутствия исходной информации о неисправностях транспортных средств на момент обращения в ремонтную организацию, исследованию, проверке и перерасчёту подлежали только  позиции, содержащиеся в представленных актах об оказании услуг. При этом экспертом учитывался необходимый минимум сопутствующих технологических операций, определённый изготовителем транспортного средства, для выполнения основного вида работ, указанного в заказ-наряде.\\

На основании предварительного изучения актов об оказании услуг эксперт приходит к заключению о необходимости проведения расчёта стоимости ремонта, так как 
во-первых, нормо-часы, указанные в Актах об оказании услуг и заказ-нарядах к государственному контракту № 1818188100962002312194450/965/18 от 08.10.2018 не соответствуют  нормо-часам, установленным заводом-изготовителем, а во-вторых, калькуляции составлены без учёта возраста транспорта средства.\\



\subsection{Расчет по заказ-нарядам}

Граничные условия расчета:\\
\noindent 1.  Стоимость одного нормо-часа согласно условиям Государственного контракта во всех расчетах  составляет \textbf{945.25} рублей.\\
2. Расчёту подлежат только позиции, указанные в представленных актах об оказании услуг.\\
3. В том случае, если Справочник трудоемкостей содержит значения трудоемкости для основной операции, а технически для выполнения операции в целом необходимо выполнить ряд сопутствующих операций, тогда в расчёт включаются сопутствующие операции.\\
4. Работы, не нормированные изготовителем принимаются по представленному нормативу, если представленный  норматив соответствует установленной рыночной стоимости, принятой в регионе.\\
5. Согласно рекомендаций изготовителя, в зависимости отвозраста автомобиля на момент ремонта, применяется допустимая для конкретного вида работ надбавка трудоемкости.

\parindent     



\csvreader[separator=semicolon, %late after line=\\\hline
]
%
{csv/Автомобили_перечень.csv}{1=\номер,2=\код,3=\машина,4=\заказнаряд,5=\цифра, 6=\процент, 7=\поакту, 8=\норматив, 9=\отклонение}%
{\begin{flushleft}\large\bfseries \номер. \, Заказ-наряд  \заказнаряд \end{flushleft}
	\large%\ifcsvstrcmp{\gender}{f}{Ms.}{Mr.}
	Автомобиль \машина, возраст  на момент расчета, лет:\,  \цифра.
	\par Допустимая надбавка трудоемкости:\, \процент\,\%. 
	\par {Расчет:}
	\фото{\код}{\код \, Перерасчет \заказнаряд}
	\par Цена работ по заказ-наряду: \поакту, руб. 
	\par Цена работ по нормативам:\,\,\, \норматив, руб. 
	\par Отклонение:\hspace{34mm}  \textbf{\отклонение, руб}.
\pagebreak
}%


\subsection{Анализ результатов исследования}

Совокупным анализом представленных актов и заказ-нарядов установлено, что отдельные трудоёмкости, указанные в документах не соответствуют нормативам изготовителя транспортного средства, а некоторые трудоёмкости  не включают необходимые сопутствующие операции.\\

Следующие операции содержат систематические ошибки:

\subsection{Сводная таблица результатов}


\begin{landscape} {\small 
	\begin{longtable}{|p{5mm}|P{72mm}|P{70mm}|P{25mm}|P{25mm}|P{25mm}|}
		\caption[]{\footnotesize {\textbf{Перечень транспортных средств и соответствующих им актов об оказании услуг и заказ-нарядов к государственному контракту № 1818188100962002312194450/965/18 от 08.10.2018. }}} \label{tab:hist}\\
		\hline%
		п/п	& Транспортное средство & Номер заказ-наряд  & Стоимость по Акту об оказании услуг  & Стоимость по нормативам  & Отклонение\\\hline\hline\endhead
		%
		\csvreader[separator=semicolon, late after line=\\\hline]%
		%
		{csv/Автомобили_перечень.csv}{1=\имя,3=\firstname,4=\matnumber, 7=\актуслуг, 8=\норативн, 9=\откл}%
		{\имя & \firstname & \matnumber & \актуслуг & \норативн & \откл}%
		
	\end{longtable}}
\end{landscape}




\section{Выводы}

Таким образом, в результате произведенного исследования, выполненного по предоставленным доказательствам   согласно рекомендаций изготовителя транспортного средства с учётом <<Общих положений. Трудоемкости работ по техническому обслуживанию и ремонту автомобилей LADA>>, [5], [6], [7], [8] cписка литературы, с учетом года выпуска транспортных средств эксперт приходи к следующим выводам:\\

\textbf{1. Нормо-часы, указанные в Актах об оказании услуг и заказ-нарядах к государственному контракту № 1818188100962002312194450/965/18 от 08.10.2018., с учётом года выпуска автотранспортных средств, не соответствуют нормо-часам, установленным   изготовителем транспортного средства.}
	
\vspace{4mm}
	
\textbf{2. Стоимость работ, указанных в прилагаемых Актах об оказании услуг и заказ-нарядах к государственному контракту  № 18181881009620023121\-94450/965/18 от 08.10.2018., составляет 881590 (Восемьсот восемьдесят одна тысяча пятьсот девяносто) рублей, стоимость работ, определённая  по нормам завода-изготовителя, действовавшим на дату проведения ремонта автотранспортных средств, с учётом года выпуска автотранспортных средств, составляет 884744 (Восемьсот восемьдесят четыре тысячи семьсот сорок четыре) рубля 55 коп, экономия при расчёте по нормам изготовителя транспортного средства составляет 3154 (Три тысячи сто пятьдесят четыре) рубля, 55 коп}.



%\vspace{15mm}
\relax
%Приложение к заключению:\\
%\textit{
%	1.
%	   }

\vspace{20mm}

\noindent{Эксперт}\hfill           {Мраморнов А.В.}

%\includepdf[pages=-]{myfile.pdf}
%\includepdf[pages=-]{calc.pdf}