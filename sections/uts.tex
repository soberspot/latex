\subsection{Расчет утраты товарной стоимости ТС}


\par Утрата товарной стоимости (УТС) обусловлена снижением товарной стоимости из-за ухудшения потребительских свойств вследствие наличия дефектов (повреждений), или следов их устранения либо наличия достоверной информации, что дефекты (повреждения) устранялись [1,п. 8].

	УТС может быть рассчитана для КТС, находящихся как в поврежденном, так и в отремонтированном состоянии (при возможности установить степень повреждения).

УТС может определяться при необходимости выполнения одного из нижеперечисленных видов ремонтных воздействий или если установлено их выполнение:

-	устранение перекоса кузова или рамы КТС;

-	замена несъемных элементов кузова КТС (полная или частичная); ремонт съемных или несъемных элементов кузова (включая оперение) КТС (в том числе пластиковых капота, крыльев, дверей, крышки багажника);

-	полная или частичная окраска наружных (лицевых) поверхностей кузова (включая оперение) КТС, бамперов;

-	полная или частичная разборка салона КТС, вызывающая нарушение качества заводской сборки.

УТС не рассчитывается:

а)	если срок эксплуатации легковых автомобилей превышает 5 лет;

б)	если легковые автомобили эксплуатируются в интенсивном режиме, а срок эксплуатации превышает 2,5 года;


в)	в случае замены кузова до оцениваемых повреждений (за исключением кузова грузового КТС, установленного на раме за кабиной);

г)	если КТС ранее подвергалось восстановительному ремонту (в том числе окраске - полной, наружной, частичной; «пятном с переходом») или имело аварийные повреждения, кроме повреждений, указанных в [1, п. 8.4];

д)	если КТС имело коррозионные повреждения кузова или кабины на момент происшествия.



Нижеприведенные повреждения не требуют расчета УТС вследствие исследуемого происшествия, а их наличие до исследуемого происшествия не обуславливает отказ от расчета УТС при таких повреждениях:

а)	эксплуатационных повреждениях ЛКП в виде меления, трещин, а также повреждений, вызванных механическими воздействиями - незначительных по площади сколов, рисок, не нарушающих защитных функций ЛКП составных частей оперения;

б)	одиночного эксплуатационного повреждения оперения кузова (кабины) в виде простой деформации, не требующего окраски, площадью не более 0,25 дм2;

в)	повреждения, которые приводят к замене отдельных составных частей, которые не нуждаются в окрашивании и не ухудшают внешний вид КТС (стекло, фары, бампера неокрашиваемые, пневматические шины, колесные диски, внешняя и внутренняя фурнитура и т. п.). Если, кроме указанных составных частей, повреждены составные части кузова, рамы, кабины или детали оперения - крылья съемные, капот, двери, крышка багажника, - то расчет величины УТС должен учитывать все повреждения составных частей в комплексе;

г)	в случае окраски молдингов, облицовок, накладок, ручек, корпусов зеркал и других мелких наружных элементов, колесных дисков.

В случае исследуемого события для автомобиля \тс\, VIN \vin\, все условия  при которых производится расчет УТС выполняются.\\


\par Величина УТС зависит от вида, характера и объема повреждений и ремонтных воздействий по их устранению.
\par Величина УТС ($ C_\text{YTC} $) определяется на дату оценки (исследования) по формуле: 

\begin{equation}\label{uts}
C_{YTC} = C_{KTC} \cdot \dfrac{\sum\limits_{i=1}^n K_{YTCi}}{100\%}, \text{руб.},
\end{equation}

\noindent где:\\
\noindent $ C_{KTC} $ -- стоимость КТС на дату оценки (исследования), руб;\\
$ K_{YTCi} $ -- коэффициент УТС по i-му элементу КТС, ремонтному воздействию, \%.
 


\par  При ремонте съемной составной части сумма стоимости ремонта (включая стоимость разборки для ремонта и при необходимости снятия детали для ремонта) и величины УТС (без учета УТС вследствие окраски) не должна превышать суммы стоимости этой составной части (с учетом коэффициента износа) и стоимости работ по ее замене.

\par   Значение коэффициента УТС $ K_{\text{утсокр}} $ при подетальной окраске наружных поверхностей кузова КТС рассчитывается с учетом количества окрашиваемых кузовных составных частей и бамперов по формуле:

\begin{equation}\label{f:yc}
K_{\text{утсокр}}=K_{\text{утсокр(1)}}+K_{\text{утсокр(N-1)}}\cdot(N-1), %/ 
\end{equation}
        
\noindent где:\\
\noindent $ \text{К}_{\text{утсокр(1)}} $ - коэффициент УТС по окраске первой кузовной составной части или бампера, \%;\\
$ \text{К}_{\text{утсокр(N-1)}} $ - коэффициент УТС по окраске второй и каждой следующей кузовной составной части или бампера, \%;\\
N - количество окрашиваемых составных частей, по которым рассчитывается УТС.\\
Значения коэффициентов УТС ($ K_{YTC} $) определены по результатам экспертой практики и приведены в приложении [1, Приложение 2.9].

\par Для исследуемого автомобиля \тс \, соответствующие ремонтным воздействиям  коэффициенты УТС приведены ниже в таблице:

\begin{table}[H]
		%\caption{}
	\begin{tabular}{|p{5mm}|p{80mm}|c|c|c|}
	\hline 
	\textbf{п/п} & \textbf{Наименование детали} &\textbf{ К-замена }& \textbf{К-ремонт }&\textbf{ К-окраска} \\ 
	\hline 
	1 & Наружная окраска кузова & -- & -- & 5 \\ 
	\hline 
	2 & Бампер задний & -- & -- & 0,35 \\ 
	\hline 
	3 & Дверь правая & -- & 0,2 & -- \\ 
	\hline 
	4 & Панель задка & 0,3 & -- & -- \\ 
	\hline 
	5 & Крыло заднее левое & 0,5 & -- & -- \\ 
	\hline 
	6 & Арка колеса наружная & 0,2 & -- & -- \\ 
	\hline 
	7 & Надставка левая задняя & 0,2 & -- & -- \\ 
	\hline 
	8 &  Устранить перекос проема левой двери & -- & 0,5 & -- \\ 
	\hline 
	8 & Нарушение целостности заводской сборки при полной разборке/сборке салона & -- & 1 & -- \\ 
	\hline 
	
\end{tabular} 

\end{table}

\vspace{7mm}

$  \sum\limits_{i=1}^n K_{YTCi} = 5+1+0.35+0.2+0.3+0.5+0.2+0.2+0.5 = 8.25 $\\
  
$   C_{KTC} = C_{KTC} \cdot \dfrac{\sum\limits_{i=1}^n K_{YTCi}}{100} = 4508000 \cdot 8.25/100 = 371910 $, или с учетом округления 372000 (Триста семьдесят две тысячи) рублей.\\

Таким образом, величина УТС автомобиля \тс\, составляет 372 000 (Триста семьдесят две тысячи) рублей.