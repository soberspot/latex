\subsection{Расчет стоимости годных остатков}

\par В случаях правовых отношений, регулируемых Гражданским кодексом РФ, расчет стоимости годных остатков и определение стоимости  реального ущерба с его уменьшением на стоимость годных остатков не предусмотрены [1, ч.II,п.9.6]. Вместе с тем, экспертная практика свидетельствует о возможности постановки перед экспертом задачи определения стоимости годных остатков вне  поля действия законодательства об ОСАГО.
\par Под годными остатками автотранспортного средства понимаются работоспособные, имеющие остаточную стоимость детали (агрегаты, узлы) поврежденного автотранспортного средства, как правило, годные к дальнейшей эксплуатации, которые можно демонтировать с поврежденного автотранспортного средства и реализовать. 
Годные остатки должны отвечать следующим условиям:

1) деталь (агрегат, узел) не должна иметь повреждений, нарушающих ее целостность и товарный вид, а агрегат (узел), кроме того, должен находиться в работоспособном состоянии;

2) деталь (агрегат, узел) не должна иметь изменений конструкции, формы, целостности и геометрии, не предусмотренных изготовителем автотранспортного средства (например, дополнительные отверстия и вырезы для крепления несерийного оборудования);

3) деталь не должна иметь следов предыдущих ремонтных воздействий (следов правки, рихтовки, следов шпатлевки, следов частичного ремонта и т.д.).

Под стоимостью годных остатков понимается наиболее вероятная стоимость, по которой они могут быть реализованы, учитывая затраты на их демонтаж, дефектовку, ремонт, хранение и продажу.
К годным остаткам не могут быть отнесены [1, ч.II,п.10.3] составные части:

- демонтаж которых требует работ, связанных с применением газосварочного и электродугового резания;

-имеющие изменения конструкции, формы, нарушения целостности, не предусмотренные изготовителем ТС;

- подвергшиеяся ранее ремонтным воздействиям (например, правке, рихтовке, шпатлеванию  и т.д.);

- влияющие на безопасность дорожного движения.Номенклатура таких составных частей приведена в приложении  2.6 методики [1];

- имеющие коррозионные повреждения;

-требующие ремонта.


Стоимость годных остатков автотранспортного средства может рассчитываться только при соблюдении следующего условия: 

- полная гибель автотранспортного средства в результате ДТП. Под полной гибелью понимается случай, когда стоимость восстановительного ремонта поврежденного ТС превышает его рыночную стоимость на момент повреждения, или проведение восстановительного ремонта технически невозможно.
 
Расчет стоимости годных остатков не производится в следующих случаях:

- когда автотранспортное средство не подлежит, с учетом технического состояния, разборке на запасные части;

- когда, в силу региональных особенностей вторичного  рынка запасных частей, годные остатки данного автотранспортного средства не пользуются спросом.

Учитывая срок эксплуатации ТС \тс \, (23 года), предельную величину физического износа (80\%), региональные особенности вторичного рынка запасных частей, общее техническое состояние и  степень повреждения исследуемого транспортного средства,  в данном случае,   под стоимостью годных остатков понимается стоимость металлической массы автомобиля (металлического лома).
 $  \text{Сго} = \text{Мтс*Сметаллолома}  $, (руб/кг).
 
Стоимость металлолома в городе Краснодаре по данным организаций, занимающихся  приемом лома цветных и черных металлов, составляет от 6,0 до 7,5 рублей за килограмм в зависимости от размеров и качественных характеристик сдаваемого лома (ниже сравнительная таблица цены лома категории 12А1, автомобильный и бытовой легковесный лом). 


\begin{table}[H]
	
\begin{tabular}{|p{8mm}|l|p{26mm}|l|p{32mm}|l|p{18mm}|l|p{10mm}|c|p{8mm}}
	\hline 
№п/п	& {\small Наименование  предприятия} &{\small Адресс } &{\small Телефон}  & 	{\small Стоимость 1 кг, прием лома (руб)} \\
	\hline 
{\small 1}	& {\small Булатов И.П.} &{\small г.Краснодар,  Восточно-Кругликовская, 38 } &	{\small +7 (918) 430-16-70}  & {\small 6,5} \\ 
	\hline 
{\small 2}	&{\small  ВТМ-ЮГПЛЮС} &	{\small г.Краснодар , ул.Текстильная, 3 } &{\small +7 (861) 227-57-07 }&{\small  7,3 }\\ 
	\hline 
{\small 3}	&{\small Евростандарт } &{\small г.Краснодар пос.Знаменский ул.Богатырская 17}  &{\small 89181535143}  & {\small 6,0} \\ 
	\hline 
{\small 4}	& {\small Метализам} & {\small г.Краснодар ул.Уралская, 141/1} &	 {\small 8(918) 467-11-68 } & {\small 	6,5} \\ 
	\hline 
{\small 5}	& {\small Промышленный ресурс} & {\small Краснодар, ул. Соколова, 54}	 &{\small (861)2701773 } & {\small 	7,2 }\\ 
	\hline
\end{tabular} 

\end{table}
		

Средняя цена приема  лома :	 		6.7 руб/кг.

%По информации, представленной организациями, занимающимися приемом и переработкой лома, данная цена не включает в себя стоимость разборки, прессовки и транспортные издержки. В дальнейших расчетах, расходы на проведение указанных выше мероприятий будут вычтены из среднерыночной стоимости приема «чистого» лома.  Поэтому в дальнейших расчетах стоимость  лома металла принята в размере 6,0 рублей за кг.
 
\noindent Если:
масса транспортного средства = 1049 руб. (по данным паспорта ТС);\\
\indent масса неметаллических изделий  ≈ 300 кг. \\
Тогда:  $ \text{Сго = (1049-300)Х6.7= 5 018} $ руб.

\par Таким образом, результаты проведенного  исследования позволяют сделать вывод о том, что стоимость годных остатков транспортного средства \тс\,, с учетом округления,   составляет 5000 (Пять тысяч) рублей.\\


