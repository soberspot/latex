\begin{center}
	\textbf{{\LARGE Новая судебная практика взыскания с виновника ДТП ущерба сверх страхового возмещения без учёта износа деталей и запчастей}}
	\end{center}
\vspace{5mm}



Многие автомобилисты, которые понесли убытки в результате дорожно-транспортного происшествия, сталкиваются с невозможностью добиться возмещения причинённого ущерба в полном объёме даже в судебном порядке.


В случае, когда страховое возмещение недостаточно для того, чтобы полностью возместить причинённый вред, потерпевший вправе обратиться в суд с требованием к виновнику ДТП о возмещении разницы между страховым возмещением и фактическим размером ущерба. Однако, до недавнего времени такая схема зачастую работала только в теории, когда как на практике возмещение ущерба в полном объёме было практически невозможно. Разница между реальным ущербом и страховым возмещением может составлять десятки тысяч рублей. 
Суды при рассмотрении таких споров указывали, что расчёт стоимости восстановительного ремонта повреждённого транспортного средства для целей определения размера ущерба, возмещаемого причинителем вреда, осуществляется в соответствии с методикой, которая в свою очередь предусматривает расчёт причинённого ущерба с учётом износа деталей и запчастей. Указанную позицию судов закрепил и Обзор практики рассмотрения судами дел, связанных с обязательным страхованием гражданской ответственности владельцев транспортных средств, утверждённый Президиумом Верховного Суда РФ 22.06.2016 г. 
Вместе с тем, \textbf{указанный подход судов не отвечает требованиям действующего законодательства, вытекающим из деликтных обязательств причинителя вреда. }
Так в силу статьи 15 Гражданского кодекса Российской Федерации под убытками понимаются расходы, которые лицо, чьё право нарушено, произвело или должно будет произвести для восстановления нарушенного права, утрата или повреждение его имущества (реальный ущерб), а также неполученные доходы, которые это лицо получило бы при обычных условиях гражданского оборота, если бы его право не было нарушено (упущенная выгода). 
Исходя из этого, лицо, которое понесло убытки в результате повреждения имущества третьими лицами, может в силу закона рассчитывать на восстановление своих нарушенных прав. Применительно к случаю причинения вреда транспортному средству это означает, что в результате возмещения убытков в полном размере потерпевший должен быть поставлен в положение, в котором он находился бы, если бы его право собственности не было нарушено, то есть ему должны быть возмещены расходы на полное восстановление эксплуатационных и товарных характеристик повреждённого транспортного средства.


В марте 2017 года Конституционный суд РФ, решения которого обязательны для всех судов на территории Российской Федерации, выразил позицию, в корне изменившую положение дел в случае взыскания ущерба, причинённого потерпевшему в результате ДТП. 
Конституционный Суд РФ разъяснил, что \textbf{институт обязательного страхования гражданской ответственности владельцев транспортных средств, введённый в действующее законодательство с целью повышения уровня защиты прав потерпевших при причинении им вреда при использовании транспортных средств иными лицами, не может подменять собой институт деликтных обязательств, регламентируемый главой 59 ГК Российской Федерации, и не может приводить к снижению размера возмещения вреда, на которое вправе рассчитывать потерпевший на основании общих положений гражданского законодательства. }
При этом, <<Единая методика определения размера расходов на восстановительный ремонт в отношении повреждённого транспортного средства>> обязательства вследствие причинения вреда не регулирует. 
Следовательно, потерпевший при недостаточности страховой выплаты на покрытие причинённого ему фактического ущерба вправе рассчитывать на восполнение образовавшейся разницы за счёт лица, в результате противоправных действий которого образовался этот ущерб, путём предъявления к нему соответствующего требования. 
Соответственно, при исчислении размера расходов, необходимых для приведения транспортного средства в состояние, в котором оно находилось до повреждения, и подлежащих возмещению лицом, причинившим вред, должны приниматься во внимание реальные, т.е. необходимые, экономически обоснованные, отвечающие требованиям завода-изготовителя, учитывающие условия эксплуатации транспортного средства и достоверно подтверждённые расходы, в том числе расходы на новые комплектующие изделия (детали, узлы и агрегаты).

Как следует \textbf{из постановления Пленума Верховного Суда Российской Федерации от 23 июня 2015 года № 25 <<О применении судами некоторых положений раздела I части первой Гражданского кодекса Российской Федерации>>, если для устранения повреждений имущества истца использовались или будут использованы новые материалы, то за исключением случаев, установленных законом или договором, расходы на такое устранение включаются в состав реального ущерба истца полностью, несмотря на то что стоимость имущества увеличилась или может увеличиться по сравнению с его стоимостью до повреждения.} 
Поскольку полное возмещение вреда предполагает восстановление повреждённого имущества до состояния, в котором оно находилось до нарушения права, в таких случаях — притом что на потерпевшего не может быть возложено бремя самостоятельного поиска деталей, узлов и агрегатов с той же степенью износа, что и у подлежащих замене, — неосновательного обогащения собственника повреждённого имущества не происходит, даже если в результате замены повреждённых деталей, узлов и агрегатов его стоимость выросла.
Впоследствии \textbf{из Обзора практики рассмотрения судами дел, связанных с обязательным страхованием гражданской ответственности владельцев транспортных средств (утв. Президиумом Верховного Суда РФ 22.06.2016 г.), был исключён пункт 22, который закреплял противоречащую принципам действующего законодательства позицию судов}.



Как следует \textbf{из постановления Пленума Верховного Суда Российской Федерации от 23 июня 2015 года № 25 <<О применении судами некоторых положений раздела I части первой Гражданского кодекса Российской Федерации>>, если для устранения повреждений имущества истца использовались или будут использованы новые материалы, то за исключением случаев, установленных законом или договором, расходы на такое устранение включаются в состав реального ущерба истца полностью, несмотря на то что стоимость имущества увеличилась или может увеличиться по сравнению с его стоимостью до повреждения.} 
Поскольку полное возмещение вреда предполагает восстановление повреждённого имущества до состояния, в котором оно находилось до нарушения права, в таких случаях — притом что на потерпевшего не может быть возложено бремя самостоятельного поиска деталей, узлов и агрегатов с той же степенью износа, что и у подлежащих замене, — неосновательного обогащения собственника повреждённого имущества не происходит, даже если в результате замены повреждённых деталей, узлов и агрегатов его стоимость выросла.
Впоследствии \textbf{из Обзора практики рассмотрения судами дел, связанных с обязательным страхованием гражданской ответственности владельцев транспортных средств (утв. Президиумом Верховного Суда РФ 22.06.2016 г.), был исключён пункт 22, который закреплял противоречащую принципам действующего законодательства позицию судов}.