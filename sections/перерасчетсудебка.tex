\setcounter{page}{1}
\clubpenalty=10000 
\widowpenalty=10000

%%%%%%%%%%%%%%%%%%%%%%%%%%%%%%%%%%%%%%%%
%      Шапка экспертной организации  
%%%%%%%%%%%%%%%%%%%%%%%%%%%%%%%%%%%%%%%%
%
%%%%%%%%%%%%%%%%%%%%%%%%%%%%%%%%%%%%%%%%%%
%
%   Экспертная организация ООО Южнорегиональная экспертная группа
%
%%%%%%%%%%%%%%%%%%%%%%%%%%%%%%%%%%%%%%%%%
\noindent %\qrcode[height=21mm]{\NomerDoc от \окончено }  %%% Добавлен QR-Code
\begin{pspicture}(21mm,21mm)
\obeylines
\psbarcode{%
	%\NomerDoc от \окончено
	BEGIN:VCARD^^J
	VERSION:4.0^^J
	%N:Мраморнов; Александр; Вчеславович^^J
	FN:Александр Мраморнов^^J
%	ORG:IP Alexandr Mramornov^^J
	TITLE: эксперт
	ORG: ИП
	URL:http://www.yourexp.ru^^J
	EMAIL:4516611@gmail.com^^J
	TEL:+7-918-451-6611^^J
	ADR:г. Краснодар, с/т № 2 А/О «Югтекс», ул. Зеленая, 472^^J
	END:VCARD
}{width=1.0 height=1.0}{qrcode}%
\end{pspicture}
\begin{center}
	\normalsize\textbf{ОБЩЕСТВО С ОГРАНИЧЕННОЙ ОТВЕТСТВЕННОСТЬЮ \\[-1.5mm] <<ЮЖНО-РЕГИОНАЛЬНАЯ\quad ЭКСПЕРТНАЯ\quad ГРУППА>> \\[-5mm]}
	%  
	\noindent\rule{\textwidth}{1pt}\\[-6mm]  % Горизонтальная линия
	% \line(1,0){460}% (1,0) -горизонтальная линия, и (0,1) - вертикальная 
\end{center}

\begin{center}
	\begin{footnotesize}\setstretch{0.3}
		%	\small\textbf\setlength   	%\raisebox{5mm}
		\vspace{-3.5mm}350072, Россия, Краснодарский край, г. Краснодар, Ростовское шоссе, 14/2, оф. 67\\[0mm]
		Телефон: \quad 8-918-451-66-11, e-mail:\quad 4516611@gmail.com\\ [-2mm]{ИНН 2311213020\quad КПП 231101001 ОГРН 1162375014560}
	\end{footnotesize}	\\[10mm]
\end{center}


\begin{flushright}
	Краснодар, 2020    \\[8mm]
\end{flushright}
\begin{center}
	\LARGE\textbf{ ЗАКЛЮЧЕНИЕ ЭКСПЕРТА}
	\bigskip\\[0mm]
	%	{\normnumxtbf{\NomerDoc}}	}{den}
\end{center}
\par
\vspace{-3mm}\noindent по гражданскому делу \delonum \, \isk \\[0mm]

%\raggedright 
%\def\hrf#1{\hbox to#1{\hrulefill}}
\noindent \textbf{№ 22-2019}\hfill           \textbf{\dataend}\\%[2mm]
%Приостановлено\hfill      \datastop\\
%Возобновлено\hfill          \datarestart\\
%Окончено\hfill                \dataend\\%[4mm]

\noindent\parbox[l][16mm]{16.5cm}
{\def\hrf#1{\hbox to#1{\hrulefill}}
	\noindent Начато\hfill            \datastart\\%[2mm]
	%	Приостановлено\hfill      \datastop\\
	%	Возобновлено\hfill          \datarestart\\
	Окончено\hfill                \dataend\\%[4mm]
}
\relax

\datastart г. ~в {\small ООО~ "ЮЖНО-РЕГИОНАЛЬНАЯ ЭКСПЕРТНАЯ ГРУППА"} \,  при определении  \, \sud  \,  от \, \dataopr \, о назначении \opr \, по гражданскому делу \delonum \, поступили:

\begin{enumerate}\setlist{nolistsep}\item  Материалы гражданского дела \delonum \, в двух томах, том 1 на 276 листах, том 2  на 143 листах.\\[-2mm]
	%	\item  
\end{enumerate}Экспертиза произведена экспертом {\small ООО "ЮЖНО-РЕГИОНАЛЬНАЯ ЭКСПЕРТНАЯ ГРУППА"} \,  Мраморновым Александром Вячеславовичем, имеющим высшее техническое образование по специальности «техническая физика», диплом РВ №311964 от 28.02.1989, квалификация -- инженер-физик, специальное образование в области оценки: Диплом ПП-1 № 037211 Российской экономической академии им. Г.В. Плеханова, квалификация -- оценка и экспертиза объектов и прав собственности, специальное образование в области независимой технической экспертизы транспортных средств: Диплом ПП-I № 424167, квалификация: эксперт-техник (специализация 150210 специальности 190601.65 – Автомобили и автомобильное хозяйство), состоящий в Государственном реестре экспертов-техников (№ в реестре 256, https://data.gov.ru/opendata/7707211418-experts,  общий трудовой  стаж 30 лет, стаж  экспертной работы  12 лет.  % Шапка организации ООО ЮРЕКСГРУП
%%%%%%%%%%%%%%%%%%%%%%%%%%%%%%%%%%%%%%%%%
%
%   Экспертная организация ИП
%
%%%%%%%%%%%%%%%%%%%%%%%%%%%%%%%%%%%%%%%%%
%\noindent\qrcode[height=20mm]{\NomerDoc от \dataend } 
\noindent %\qrcode[height=21mm]{\NomerDoc от \окончено }  %%% Добавлен QR-Code
\begin{pspicture}(21mm,21mm)
\obeylines
\psbarcode{%
	%\NomerDoc от \окончено
	BEGIN:VCARD^^J
	VERSION:4.0^^J
	%N:Мраморнов; Александр; Вчеславович^^J
	FN:Александр Мраморнов^^J
%	ORG:IP Alexandr Mramornov^^J
	TITLE: эксперт
	ORG: ИП
	URL:http://www.yourexp.ru^^J
	EMAIL:4516611@gmail.com^^J
	TEL:+7-918-451-6611^^J
	ADR:г. Краснодар, с/т № 2 А/О «Югтекс», ул. Зеленая, 472^^J
	END:VCARD
}{width=1.0 height=1.0}{qrcode}%
\end{pspicture}

 %%% Добавлен QR-Code
\vspace{-4mm}
\begin{center}
	\large\textbf{ИНДИВИДУАЛЬНЫЙ\quad ПРЕДПРИНИМАТЕЛЬ  \\[-1.5mm] МРАМОРНОВ  АЛЕКСАНДР ВЯЧЕСЛАВОВИЧ \\[-5.5mm]}
	%  
	\noindent\rule{\textwidth}{2pt}\\[-6mm]  % Горизонтальная линия
	% \line(1,0){460}% (1,0) -горизонтальная линия, и (0,1) - вертикальная 
\end{center}

\begin{center}
	\begin{footnotesize}\setstretch{0.3}
		%	\small\textbf\setlength   	%\raisebox{5mm}
		\vspace{-2.5mm}г. Краснодар,
		 %с/т № 2 А/О «Югтекс»,
		  ул. Зеленая, 472, 
		Телефон: 8-918-451-66-11, e-mail: 4516611@gmail.com\\ [-2mm]{ИНН\quad 231200665168\quad ОГРНИП \quad 310231220400043}
	\end{footnotesize}	\\[10mm]
\end{center}


\begin{flushright}
% 
	 \hfill	Краснодар, 2020    \\[8mm]
\end{flushright}  
\begin{center}
	\LARGE\textbf{ЗАКЛЮЧЕНИЕ ЭКСПЕРТА}
\end{center}
\par
\vspace{4mm}


\par
\vspace{-3mm}\noindent по гражданскому делу \delonum \, \isk \\[0mm]

%\raggedright 
%\def\hrf#1{\hbox to#1{\hrulefill}}
\noindent \textbf{№ 22-2019}\hfill           \textbf{\dataend}\\%[2mm]
%Приостановлено\hfill      \datastop\\
%Возобновлено\hfill          \datarestart\\
%Окончено\hfill                \dataend\\%[4mm]

\noindent\parbox[l][28mm]{16.5cm}
{\def\hrf#1{\hbox to#1{\hrulefill}}
	\noindent Начато\hfill            \datastart\\%[2mm]
Приостановлено  \hfill      \datastop\\
Возобновлено  \hfill          \datarestart\\
Окончено \hfill                \dataend\\[4mm]
}
\relax
%\vspace{15mm}
\noindent\datastart г. ~ ИП Мраморнову Александру Вячеславовичу \,  при определении  \, \sud  \,  от \, \dataopr \, о назначении \opr \, по гражданскому делу \delonum \, поступили:

\begin{enumerate}\setlist{nolistsep}\item  Материалы гражданского дела \delonum \, в двух томах, том 1 на 276 листах, том 2  на 143 листах.\\[-2mm]
	%	\item  
	\end{enumerate}

\paragraph*{}
Экспертиза произведена  экспертом
\,  Мраморновым Александром Вячеславовичем, имеющим высшее  образование по специальности «техническая физика», диплом РВ №311964 от 28.02.1989, квалификация -- инженер-физик, специальное образование в области оценки: Диплом ПП-1 № 037211 Российской экономической академии им. Г.В. Плеханова, квалификация -- оценка и экспертиза объектов и прав собственности, специальное образование в области независимой технической экспертизы транспортных средств: Диплом ПП-I № 424167, квалификация: эксперт-техник (специализация 150210 специальности 190601.65 – Автомобили и автомобильное хозяйство), состоящий в Государственном реестре экспертов-техников (№ в реестре 256, https://data.gov.ru/opendata/7707211418-experts,  общий трудовой  стаж 30 лет, стаж  экспертной работы  12 лет. 
\par Заключение подготовлено по месту фактического расположения ИП по адресу: г. Краснодар, 
%с/т № 2 А/О «Югтекс», 
ул. Зеленая, 472.
\vspace{4mm}
%
%%   вопросы экспертизы
\subsection{Вопросы экспертизы}

\begin{enumerate}
	\item Соответствуют ли нормо-часы, указанные в Актах об оказании услуг и заказ-нарядах к государственному контракту № 1818188100962002312194450/965/18 от 08.10.2018, нормо-часам, с учётом года выпуска автотранспортных средств, установленных заводом-изготовителем? Если - нет, то рассчитать стоимость работ, указанных в прилагаемых Актах об оказании услуг и заказ-нарядах к государственному контракту  № 1818188100962002312194450/965/18 от 08.10.2018., по нормам завода-изготовителя, действовавшим на дату проведения ремонта автотранспортных средств.
\end{enumerate}


\subsection{Для производства исследования представлено} %Название по шаблону минюста
\begin{enumerate}
	\item Копия материалов гражданского дела в том числе акты об оказании услуг и заказ-наряды к государственному контракту  № 1818188100962002312194450/965/18 от 08.10.2018  в количестве 55 шт. 
	\end{enumerate}
%
%
%\vspace{-275mm}
\addcontentsline{toc}{section}{Использованные нормативы и источники информации}
%
%\left( \addcontentsline{toc}{section}{Использованные нормативы и источники информации}

\subsection{Использованные нормативы и источники информации}
%
\begin{enumerate}
%	
%\vspace{-10mm}
%
%\item
%Федеральный закон от 31 мая 2001 г. N 73-ФЗ \emph{О государственной судебно-экспертной деятельности в Российской Федерации} 
%
%\item
%Федеральный закон от 25.04.2002 N 40-ФЗ \emph{Об обязательном страховании гражданской ответственности владельцев транспортных средств} // Собрание законодательства РФ  06.05.2002, N 18, ст. 1720
%
%\item
%Федеральный закон от 28 марта 2017 г. N 49-ФЗ \emph{О внесении изменений в Федеральный закон "Об обязательном страховании гражданской ответственности владельцев транспортных средств} // Российская газета - Федеральный выпуск №7234 (68)
%
%
%
\item 
Махнин\,Е.\,Л., Новоселецкий\, И.\,Н., Федотов\, С.\,В. \emph{Методические рекомендации по проведению судебных автотехнических экспертиз и исследований колесных транспортных средств в целях определения размера ущерба, стоимости восстановительного ремонта и оценки} // --М.: ФБУ РФЦСЭ при Минюсте России, 2018.-326 с.  ISBN 978-5-91133-185-6
%
%\item  «Исследование автомототранспортных средств в целях определения стоимости восстановительного ремонта и оценки» (Методические рекомендации для судебных экспертов, утверждено  научно-методическим советом РФЦСЭ 13 марта 2013 г., протокол №35.), Минюст России, 20013г.
%%
%\item  Судебная автотехническая экспертиза. Институт повышения квалификации Российского Федерального Центра Судебной Экспертизы, М. 2007.
%
%\item
%Положение Банка России от 19 сентября 2014 года № 432-П \emph{О единой методике определения размера расходов на восстановительный ремонт в отношении повреждённого транспортного средства} // Вестник банка России, № 93 (1571). Нормативные акты и оперативная информация
%Центрального банка Российской Федерации. Москва, 2014
%
%\item
%Положение ЦБ РФ от 19 сентября 2014 года №433-П \emph{О правилах проведения независимой технической экспертизы транспортного средства} // Вестник банка России, № 93 (1571). Нормативные акты и оперативная информация Центрального банка Российской Федерации. Москва, 2014
%
\item ГОСТ 58197-2018 \emph{Порядок проведения экспертизы качества автомототранспортных средств}
\item ГОСТ 7593-80 \emph{Покрытия лакокрасочные грузовых автомобилей. Технические требования}
\item ГОСТ 9.414-2012 \emph{Покрытия лакокрасочные. Метод оценки внешнего вида}
\item ГОСТ Р 54586-2011 \emph{Материалы лакокрасочные. Метод определения твёрдости покрытия по 	карандашу}
\item ГОСТ Р 51694-2000 \emph{Материалы лакокрасочные. Определение толщины покрытия}
\item ГОСТ 31149—2014 \emph{Материалы лакокрасочные. Определение адгезии методом решетчатого  надреза}
\item ГОСТ 9.407-2017 \emph{Единая система защиты от коррозии и старения} ПОКРЫТИЯ ЛАКОКРАСОЧНЫЕ. \emph {Термины и определения}
\item ГОСТ 9.407-2015 \emph{Единая система защиты от коррозии и старения} ПОКРЫТИЯ ЛАКОКРАСОЧНЫЕ. \emph {Метод оценки внешнего вида}
\item ГОСТ 857-1-2009 \emph{Сварка и родственные процессы}
%\item ГОСТ 15878-79  \emph{Контактная сварка. Соединения сварные. Конструктивные элементы и размеры}
%\item ГОСТ 23518-79 \emph{Дуговая сварка в защитных газах. Соединения сварные под острыми и тупыми углами. Основные типы, конструктивные элементы и размеры}
%
%\item
%ПОТ РМ-027-2003  \emph{Межотраслевые правила по охране труда на автомобильном транспорте} //
%Спб.: ЦОТПБСП, 2003. - 200 с.  ISBN 5-326-00140-3
%
%\item ТУ 017207-255-00232934-2014 \emph{Кузова автомобилей LADA. Технические требования при приемке в ремонт, ремонте и выпуске из ремонта предприятиями дилерской сети ОАО "АВТОВАЗ"}//  ОАО НВП "ИТЦ АВТО", 2014
%
%\item В.Л. Смирнов, Ю.С. Прохоров, В.С. Боюр и др. \emph{Автомобили ВАЗ. Кузова. Технология ремонта, окраски и  антикоррозионной защиты. Часть II}// -Н.Новгород: АТИС, 2001.-241с.
%
%\item ГОСТ 27.002-2015  \emph{«Надежность в технике. Термины и определения» }
%
%\item ГОСТ 2.610-2006 \emph{"Единая система конструкторской документации (ЕСКД). Правила выполнения эксплуатационных документов"}
%
%\bibitem{ca:x}
%Хусаинов\, А.\,Ш. \emph{Теория автомобиля: конспект лекций } //
%А.\,Ш.\,Хусаинов, В.\,В.\,Селифонов. – Ульяновсk: УлГТУ, 2008. – 121 с.
% 
\item \emph{Судебная автотехническая экспертиза} Институт повышения квалификации Российского Федерального Центра Судебной Экспертизы, М. 2007.
%
%\item
%\emph{Марочник сталей и сплавов} Под редакцией Зубченко А.С. — М.:
%Машиностроение, 2003 г
\item \emph Л.П. Шестопалова, Т.Е. Лихачева {Методы исследования материалов и деталей машин при проведении автотехнической экспертизы}// учеб. пособие /
. – М.: МАДИ, 2017. – 180 с.
%
%\item  Кутателадзе С.С. \emph{Основы теории теплообмена}. – М.: Атомиздат, 1979. – 416с.
%
%\item \emph{Двигатели внутреннего сгорания. Теория поршневых и комбинированных двигателей}. – Под ред. С. Орлина,  М .Круглова. М.: Машиностроение, 1983.- 372с.
%
%\item
%Корухов\,Ю.\,Г., Замиховский\, М.\,И. \emph {Криминалистическая фотография и видеозапись для экспертов-автотехников} // Практическое пособие М.: ИПК РФЦСЭ при МЮ РФ, 2006г.
%
\item
Чава\,И.\,И. \emph {Судебная автотехническая экспертиза} // Учебно-методическое пособие для  экспертов,    судей, следователей, дознавателей и адвокатов. НП «Судэкс», Москва, 2014.
%
%\item
%Краснопевцев\,Б.,В. \emph{Фотограмметрия} // Учебное пособие. МИИГАиК, 2008.
%
%\item Хрулев\,А.\,Э. \emph{Ремонт двигателей зарубежных автомобилей} // Производственно-практ. издание --М.: Издательство "За рулем", 1999. --440 с., ил., табл.  ISBN 5-85907-084-5
%
%\item \emph{Исследование транспортных средств в целях определения стоимости восстановительного ремонта и оценки: курс лекций} // под общ. ред. д-ра юрид. наук, профессора С.А. Смирновой; Министерство юстиции Российской Федерации, Федеральное бюджетное учреждение Рос. Федер. центр судеб экспертизы. - М.: ФБУ РФЦСЭ при Минюсте России, 2017. - 286 с.
%
%\item  \emph{Технологическое руководство «Приемка, ремонт и выпуск из ремонта кузовов легковых автомобилей предприятиями автотехобслуживания»} РД 37.009.024-92
%
%\item Карасев А.\,И. \emph{Теория вероятностей и математическая статистика} : учебник. М.: Статистика, 1979. 279 с.
%
%\item А.\,П. Ковалев, А.\,А. Кушель, И.\,В. Королев, П.\,В. Фадеев	\emph{Практика оценки стоимости машин и оборудования} : учебник //  ; под ред. М. А. Федотовой. М. : Финансы и статистика, 2005. 272 с.
%\item
%Утакаева И.Х. \emph{Применение пакета статистического анализа Python для анализа данных автомобильного рынка}// Вестник Алтайской академии экономики и права. – 2019. – № 2-2. – С. 346-351; 
%
%\item Макушев\,Ю.П., Иванов\,А.Л. \emph{Упрощенный расчет турбокомпрессора для двигателя внутреннего сгорания} // Сибирская государственная автомобильно-дорожная академия, г. Омск, Омский научный вестник № 4 (73), 2008
%
%\item Герт Хак \emph{Турбодвигатели и компрессоры} // Справ. пособие : Пер. с нем., Лангкабель. - М. : АСТ : Астрель, 2003. - 350 с. : ил., 25 см. ISBN 5170193777
%
%\item Повреждения поршней // Группа Motorservice, Kolbenschmidt Pierburg, www.ms-motorservice.com
%
%\item 	\emph{Повреждения подшипников скольжения}. Kolbenschmidt / MS Motorservice International Gmbh
%
%\item Хрулев А.Э., Кротов М.В. \emph{Влияние неисправностей в системе смазки на характер повреждения подшипников ДВС}. Научно-технический журнал "Двигатели внутреннего сгорания" №1/2018, DOI: 10.20998/0419-8719.2018.1.13
%
%\item  \emph{Анализ и предотвращение отказов подшипников}. https://access.cummins.com
%
%\item 	\emph{Руководство по замене вкладышей и устранению повреждений вкладышей}. Federal-Mogul Corportion/ General Distribution Centre, Belgium
%
\item
Автомобильный справочник //  Пер. с англ.  -- 2-е изд., перераб. и доп. --М.:ЗАО "КЖИ "За рулем", 2004. --992 с.: ил.  ISBN 5-85907-327-5
%
%\item
%Справочник физических свойств материалов //  www.matweb.com   by MatWeb, LLC
%\item
%Суворов\,Ю.\,Б. \emph{Судебная дорожно-транспортная экспертиза} // М.\,«Экзамен»
%
\item
Чалкин\,А.\,В.  \emph{Осмотр, фиксация и моделирование механизма образования внешних повреждений автомобилей с использованием их масштабных изображений} / А.\,В. Чалкин, А.\,Л. Пушнов, В.\,В. Чубченко // Учебное пособие -- М.:ВНКЦ МВД СССР 1991г.
%
%URL: https://vaael.ru/ru/article/view?id=334
%
%\bibitem{torm}
%Булгаков \,Н.\,А.  \emph{Исследование динамики торможения автомобиля} / Н.\,А. Булгаков, А.Б. Гредескул, С.\,И Ломака // Научное сообщение № 18.- Харьков: Изд-во Харьковского госуниверситета, 1962. - 36 с.
%
%\bibitem{prich}
%Бутырин\, А.\,Ю. \emph{Характеристика обстоятельств, имеющих значение при проведении казуальных автотехнических и строительно-технических судебно-экспертных исследований} / А.\,Ю. Бутырин, Д.\.С. Дубровский, Е.\,А. Холина, И.\,И. Чава // Юридический журнал № 4,- М.:  2010

\item Петер фон ден Керкхофф, Гельмут Хааген. \emph{«Каталог повреждений лакокрасочных покрытий»}//-- М., Издательский дом Третий Рим, 2004.
\item Технические бюллетени авторемонтных материалов CarSYSTEM, \url{https://www.carsystem.org}, \url{http://www.carsystem.ru}
%
\item Шпатлевки полиэфирные «Novol». Технические карты. \url{http://professional.novol.pl/ru/}
%\item
%\emph{Сервис по автоматической расшифровке VIN номеров – AudaVIN} // Audatex
%
\item
\emph{Методика окраски и расчета стоимости лакокрасочных материалов для проведения окраски транспортных средств   AZT}// http://azt-automotive.com,   http://www.schwacke.ru/\-ereonline.htm

%\item \emph{CRASH 3 Technical Manual}. National Highway Traffic Safety Administration, \url{www-nass.nhtsa.gov}
%%
%\item
%\emph{Информационно-справочная система по ремонту и сервисному обслуживанию автомобилей Autodata}// Autodata online, лицензионный код GA0288799, Autodata Group Ltd, Великобритания. Дистрибьютор в России ЗАО «Легион-Автодата», www.autodata.ru, www.autodata-online.ru
%
\item
\emph{Специализированное программное обеспечение для расчета стоимости  восстановительного ремонта, содержащее нормативы трудоемкости работ, регламентируемые изготовителями транспортного средства}//   AudaPadWeb, лицензионное соглашение № AS/APW-658  RU-P-409-409435

%\item
%\emph{Электронный каталог деталей автомобилей}// www.elcats.ru
%
\item
\emph{Электронный каталог деталей автомобилей}// \url{http://zavod-nn.ru/}

%\item
%\emph{Электронный каталог деталей автомобилей}//http://lada-original.ru/catalog/
%%
%\item  \emph{Электронная база данных РСА  средней стоимости запасных частей и нормо-часа работ} //http://prices.autoins.ru/spares/
%%
%\item
%\emph{Цены оригинальных деталей на автомобили Мерседес Бенц}https://partsprice.mercedes-benz.ru/?owda=misc
%%
%\item Интернет-ресурсы:\\
%\url{emex.ru}\\
%\url{exist.ru}\\
%\url{autopiter.ru}\\
%\url{autostels.ru}\\
%\url{autokorea22.ru}\\
%\url{chevrolet-daewoo-parts.ru}
%
%Аверьянова Т.В., Белкин Р.С., Корухов Ю.Г., Россинская Е.Р. Криминалистика. – М.: Норма, 2003.
%Акимов С.В. Электрооборудование автомобилей: Учебник для вузов/ С.В.Акимов, Ю.П.Чижков./ М.:Аист Принт Изд-во ООО, 2003.
%Андрианов Ю.В. Как оценить и возместить ущерб от дорожно-транспортного происшествия. - М.: «Дело», 2002.
%Арбитражный процессуальный кодекс Российской Федерации.
%Арсеньев В.Д. Соотношение понятий предмета и объекта в судебной экспертизе // Проблемы теории судебной экспертизы: Сб. науч. тр. ВНИИСЭ. – М., вып. 44, 1980.
%Бабков В.Ф. Автомобильные дороги. – М.: Транспорт, 1983.
%Бабков В.Ф. Дорожные условия и безопасность движения. – М.: Транспорт, 1993.
%Байэтт Р., Уоттс Р. Расследование дорожно-транспортных происшествий / Пер. с англ. – М.: Транспорт, 1983.
%Белкин Р.С. Курс криминалистики. – М.: Закон и право, 2001.
%Бирюков Б.М.. Интернет-справочник автомобилиста 2001: Автосправка; Маркировка автомобилей; Автомобильные каталоги и базы данных; Продажа и покупка автомобилей; Запчасти к автомобилям; Тюнинг, ремонт и ТО автомоб. – М.:ТИД КОНТИНЕНТ-Пресс, 2001.
%Боровских Ю.И. Техническое обслуживание и ремонт автомобилей. – М.: Высшая школа, 1988.
%Винберг А.И., Мирский Д.Я., Ростов М.Н. Гносеологический, информационный и процессуальный аспекты учения об объекте судебной экспертизы // Вопросы теории и практики судебной экспертизы: Сб. науч. тр. ВНИИСЭ. – М., 1983.
%Возможности производства судебной экспертизы в государственных судебно-экспертных учреждениях Минюста России – М., Антидор, 2004.
%Гаврилов К.Л. Диагностика электрооборудования автомобилей: Практическое руководство /Константин Гаврилов./ М.:НЦ ЭНАС Изд-во ЗАО, 2001.
%Гражданский процессуальный кодекс Российской Федерации.
%Грановский Г.Л., Поляков В.З., Майлис Н.П. Математическое моделирование в производстве трасологических экспертиз // Моделирование при производстве трасологических экспертиз: Сб. науч. тр. ВНИИСЭ. – М., вып. 49, 1981.
%Грановский Г.Л., Поляков В.З., Майлис Н.П. Математическое моделирование в производстве трасологических экспертиз // Моделирование при производстве трасологических экспертиз: Сб. науч. тр. ВНИИСЭ. – М., вып. 49, 1981.
%Григорян В.Г. Определение наличия (отсутствия) у водителя ТС технической возможности предотвратить наезд на пешехода // Проблемы судебной автотехнической экспертизы. – М. ВНИИСЭ, 1988.
%Григорян В.Г. Применение в экспертной практике параметров торможения автотранспортных средств: Методические рекомендации. – М.: РФЦСЭ, 1995.
%Дефекты автомобильных шин. Каталог. – М.: НИИШП, 2000.
%Дорожная терминология: Справочник / Под ред. М.И. Вейцмана. – М.: Транспорт, 1985.
%Жилинский Г.В., Суворов Ю.Б. Особенности исследования технического состояния транспортных средств, участвовавших в ДТП // Автомобильный транспорт. – 1986. – № 9.
%Жулев В.И. Предупреждение дорожно-транспортных происшествий. – М.: Юридическая литература, 1989.
%Замиховский М.И. и др. Определение характера движения ТС по следам колес: Методическое письмо. – М.: ВНИИСЭ, 1993.
%Замиховский М.И. и др. Экспертное исследование следов на ТС, возникших при ДТП: Метод. письмо. – М.: ВНИИСЭ, 1994.
%Замиховский М.И. Экспертная реконструкция механизма ДТП по его следам: Автореф. канд. дис. – М.: ВНИИСЭ, 1991.
%Замиховский М.И., Самарина Т.М. Комплексное изучение следов в целях идентификации человека, находившегося на месте водителя в момент ДТП // Экспертная техника. – М.: ВНИИСЭ, 1990. – Вып.114.
%Зинин А.М., Майлис Н.П. Судебная экспертиза (учебник для вузов). – М., 2002.
%Идентификация легковых автомобилей. Европа, Азия, Америка. — М., 1999 — 2000.
%Иларионов В.А., Чернов В.И., Дадашев Ф.А. Расчет параметров маневра транспортных средств: Метод. письмо для экспертов. – М.: ВНИИСЭ, 1988.
%Исаев А.А., Иванова Н.Ю., Замиховский М.И. Трасологическое определение механизма наезда ТС на пешехода: Метод реком. – М.: ВННИСЭ, 1990.
%Исследование обстоятельств дорожно-транспортного происшествия», Чава И.И., Москва, 2007 г.
%Каталоги запасных частей автомобилей ЗАЗ, ВАЗ, АЗЛК и Иж, ГАЗ (легковые), ГАЗ (грузовые), УАЗ, ЗиЛ (грузовые), МАЗ, КамАЗ, КРАЗ, «Урал», автобусов РАФ, ЛИАЗ, ПАЗ, тракторов. — М.: 2000 «Прайс - Н».
%Классификация основных методов судебной экспертизы. – М.: ВНИИСЭ, 1982.
%Кодекс Российской Федерации об административных правонарушениях.
%Колдин В.Я., Полевой Н.С. Информационные процессы и структуры в криминалистике. – М.: МГУ, 1985.
%Колеса и шины: Краткий справочник: Выпуск 2/Сост. А.М.Ладыгина./ М.:Катера и Яхты Журнал ЗАО КП, 2003.
%Комментарий к законодательству о судебной экспертизе – М., Норма, 2004.
%Комментарий к Федеральному закону «О государственной судебно-экспертной деятельности в Российской Федерации». – М., Проспект, 2002.
%Корухов Ю.Г. Криминалистическая диагностика при расследовании преступлений. – М.: Норма, 1998.
%Корухов Ю.Г. Трасологическая диагностика: Метод пособ. – М.: ВНИИСЭ, 1983.
%Косенков А.А. Диагностика неисправностей автоматических коробок передач и трансмиссий – Ростов-на-Дону: Изд-во ЗАО НЦ ЭНАС, 2003.
%Косенков А.А. Устройство тормозных систем иномарок и отечественных автомобилей – Ростов-на-Дону: Изд-во ООО Аист Принт, 2003.
%Кренцель Р.Б. Применение в экспертной практике экспериментально-расчетных значений параметров легкости рулевого управления автомобилей: Метод. реком. – М.: ВНИИСЭ, 1993.
%Курзуков Н.И. Аккумуляторные батареи: Краткий справочник/Н.И.Курзуков, В.М.Ягнятинский./ 2003.
%Майлис Н.П. Судебная трасология. М.: Право и закон, 2003.
%Методические рекомендации по применению нормативных документов (актов) в автотехнической экспертизе. РФЦСЭ, 2004.
%Методические рекомендации: Установление признаков дифференциации и характеристик покрытий автомобильных дорог на месте дорожно-транспортных происшествий. – М.: РФЦСЭ при Минюсте России, 1995.
%Методическое руководство Определение стоимости, затрат на восстановление и утраты товарной стоимости автотранспортных средств. - СПб, СЗРЦСЭ Минюста России, РФЦСЭ при Минюсте России, 2001.
%Назначение и производство судебных экспертиз: Пособ. для следователей, судей и экспертов. – М.: Юридическая литература, 1988
%Назначение и производство судебных экспертиз: Пособие для следователей. /Под ред. Аринушкина Г.П., Шляхова А.Р. – М.: Юридическая литература, 1988.
%Немчинов М.В. Сцепные качества дорожных покрытий и безопасность движения автомобиля. – М.: Транспорт, 1985.
%Орлов Ю.К. Судебная экспертиза как средство доказывания в уголовном судопроизводстве – М., ИПК РФЦСЭ, 2005.
%Основные положения по допуску транспортных средств к эксплуатации и обязанности должностных лиц по обеспечению безопасности дорожного движения. – М.: Транспорт, 1995.
%Руководства изготовителей по эксплуатации транспортных средств.
%Сильянов В.В. Транспортно-эксплуатационные качества автомобильных дорог. – М.: Транспорт, 1984.
%Синельников Р.А., Лосавио С.К. Ремонт аварийных кузовов легковых автомобилей отечественного и иностранного производства. – М., Транспорт. 2001
%Сова Ф.П. Определение типов и моделей автотранспортных средств по следам шин: Учеб. пособ. – М.: ВШ МВД СССР, 1973.
%Современные возможности судебных экспертиз. Под ред. Ю.Г. Корухова – М., 2000.
%Солохин А.А. Судебно-медицинская экспертиза в случаях автомобильной травмы. – М., 1968.
%Соснин Д.А. Автотроника: Электрооборудование и системы бортовой автоматики современных легковых автомобилей: Учебное пособие - специалисту по ремонту и владельцам автомобилей/Дмитрий Соснин./ 2001.
%Справочник по месторасположению идентификационных номеров на легковых автомобилях. — М., 1996. — Вып. 2.
%Строительная, дорожная и специальная техника: Краткий справочник. А.А. Глазов и др. — М., 1998.
%Суворов Ю.Б. Анализ влияния эксплуатационных факторов системы ВАД для экспертного исследования причин ДТП // Теоретические и методические вопросы судебной экспертизы: Сб. науч. тр. ВНИИСЭ. – М., 1988.
%Суворов Ю.Б. Анализ влияния эксплуатационных факторов системы ВАД для экспертного исследования причин ДТП В сб. Теоретические и методические вопросы судебной экспертизы. – М.: ВНИИСЭ, 1988.
%Суворов Ю.Б. и др. Диагностическое исследование элементов автомобильных дорог, влияющих на безопасность дорожного движения (дорожных условий), на участках ДТП: Метод. пособ. для экспертов, следователей и судей. – М.: ВНИИСЭ, 1990.
%Суворов Ю.Б. Интерпретация и использование выводов экспертов по результатам производства экспертиз водителя и дороги. В сб. Вопросы теории и практики судебной экспертизы. – М.: ВНИИСЭ, 1990.
%Суворов Ю.Б. Комплексное экспертное исследование причин ДТП. Учет факторов системы ВАД при установлении непосредственных причин ДТП экспертом // Экспертная практика и новые методы исследования. – М.: ВНИИСЭ, 1993. – Вып. 9.
%Суворов Ю.Б. Новые виды, состояние и перспективы развития САТЭ // Проблемы судебной автотехнической экспертизы: Сб. науч. тр. ВНИИСЭ. – М., 1988.
%Суворов Ю.Б. Судебная дорожно-транспортная экспертиза. Судебно-экспертная оценка действий водителей и других лиц, ответственных за обеспечение безопасности дорожного движения на участках ДТП. Учебное пособие. – М.: Экзамен, 2003.
%Суворов Ю.Б. Судебная дорожно-транспортная экспертиза. Технико-юридический анализ причин дорожно-транспортных происшествий и причинно-действующих факторов. Учебное пособие. – М.: Приор, 1998.
%Суворов Ю.Б. Установление признаков дифференциации покрытий и характеристик автомобильных дорог на месте дорожно-транспортного происшествия: Метод. реком. – М.: РФЦСЭ, 1995.
%Суворов Ю.Б., Кочнев В.А. Методика и результаты экспериментального исследования влияния неравномерности снижения сцепных свойств дороги на параметры движения автомобиля при торможении. Инф. сборник. Экспертная практика и новые методы исследования. – М.: ВНИИСЭ, 1993. вып. 2.
%Суворов Ю.Б., Решетников Б.М., Кочнев В.А. Результаты экспериментального определения коэффициентов сцепления дорожных покрытий. В сб. «Экспертная техника», № 117. – М.: ВНИИСЭ, 1990.
%Судебная автотехническая экспертиза: Методическое пособие для экспертов-автотехников, следователей и судей /Под ред. В.А. Иларионова. – М.: ВНИИСЭ, 1980. – Ч. 2.
%Судебная дорожно-транспортная экспертиза». Экспертное исследование технического состояния дорог, дорожных условий на месте дорожно-транспортного происшествия, Суворов Ю.Б., Панина А.С., Москва, 2007 г.
%Транспортно-трасологическая экспертиза о дорожно-транспортных происшествиях», выпуск 1, методическое руководство для экспертов. Москва 2006 г.
%Транспортно-трасологическая экспертиза о дорожно-транспортных происшествиях», выпуск 2, методическое руководство для экспертов. Москва 2006 г.
%Транспортно-трасологическая экспертиза по делам о дорожно-транспортных происшествиях (Диагностические исследования): Метод. пособ. для экспертов, следователей и судей /Под редакцией Ю.Г. Корухова. – М.: ВНИИСЭ, 1988. – Вып. 1 и 2.
%Уголовно-процессуальный кодекс Российской Федерации.
%Федеральный закон от 31 мая 2001 г. № 73-ФЗ «О государственной судебно-экспертной деятельности в Российской Федерации».
%Фотография и видеозапись для экспертов-автотехников». Ю.Г.Корухов, Москва 2006 г.
%Чубченко А.П. и др. Отпечатки протекторов автотранспортных средств: Учеб. пособ. – М.: ВНИИ МВД СССР, 1987.
%Шляхов А.Р. Судебная экспертиза: организация и проведение. – М.: Юридическая литература, 1979.
%Эксперт // Руководство для экспертов органов внутренних дел и юстиции. Под ред. Т.В. Аверьяновой, В.Ф. Статкуса. – М., 2002.
%%
\end{enumerate}


%%%%%%%%%%%%%%%%%%%%%%%%%%%%%%%%%%%%%%%%%%%%%%%%%%%%%%%%%%%%%%%%%%%%%%%%%%%%%%%%%
\subsection{Технические средства}  %% Список не удалять!!!
\begin{itemize}

%
%\item  Специализированное программное обеспечение для расчёта стоимости  восстановительного ремонта, содержащее нормативы трудоёмкости работ, регламентируемые изготовителями транспортного средства     AudaPadWeb, лицензионное соглашение № AS/\- APW-658  RU-P-409-409435

\item  ПЭВМ под управлением операционной системы Windows 10 с установленным набором макрорасширений LaTeX системы компьютерной вёрстки TeX, cвободная лицензия LaTeX Project Public License (LPPL)
%	
\end{itemize}
%%%%%%%%%%%%%%%%%%%%%%%%%%%%%%%%%%%%%%%%%%%%%%%%%%%%%%%%%%%%%%%%%%%%%%%%%%%%%%%%%%%%%%%%%%%%%%%%%%%%%%
\subsection{Условные обозначения и принятые сокращения}
\begin{description}
%	 
%%\item[АВС] --Антиблокировочная система
\item[АМТС] -- автомототранспортное средство
\item[ДВС] -- двигатель внутреннего сгорания
\item[ГБЦ] -- головка блока цилиндров
\item[ДТП] -- дорожно--транспортное происшествие
\item[гос.\,рег.\,знак] -- государственный регистрационный знак
\item[КТС] -- колесо-транспортное средство 
\item[ЛКП] -- лакокрасочное покрытие
\item[МКПП] -- механическая коробка перемены передач
%\item[л.д.] --Лист дела
%%\item[Колесо турбины]  -- крыльчатка турбины
\item[ТС] -- транспортное средство
%\item[ТK, ТКР] -- Турбокомпрессор. Состоит из двух частей: турбины и компрессора, объединенных общим валом. Вал вращается в подшипниках, размещенных в центральном корпусе ТК
\item[ЭБУ] -- электронный блок управления
%%\item[FRAME] "--*Номер кузова транспортного средства, выпущенного для продажи на внутреннем рынке Японии и содержащий информацию производителя о транспортном средстве
\item[DTC] --Diagnostic Trouble Codes, диагностические коды неисправностей
\item[VIN] -- vehicle identification number, 17--значный идентификационный номер транспортного средства, соответствующий стандарту ISO 3779--2012
\item[н/н мехобработка] -- не нормированная изготовителем транспортного средства трудоёмкость механической обработки детали
\item[н/у] -- не определена изготовителем  
%
\end{description}
\subsection{Методы исследования}
\begin{itemize}
\item Экспертный метод (метод экспертной оценки) -- совокупность операций по выбору комплекса или единичных характеристик объекта, определение их действительных значений и оценка экспертом их соответствия установленным требованиям и/или технической информации;
\item Метод сравнения -- сопоставление фактических данных и данных источников.
\end{itemize}

\subsection{Исходные данные}

Акты об оказании услуг и заказ-наряды к государственному контракту № 1818188100962002312194450/965/18 от 08.10.2018г. в количестве 55 шт. л.д. (...)

Стоимость  нормо-часа работ и стоимость запасных частей определена условиями государственного контракта  № 1818188100962002312194450/965/18 от 08.10.2018 945.25 р/час.\\

   \pagebreak      
%\subsection{Обстоятельства дела}
%\begin{itemize}
%\item 03.09.2019 автомобиль \тс на эвакуаторе был доставлен в сервисный центр. При работе двигателя при увеличении оборотов до 2000 об/мин слышен стук в ДВС. 
%%
%\end{itemize}
% 
%
%
\section{Исследование}

Исследование проводится на основании  трудоемкости работ (услуг) на техническое обслуживание (ТО) и ремонт автомобилей LADA
Priora, разработаных ООО "ИТЦ АВТОСФЕРА" на основании действующей нормативно-технической и
технологической документации на ТО и ремонт автомобилей LADA, [],[],[],[].  Указанные трудоемкости  распространяются на работы (услуги) по ТО и ремонту автомобилей LADA, выполняемые
предприятиями технического обслуживания автомобилей (ПТОА).  Согласно Общих положений 


\subparagraph{Позиции трудоемкостей.} Каждой позиции трудоемкости присвоен неповторяющийся пятизначный номер. Первые две цифры
номера позиции указывают на раздел, к которому она относится:\\
00 - техническое обслуживание\\
10 - 84 - ремонт\\
85 - окраска\\
86- антикоррозионная обработка\\
87– поиск не явно выраженных неисправностей\\
Три последние цифры в номерах позиций представляют собой собственно порядковый номер позиции
внутри разделов и подразделов.

\subparagraph{Применение надбавок.}  Предприятие может применять надбавки к настоящим трудоемкостям при ремонте автомобилей старше
5 лет до 10%, старше 8 лет до 20% (кроме работ разделов: техническое обслуживание, окраска,
антикоррозионная обработка, поиск не явно выраженных неисправностей).







Согласно поставленных судом вопросов экспертизы, исследование должно быть проведено по действующим на момент ремонта  нормам завода-изготовителя с учётом года выпуска автотранспортных средств. Таким образом, настоящее исследование проводится по предоставленным доказательствам   согласно рекомендаций изготовителя транспортного средства с учётом <<Общих положений. Трудоемкости работ по техническому обслуживанию и ремонту автомобилей LADA>>, [6], [7], [8], [9] cписка литературы.\\
 Каждой позиции трудоемкости соответствует:\\
	• код детали/узла (семь цифр) и код работы (две цифры);\\
%	• наименование детали, краткая характеристика выполняемой работы и при необходимости полная характеристика;
%	• норма времени на выполнение работ и максимально допустимое количество повторов выполнения этой работы на одном автомобиле (указывается через звездочку *). Если звездочка отсутствует, то количество равно "1".
%	Прочерк вместо нормы времени указывает на то, что данная работа не выполняется на соответствующей модели автомобиля.
    • Трудоемкость работ (услуг) по ТО и ремонту автомобилей, не предусмотренных изготовителем, отсутствующим в справочниках трудоемкостей завода-изготовителя, определяется по согласованию с заказчиком.
    
    Каждой позиции трудоемкости присвоен неповторяющийся пятизначный номер. Первые
    две цифры номера позиции указывают на раздел, к которому она относится:\\
    00 - техническое обслуживание\\
    10 - 84 - ремонт\\
    85 - окраска\\
    86 - антикоррозионная обработка\\
    87 – позиции для оформления непредусмотренных работ.\\
    
    \фото{коды}{Таблица кодов работ семейства автомобилей LADA. Фрагмент.} 
%	6. В трудоемкостях учтены операции по ТО и ремонту автомобилей в условиях предприятия. Оплата за работы вне предприятия или доставка автомобилей на предприятие для ТО и ремонта взимается с заказчика отдельно.
%	Разработка норм времени производилась с использованием технологического оборудования, оснастки и механизированного инструмента в условиях аналогичных условиям предприятия сервисно-сбытовой сети с привлечением механиков имеющих квалификацию на проведение ремонта и технического обслуживания не специализирующихся на какой-либо конкретной системе автомобиля.

Раздел "Общие положения" содержит
    • Предприятие может применять надбавки к настоящим трудоёмкостям при ремонте автомобилей старше 5 лет до 10\%, старше 8 лет до 20\% (кроме работ разделов: техническое обслуживание, окраска, антикоррозионная обработка, поиск не явно выраженных неисправностей).
    
    
    \фото{надбавка}{Фрагмент  <<{\footnotesize Автомобили LADA PRIORA. Трудоемкости работ (услуг) по техническому обслуживанию и ремонту /Куликов А.В., Христов П.Н. и др. -- ООО "ИТЦТ АВОСФЕРА", Тольяти}>>}
%	7. Стоимость выполнения работ рассчитывается на основании приведённых норм времени и действующей на предприятии на дату ремонта стоимости нормо-часа и представляет собой только стоимость услуги, а стоимость заменяемых агрегатов, узлов и деталей, а также основных материалов, используемых при ТО и ремонте автомобилей, оплачивается заказчиком отдельно.
%	  ТРУДОЕМКОСТИ РАБОТ ПО ТЕХНИЧЕСКОМУ ОБСЛУЖИВАНИЮ И РЕМОНТУ АВТОМОБИЛЕЙ LADA содержат следующие ОБЩИЕ ПОЛОЖЕНИЯ, идентичные для исследуемых моделей транспортных средств.   В данных трудоемкостях указано время (в часах и десятых долях часа) для выполнения операций мойки, сма­зочно-заправочных, ремонта, замены, диагностики, регулировки, окраски, антикоррозионной обработки.2  Трудоемкости  распространяются  на  работы  (услуги)  по  ТО  и  ремонту  автомобилей  LADA,  выполняемые предприятиями технического обслуживания автомобилей (ПТОА).Трудоемкости работ на  предпродажную  подготовку  автомобилей  LADA  приведены в  технологической инст­рукции "Автомобили LADA  -  предпродажная подготовка".3  Каждой позиции трудоемкости присвоен неповторяющийся пятизначный номер.  Первые две цифры номера позиции указывают на раздел, к которому она относится:00 - техническое обслуживание10 - 84 - ремонт85 - окраска86 - антикоррозионная обработка87 - поиск не явно выраженных неисправностейТри последние  цифры в номерах позиций представляют собой собственно порядковый номер позиции внутри разделов и подразделов.В разделе "Техническое обслуживание"  позиции сгруппированы по подразделам в зависимости от вида выпол­няемой работы (по кодам работ).В разделе "Ремонт"  первые две цифры дополнительно указывают на систему автомобиля, к которой относится объект работы, и позиции сгруппированы по подразделам в зависимости от принадлежности к соответствующей системе/подсистеме автомобиля.Внутри разделов и подразделов позиции  трудоемкостей расположены в порядке возрастания номеров деталей (узлов) по каталогу LADA.4 Каждой позиции трудоемкости соответствует:- код детали/узла (семь цифр) и код работы (две цифры), см.таблицу 1;- наименование детали, краткая характеристика выполняемой работы и при необходимости полная характери­стика;- норма времени на выполнение работ и максимально допустимое количество повторов выполнения этой рабо­ты на одном автомобиле (указывается через звездочку *).
%}

Обозначения столбцов в расчётных таблицах:
\begin{description}
	\item[N] -- номер позиции
	\item[Наименование] -- наименование ремонтной операции 
	\item[Кол.оп.] -- количество операций
	\item[Фактическая трудоемкость, н/ч] -- норма времени по заказ-наряду
	\item[Всего, руб] -- цена операции, произведение норматива и единицы нормо-часа (945,25 р.)
	\item[Код детали, код работы] -- соответствующий  позиции трудоемкости  семизначный код детали и двухзначный код  работы по каталогу LADA
	\item[№ позиции] -- уникальный пятизначный номер позиции	
	\item[Нормативная трудоемкость, н/ч] -- трудоемкость операции, расчитанная по справочнику
	\item[Допустимая надбавка] -- допустимое изготовителем увеличение времени ремонта  в  \% в зависимости от возраста автомобиля
%	\item[label] description 
\end{description} 

\vspace{3mm}

Совокупным анализом представленных актов и заказ-нарядов, установлено, что отдельные трудоемкости, указанные в документах не соответствуют нормативам изготовителя и не включают необходимые сопутствующие операции.\\

\begin{tabular}{|c|c|l|c|c|c|c|}
	\hline
	\multicolumn{2}{|c|}{Работа} & Состав работ & Код детали, код работы & № позиции & Трудоемкость &  \\
	\hline
	\multicolumn{2}{|c|}{Замена амортизатора передней подвески} & Стойка телескопическая передей подвески левая в сборе с-у & 2901031,20 & 29006 & 0,55 &  \\
	\hline
	\multicolumn{2}{|c|}{} & Стойка телескопическая переней подвески левая в сборе - разборка снятой стойи в сборе до 2905003 & 2901031.22 & 29007 & 0,55 &  \\
	\hline
	&  &  &  &  &  &  \\
	\hline
	&  &  &  &  &  &  \\
	\hline
	&  &  &  &  &  &  \\
	\hline
	&  &  &  &  &  &  \\
	\hline
\end{tabular}



\begin{enumerate}


%2455

\item \par\textbf{{Работы по заказ-наряду № 1818188100962002312194450/\-АСЮ0002455 от 19.11.2018, автомобиль Лада Приора,  VIN XTA217030B0327141}}

Исследуемый автомобиль:  Лада Приора, гос. номер: Т 4106 23 VIN: ХТА217030В0327141 год вып. 2011



\фото{foto/2455}{}

Нормо-часы, указанные в Акте об оказании услуг и заказ-наряде не соответствуют нормо-часам,  установленным заводом-изготовителем.\\
Стоимость работ по нормам завода-изготовителя, действовавшим на дату проведения ремонта автотранспортных средств составляет: 945.25 * 20.92 = 19773,68 (Девятнадцать тысяч семьсот семьдесят три) рубля, 68 коп.

Экономия по заказ-наряду составляет -65.21 (Шестьдесят пять) рублей, 21 коп.  
\vspace{3mm}



%2456

\item \par\textbf{{Работы по заказ-наряду № 1818188100962002312194450/\-АСЮ0002456 от 19.11.2018, автомобиль Лада Приора,  VIN  XTA217, регистрационный знак Т4106123, год вып. 2011}}

Исследуемый автомобиль:  Лада Приора гос. номер: Т4106 23 VIN: ХТА217 год вып. 2011


\фото{foto/2456}{}

Нормо-часы, указанные в Акте об оказании услуг и заказ-наряде не соответствуют нормо-часам,  установленным заводом-изготовителем.\\
Стоимость работ по нормам завода-изготовителя, действовавшим на дату проведения ремонта автотранспортных средств составляет: 945.25 * 12.25 = 11579.31 (Одиннадцать тысяч пятьсот семьдесят девять) рубль, 31 коп.

Экономия по заказ-наряду составляет -425.34 (Четыреста двадцать пять) рублей, 34 коп.  
\vspace{3mm}


%2457

\item \par\textbf{{Работы по заказ-наряду № 1818188100962002312194450/\-АСЮ0002457 от 20.11.2018, автомобиль ВАЗ-21150, VIN XTA21150B5037281
}}


Исследуемый автомобиль:  ВАЗ-21150 гос. номер: О171ЕА 123 VIN: XTA211540B5037281 год вып. 2009.



\фото{foto/2457}{}
\фото{foto/24571}{}

Нормо-часы, указанные в Акте об оказании услуг и заказ-наряде не соответствуют нормо-часам,  установленным заводом-изготовителем.\\
Стоимость работ по нормам завода-изготовителя, действовавшим на дату проведения ремонта автотранспортных средств составляет: 945.25 * 26.70 = 25238.18 (Двадцать пять тысяч двести тридцать восемь) рублей, 18 коп.

Перерасход по заказ-наряду составляет 425.40 (Четыреста двадцать пять) рублей, 40 коп.  
\vspace{3mm}


\z{}{}


% 2458

\item \par\textbf{{Работы по заказ-наряду № 1818188100962002312194450/\-АСЮ0002458 от 22.11.2018, автомобиль ВАЗ 213100, VIN XTA213100G0179098}}

Исследуемый автомобиль:  ВАЗ 213100 гос. номер: А05В5 23 VIN: XTA213100G0179098 год вып. 2015



\фото{foto/2458}{}

Нормо-часы, указанные в Акте об оказании услуг и заказ-наряде не соответствуют нормо-часам,  установленным заводом-изготовителем.\\
Стоимость работ по нормам завода-изготовителя, действовавшим на дату проведения ремонта автотранспортных средств составляет: 945.25 * 19.66 = 18583.62 (Восемнадцать тысяч пятьсот восемьдесят три) рублей, 62 коп.

Перерасход по заказ-наряду составляет 3535,26 (Три тысячи пятьсот тридцать пять) рублей, 26 коп.  
\vspace{3mm}

\z{}{}
%2461

\item \par\textbf{{Работы по заказ-наряду № 1818188100962002312194450/\-АСЮ0002461 от 21.11.2018, Шевролет Нива VIN X9L212300C0}}

\фото{foto/2461}{}

Нормо-часы, указанные в Акте об оказании услуг и заказ-наряде не соответствуют нормо-часам,  установленным заводом-изготовителем.\\
Стоимость работ по нормам завода-изготовителя, действовавшим на дату проведения ремонта автотранспортных средств составляет: 945.25 * 25.72 = 24311.83 (Двадцать четыре тысячи триста одиннадцать) рублей, 83 коп.

Экономия по заказ-наряду составляет -2617.54 (Две тысячи шестьсот семнадцать) рублей, 64 коп.  
\vspace{3mm}


%2494


\item \par\textbf{{Работы по заказ-наряду 1818188100962002312194450/\-АСЮ0002494 от 01.11.2018, Лада Гранта VIN XTA219010H0474947}}

Исследуемый автомобиль:  Лада Гранта гос. номер: С978ТЕ123 VIN: ХТА219010Н0474947 год вып. 2017


\фото{foto/2494}{}

Нормо-часы, указанные в Акте об оказании услуг и заказ-наряде не соответствуют нормо-часам,  установленным заводом-изготовителем.\\
Стоимость работ по нормам завода-изготовителя, действовавшим на дату проведения ремонта автотранспортных средств составляет: 945.25 * 3.23 = 3053,16 (Три тысячи пятьдесят три) рубля, 16 коп.

Перерасход по заказ-наряду составляет 113,44 (Сто тринадцать) рублей, 44 коп.  
\vspace{3mm}


%2496


\item \par\textbf{{Работы по заказ-наряду  № 1818188100962002312194450/\-АСЮ0002496 от 02.11.2018, Лада Приора VIN XTA217050G0528754 }}

Исследуемый автомобиль:  Лада Приора гос, номер: К800РР 123 VIN: ХТА21705000528754 год вып. 2015


\фото{foto/2496}{}

Нормо-часы, указанные в Акте об оказании услуг и заказ-наряде не соответствуют нормо-часам,  установленным заводом-изготовителем.\\
Стоимость работ по нормам завода-изготовителя, действовавшим на дату проведения ремонта автотранспортных средств составляет: 945.25 * 4.74 = 4480.49 (Четыре тысячи четыреста восемьдесят) рублей, 49 коп.

Перерасход по заказ-наряду составляет 387,57 (Триста восемьдесят семь) рублей, 57 коп.  
\vspace{3mm}


\item \par\textbf{{Работы по заказ-наряду  № 1818188100962002312194450/\-АСЮ0002497 от 21.11.2018, Лада Приора VIN XTA217230A0113273}}





%2497



\фото{foto/2497}{}

Нормо-часы, указанные в Акте об оказании услуг и заказ-наряде не соответствуют нормо-часам,  установленным заводом-изготовителем.\\
Стоимость работ по нормам завода-изготовителя, действовавшим на дату проведения ремонта автотранспортных средств составляет: 945.25 * 3.78 = 3573.05 (Три тысячи пятьсот семьдесят три) рубля, 05 коп.

Экономия по заказ-наряду составляет -453.72 (Четыреста пятьдесят три) рубля, 72 коп.  
\vspace{3mm}




%2498

\z{№ 1818188100962002312194450/\-АСЮ0002498 от 02.11.2018}{Лада Приора  VIN  XTA217050F0511839}


\item \par\textbf{{Работы по заказ-наряду № 1818188100962002312194450/\-АСЮ0002498 от 02.11.2018, Лада Приора  VIN  XTA217050F0511839}}

\фото{foto/2498}{}

Нормо-часы, указанные в Акте об оказании услуг и заказ-наряде не соответствуют нормо-часам,  установленным заводом-изготовителем.\\
Стоимость работ по нормам завода-изготовителя, действовавшим на дату проведения ремонта автотранспортных средств составляет: 945.25 * 16.28 = 15388.67 (Пятнадцать тысяч триста восемьдесят восемь) рублей, 67 коп.

Экономия по заказ-наряду составляет -122.85 (Сто двадцать два) рубля, 85 коп.  
\vspace{3mm}



%2499


\item \par\textbf{{Работы по заказ-наряду  № 1818188100962002312194450/\-АСЮ0002499 от 05.11.2018, Лада Приора   VIN XTA217030B0326604}}

Исследуемый автомобиль: Лада Приора гос. номер: К0940 23 VIN: XTА217030В0326604 год вып. 2011


\фото{foto/2499}{}
\фото{foto/24991}{}

Нормо-часы, указанные в Акте об оказании услуг и заказ-наряде не соответствуют нормо-часам,  установленным заводом-изготовителем.\\
Стоимость работ по нормам завода-изготовителя, действовавшим на дату проведения ремонта автотранспортных средств составляет: 945.25 * 29.74 = 28111.74 (Двадцать восемь тысяч сто одиннадцать) рублей, 74 коп.

Экономия по заказ-наряду составляет -84.99 (Восемьдесят четыре ) рубля, 99 коп.  
\vspace{3mm}





%2500


\item \par\textbf{{Работы по заказ-наряду   № 1818188100962002312194450/\-АСЮ0002500 от 05.11.2018, Шевролет Нива  VIN  X9L21230050104236}}

Исследуемый автомобиль:  Лада Приора гос. номер: К0969 23 VIN: ХТА217030В0326953 год вып. 2011


\фото{foto/2500}{}
\фото{foto/25001}{}

Нормо-часы, указанные в Акте об оказании услуг и заказ-наряде не соответствуют нормо-часам,  установленным заводом-изготовителем.\\
Стоимость работ по нормам завода-изготовителя, действовавшим на дату проведения ремонта автотранспортных средств составляет: 945.25 * 28,07 = 26 533,17
 (Двадцать шесть тысяч пятьсот тридцать три) рублей, 17 коп.

Перерасход по заказ-наряду составляет 75.69 (Семьдесят пять ) рублей, 69 коп.  
\vspace{3mm}



%2502



\item \par\textbf{{Работы по заказ-наряду   № 1818188100962002312194450/\-АСЮ0002502 от 05.11.2018, Лада Приора VIN XTA217030B0326953}}

Исследуемый автомобиль:  Лада Приора гос. номер: К0969 23 VIN: ХТА217030В0326953 год вып. 2011




\фото{foto/2502}{}
\фото{foto/25021}{}

Нормо-часы, указанные в Акте об оказании услуг и заказ-наряде не соответствуют нормо-часам,  установленным заводом-изготовителем.\\
Стоимость работ по нормам завода-изготовителя, действовавшим на дату проведения ремонта автотранспортных средств составляет: 945.25 * 17,05 = 16116.51 (Шестнадцать тысяч сто шестнадцать) рубля, 51 коп.

Перерасход по заказ-наряду составляет 756.24 (Семьсот пятьдесят шесть ) рублей, 24 коп.  
\vspace{3mm}






%2503

\item \par\textbf{{Работы по заказ-наряду   Выполненные работы по заказ-наряду № 1818188100962002312194450/\-АСЮ0002503 от 06.11.2018
		, Лада Приора  VIN  XTA217030B0327073}}
	
	
Исследуемый автомобиль:  Лада Приора гос. номер: К0346 23 VIN: XTA217030B0327073 год вып. 2011

	
	

\фото{foto/2503}{}

Нормо-часы, указанные в Акте об оказании услуг и заказ-наряде не соответствуют нормо-часам,  установленным заводом-изготовителем.\\
Стоимость работ по нормам завода-изготовителя, действовавшим на дату проведения ремонта автотранспортных средств составляет: 945.25 * 10,26 = 9698,27 (Девять тысяч шестьсот девяносто восемь) рублей, 26 коп.

Экономия по заказ-наряду составляет -1238.25 (Одна тысяча двести тридцать восемь) рублей, 25 коп.  
\vspace{3mm}



%2504


\item \par\textbf{{Работы по заказ-наряду   Выполненные работы по заказ-наряду № 1818188100962002312194450/\-АСЮ0002504 от 06.11.2018, Лада Приора VIN XTA217030E0469542}}



Исследуемый автомобиль:  Лада Приора гос, номер: М876ОВ 123 VIN: : ХТА217030Е0469542 год вып, 2014



\фото{foto/2504}{}

Нормо-часы, указанные в Акте об оказании услуг и заказ-наряде не соответствуют нормо-часам,  установленным заводом-изготовителем.\\
Стоимость работ по нормам завода-изготовителя, действовавшим на дату проведения ремонта автотранспортных средств составляет: 945.25 * 8,67 = 8195.32 (Восемь тысяч сто девяносто пять) рублей, 32 коп.

Перерасход по заказ-наряду составляет 3194.97 ( Три тысячи сто девяносто четыре) рубля, 97 коп.  
\vspace{3mm}


\z{}{}



%2505


\item \par\textbf{{№ 1818188100962002312194450/\-АСЮ0002505 от 01.11.2018, Шевролет Нива  VIN X9L212300F0554302}}


Исследуемый автомобиль:  Шевролет Нива гос. номер Н438ОХ 123 VIN: X9L212300F0554302 год вып. 2015


\фото{foto/2505}{}

Нормо-часы, указанные в Акте об оказании услуг и заказ-наряде не соответствуют нормо-часам,  установленным заводом-изготовителем.\\
Стоимость работ по нормам завода-изготовителя, действовавшим на дату проведения ремонта автотранспортных средств составляет: 945.25 * 9,33 = 8819,18 (Восемь тысяч восемьсот девятнадцать) рублей, 18 коп.

Перерасход по заказ-наряду составляет 1342.29 ( Одна тысяча триста сорок два) рубля, 29 коп.  
\vspace{3mm}

\z{}{}

%2513



\item \par\textbf{{Работы по заказ-наряду № 1818188100962002312194450/\-АСЮ0002513 от 07.11.2018, Лада Гранта XTA219010J0547457
}}


Исследуемый автомобиль:  Лада Гранта гос. номер: У087УК 123 VIN: ХТА219010J0547457 год вып. 2018


\фото{foto/2513}{}

Нормо-часы, указанные в Акте об оказании услуг и заказ-наряде не соответствуют нормо-часам,  установленным заводом-изготовителем.\\
Стоимость работ по нормам завода-изготовителя, действовавшим на дату проведения ремонта автотранспортных средств составляет: 945.25 * 1,73 = 1635,28 (Одна тысяча шестьсот тридцать пять) рублей, 28 коп.

Перерасход по заказ-наряду составляет 207.96 ( Двести семь) рублей, 96 коп.  
\vspace{3mm}




%2514

\item \par\textbf{{Работы по заказ-наряду № 1818188100962002312194450/\-АСЮ0002514 от 02.11.2018, Лада Гранта XTA219010J0547457
}}

\фото{foto/2514}{}

Нормо-часы, указанные в Акте об оказании услуг и заказ-наряде не соответствуют нормо-часам,  установленным заводом-изготовителем.\\
Стоимость работ по нормам завода-изготовителя, действовавшим на дату проведения ремонта автотранспортных средств составляет: 945.25 * 4.15 = 3922.79 (Три тысячи девятьсот двадцать два) рубля, 79 коп.

Перерасход по заказ-наряду составляет 897,99 ( Восемьсот девяносто семь) рублей, 99 коп.  
\vspace{3mm}




%2523


\item \par\textbf{{Работы по заказ-наряду № 1818188100962002312194450/\-АСЮ0002523 от 07.11.2018, Лада Приора  VIN  XTA217050F495161
}}


Исследуемый автомобиль:  



\фото{foto/2523}{}

Нормо-часы, указанные в Акте об оказании услуг и заказ-наряде не соответствуют нормо-часам,  установленным заводом-изготовителем.\\
Стоимость работ по нормам завода-изготовителя, действовавшим на дату проведения ремонта автотранспортных средств составляет: 945.25 * 17.41 = 16456.80 (Шестнадцать тысяч четыреста пятьдесят шесть) рублей, 80 коп.

Перерасход по заказ-наряду составляет 510.46 ( Пятьсот десять) рублей, 46 коп.  
\vspace{3mm}





%2524

\item \par\textbf{{Работы по заказ-наряду № № 1818188100962002312194450/\-АСЮ0002524 от 09.11.2018, Лада Приора VIN  XTA217030A0263211
}}


Исследуемый автомобиль:  Лада Приора гос. номер: К0804 23 VIN: ХТА217030А0263211 год вып. 2010



\фото{foto/2524}{}
\фото{foto/25241}{}


Нормо-часы, указанные в Акте об оказании услуг и заказ-наряде не соответствуют нормо-часам,  установленным заводом-изготовителем.\\
Стоимость работ по нормам завода-изготовителя, действовавшим на дату проведения ремонта автотранспортных средств составляет: 945.25 * 27.64 = 26126,71 (Двадцать шесть тысяч сто двадцать шесть) рубля, 71 коп.

Экономия по заказ-наряду составляет -84.99 ( Восемьдесят четыре) рубля, 99 коп.  
\vspace{3mm}


%2532


\item \par\textbf{{Работы по заказ-наряду № 1818188100962002312194450/\-АСЮ0002532 от 08.11.2018, Лада Приора  VIN XTA217230A0107527
}}


Исследуемый автомобиль:  Лада Приора гос. номер: N1904ТО 123 VIN: ХТА217230А0107527 год вып. 2010


Исследуемый автомобиль:  Лада Приора гос. номер: N1904ТО 123 VIN: ХТА217230А0107527 год вып. 2010


\фото{foto/2532}{}


Нормо-часы, указанные в Акте об оказании услуг и заказ-наряде не соответствуют нормо-часам,  установленным заводом-изготовителем.\\
Стоимость работ по нормам завода-изготовителя, действовавшим на дату проведения ремонта автотранспортных средств составляет: 945.25 * 12.80 = 12174.82 (Двенадцать тысяч сто семьдесят четыре ) рубля, 82 коп.

Перерасход по заказ-наряду составляет 4272.57 ( Четыре тысячи двести семьдесят два) рубля, 57 коп.  
\vspace{3mm}


%2533

\item \par\textbf{{Работы по заказ-наряду  № 1818188100962002312194450/АСЮ0002533 от 09.11.2018, Лада Приора   VIN   XTA217030B0327085
}}


\фото{foto/2533}{}
\фото{foto/25331}{}
\фото{foto/25332}{}


Нормо-часы, указанные в Акте об оказании услуг и заказ-наряде не соответствуют нормо-часам,  установленным заводом-изготовителем.\\
Стоимость работ по нормам завода-изготовителя, действовавшим на дату проведения ремонта автотранспортных средств составляет: 945.25 * 29.09 = 2749.32 (Две тысячи семьсот сорок девять ) рублей, 32 коп.

Перерасход по заказ-наряду составляет 340.36 ( Триста сорок) рублей, 36 коп.  
\vspace{3mm}




%2534

\item \par\textbf{{Работы по заказ-наряду № 1818188100962002312194450/\-АСЮ0002534 от 06.11.2018, Лада Приора   VIN XTA217030B0327119
}}


Исследуемый автомобиль:  Лада Приора гос. номр К0065 23 VIN: XTA217030B0327119 год вып. 2015


\фото{foto/2534}{}
\фото{foto/25341}{}

Нормо-часы, указанные в Акте об оказании услуг и заказ-наряде не соответствуют нормо-часам,  установленным заводом-изготовителем.\\
Стоимость работ по нормам завода-изготовителя, действовавшим на дату проведения ремонта автотранспортных средств составляет: 945.25 * 14.970 = 14150.39 (Четырнадцать тысяч сто пятьдесят ) рублей, 39 коп.

Перерасход по заказ-наряду составляет 3762.14 ( Три тысячи семьсот шестьдесят два) рубля, 14 коп.  
\vspace{3mm}



%2558

\item \par\textbf{{Работы по заказ-наряду № 1818188100962002312194450/\-АСЮ0002558 от 07.11.2018, автомобиль Лада Приора  VIN XTA217030B0327995}}


Исследуемый автомобиль:  Лада Приора гос, номер: H347AX 123 VIN: ХТА207030В0327995 год вып. 2011


\фото{foto/2558}{}

Нормо-часы, указанные в Акте об оказании услуг и заказ-наряде не соответствуют нормо-часам,  установленным заводом-изготовителем.\\
Стоимость работ по нормам завода-изготовителя, действовавшим на дату проведения ремонта автотранспортных средств составляет: 945.25 * 6.53 = 6172.25 (Двадцать девять тысяч сто четыре) рубля, 25 коп.

Перерасход по заказ-наряду составляет 66.17 (Шестьдесят шесть) рублей, 17 коп.  
\vspace{3mm}



%2560



\item \par\textbf{{Работы по заказ-наряду  № 1818188100962002312194450/\-АСЮ0002560 от 13.11.2018, Лада Приора VIN  XTA217030B0062028
}}

Исследуемый автомобиль:  Лада Приора гос. номер: А395ВМ123 V1N: ХТА217130В0062028 год выл. 2011


\фото{foto/2560}{}


Нормо-часы, указанные в Акте об оказании услуг и заказ-наряде не соответствуют нормо-часам,  установленным заводом-изготовителем.\\
Стоимость работ по нормам завода-изготовителя, действовавшим на дату проведения ремонта автотранспортных средств составляет: 945.25 * 13.88 = 13120.07 (Тринадцать тысяч сто двадцать ) рублей, 07 коп.

Экономия по заказ-наряду составляет -75,60 (Семьдесят шесть) рублей, 60 коп.  
\vspace{3mm}



%2561

\item \par\textbf{{Работы по заказ-наряду  № 1818188100962002312194450/\-АСЮ0002561 от 08.11.2018, Лада Приора  VIN  XTA21703090200017
}}


Исследуемый автомобиль:  Лада Приора гос. номер: Н350ХМ 93 VIN: ХТА21703090200017 год вып. 2009






\фото{foto/2561}{}


Нормо-часы, указанные в Акте об оказании услуг и заказ-наряде не соответствуют нормо-часам,  установленным заводом-изготовителем.\\
Стоимость работ по нормам завода-изготовителя, действовавшим на дату проведения ремонта автотранспортных средств составляет: 945.25 * 5.580 = 5274.50 (Пять тысяч двести семьдесят четыре ) рублей, 50 коп.

Экономия по заказ-наряду составляет -75.60 ( Семьдесят пять) рублей, 60 коп.  
\vspace{3mm}


%2562


\item \par\textbf{{Работы по заказ-наряду   № 1818188100962002312194450/\-АСЮ0002562 от 09.11.2018, ВАЗ 213100 VIN XTA21310070086610
}}


Исследуемый автомобиль:  ВАЗ 213100 гос. номер: Т078МВ 93 VIN: ХТА21310070086610 год вып. 2007


\фото{foto/2562}{}


Нормо-часы, указанные в Акте об оказании услуг и заказ-наряде не соответствуют нормо-часам,  установленным заводом-изготовителем.\\
Стоимость работ по нормам завода-изготовителя, действовавшим на дату проведения ремонта автотранспортных средств составляет: 945.25 * 8.4 = 7940.10 (Семь тысяч девятьсот сорок) рублей, 10 коп.

Перерасход по заказ-наряду составляет -1512.39 (Одна тысяча пятьсот двенадцать) рублей, 39 коп.  
\vspace{3mm}



%2563


\item \par\textbf{{Работы по заказ-наряду   № 1818188100962002312194450/\-АСЮ0002563 от 12.11.2018, Лада Приора VIN  XTA217030B0327241
}}

Исследуемый автомобиль:  Лада Приора гос. номер: М3026 23 ViN: XTA21703DB0327241 год вып. 2011


\фото{foto/2563}{}


Нормо-часы, указанные в Акте об оказании услуг и заказ-наряде не соответствуют нормо-часам,  установленным заводом-изготовителем.\\
Стоимость работ по нормам завода-изготовителя, действовавшим на дату проведения ремонта автотранспортных средств составляет: 945.25 * 10.37 = 9802.24 (Девять тысяч восемьсот два ) рублей, 24 коп.

Экономия по заказ-наряду составляет -207.94 (Двести семь) рублей, 94 коп.  
\vspace{3mm}


%2564


\item \par\textbf{{Работы по заказ-наряду  № 1818188100962002312194450/\-АСЮ0002564 от 12.11.2018, Лада Приора  VIN  XTA217030B0327391
}}

Исследуемый автомобиль:  Лада Приора гос. номер: М3279 23 VIN: ХТА217030В0327391 год вып. 2011



\фото{foto/2564}{}


Нормо-часы, указанные в Акте об оказании услуг и заказ-наряде не соответствуют нормо-часам,  установленным заводом-изготовителем.\\
Стоимость работ по нормам завода-изготовителя, действовавшим на дату проведения ремонта автотранспортных средств составляет: 945.25 * 20.02 = 18923.32 (Восемнадцать тысяч девятьсот двадцать три ) рублей, 32 коп.

Перерасход по заказ-наряду составляет 1635,32 ( Одна тысяча шестьсот тридцать пять) рублей, 32 коп.  
\vspace{3mm}


%2569


\item \par\textbf{{Работы по заказ-наряду  № 1818188100962002312194450/\-АСЮ0002569 от 14.11.2018,Лада Приора  VIN XTA217130B0062022
}}

Исследуемый автомобиль:  Лада Приора гос. номер: С374ЕЕ 123 VIN; ХТА217130В0062022 год вып. 2011


\фото{foto/2569}{}


Нормо-часы, указанные в Акте об оказании услуг и заказ-наряде не соответствуют нормо-часам,  установленным заводом-изготовителем.\\
Стоимость работ по нормам завода-изготовителя, действовавшим на дату проведения ремонта автотранспортных средств составляет: 945.25 * 27.21 = 25720.25 (Двадцать пять тысяч семьсот двадцать ) рублей, 25 коп.

Экономия по заказ-наряду составляет -1 049.16 ( Четыреста шестьдесят три) рубля, 24 коп.  
\vspace{3mm}

%2570


\item \par\textbf{{Работы по заказ-наряду   № 1818188100962002312194450/\-АСЮ0002570 от 12.11.2018, Лада Приора   VIN  XTA217050F0508648
}}

\фото{foto/2570}{}


Нормо-часы, указанные в Акте об оказании услуг и заказ-наряде не соответствуют нормо-часам,  установленным заводом-изготовителем.\\
Стоимость работ по нормам завода-изготовителя, действовавшим на дату проведения ремонта автотранспортных средств составляет: 945.25 * 12,81 = 12108,65 (Двенадцать тысяч сто восемь ) рублей, 65 коп.

Экономия по заказ-наряду составляет -623 82( Шестьсот двадцать три) рубля, 82коп.  
\vspace{3mm}


%2571

\item \par\textbf{{Работы по заказ-наряду    № 1818188100962002312194450/\-АСЮ0002571 от 13.11.2018, Лада Приора  VIN  XTA217030B0327120
}}

Исследуемый автомобиль:  Лада Приора гос. номер: Т4036 23 VJN: ХТА217030В0327120 год вып. 2011


\фото{foto/2571}{}
\фото{foto/25711}{}

Нормо-часы, указанные в Акте об оказании услуг и заказ-наряде не соответствуют нормо-часам,  установленным заводом-изготовителем.\\
Стоимость работ по нормам завода-изготовителя, действовавшим на дату проведения ремонта автотранспортных средств составляет: 945.25 * 21.720 = 20530.83 (Двадцать тысяч пятьсот тридцать ) рублей, 83 коп.

Перерасход по заказ-наряду составляет 1918.90 ( Одна тысяча девятьсот восемнадцать) рублей, 90 коп.  
\vspace{3mm}

%%%%%%%%%%%%%%%%%%%%%%%%%%%%%%%%%%%%%%%%%

%2752





%%%%%%%%%%%%%%%%%%%%%%%%%%%%%%%%%%%%%%%%%
%2573

\item \par\textbf{{Работы по заказ-наряду    № 1818188100962002312194450/\-АСЮ0002573 от 14.11.2018,Лада Приора   VIN  XTA217030E0469310
}}

\фото{foto/2573}{}

Нормо-часы, указанные в Акте об оказании услуг и заказ-наряде не соответствуют нормо-часам,  установленным заводом-изготовителем.\\
Стоимость работ по нормам завода-изготовителя, действовавшим на дату проведения ремонта автотранспортных средств составляет: 945.25 *9,670  = 9140.50  (Девять тысяч сто сорок ) рублей, 50 коп.

Перерасход по заказ-наряду составляет 595,52 ( Пятьсот девяносто пять ) рублей, 52коп.  
\vspace{3mm}




%2601

\item \par\textbf{{Работы по заказ-наряду    № 1818188100962002312194450/\-АСЮ0002601 от 14.11.2018, Лада Приора   VIN   XTA217050F0509727
}}


Исследуемый автомобиль:  Лада Приора гос. номер: М0606 23 VIN: XTA217050F0509727 год вып. 2015	1818188100962002312194450/АСЮ0002601-29 от 14.11.2018	3


\фото{foto/2601}{}

Нормо-часы, указанные в Акте об оказании услуг и заказ-наряде не соответствуют нормо-часам,  установленным заводом-изготовителем.\\
Стоимость работ по нормам завода-изготовителя, действовавшим на дату проведения ремонта автотранспортных средств составляет: 945.25 *29,70  = 26703,31 (Двадцать шесть тысяч семьсот три ) рубля, 31 коп.

Экономия по заказ-наряду составляет -1370,60 ( Одна тысяча триста семьдесят ) рублей, 60коп.  
\vspace{3mm}

%%%%%%%%%%%%%%%%%%%%%%%%%%%%%%%%%%%%%%%%%%%%%%
%2602


\item \par\textbf{{Работы по заказ-наряду    № 1818188100962002312194450/АСЮ0002602 от 16.11.2018, Лада Приора   VIN XTA217030B0327722
}}

\фото{foto/2602}{}
\фото{foto/26021}{}


Нормо-часы, указанные в Акте об оказании услуг и заказ-наряде не соответствуют нормо-часам,  установленным заводом-изготовителем.\\
Стоимость работ по нормам завода-изготовителя, действовавшим на дату проведения ремонта автотранспортных средств составляет: 945.25 *34,71  = 32809.63 (Тридцать две тысячи восемьсот девять) рублей, 63 коп.

Перерасход по заказ-наряду составляет 3440.81 (Три тысячи четыреста сорок ) рублей, 81 коп.  
\vspace{3mm}


%2604

\item \par\textbf{{Работы по заказ-наряду    № 1818188100962002312194450/\-АСЮ0002604 от 15.11.2018, Лада Гранта  VIN  XTA219010D0152058
}}

Исследуемый автомобиль:  Лада Гранта гос. номер: Т743СВ 123 VIN: ХТА219010D0152058 год вып. 2013


\фото{foto/2604}{}
\фото{foto/26041}{}


Нормо-часы, указанные в Акте об оказании услуг и заказ-наряде не соответствуют нормо-часам,  установленным заводом-изготовителем.\\
Стоимость работ по нормам завода-изготовителя, действовавшим на дату проведения ремонта автотранспортных средств составляет: 945.25 * 20,91  = 19765.15 (Девятнадцать тысяч семьсот шестьдесят пять ) рубль, 15 коп.

Перерасход по заказ-наряду составляет 1455.77 (Одна тысяча четыреста пятьдесят пять) рублей, 74 коп.  
\vspace{3mm}

%%%%%%%%%%%%%%%%%%%%%%%%%%%%%%%%%%%%%%%%%%%%%
%%2623



%%%%%%%%%%%%%%%%%%%%%%%%%%%%%%%%%%%%%%%%%%%%%





%2648

\item \par\textbf{{Работы по заказ-наряду  № 1818188100962002312194450/\-АСЮ0002648 от 23.11.2018, Лада Приора  VIN  XTA217230A0113273
}}

\фото{foto/2648}{}


Нормо-часы, указанные в Акте об оказании услуг и заказ-наряде не соответствуют нормо-часам,  установленным заводом-изготовителем.\\
Стоимость работ по нормам завода-изготовителя, действовавшим на дату проведения ремонта автотранспортных средств составляет: 945.25 * 28.41  = 26854.55 (Двадцать шесть тысяч восемьсот пятьдесят четыре) рубль, 55 коп.

Экономия по заказ-наряду составляет -1380,02 (Одна тысяча триста восемьдесят) рублей, 02 коп.  
\vspace{3mm}




%2649

\item \par\textbf{{Работы по заказ-наряду  № 1818188100962002312194450/\-АСЮ0002649 от 23.11.2018, Лада Приора  VIN  XTA217230A0113273
}}

\фото{foto/2649}{}


Нормо-часы, указанные в Акте об оказании услуг и заказ-наряде не соответствуют нормо-часам,  установленным заводом-изготовителем.\\
Стоимость работ по нормам завода-изготовителя, действовавшим на дату проведения ремонта автотранспортных средств составляет: 945.25 * 10,75  = 10161.44 (Десять тысяч сто шестьдесят один ) рубль, 44 коп.

Перерасход по заказ-наряду составляет 425,37 (Четыреста двадцать пять) рублей, 37 коп.  
\vspace{3mm}

%2650

\item \par\textbf{{Работы по заказ-наряду   № 1818188100962002312194450/\-АСЮ0002650 от 20.11.2018, ВАЗ-21150 VIN  XTA211540B5037281
}}

Исследуемый автомобиль:  ВАЗ-21150 гос. номер: О171ЕА/123 VIN: ХТА211540В5037281 год вып. 2009


\фото{foto/2650}{}


Нормо-часы, указанные в Акте об оказании услуг и заказ-наряде не соответствуют нормо-часам,  установленным заводом-изготовителем.\\
Стоимость работ по нормам завода-изготовителя, действовавшим на дату проведения ремонта автотранспортных средств составляет: 945.25 * 10.07  = 9518.67 (Девять тысяч пятьсот восемнадцать ) рубль, 67 коп.

Перерасход по заказ-наряду составляет 217.42 (Двести семнадцать) рублей, 42 коп.  
\vspace{3mm}



%2561


	\item \par\textbf{{Работы по заказ-наряду № 1818188100962002312194450/\-АСЮ0002651 от 28.11.2018, автомобиль Лада Приора, VIN XTA217030B0327369}} соответствуют комплексу работ капитального ремонта двигателя и головки блока цилиндров  (слить масло,произвести полную разборку,(кроме узлов электрооборудования и системы питания), промыть, обдуть сжатым воздухом, продефектовать, заменить узлы и детали, собрать (с выполнением смазочных работ), залить масло, произвести обкатку на стенде. Код 100260 1004 в системе ОАО "АВТОВАЗ") с включением работ механической обработки ДВС и ГБЦ, выполняющихся на соответствующем специализированном оборудовании и   не нормируемых изготовителем ТС.\\
\фото{foto/2651}{}


Нормо-часы, указанные в Акте об оказании услуг и заказ-наряде не соответствуют нормо-часам,  установленным заводом-изготовителем.\\
Стоимость работ по нормам завода-изготовителя, действовавшим на дату проведения ремонта автотранспортных средств составляет: 945.25 * 32,79 = 29104,25 (Двадцать девять тысяч сто четыре) рубля.

Работы с кодом 10004 (13.6 ч)  включают трудоемкости перекрёстных  работ с кодами:  10049 (0,5ч), 10043 (0,3ч), 10056 (0,32ч),  10081(0,9ч), 12003 (0,5ч), 10202 (0,45ч), 13012 (0,6ч)\\
Экономия по данному заказ-наряду  составляет: -1739,22 (Одна тысяча семьсот тридцать девять) рублей, 22 коп.
\vspace{3mm}

%2652

\item \par\textbf{{Работы по заказ-наряду   № 1818188100962002312194450/\-АСЮ0002652 от 26.11.2018, Лада Приора  VIN   XTA21700B0327369
}}

Исследуемый автомобиль:  Лада Приора гос. номер: У7150/23 VIN: ХТА21723080032119 год вып. 2008


\фото{foto/2652}{}
\фото{foto/26521}{}

Нормо-часы, указанные в Акте об оказании услуг и заказ-наряде не соответствуют нормо-часам,  установленным заводом-изготовителем.\\
Стоимость работ по нормам завода-изготовителя, действовавшим на дату проведения ремонта автотранспортных средств составляет: 945.25 * 28.16  = 26618.24 (Двадцать шесть тысяч шестьсот восемнадцать) рубль, 24 коп.

Перерасход по заказ-наряду составляет 1361.22 (Одна тысяча триста шестьдесят один) рублей, 22 коп.  
\vspace{3mm}



%2653

\item \par\textbf{{Работы по заказ-наряду  № 1818188100962002312194450/\-АСЮ0002653 от 27.11.2018, Лада Приора  VIN  XTA21723080032119
}}

\фото{foto/2653}{}
\фото{foto/26531}{}

Нормо-часы, указанные в Акте об оказании услуг и заказ-наряде не соответствуют нормо-часам,  установленным заводом-изготовителем.\\
Стоимость работ по нормам завода-изготовителя, действовавшим на дату проведения ремонта автотранспортных средств составляет: 945.25 * 26.97  = 25493,39 (Двадцать пять тысяч четыреста девяносто три) рубль, 39 коп.

Экономия по заказ-наряду составляет -302.40 (Триста два) рублей, 40 коп.  
\vspace{3mm}



%2654


\item \par\textbf{{Работы по заказ-наряду  № 1818188100962002312194450/\-АСЮ0002654 от 27.11.2018, Лада Приора XTA21723080032119
}}

\фото{foto/2654}{}

Нормо-часы, указанные в Акте об оказании услуг и заказ-наряде не соответствуют нормо-часам,  установленным заводом-изготовителем.\\
Стоимость работ по нормам завода-изготовителя, действовавшим на дату проведения ремонта автотранспортных средств составляет: 945.25 * 18,21  = 17213,00 (Семнадцать тысяч двести тринадцать ) рубль, 00 коп.

Экономия составляет по заказ-наряду составляет -1049,21 (Одна тысяча сорок девять) рублей, 21 коп.  
\vspace{3mm}

%2683

\item \par\textbf{{Работы по заказ-наряду   № 1818188100962002312194450/\-АСЮ0002683 от 22.11.2018, Лада Приора  VIN XTA217030E0468978
}}


Исследуемый автомобиль:  


\фото{foto/2683}{}


Нормо-часы, указанные в Акте об оказании услуг и заказ-наряде не соответствуют нормо-часам,  установленным заводом-изготовителем.\\
Стоимость работ по нормам завода-изготовителя, действовавшим на дату проведения ремонта автотранспортных средств составляет: 945.25 * 2.41  = 2315.86 (2315) рублей, 86 коп.

Перерасход  по заказ-наряду составляет 37,82 (Тридцать семь) рублей, 82 коп.  
\vspace{3mm}

%2686
%%%%%%%%%%%%%%%%%%%%%%%%%%%%%%%%%%%%%%%%%%%%%%



%%%%%%%%%%%%%%%%%%%%%%%%%%%%%%%%%%%%%%%%%%%%%%



%2687

\item \par\textbf{{Работы по заказ-наряду   № 1818188100962002312194450/\-АСЮ0002687 от 03.12.2018, Лада Приора  VIN  XTA217030A0263506
}}

\фото{foto/2687}{}

Нормо-часы, указанные в Акте об оказании услуг и заказ-наряде не соответствуют нормо-часам,  установленным заводом-изготовителем.\\
Стоимость работ по нормам завода-изготовителя, действовавшим на дату проведения ремонта автотранспортных средств составляет: 945.25 * 16.21  = 15332.50 (Пятнадцать тысяч триста тридцать два) рубля, 50 коп.

Экономия  по заказ-наряду составляет -151.19 (Сто пятьдесят один) рубль, 19 коп.  
\vspace{3mm}



%2688


\item \par\textbf{{Работы по заказ-наряду    № 1818188100962002312194450/\-АСЮ0002688 от 04.12.2018, Лада Приора  VIN  XTA217050F0511568
}}

\фото{foto/2688}{}
\фото{foto/26881}{}

Нормо-часы, указанные в Акте об оказании услуг и заказ-наряде не соответствуют нормо-часам,  установленным заводом-изготовителем.\\
Стоимость работ по нормам завода-изготовителя, действовавшим на дату проведения ремонта автотранспортных средств составляет: 945.25 * 17.05  = 16116.51 (Шестнадцать тысяч сто шестнадцать) рубль, 51 коп.

Перерасход  по заказ-наряду составляет 2457.72 (Две тысячи четыреста пятьдесят семь) рубля, 72 коп.  
\vspace{3mm}




%2693


\item \par\textbf{{Работы по заказ-наряду    № 1818188100962002312194450/\-АСЮ0002693 от 05.12.2018, Лада Приора  VIN  XTA217050F0509543
}}

\фото{foto/2693}{}


Нормо-часы, указанные в Акте об оказании услуг и заказ-наряде не соответствуют нормо-часам,  установленным заводом-изготовителем.\\
Стоимость работ по нормам завода-изготовителя, действовавшим на дату проведения ремонта автотранспортных средств составляет: 945.25 * 17.62  = 16655.31 (Шестнадцать тысяч шестьсот пятьдесят пять) рубль, 31 коп.

Перерасход  по заказ-наряду составляет 3289.50 (Три тысячи двести восемьдесят девять) рубля, 50 коп.  
\vspace{3mm}

%2703


\item \par\textbf{{Работы по заказ-наряду     № 1818188100962002312194450/\-АСЮ0002703 от 04.12.2018, Лада Гранта   VIN  XTA219010H0475090
}}

\фото{foto/2703}{}


Нормо-часы, указанные в Акте об оказании услуг и заказ-наряде не соответствуют нормо-часам,  установленным заводом-изготовителем.\\
Стоимость работ по нормам завода-изготовителя, действовавшим на дату проведения ремонта автотранспортных средств составляет: 945.25 * 3.36  = 3176.04 (Три тысячи сто семьдесят шесть) рубль, 04 коп.

Экономия  по заказ-наряду составляет -9.44 (Девять) рублей, 44 коп.  
\vspace{3mm}



%2721



\item \par\textbf{{Работы по заказ-наряду     № 1818188100962002312194450/\-АСЮ0002721 от 06.12.2018, Лада Приора  VIN   XTA217030B0327885
}}

\фото{foto/2721}{}


Нормо-часы, указанные в Акте об оказании услуг и заказ-наряде не соответствуют нормо-часам,  установленным заводом-изготовителем.\\
Стоимость работ по нормам завода-изготовителя, действовавшим на дату проведения ремонта автотранспортных средств составляет: 945.25 * 2.41  = 2278.05 (Две тысячи двести семьдесят восемь) рубль, 05 коп.

Перерасход  по заказ-наряду составляет 37.82 (Тридцать семь) рубля, 82 коп.  
\vspace{3mm}


%2722


\item \par\textbf{{Работы по заказ-наряду     № 1818188100962002312194450/\-АСЮ0002722 от 06.12.2018, Лада Приора  VIN  XTA217030B0289539
}}

Исследуемый автомобиль:  Лада Приора гос. номер: Е335АВ 123 VIN: ХТА217030В0289539 год вып. 2011


\фото{foto/2722}{}


Нормо-часы, указанные в Акте об оказании услуг и заказ-наряде не соответствуют нормо-часам,  установленным заводом-изготовителем.\\
Стоимость работ по нормам завода-изготовителя, действовавшим на дату проведения ремонта автотранспортных средств составляет: 945.25 * 13.74  = 12987.74 (Двенадцать тысяч девятьсот восемьдесят семь) рублей, 74 коп.

Экономия  по заказ-наряду составляет -1833.76 (Одна тысяча восемьсот тридцать три) рубля, 76 коп.  
\vspace{3mm}


%2723


\item \par\textbf{{Работы по заказ-наряду    № 1818188100962002312194450/\-АСЮ0002723 от 14.11.2018, Лада Приора    VIN  XTA217030E0469310
}}


Исследуемый автомобиль:  Лада Приора гос. номер: Х370ОС 123 VIN; ХТА217030Е0469310 год вып. 2014



\фото{foto/2723}{}
\фото{foto/27231}{}

Нормо-часы, указанные в Акте об оказании услуг и заказ-наряде не соответствуют нормо-часам,  установленным заводом-изготовителем.\\
Стоимость работ по нормам завода-изготовителя, действовавшим на дату проведения ремонта автотранспортных средств составляет: 945.25 * 23.16  = 22969.58 (Двадцать две тысячи девятьсот шестьдесят девять) рублей, 58 коп.

Перерасход  по заказ-наряду составляет 3355.69 (Три тысячи триста пятьдесят пять) рубля, 69 коп.  
\vspace{3mm}





\item \par\textbf{{Работы по заказ-наряду    № 1818188100962002312194450/\-АСЮ0002724 от 03.12.2018, Лада Приора    VIN  XTA217030A0263506
}}

\фото{foto/2724}{}

Нормо-часы, указанные в Акте об оказании услуг и заказ-наряде не соответствуют нормо-часам,  установленным заводом-изготовителем.\\
Стоимость работ по нормам завода-изготовителя, действовавшим на дату проведения ремонта автотранспортных средств составляет: 945.25 * 9.37  = 8856.99 (Восемь тысяч восемьсот пятьдесят шесть) рублей, 99 коп.

Перерасход  по заказ-наряду составляет 595.52 (Пятьсот девяносто два) рубля, 52 коп.  
\vspace{3mm}



\item \par\textbf{{Работы по заказ-наряду    № 1818188100962002312194450/\-АСЮ0002726 от 05.12.2018, Лада Приора VIN  XTA217050F0509543
}}

\фото{foto/2726}{}

Нормо-часы, указанные в Акте об оказании услуг и заказ-наряде не соответствуют нормо-часам,  установленным заводом-изготовителем.\\
Стоимость работ по нормам завода-изготовителя, действовавшим на дату проведения ремонта автотранспортных средств составляет: 945.25 * 27.25  = 25758.06 (Двадцать пять тысяч семьсот пятьдесят восемь) рублей, 06 коп.

Экономия  по заказ-наряду составляет  (Три тысячи семьдесят два) рубля, 04 коп.  
\vspace{3mm}



\item \par\textbf{{Работы по заказ-наряду     № 1818188100962002312194450/\-АСЮ0002727 от 04.12.2018, Лада Приора  VIN   XTA217050F0511568
}}

\фото{foto/2727}{}

Нормо-часы, указанные в Акте об оказании услуг и заказ-наряде не соответствуют нормо-часам,  установленным заводом-изготовителем.\\
Стоимость работ по нормам завода-изготовителя, действовавшим на дату проведения ремонта автотранспортных средств составляет: 945.25 * 13.19  = 12467.85 (Двенадцать тысяч четыреста шестьдесят семь) рублей, 85 коп.

Перерасход  по заказ-наряду составляет 860.20  (Восемьсот шестьдесят) рублей, 20 коп.  
\vspace{3mm}



\end{enumerate}

\subsection{Анализ результатов исследования}

Представленные 

\begin{landscape}
 (Таблица  \ref{tab:4})\\
%\small
\fontsize{8.5pt}{9.25pt}\selectfont
\addtolength{\tabcolsep}{-2pt}
\begin{longtable}{p{4mm}|G{34mm}|G{72mm}|G{30mm}|G{30mm}|G{30mm}}
	%
	%
	% 
	\caption[]{Акты об оказании услуг и заказ-наряды на выполнение работ по ремонту транспортного средства}\label{tab:4}\\ 
	\hline\hline 
	
	\text{n/n} & \textbf{Подлежащее ремонту транспортное средство} & \textbf{Заказ-наряд} & \textbf{{Стоимость работ по заказ-наряду фактическая, руб}} & \textbf{{Стоимость работ нормативная, руб}}&\textbf{{Отклонение, руб}}\\ \hline\endhead
	

	%1

	\пять{Лада Приора, VIN XTA217030B0327141}{Заказ-наряд № 1818188100962002312194450/\-АСЮ0002455 от 19.11.2018}{19708.47}{18025.92}{1682.55}
	
	
	
	
\end{longtable} \setcounter{rownum}{0}

\end{landscape}




\section{Выводы}


\begin{enumerate}
	\item 
	\textbf{"Нормо-часы, указанные в  нормо-часам, с учётом года выпуска автотранспортных средств, не соответствуют нормо-часам, установленным   изготовителем транспортного средства.}
	\item 
	\textbf{Стоимость работ, указанных в прилагаемых Актах об оказании услуг и заказ-нарядах к государственному контракту  № 1818188100962002312194450/965/18 от 08.10.2018., по нормам завода-изготовителя, действовавшим на дату проведения ремонта автотранспортных средств, с учётом года выпуска автотранспортных средств, составляет ....}
\end{enumerate}






\vspace{15mm}
\relax
Приложение к заключению:\\
\textit{
	1.
	   }

\vspace{20mm}

{Эксперт}\hfill           {Мраморнов А.В.}

%\includepdf[pages=-]{myfile.pdf}
%\includepdf[pages=-]{calc.pdf}