\setcounter{page}{1}
\clubpenalty=10000 
\widowpenalty=10000

%%%%%%%%%%%%%%%%%%%%%%%%%%%%%%%%%%%%%%%%
%      Шапка экспертной организации  
%%%%%%%%%%%%%%%%%%%%%%%%%%%%%%%%%%%%%%%%
%
%%%%%%%%%%%%%%%%%%%%%%%%%%%%%%%%%%%%%%%%%%
%
%   Экспертная организация ООО Южнорегиональная экспертная группа
%
%%%%%%%%%%%%%%%%%%%%%%%%%%%%%%%%%%%%%%%%%
\noindent %\qrcode[height=21mm]{\NomerDoc от \окончено }  %%% Добавлен QR-Code
\begin{pspicture}(21mm,21mm)
\obeylines
\psbarcode{%
	%\NomerDoc от \окончено
	BEGIN:VCARD^^J
	VERSION:4.0^^J
	%N:Мраморнов; Александр; Вчеславович^^J
	FN:Александр Мраморнов^^J
%	ORG:IP Alexandr Mramornov^^J
	TITLE: эксперт
	ORG: ИП
	URL:http://www.yourexp.ru^^J
	EMAIL:4516611@gmail.com^^J
	TEL:+7-918-451-6611^^J
	ADR:г. Краснодар, с/т № 2 А/О «Югтекс», ул. Зеленая, 472^^J
	END:VCARD
}{width=1.0 height=1.0}{qrcode}%
\end{pspicture}
\begin{center}
	\normalsize\textbf{ОБЩЕСТВО С ОГРАНИЧЕННОЙ ОТВЕТСТВЕННОСТЬЮ \\[-1.5mm] <<ЮЖНО-РЕГИОНАЛЬНАЯ\quad ЭКСПЕРТНАЯ\quad ГРУППА>> \\[-5mm]}
	%  
	\noindent\rule{\textwidth}{1pt}\\[-6mm]  % Горизонтальная линия
	% \line(1,0){460}% (1,0) -горизонтальная линия, и (0,1) - вертикальная 
\end{center}

\begin{center}
	\begin{footnotesize}\setstretch{0.3}
		%	\small\textbf\setlength   	%\raisebox{5mm}
		\vspace{-3.5mm}350072, Россия, Краснодарский край, г. Краснодар, Ростовское шоссе, 14/2, оф. 67\\[0mm]
		Телефон: \quad 8-918-451-66-11, e-mail:\quad 4516611@gmail.com\\ [-2mm]{ИНН 2311213020\quad КПП 231101001 ОГРН 1162375014560}
	\end{footnotesize}	\\[10mm]
\end{center}


\begin{flushright}
	Краснодар, 2020    \\[8mm]
\end{flushright}
\begin{center}
	\LARGE\textbf{ ЗАКЛЮЧЕНИЕ ЭКСПЕРТА}
	\bigskip\\[0mm]
	%	{\normnumxtbf{\NomerDoc}}	}{den}
\end{center}
\par
\vspace{-3mm}\noindent по гражданскому делу \delonum \, \isk \\[0mm]

%\raggedright 
%\def\hrf#1{\hbox to#1{\hrulefill}}
\noindent \textbf{№ 22-2019}\hfill           \textbf{\dataend}\\%[2mm]
%Приостановлено\hfill      \datastop\\
%Возобновлено\hfill          \datarestart\\
%Окончено\hfill                \dataend\\%[4mm]

\noindent\parbox[l][16mm]{16.5cm}
{\def\hrf#1{\hbox to#1{\hrulefill}}
	\noindent Начато\hfill            \datastart\\%[2mm]
	%	Приостановлено\hfill      \datastop\\
	%	Возобновлено\hfill          \datarestart\\
	Окончено\hfill                \dataend\\%[4mm]
}
\relax

\datastart г. ~в {\small ООО~ "ЮЖНО-РЕГИОНАЛЬНАЯ ЭКСПЕРТНАЯ ГРУППА"} \,  при определении  \, \sud  \,  от \, \dataopr \, о назначении \opr \, по гражданскому делу \delonum \, поступили:

\begin{enumerate}\setlist{nolistsep}\item  Материалы гражданского дела \delonum \, в двух томах, том 1 на 276 листах, том 2  на 143 листах.\\[-2mm]
	%	\item  
\end{enumerate}Экспертиза произведена экспертом {\small ООО "ЮЖНО-РЕГИОНАЛЬНАЯ ЭКСПЕРТНАЯ ГРУППА"} \,  Мраморновым Александром Вячеславовичем, имеющим высшее техническое образование по специальности «техническая физика», диплом РВ №311964 от 28.02.1989, квалификация -- инженер-физик, специальное образование в области оценки: Диплом ПП-1 № 037211 Российской экономической академии им. Г.В. Плеханова, квалификация -- оценка и экспертиза объектов и прав собственности, специальное образование в области независимой технической экспертизы транспортных средств: Диплом ПП-I № 424167, квалификация: эксперт-техник (специализация 150210 специальности 190601.65 – Автомобили и автомобильное хозяйство), состоящий в Государственном реестре экспертов-техников (№ в реестре 256, https://data.gov.ru/opendata/7707211418-experts,  общий трудовой  стаж 30 лет, стаж  экспертной работы  12 лет.  % Шапка организации ООО ЮРЕКСГРУП
%%%%%%%%%%%%%%%%%%%%%%%%%%%%%%%%%%%%%%%%%
%
%   Экспертная организация ИП
%
%%%%%%%%%%%%%%%%%%%%%%%%%%%%%%%%%%%%%%%%%
%\noindent\qrcode[height=20mm]{\NomerDoc от \dataend } 
\noindent %\qrcode[height=21mm]{\NomerDoc от \окончено }  %%% Добавлен QR-Code
\begin{pspicture}(21mm,21mm)
\obeylines
\psbarcode{%
	%\NomerDoc от \окончено
	BEGIN:VCARD^^J
	VERSION:4.0^^J
	%N:Мраморнов; Александр; Вчеславович^^J
	FN:Александр Мраморнов^^J
%	ORG:IP Alexandr Mramornov^^J
	TITLE: эксперт
	ORG: ИП
	URL:http://www.yourexp.ru^^J
	EMAIL:4516611@gmail.com^^J
	TEL:+7-918-451-6611^^J
	ADR:г. Краснодар, с/т № 2 А/О «Югтекс», ул. Зеленая, 472^^J
	END:VCARD
}{width=1.0 height=1.0}{qrcode}%
\end{pspicture}

 %%% Добавлен QR-Code
\vspace{-4mm}
\begin{center}
	\large\textbf{ИНДИВИДУАЛЬНЫЙ\quad ПРЕДПРИНИМАТЕЛЬ  \\[-1.5mm] МРАМОРНОВ  АЛЕКСАНДР ВЯЧЕСЛАВОВИЧ \\[-5.5mm]}
	%  
	\noindent\rule{\textwidth}{2pt}\\[-6mm]  % Горизонтальная линия
	% \line(1,0){460}% (1,0) -горизонтальная линия, и (0,1) - вертикальная 
\end{center}

\begin{center}
	\begin{footnotesize}\setstretch{0.3}
		%	\small\textbf\setlength   	%\raisebox{5mm}
		\vspace{-2.5mm}г. Краснодар, с/т № 2 А/О «Югтекс», ул. Зеленая, 472, 
		Телефон: 8-918-451-66-11, e-mail: 4516611@gmail.com\\ [-2mm]{ИНН\quad 231200665168\quad ОГРНИП \quad 310231220400043}
	\end{footnotesize}	\\[10mm]
\end{center}


\begin{flushright}
% 
	 \hfill	Краснодар, 2020    \\[8mm]
\end{flushright}  
\begin{center}
	\LARGE\textbf{ЗАКЛЮЧЕНИЕ ЭКСПЕРТА}
\end{center}
\par
\vspace{4mm}


\par
\vspace{-3mm}\noindent по гражданскому делу \delonum \, \isk \\[0mm]

%\raggedright 
%\def\hrf#1{\hbox to#1{\hrulefill}}
\noindent \textbf{№ 22-2019}\hfill           \textbf{\dataend}\\%[2mm]
%Приостановлено\hfill      \datastop\\
%Возобновлено\hfill          \datarestart\\
%Окончено\hfill                \dataend\\%[4mm]

\noindent\parbox[l][16mm]{16.5cm}
{\def\hrf#1{\hbox to#1{\hrulefill}}
	\noindent Начато\hfill            \datastart\\%[2mm]
	%	Приостановлено\hfill      \datastop\\
	%	Возобновлено\hfill          \datarestart\\
	Окончено\hfill                \dataend\\%[4mm]
}
\relax

\datastart г. ~в {\small ООО~ "ЮЖНО-РЕГИОНАЛЬНАЯ ЭКСПЕРТНАЯ ГРУППА"} \,  при определении  \, \sud  \,  от \, \dataopr \, о назначении \opr \, по гражданскому делу \delonum \, поступили:

\begin{enumerate}\setlist{nolistsep}\item  Материалы гражданского дела \delonum \, в двух томах, том 1 на 276 листах, том 2  на 143 листах.\\[-2mm]
	%	\item  
	\end{enumerate}

\paragraph*{}
Экспертиза произведена  экспертом
\,  Мраморновым Александром Вячеславовичем, имеющим высшее  образование по специальности «техническая физика», диплом РВ №311964 от 28.02.1989, квалификация -- инженер-физик, специальное образование в области оценки: Диплом ПП-1 № 037211 Российской экономической академии им. Г.В. Плеханова, квалификация -- оценка и экспертиза объектов и прав собственности, специальное образование в области независимой технической экспертизы транспортных средств: Диплом ПП-I № 424167, квалификация: эксперт-техник (специализация 150210 специальности 190601.65 – Автомобили и автомобильное хозяйство), состоящий в Государственном реестре экспертов-техников (№ в реестре 256, https://data.gov.ru/opendata/7707211418-experts,  общий трудовой  стаж 30 лет, стаж  экспертной работы  12 лет. 
\par Заключение подготовлено по месту фактического расположения ИП по адресу: г. Краснодар, с/т № 2 А/О «Югтекс», ул. Зеленая, 472.
\vspace{4mm}
%
%%   вопросы экспертизы
\subsection{Вопросы экспертизы}

\begin{enumerate}
	\item Соответствуют ли нормо-часы, указанные в Актах об оказании услуг и заказ-нарядах к государственному контракту № 1818188100962002312194450/965/18 от 08.10.2018, нормо-часам, с учетом года выпуска автотранспортных средств, установленных заводом-изготовителем? Если - нет, то рассчитать стоимость работ, указанных в прилагаемых Актах об оказании услуг и заказ-нарядах к государственному контракту  № 1818188100962002312194450/965/18 от 08.10.2018., по нормам завода-изготовителя, действовавшим на дату проведения ремонта автотранспортных средств.
\end{enumerate}


\subsection{Для производства исследования представлено} %Название по шаблону минюста
\begin{enumerate}
	\item копия материалов гражданского дела в том числе: в количестве 55 шт. 
	\end{enumerate}
%
%
%\vspace{-275mm}
\addcontentsline{toc}{section}{Использованные нормативы и источники информации}
%
%\left( \addcontentsline{toc}{section}{Использованные нормативы и источники информации}

\subsection{Использованные нормативы и источники информации}
%
\begin{enumerate}
\item 
Махнин\,Е.\,Л., Новоселецкий\, И.\,Н., Федотов\, С.\,В. \emph{Методические рекомендации по проведению судебных автотехнических экспертиз и исследований колёсных транспортных средств в целях определения размера ущерба, стоимости восстановительного ремонта и оценки} // -- М.: ФБУ РФЦСЭ при Минюсте России, 2018.-326 с.  ISBN 978-5-91133-185-6.
%
%
%
%
\item ТУ 017207-255-00232934-2014 \emph{Кузова автомобилей LADA. Технические требования при приёмке в ремонт, ремонте и выпуске из ремонта предприятиями дилерской сети ОАО "АВТОВАЗ"}//  ОАО НВП "ИТЦ АВТО", 2014
%
\item Смирнов  В.Л., Прохоров  Ю.С., Боюр В.С.  и др. \emph{Автомобили ВАЗ. Кузова. Технология ремонта, окраски и  антикоррозионной защиты. Часть II}// - Н.Новгород: АТИС, 2001.- 241с.
%
\item 
Савич Е.Л. \emph{Техническое  обслуживание  и  ремонт  легковых  автомобилей} : учеб. пособие / Е.Л. Савич, М.М. Болбас, В.К. Ярошевич ; под общ. ред. Е.Л. Савича. -Мн. : Вышэйшая школа,  2001. - 479 с. - ISBN985-06-0502-2.
%
\item 
Автомобили ВАЗ-2121, 21213, 21214, 2131 и их модификации: <<Трудоемкости работ (услуг) по техническому обслуживанию и ремонту>> /Куликов А.В., Христов П.Н., Климов В.Е.,  Боюр В.С., Рева В.В., Зимин В.А., Завьялова Н.Н., Хлыненкова Г.А. -- ИТЦТ "АвтоВАЗтехобслуживание", Тольяти -- 2005. 
%
\item
Автомобили LADA SAMARA и их модификации: <<Трудоемкости работ (услуг) по техническому обслуживанию и ремонту>> /Куликов А.В., Христов П.Н., Климов В.Е., Рева В.В., Боюр В.С., Васильев М.В., Фахрутдинов Р.В.,  Прудских Д.А., Гирко В.Б., Шмелева В.А., Зимин В.А. --  ОАО НВП "ИТЦ АВТО",  -- 2006. - 252 стр.
%
\item 
Автомобили LADA PRIORA. Трудоемкости работ (услуг) по техническому обслуживанию и ремонту /Куликов А.В., Христов П.Н., Климов В.Е., Рева В.В., Козлов П.Л., Боюр В.С., Прудских Д.А., Шмелева В.А., Зимин В.А. -- ООО "ИТЦТ АВОСФЕРА", Тольяти -- 2009. -- 344 с.
%
\item 
{Трудоемкости работ по техническому обслуживанию и ремонту автомобилей автомобилей Lada  Granta}/   \url{https://docplayer.ru/30250248-Trudoemkosti-rabot-po-teh\-nicheskomu-obsluzhivaniyu-i-remontu-avtomobiley-lada- granta.html}.
%
%
\item
{Специализированное программное обеспечение для расчёта стоимости  восстановительного ремонта, содержащее нормативы трудоёмкости работ, регламентируемые изготовителями транспортного средства}//   AudaPadWeb, лицензионное соглашение № AS/APW-658  RU-P-409-409435.
%
%
%
\item
{Специализированное программное обеспечение для расчёта стоимости  восстановительного ремонта, содержащее нормативы трудоёмкости работ, регламентируемые изготовителями транспортного средства ОАО «АвтоВАЗ», ЗАО «Джи-Эм-АвтоВАЗ», ОАО «СеАЗ» и ОАО «ЗМА»}//   Автосфера АС:Смета, v.3.9.11// ООО "ИТЦ «ИнтегроМаш», \url{https://autosmeta.pro}.
%
%
%
\item Информационный портал по техническому обслуживанию и ремонту автомобилей	 ВАЗ:\\ \url{www.autosphere.ru}.

%%
\end{enumerate}


%%%%%%%%%%%%%%%%%%%%%%%%%%%%%%%%%%%%%%%%%%%%%%%%%%%%%%%%%%%%%%%%%%%%%%%%%%%%%%%%%
\subsection{Технические средства}  %% Список не удалять!!!
\begin{itemize}

%
%\item  Специализированное программное обеспечение для расчёта стоимости  восстановительного ремонта, содержащее нормативы трудоёмкости работ, регламентируемые изготовителями транспортного средства     AudaPadWeb, лицензионное соглашение № AS/\- APW-658  RU-P-409-409435

\item  ПЭВМ под управлением операционной системы Windows 10 с установленным набором макрорасширений LaTeX системы компьютерной вёрстки TeX, cвободная лицензия LaTeX Project Public License (LPPL)
%	
\end{itemize}
%%%%%%%%%%%%%%%%%%%%%%%%%%%%%%%%%%%%%%%%%%%%%%%%%%%%%%%%%%%%%%%%%%%%%%%%%%%%%%%%%%%%%%%%%%%%%%%%%%%%%%
\subsection{Условные обозначения и принятые сокращения}
\begin{description}
%	 
%%\item[АВС] --Антиблокировочная система
\item[АМТС] -- автомототранспортное средство
\item[ДВС] -- двигатель внутреннего сгорания
\item[ГБЦ] -- головка блока цилиндров
\item[ДТП] -- дорожно--транспортное происшествие
\item[гос.\,рег.\,знак] -- государственный регистрационный знак
\item[КТС] -- колесо-транспортное средство 
\item[ЛКП] -- лакокрасочное покрытие
\item[МКПП] -- механическая коробка перемены передач
%\item[л.д.] --Лист дела
%%\item[Колесо турбины]  -- крыльчатка турбины
\item[ТС] -- транспортное средство
%\item[ТK, ТКР] -- Турбокомпрессор. Состоит из двух частей: турбины и компрессора, объединенных общим валом. Вал вращается в подшипниках, размещенных в центральном корпусе ТК
\item[ЭБУ] -- электронный блок управления
%%\item[FRAME] "--*Номер кузова транспортного средства, выпущенного для продажи на внутреннем рынке Японии и содержащий информацию производителя о транспортном средстве
\item[DTC] --Diagnostic Trouble Codes, диагностические коды неисправностей
\item[VIN] -- vehicle identification number, 17--значный идентификационный номер транспортного средства, соответствующий стандарту ISO 3779--2012
\item[н/н мехобработка] -- не нормированная изготовителем транспортного средства трудоёмкость механической обработки детали
\item[н/у] -- не определена изготовителем  
%
\end{description}
\subsection{Методы исследования}
\begin{itemize}
\item Экспертный метод (метод экспертной оценки) -- совокупности операций по выбору комплекса или единичных характеристик объекта, определению их действительных значений и оценкой экспертом соответствия их установленным требованиям и/или технической информации;
\item Метод сравнения -- сопоставление фактических данных и данных источников.
\end{itemize}

\subsection{Исходные данные}

Стоимость  нормо-часа работ и стоимость запасных частей определена условиями государственного контракта  № 1818188100962002312194450/965/18 от 08.10.2018 945.25 р/час.\\
Акты об оказании услуг и заказ-наряды к государственному контракту № 1818188100962002312194450/965/18 от 08.10.2018г. в количестве 55 шт. (Таблица  \ref{tab:4}):\\

\begin{longtable}{p{4mm}|G{58mm}|G{85mm}} 
	\caption[]{Акты об оказании услуг и заказ-наряды на выполнение работ по ремонту транспортного средства}\label{tab:4}\\ 
	\hline\hline 
	
		\text{n/n} & \textbf{Подлежащее ремонту транспортное средство} &  \textbf{Заказ-наряд}\\ \hline\endhead
	
	\два{Лада Приора, VIN XTA217030B0327369}{Заказ-наряд № 1818188100962002312194450/АСЮ0002651 от 28.11.2018}
	
	\два{Лада Приора  VIN XTA217030B0327995}{Заказ-наряд № 1818188100962002312194450/АСЮ0002558 от 07.11.2018}
	
	
\end{longtable} \setcounter{rownum}{0}
	
%	

         
\subsection{Обстоятельства дела}
\begin{itemize}
\item 03.09.2019 автомобиль \тс на эвакуаторе был доставлен в сервисный центр. При работе двигателя при увеличении оборотов до 2000 об/мин слышен стук в ДВС. 
%
\end{itemize}
% 
%
%
\section{Исследование}

Настоящее исследование проводится по предоставленным доказательствам   согласно рекомендаций изготовителя транспортного средства с учётом следующих положений:
%\цитата
%{ 4. Каждой позиции трудоемкости соответствует:
%	• код детали/узла (семь цифр) и код работы (две цифры), см.таблицу 1;
%	• наименование детали, краткая характеристика выполняемой работы и при необходимости полная характеристика;
%	• норма времени на выполнение работ и максимально допустимое количество повторов выполнения этой работы на одном автомобиле (указывается через звездочку *). Если звездочка отсутствует, то количество равно "1".
%	Прочерк вместо нормы времени указывает на то, что данная работа не выполняется на соответствующей модели автомобиля.
%	5. Трудоемкость работ (услуг) по ТО и ремонту автомобилей, не предусмотренных настоящим перечнем, определяется по согласованию с заказчиком.
%	6. В трудоемкостях учтены операции по ТО и ремонту автомобилей в условиях предприятия. Оплата за работы вне предприятия или доставка автомобилей на предприятие для ТО и ремонта взимается с заказчика отдельно.
%	Разработка норм времени производилась с использованием технологического оборудования, оснастки и механизированного инструмента в условиях аналогичных условиям предприятия сервисно-сбытовой сети с привлечением механиков имеющих квалификацию на проведение ремонта и технического обслуживания не специализирующихся на какой-либо конкретной системе автомобиля.
%	Предприятие может применять надбавки к настоящим трудоемкостям при ремонте автомобилей старше 5 лет до 10\%, старше 8 лет до 20\% (кроме работ разделов: техническое обслуживание, окраска, антикоррозионная обработка, поиск не явно выраженных неисправностей).
%	Предприятие имеет право так же применять надбавки к настоящим трудоемкостям работ до 10\% от трудоемкости выполнения операции по снятию/установке навесных панелей кузова (двери, капот, крышка багажника) в случае аварийного повреждения кузова автомобиля и невозможности снятия узла с автомобиля методами предусмотренными технологической документацией.
%	Предприятие имеет право применять надбавки к настоящим трудоемкостям работ до 20\% на снятие/установку детали (узла) с аварийно поврежденной навесной панели (капот, двери, крышка багажника), требующей ее замены или ремонта №3, если снятие/установка детали (узла) невозможна методами предусмотренными технологической документацией.
%	Предприятие имеет право обратиться в ОАО "АВТОВАЗ" по вопросу введения дополнительной операции или пересмотра действующего норматива с обязательным заполнением бланка - "Запрос на изменение (введения) норматива трудоёмкости работ". Форма бланка приведена в приложении 1 настоящего сборника.
%	7. Стоимость выполнения работ рассчитывается на основании приведённых норм времени и действующей на предприятии на дату ремонта стоимости нормо-часа и представляет собой только стоимость услуги, а стоимость заменяемых агрегатов, узлов и деталей, а также основных материалов, используемых при ТО и ремонте автомобилей, оплачивается заказчиком отдельно.
%	  ТРУДОЕМКОСТИ РАБОТ ПО ТЕХНИЧЕСКОМУ ОБСЛУЖИВАНИЮ И РЕМОНТУ АВТОМОБИЛЕЙ LADA содержат следующие ОБЩИЕ ПОЛОЖЕНИЯ, идентичные для исследуемых моделей транспортных средств.   В данных трудоемкостях указано время (в часах и десятых долях часа) для выполнения операций мойки, сма­зочно-заправочных, ремонта, замены, диагностики, регулировки, окраски, антикоррозионной обработки.2  Трудоемкости  распространяются  на  работы  (услуги)  по  ТО  и  ремонту  автомобилей  LADA,  выполняемые предприятиями технического обслуживания автомобилей (ПТОА).Трудоемкости работ на  предпродажную  подготовку  автомобилей  LADA  приведены в  технологической инст­рукции "Автомобили LADA  -  предпродажная подготовка".3  Каждой позиции трудоемкости присвоен неповторяющийся пятизначный номер.  Первые две цифры номера позиции указывают на раздел, к которому она относится:00 - техническое обслуживание10 - 84 - ремонт85 - окраска86 - антикоррозионная обработка87 - поиск не явно выраженных неисправностейТри последние  цифры в номерах позиций представляют собой собственно порядковый номер позиции внутри разделов и подразделов.В разделе "Техническое обслуживание"  позиции сгруппированы по подразделам в зависимости от вида выпол­няемой работы (по кодам работ).В разделе "Ремонт"  первые две цифры дополнительно указывают на систему автомобиля, к которой относится объект работы, и позиции сгруппированы по подразделам в зависимости от принадлежности к соответствующей системе/подсистеме автомобиля.Внутри разделов и подразделов позиции  трудоемкостей расположены в порядке возрастания номеров деталей (узлов) по каталогу LADA.4 Каждой позиции трудоемкости соответствует:- код детали/узла (семь цифр) и код работы (две цифры), см.таблицу 1;- наименование детали, краткая характеристика выполняемой работы и при необходимости полная характери­стика;- норма времени на выполнение работ и максимально допустимое количество повторов выполнения этой рабо­ты на одном автомобиле (указывается через звездочку *). Если звездочка отсутствует, то количество равно " 1".Прочерк вместо нормы времени указывает на то, что данная работа не выполняется на соответствующей моде­ли автомобиля.5 Трудоемкость работ (услуг) по ТО и ремонту автомобилей, не предусмотренных настоящим перечнем, опре­деляется по согласованию с заказчиком.6 В трудоемкостях учтены операции по ТО и ремонту  автомобилей в условиях предприятия.  Оплата за работы вне предприятия или доставка автомобилей на предприятие для ТО и ремонта взимается с заказчика отдельно.Разработка норм времени производилась с  использованием технологического оборудования, оснастки и меха­низированного инструмента в условиях аналогичных условиям предприятия сервисно-сбытовой сети с привлече­нием  механиков, имеющих квалификацию  на проведение ремонта и технического  обслуживания не  специализи­рующихся на какой-либо конкретной системе автомобиля.Предприятие может применять надбавки к настоящим трудоемкостям  при ремонте  автомобилей старше  5 лет до  10\%, старше 8 лет до 20\% (кроме работ разделов: техническое обслуживание, окраска, антикоррозионная обра­ботка, поиск не явно выраженных неисправностей).Предприятие имеет право так же применять надбавки к настоящим трудоемкостям работ до  10\% от трудоемко­сти выполнения операции по снятию/установке навесных панелей кузова (двери, капот, крышка багажника) в слу­чае аварийного повреждения кузова автомобиля и невозможности снятия узла с автомобиля методами предусмот­ренными технологической документацией.Предприятие имеет право применять надбавки к настоящим трудоемкостям работ до  20\%  на снятие/установку детали (узла) с аварийно поврежденной навесной панели (капот, двери, крышка багажника), требующей ее замены или ремонта No3, если снятие/установка детали (узла) невозможна методами предусмотренными технологической документацией.
%}


\begin{enumerate}
	\item \par\textbf{{ Выполненные работы по заказ-наряду № 1818188100962002312194450/АСЮ0002651 от 28.11.2018, автомобиль Лада Приора, VIN XTA217030B0327369}} соответствуют комплексу работ капитального ремонта двигателя и головки блока цилиндров  (слить масло,произвести полную разборку,(кроме узлов электрооборудования и системы питания), промыть, обдуть сжатым воздухом, продефектовать, заменить узлы и детали, собрать (с выполнением смазочных работ), залить масло, произвести обкатку на стенде. Код 100260 1004 в системе ОАО "АВТОВАЗ") с включением работ механической обработки ДВС и ГБЦ, выполняющихся на соответствующем специализированном оборудовании и   не нормируемых изготовителем ТС.\\
\фото{foto/2651}{}


Нормо-часы, указанные в Акте об оказании услуг и заказ-наряде не соответствуют нормо-часам, с учетом года выпуска автотранспортных средств, установленных заводом-изготовителем.\\
Рассчитанная стоимость работ, по нормам завода-изготовителя, действовавшим на дату проведения ремонта автотранспортных средств составляет 945.25 * 30.78 =
29104.25 (Двадцать девять тысяч сто четыре) рубля.

Работы код 10004 (13.6 ч)  включают трудоемкости перекрёстных  работ с кодами:  10049 (0,5ч), 10043 (0,3ч), 10056 (0,32ч),  10081(0,9ч), 12003 (0,5ч), 10202 (0,45ч), 13012 (0,6ч)\\
Фактическая разница (удорожание)  по данному заказ-наряду составляет 30.79 - 29.0 = 1.79 часа или  945.25 * 1.79 = 1739.26 рублей.


\item \par\textbf{{ Выполненные работы по заказ-наряду № 1818188100962002312194450/АСЮ0002558 от 07.11.2018, автомобиль Лада Приора  VIN XTA217030B0327995}}




\end{enumerate}

\subsection{Анализ результатов исследования}

Представленные 

\section{Выводы}


\begin{enumerate}
	\item 
	\textbf{"Нормо-часы, указанные в  нормо-часам, с учетом года выпуска автотранспортных средств, не соответствуют нормо-часам, установленным   изготовителем транспортного средства.}
	\item 
	\textbf{Стоимость работ, указанных в прилагаемых Актах об оказании услуг и заказ-нарядах к государственному контракту  № 1818188100962002312194450/965/18 от 08.10.2018., по нормам завода-изготовителя, действовавшим на дату проведения ремонта автотранспортных средств составляет ....}
\end{enumerate}

\vspace{15mm}
\relax
Приложение к заключению:\\
\textit{
	1.
	   }

\vspace{20mm}

{Эксперт}\hfill           {Мраморнов А.В.}

%\includepdf[pages=-]{myfile.pdf}
%\includepdf[pages=-]{calc.pdf}