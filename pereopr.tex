\renewcommand{\chaptername}{Заключение эксперта}
\renewcommand{\refname}{Список}
\renewcommand{\bibname}{\large {Использованные нормативы и источники информации}}

\renewcommand{\epsilon}{\ensuremath{\varepsilon}}
\renewcommand{\phi}{\ensuremath{\varphi}}
\renewcommand{\kappa}{\ensuremath{\varkappa}}
\renewcommand{\le}{\ensuremath{\leqslant}}
\renewcommand{\leq}{\ensuremath{\leqslant}}
\renewcommand{\ge}{\ensuremath{\geqslant}}
\renewcommand{\geq}{\ensuremath{\geqslant}}
\renewcommand{\emptyset}{\varnothing}


%%%%%%%%%%%%%%%%%%%%%%%  Подсчет строк в таблице
\newcounter{rownum}
\setcounter{rownum}{0}
\newcommand{\Rownum}{\stepcounter{rownum}%
\arabic{rownum}}
			

%\def\contentsname{Содержание}
%Аннотация  \abstractname
%Часть       \partname
%Глава        \chaptername
%Список литературы  \refname
%Рис.                \figurename
%Таблица           \tablename
%Литература       \bibname

%Предметный указатель  \indexname
%Приложение                \appendixname
%Содержание          \contentsname
%Список иллюстраций \listfigurename
%Список таблиц        \listtablename
%\addto\captionsrussian{\def\refname{Список используемой литературы}}

%%%%%%%%%%%%%%   Размещение изображений
%\textfloatsep — расстояние между флоатс (в верхней или нижней части страницы) и текстом (по умолчанию, около 20pt)
%\floatsep — вертикальное расстояние между двумя флоатс (около 12pt)
%\intextsep — расстояние между флоатс вставленным "прямо здесь" (параметр h) и текстом (около 12pt)
%\abovecaptionskip и \belowcaptionskip — расстояние над и под подписью к флоат
\setcounter{totalnumber}{10}
\setcounter{topnumber}{10}
\renewcommand{\topfraction}{1}
\renewcommand{\textfraction}{0}
%%%%%%  Больше плавающих объектов на страницу
 \setlength{\textfloatsep}{10pt plus 1.0pt minus 2.0pt}
 \setlength{\floatsep}{5pt plus 1.0pt minus 1.0pt}
 \setlength{\intextsep}{5pt plus 1.0pt minus 1.0pt}

%%%%%%%%%%%%%%%%%%%%%%%%%%%%%%%%%%%%%%%%%%%%%%%%%%%%%%%%%%%%%%%
%
%   Заметка на полях  (ремарка)
%
%%%%%%%%%%%%%%%%%%%%%%%%%%%%%%%%%%%%%%%%%%%%%%%%%%%%%%%%%%%%%%%%
\newcommand{\rem}[1]
{
\marginpar{\scriptsize\textcolor{red}{#1}}
}
\newcommand{\рем}[1]
{
	\marginpar{\scriptsize\textcolor{red}{#1}}
}

%%%%%%%%%%%%%%%%%%%%%%%%%%%%%%%%%%%%%%%%%%%%%%%%%%%% ПЕРЕОПРЕДЕЛЕНИЕ ФОРМАТИРОВАНИЯ ЯЧЕЕК ТАБЛИЦЫ%%%%%%%%%%
%
\newcolumntype{P}[1]{>{\centering\arraybackslash}p{#1}}   %  \centering   \raggedleft  \raggedright
\newcolumntype{M}[1]{>{\raggedright\arraybackslash}m{#1}} %
\newcolumntype{G}[1]{>{\centering\arraybackslash}m{#1}} %

%%%%%%%%%%%% ВСАВКА с масштабированием ИЗОБРАЖЕНИЯ 2х3  В ТАБЛИЦУ 
\newcommand{\imt}[1]
{\includegraphics[width=35mm, height=23mm, keepaspectratio=false]{#1}}

%%%% Переопределение команды для 
%  Её вызов — \imgh{45.25mm}{zb}{Пример}
%  Первый параметр — ширина
%  Второй параметр — название файла
%  Третий параметр — название подписи к изображению
\newcommand{\imgh}[3]
{
	\begin{figure}[hpt!]
		\center{\includegraphics[width=#1]{foto/#2}}
		\caption{\small {#3}}
		\label{ris:#2}
	\end{figure}
}

\newcommand{\imgroot}[4]
{
	\begin{figure}[hpt!]
		\center{\includegraphics[angle=#4,width=#1]{foto/#2}}
		\caption{\small {#3}}
		\label{ris:#2}
	\end{figure}
}



%%Собственный  простейший список без нумерации и с обычными межстрочными интервалами
\newenvironment{compactlist}{
    \begin{list}{{$\bullet$}}{
            \setlength\partopsep{0pt}
            \setlength\parskip{0pt}
            \setlength\parsep{0pt}
            \setlength\topsep{0pt}
            \setlength\itemsep{0pt}
            \setlength{\itemindent}{\leftmargin}
            \setlength{\leftmargin}{0pt}
        }
    }{
    \end{list}
}
%%%%%%%%%%%%%%%%%%%%%%%%%%%%%%%%%%%%
%%
%% ПЕРЕОПРЕДЕЛЕНИЕ ДЛЯ ЗАПИСИ СТРОКИ АКТА ОСМОТРА
%%

\newcommand{\акт}[4]{\Rownum  & {\small #1}& #2  & #3 & #4\\  \toprule}

%%%%%%%%%%%%%%%
%% Переопределение для ЗАКЛЮЧЕНИЯ. Таблица ввода повреждений  с фото

\newcommand{\пов}[2]{\Rownum  & {\small #1 }&  \imt{foto/#2}\\ \hline \toprule}

%%%%%%%%%%%%%%%%
%%%  Переопределение длятаблицы ИСТОРИИ РЕМОНТА и сервисного обслуживания

\newcommand{\ист}[5]{#1 & #2  & #2 & #4   & #5 \\ \hline}

%%%%%%%%%%%%%%%%%%%%%%%%%%%%%%%%%%%5
\newcommand{\dee}{
	% вертикальные промежутки:
	\topsep=0pt % вокруг списка
	\parsep=0pt % между абзацами
	\itemsep=0pt % между пунктами % горизонтальные промежутки: \itemindent=0pt % абзацный выступ
	\labelsep=1ex % расстояние до метки
	\leftmargin=\parindent % отступ слева
	\rightmargin=0pt} % отступ справа
%%

%%%%%%%%%%%% Нумерованный список
\newcommand{\be}{\begin{enumerate}}
\newcommand{\en}{\end{enumerate}}


%%%% Вставить цитату
\newcommand{\цитата}[1]
{ 
	\begin{quote} 
		\textcolor{gray}{#1} 
	\end{quote} 
}

\newcommand{\блеклый}[1]
{\textcolor{gray}{#1}[0.7]}

\newcommand{\сноска}[1]{\footnote{#1}}

\newcommand{\икс}{$x$}
\newcommand{\игрек}{$y$}
\newcommand{\зет}{$z$}
%\newcommand{\auda}{Audatex AudaWeb, в модуле ОСАГО ПРО}
\newcommand{\auda}{Audatex AudaWeb}
\newcommand{\г}{$\checkmark $}
\newcommand{\7}{$\checkmark $}
\newcommand{\градус}{\circ}

\newcommand{\угол}[1]{$ #1^\circ $}
%%%% Стиль для колонтитулов

\newcommand{\грз}{\grz}
\newcommand{\вин}{\vin}
\newcommand{\датадтп}{\datadtp}
\newcommand{\датадоговора}{\dog}
\newcommand{\начато}{\datastart}
\newcommand{\датаосмотра}{\osm}
\newcommand{\датазаключения}{\zkl}
\newcommand{\страховойполис}{\polis}
\newcommand{\протокол}{\pr}
\newcommand{\повреждения}{\pov}
\newcommand{\иск}{\isk}
\newcommand{\тс}{\tc}
\newcommand{\окончено}{\dataend}
\newcommand{\прибл}{$ \approx $}
\newcommand{\тса}{\tca}
\newcommand{\тсб}{\tcb}
\newcommand{\ссылка}{\ref}
%\newcommand{}{}

%%%%%%%%%%%%%%%% ПЕРЕОПРЕДЕЛЕНИЕ  "По вопросу"     \повопросу{вопрос}
\newcommand{\повопросу}[1]{\,{\renewcommand\baselinestretch{0.86}\small\normalsize 
\subsection{\underline{По  вопросу}\,\,\textbf{\small{<<#1>>}}}}
\renewcommand\baselinestretch{1.2}\small\normalsize}
%%%%%%%%%%%%%%%%%%%%%%%%%%%%%%%%%%%%%%%%%%%%%


\newcommand{\фото}[2]
{
    \begin{figure}[H]
        \center{\includegraphics[width=0.99\textwidth]{foto/#1}}
        \caption{\small {#2}}
        \label{рис:#1}
    \end{figure}
}


%%%%%%%%%%%%%%% ДВА РИСУНКА РЯДОМ            \дварядом{файл1}{подпись1}{файл2}{подпись2}
\newcommand{\дварядом}[4]{\begin{figure}[H]\centering
        \parbox[t]{0.49\textwidth}
        {\centering
            \includegraphics[width=.49\textwidth,  height=.32\textwidth]{foto/#1}
            \caption{\footnotesize {#2}}
            \label{рис:#1}}
        \hfil \hfil
        \parbox[t]{0.49\textwidth}
        {\centering
            \includegraphics[width=.49\textwidth, height=.32\textwidth]{foto/#3}
            \caption{\footnotesize {#4}}
            \label{рис:#3}}
        
\end{figure}}


%%%% СТС две стороны рядом
\newcommand{\стс}[4]
{\begin{figure}[H]
        \centering
        \parbox[t]{0.49\textwidth}
        {\centering
            \includegraphics[width=.49\textwidth]{foto/#1}
            \caption{\footnotesize {#2}}
            \label{рис:#1}}
        \hfil \hfil
        \parbox[t]{0.49\textwidth}
        {\centering
            \includegraphics[width=.49\textwidth]{foto/#3}
            \caption{\footnotesize {#4}}
            \label{рис:#3}}
        
\end{figure}}


%%%%% ФОТО РЯДОМ С ТЕКСТОМ
%
%\newcommand{\фотосправа}[2]{
%    \begin{SCfigure}
%        \centering {\footnotesize \caption{#2} 
%            \includegraphics[width = 0.6 \textwidth]{foto/#1}
%            \label{рис:#1}
%    \end{SCfigure}}



%%%% Переопределение команды для 
%  Её вызов — \фотомасштаб{45.25mm}{название файла}{подпись рисунка}

%  Первый параметр — название файла
%  Второй параметр — название подписи к изображению
%  Третий параметр — ширина
\newcommand{\фотомасштаб}[3]
{
    \begin{figure}[H]
        \center{\includegraphics[width=#3]{foto/#1}}
        \caption{\small{#2}}
        \label{рис:#1}
    \end{figure}
}



%  Её вызов — \фотоповорот{45.25mm}{название файла}{подпись рисунка}{угол поворота}
%  Первый параметр — ширина
%  Второй параметр — название файла
%  Третий параметр — название подписи к изображению
\newcommand{\фотоповорот}[4]
{
    \begin{figure}[hpt!]
        \center{\includegraphics[angle=#4,width=#1]{foto/#2}}
        \caption{\small {#3}}
        \label{рис:#2}
    \end{figure}
}

%%% ИЗМЕРИТЬ ШИРИНУ СТРАНИЦЫ
\newcommand{\ширина}{\the\textwidth\\
    \printinunitsof{mm}\prntlen{\textwidth}}


