
\documentclass[a4paper,10pt]{report}

\usepackage{xltxtra} %выполняет основные настройки XeLaTeX и загружает пакет fontspec

\usepackage{xunicode}

\usepackage{polyglossia}

\usepackage{ccaption}

\usepackage{subfigure}

\usepackage{float}

%\setdefaultlanguage{russian}
\setmainlanguage{russian}
\setotherlanguage{english}

\defaultfontfeatures{Ligatures={TeX},Renderer=Basic}  %% свойства шрифтов по умолчанию
\setkeys{russian}{babelshorthands=true} %поддержка специальных команд пакета babel

\setmainfont[Ligatures={TeX,Historic}]{Arial} %% задаёт основной шрифт документа
%
%\setsansfont{Comic Sans MS}           %% задаёт шрифт без засечек
%%\setsansfont{Arial} 

% В качестве аргумента можно указывать системное имя шрифта, имя файла или файлов шрифт
\setsansfont[
BoldFont=arialbd.ttf,
ItalicFont=ariali.ttf,
BoldItalicFont=arialbi.ttf
]{arial.ttf}
\setmonofont{Courier New} 

\newfontfamily{\cyrillicfont}{Arial} 
\newfontfamily{\cyrillicfontrm}{Arial}
\newfontfamily{\cyrillicfonttt}{Arial}
\newfontfamily{\cyrillicfontsf}{Arial}


%переопределения терминов в пакете polyglossia
\addto\captionsrussian{%
	\renewcommand{\figurename}{Фото }%
	\renewcommand{\tablename}{Табл.}%
}

\frenchspacing

\renewcommand{\epsilon}{\ensuremath{\varepsilon}}
\renewcommand{\phi}{\ensuremath{\varphi}}
\renewcommand{\kappa}{\ensuremath{\varkappa}}
\renewcommand{\le}{\ensuremath{\leqslant}}
\renewcommand{\leq}{\ensuremath{\leqslant}}
\renewcommand{\ge}{\ensuremath{\geqslant}}
\renewcommand{\geq}{\ensuremath{\geqslant}}
\renewcommand{\emptyset}{\varnothing}

%%%% Дополнительная работа с математикой
\usepackage{amsmath,amsfonts,amssymb,amsthm,mathtools} % AMS
\usepackage{icomma} % "Умная" запятая: $0,2$ --- число, $0, 2$ --- перечисление
%%
%%% Номера формул
%\mathtoolsset{showonlyrefs=true} % Показывать номера только у тех формул, на которые есть \eqref{} в тексте.
%%\usepackage{leqno} % Нумерация формул слева

%% Свои команды
%\DeclareMathOperator{\sgn}{\mathop{sgn}}

%%% Перенос знаков в формулах (по Львовскому)
%\newcommand*{\hm}[1]{#1\nobreak\discretionary{}
%	{\hbox{$\mathsurround=0pt #1$}}{}}

%%% Работа с картинками
\usepackage{graphicx}  % Для вставки рисунков
\graphicspath{{foto/}{images/}{jpg/}}  % папки с картинками
\setlength\fboxsep{0pt} % Отступ рамки \fbox{} от рисунка
\setlength\fboxrule{0pt} % Толщина линий рамки \fbox{}
%\usepackage{wrapfig} % Обтекание рисунков текстом

%%% Страница
%\special{papersize=210mm,297mm}

%\usepackage{extsizes} % Возможность сделать 14-й шрифт
\usepackage{geometry} % Простой способ задавать поля
\geometry{top=15mm}
\geometry{bottom=15mm}
\geometry{left=15mm}
\geometry{right=15mm}
%


\usepackage{tikz} % Работа с графикой
\usepackage{pgfplots}
\pgfplotsset{compat=newest}
\usepackage{pgfplotstable}

% Убираем нумерацию страниц
\pagenumbering{gobble}

%\pagestyle{empty}

\clubpenalty=10000 
\widowpenalty=10000



\captiondelim{. }



\begin{document}
%
%
%

%pagenumbering{Roman}
%%%%%%%%%%%%%%%%%%%%%%%%%%%%%%%%%%%%%%%%%%%%%%%%
%
%    Титульные данные
%
%%%%%%%%%%%%%%%%%%%%%%%%%%%%%%%%%%%%%%%%%%%%%%%%
\newcommand{\NomerDoc}{27-03/2020}  % номер заключения
%
\newcommand{\dog}{25.02.2020} % Дата договора
%
\newcommand{\datastart}{25.02.2020} % Дата начала исследования
%	
%\newcommand{\datadtp}{\ldots}  % Дата ДТП        
%         
\newcommand{\osm}{25.02.2020} % ОСМОТР % Дата осмотра
%
\newcommand{\dataend}{25.03.2020} % Дата окончания
%
%\newcommand{\datastop}{\ldots}  % Дата приостановления
%\newcommand{\datarestart}{\ldots}
%
%%%%%%%%%%%   ДОКУМЕНТЫ  
%  
% Свидетельство о регистрации ТС
\newcommand{\свид}{ВК № 384784 }     
% Паспорт транспорного средства
\newcommand{\птс}{ТС № 001742 }     
\newcommand{\владелец}{\ldots}
\newcommand{\адресвладельца}{\ldots}
\newcommand{\заказчик}{ООО «Логистик-Юг», в лице генерального 
    директора Тарасова Андрея Ивановича}
\newcommand{\адресзаказчика}{г. Краснодар, ул. Стасова, д.178/1, офис 81}
% Страховой полис
\newcommand{\polis}{\ldots}  
%  Протокол
\newcommand{\prt}{\ldots}                        
 % Постановление
\newcommand{\постановление}{\ldots}
 % Определение
\newcommand{\определение}{\ldots}
% Повреждения
\newcommand{\pov}{\ldots} %Перечень повреждений  
% Присутствовали   
\newcommand{\присутствовали}{\ldots}
% Место осмотра
\newcommand{\местоосмотра}{г. Краснодар, хутор Ленина, МФТ-1, отделение 4, литер Г- 46}
%
%%%%%%%%%%%  ТРАНСПОРТНОЕ СРЕДСТВО
%
\newcommand{\tc}{KOMATSU FG15T-20}   % Транспортное сердство
\newcommand{\grz}{\ldots} % Регистрационный знак
\newcommand{\vin}{\ldots}  % VIN
%
\newcommand{\типкузова}{\ldots}
\newcommand{\двигатель}{К15-019305Х} 
%
\newcommand{\colr}{\ldots}  % Какого  цвета кузов
\newcommand{\цвет}{\ldots}  
\newcommand{\типлкп}{\ldots} 
%
\newcommand{\пробег}{\ldots}
\newcommand{\год}{2007}          % Год выпуска
\newcommand{\началоэкспл}{\ldots}
\newcommand{\датаизготовления}{\ldots}


%%%%%%%%% Другие транспортные средства

%
%\newcommand{\tca}{\ldots}
%\newcommand{\тса}{\ldots}
%%
\newcommand{\tcb}{\ldots}
%\newcommand{\тсв}{\cdod}



 
%
%%%%%%%%%%%%%%%%%%%%%                   % Если судебка 
%
\newcommand{\delosud}{гражданское дело 2-1033/19} % 

\newcommand{\delonum}{2-1033/19}  % номер дела

\newcommand{\opr}{дополнительной судебной автотехнической экспертизы}

\newcommand{\sud}{мирового судьи судебного участка № 11 г. Белореченска Краснодарского края Мозер Г.Л.}

\newcommand{\dataopr}{14.10.2019}

\newcommand{\isk}{Ильенко Р.В. к Шамояну Р.О. и САО "ВСК" о взыскании страхового возмещения}

%
\newcommand{\hod}{\ldots} % № ходатайство
%\newcommand{\hod2}{№ 3567 от 01.01.2008} % № ходатайство
%%%%%%%%%%%%%%%%%%%%%%%%%
% РЕЦЕНЗИЯ
\newcommand{\чел}{\ldots}
\newcommand{\назакл}{\ldots}

% example-image

\renewcommand{\chaptername}{Заключение эксперта}
\renewcommand{\refname}{Список}
\renewcommand{\bibname}{\large {Использованные нормативы и источники информации}}

\renewcommand{\epsilon}{\ensuremath{\varepsilon}}
\renewcommand{\phi}{\ensuremath{\varphi}}
\renewcommand{\kappa}{\ensuremath{\varkappa}}
\renewcommand{\le}{\ensuremath{\leqslant}}
\renewcommand{\leq}{\ensuremath{\leqslant}}
\renewcommand{\ge}{\ensuremath{\geqslant}}
\renewcommand{\geq}{\ensuremath{\geqslant}}
\renewcommand{\emptyset}{\varnothing}

%%%%%%%%%%%%%%%%%%%%%%%%%%%%%%%%%   ПРОИЗВОЛЬНЫЙ СЧЕТЧИК

\newcounter{@nnn}  % задаём имя счёчика 
\setcounter{@nnn}{0}  % устанавливаем его первое значение

\newcommand{\z}[2]{\par\addtocounter{@nnn}{1}  % формируем комманду 
	{\bf \arabic{@nnn}.   Работы по заказ-наряду  #1, произведеные на автомобиле #2:}}


%%%%%%%%%%%%%%%%%%%%%%%  Подсчет строк в таблице
\newcounter{rownum}
\setcounter{rownum}{0}
\newcommand{\Rownum}{\stepcounter{rownum}%
\arabic{rownum}}

%\def\contentsname{Содержание}
%Аннотация  \abstractname
%Часть       \partname
%Глава        \chaptername
%Список литературы  \refname
%Рис.                \figurename
%Таблица           \tablename
%Литература       \bibname

%Предметный указатель  \indexname
%Приложение                \appendixname
%Содержание          \contentsname
%Список иллюстраций \listfigurename
%Список таблиц        \listtablename
%\addto\captionsrussian{\def\refname{Список используемой литературы}}

%%%%%%%%%%%%%%   Размещение изображений
%\textfloatsep — расстояние между флоатс (в верхней или нижней части страницы) и текстом (по умолчанию, около 20pt)
%\floatsep — вертикальное расстояние между двумя флоатс (около 12pt)
%\intextsep — расстояние между флоатс вставленным "прямо здесь" (параметр h) и текстом (около 12pt)
%\abovecaptionskip и \belowcaptionskip — расстояние над и под подписью к флоат
\setcounter{totalnumber}{10}
\setcounter{topnumber}{10}
\renewcommand{\topfraction}{1}
\renewcommand{\textfraction}{0}
%%%%%%  Больше плавающих объектов на страницу
 \setlength{\textfloatsep}{10pt plus 1.0pt minus 2.0pt}
 \setlength{\floatsep}{5pt plus 1.0pt minus 1.0pt}
 \setlength{\intextsep}{5pt plus 1.0pt minus 1.0pt}

%%%%%%%%%%%%%%%%%%%%%%%%%%%%%%%%%%%%%%%%%%%%%%%%%%%%%%%%%%%%%%%
%
%   Заметка на полях  (ремарка)
%
%%%%%%%%%%%%%%%%%%%%%%%%%%%%%%%%%%%%%%%%%%%%%%%%%%%%%%%%%%%%%%%%
\newcommand{\rem}[1]
{
\marginpar{\scriptsize\textcolor{red}{#1}}
}
\newcommand{\рем}[1]
{
	\marginpar{\scriptsize\textcolor{red}{#1}}
}

%%%%%%%%%%%%%%%%%%%%%%%%%%%%%%%%%%%%%%%%%%%%%%%%%%%% ПЕРЕОПРЕДЕЛЕНИЕ ФОРМАТИРОВАНИЯ ЯЧЕЕК ТАБЛИЦЫ%%%%%%%%%%
%
\newcolumntype{P}[1]{>{\centering\arraybackslash}p{#1}}   %  \centering   \raggedleft  \raggedright
\newcolumntype{M}[1]{>{\raggedright\arraybackslash}m{#1}} %
\newcolumntype{G}[1]{>{\centering\arraybackslash}m{#1}} %

%%%%%%%%%%%% ВСАВКА с масштабированием ИЗОБРАЖЕНИЯ 2х3  В ТАБЛИЦУ
\newcommand{\imt}[1]
{\includegraphics[width=35mm, height=23mm, keepaspectratio=false]{#1}}

%%%% Переопределение команды для
%  Её вызов — \imgh{45.25mm}{zb}{Пример}
%  Первый параметр — ширина
%  Второй параметр — название файла
%  Третий параметр — название подписи к изображению
\newcommand{\imgh}[3]
{
	\begin{figure}[hpt!]
		\center{\includegraphics[width=#1]{foto/#2}}
		\caption{\small {#3}}
		\label{ris:#2}
	\end{figure}
}


\newcommand{\imgroot}[4]
{
	\begin{figure}[hpt!]
		\center{\includegraphics[angle=#4,width=#1]{foto/#2}}
		\caption{\small {#3}}
		\label{ris:#2}
	\end{figure}
}

%%Собственный  простейший список без нумерации и с обычными межстрочными интервалами
\newenvironment{compactlist}{
    \begin{list}{{$\bullet$}}{
            \setlength\partopsep{0pt}
            \setlength\parskip{0pt}
            \setlength\parsep{0pt}
            \setlength\topsep{0pt}
            \setlength\itemsep{0pt}
            \setlength{\itemindent}{\leftmargin}
            \setlength{\leftmargin}{0pt}
        }
    }{
    \end{list}
}
%%%%%%%%%%%%%%%%%%%%%%%%%%%%%%%%%%%%
%%
%% ПЕРЕОПРЕДЕЛЕНИЕ ДЛЯ ЗАПИСИ СТРОКИ АКТА ОСМОТРА
%%

\newcommand{\акт}[4]{\Rownum  & {\small #1}& #2  & #3 & #4\\  \toprule}

%%%%%%%%%%%%%%%
%% Переопределение для ЗАКЛЮЧЕНИЯ. Таблица ввода повреждений  с фото

\newcommand{\пов}[2]{\Rownum  & {\small #1 }&  \imt{foto/#2}\\ \hline \toprule}

%%%%%%%%%%%%%%%%
%%%  Переопределение длятаблицы ИСТОРИИ РЕМОНТА и сервисного обслуживания

\newcommand{\ист}[5]{#1 & #2  & #2 & #4   & #5 \\ \hline}

%%%%%%%%%%%%%%%%%%%%%%%%%%%%%%%%%%%
%%% ПЕРЕОПРЕДЕЛЕНИЕ ДЛЯ ТАБЛИЦЫ с Игдексом и Двумя Столбцами

\newcommand{\два}[2]{\small \Rownum  & {\small #1 }&  \small #2\\ \hline \toprule}

%%% ПЕРЕОПРЕДЕЛЕНИЕ ДЛЯ ТАБЛИЦЫ с Индексом и Пятью Столбцами

\newcommand{\пять}[5]{\small \Rownum  & \small #1 &  \small #2&\small #3&\small #4&\small #5\\ \hline \toprule}

%%%%%%%%%%%%%%%%%%%%%%%%%%%%%%%%%%%
\newcommand{\dee}{
	% вертикальные промежутки:
	\topsep=0pt % вокруг списка
	\parsep=0pt % между абзацами
	\itemsep=0pt % между пунктами % горизонтальные промежутки: \itemindent=0pt % абзацный выступ
	\labelsep=1ex % расстояние до метки
	\leftmargin=\parindent % отступ слева
	\rightmargin=0pt} % отступ справа
%%

%%%%%%%%%%%% Нумерованный список
\newcommand{\be}{\begin{enumerate}}
\newcommand{\en}{\end{enumerate}}

%%%% Вставить цитату
\newcommand{\цитата}[1]
{
	\begin{quote}
		\textcolor{gray}{#1}
	\end{quote}
}

\newcommand{\блеклый}[1]
{\textcolor{gray}{#1}[0.7]}

\newcommand{\сноска}[1]{\footnote{#1}}

\newcommand{\икс}{$x$}
\newcommand{\игрек}{$y$}
\newcommand{\зет}{$z$}
\newcommand{\audaОСАГО}{Audatex AudaWeb, в модуле ОСАГО ПРО}
\newcommand{\auda}{Audatex AudaWeb}

%%%%%%%%%%%%%%%%%%%%%%%%%%% ЧЕК БОКСЫ
\newcommand{\cmark}{\ding{51}}%$\checkmark $
\newcommand{\xmark}{\ding{55}}%
\newcommand{\done}{{$\square$}{\hspace{-6.5pt}\cmark}}
\newcommand{\wontfix}{{$\square$}{\hspace{-6.5pt}\xmark}}
%%%%%%%%%%%%%%%%%%%%%%%%%%%%%%%%%%%%%%%%%%

\newcommand{\г}{$\checkmark $}
\newcommand{\7}{$\checkmark $}
\newcommand{\галка}{\ding{51}}
\newcommand{\х}{\ding{55}}
\newcommand{\градус}{\circ}
\newcommand{\чек}{$\square$}

\newcommand{\чекг}{\done}
\newcommand{\чекх}{\wontfix}


% Площадь пореждений, М2
\newcommand{\s}[1]{$S_{\text{повр}} \approx#1\, m^2$}

\newcommand{\угол}[1]{$ #1^\circ $}
%%%% Стиль для колонтитулов

\newcommand{\грз}{\grz}
\newcommand{\вин}{\vin}
\newcommand{\датадтп}{\datadtp}
\newcommand{\датадоговора}{\dog}
\newcommand{\начато}{\datastart}
\newcommand{\датаосмотра}{\osm}
\newcommand{\датазаключения}{\zkl}
\newcommand{\страховойполис}{\polis}
\newcommand{\протокол}{\pr}
\newcommand{\повреждения}{\pov}
\newcommand{\иск}{\isk}
\newcommand{\тс}{\tc}
\newcommand{\окончено}{\dataend}
\newcommand{\прибл}{$ \approx $}
\newcommand{\тса}{\tca}
\newcommand{\тсб}{\tcb}
\newcommand{\ссылка}{\ref}
%\newcommand{}{}

%%%%%%%%%%%%%%%% ПЕРЕОПРЕДЕЛЕНИЕ  "По вопросу"     \повопросу{вопрос}
\newcommand{\повопросу}[1]{\,{\renewcommand\baselinestretch{0.86}\small\normalsize
\subsection{\underline{По  вопросу}\,\,\textbf{\small{<<#1>>}}}}
\renewcommand\baselinestretch{1.2}\small\normalsize}
%%%%%%%%%%%%%%%%%%%%%%%%%%%%%%%%%%%%%%%%%%%%%


\newcommand{\фото}[2]
{
    \begin{figure}[H]
        \center{\includegraphics[width=0.99\textwidth]{#1}}
        \caption{\small {#2}}
        \label{рис:#1}
    \end{figure}
}



\newcommand{\фотоб}[2]
{
	\begin{figure}[H]
		\center{\includegraphics[width=0.99\textwidth]{#1}}
		\caption*{\small {#2}}
		\label{рис:#1}
	\end{figure}
}



\newcommand{\фот}[2]
{
	\begin{figure}[H]
		\center{\includegraphics[width=0.99\textwidth]{#1}}
		\caption{\small {#2}}
	%	\label{рис:#1}
	\end{figure}
}

%%%%%%%%%%%%%%% ДВА РИСУНКА РЯДОМ            \дварядом{файл1}{подпись1}{файл2}{подпись2}
\newcommand{\дварядом}[4]{\begin{figure}[H]\centering
        \parbox[t]{0.49\textwidth}
        {\centering
            \includegraphics[width=.49\textwidth,  height=.32\textwidth]{foto/#1}
            \caption{\footnotesize {#2}}
            \label{рис:#1}}
        \hfil \hfil
        \parbox[t]{0.49\textwidth}
        {\centering
            \includegraphics[width=.49\textwidth, height=.32\textwidth]{foto/#3}
            \caption{\footnotesize {#4}}
            \label{рис:#3}}

\end{figure}}

%%%%%%%%%%%%%%%%%%%%%%%%%%%%%%%%%%%
% Два рядом с одной общей подписью
%%%%%%%%%%%%%%%%%%%%%%%%%%%%%%%%%%%

\newcommand{\дварисунка}[5]{\begin{figure}[H]
	\begin{minipage}{0.49\textwidth}
		\includegraphics[width=\linewidth,  height=.64\linewidth]{foto/#1}
		\subcaption{#2}
	\end{minipage}
	\hfill
	\begin{minipage}{0.49\textwidth}
		\includegraphics[width=\linewidth,  height=.64\linewidth]{foto/#3}
		\subcaption{#4}
	\end{minipage}

	\caption{#5}
	\label{рис:#1}
:\end{figure}}




%%%% СТС две стороны рядом
\newcommand{\стс}[4]{\begin{figure}[H]
        \centering
        \parbox[t]{0.49\textwidth}
        {\centering
            \includegraphics[width=.49\textwidth]{foto/#1}
            \caption{\footnotesize {#2}}
            \label{рис:#1}}
        \hfil \hfil
        \parbox[t]{0.49\textwidth}
        {\centering
            \includegraphics[width=.49\textwidth]{foto/#3}
            \caption{\footnotesize {#4}}
            \label{рис:#3}}

\end{figure}}


%%%%% ФОТО РЯДОМ С ТЕКСТОМ
%
%\newcommand{\фотосправа}[2]{
%    \begin{SCfigure}
%        \centering {\footnotesize \caption{#2}
%            \includegraphics[width = 0.6 \textwidth]{foto/#1}
%            \label{рис:#1}
%    \end{SCfigure}}



%%%% Переопределение команды для
%  Её вызов — \фотомасштаб{45.25mm}{название файла}{подпись рисунка}

%  Первый параметр — название файла
%  Второй параметр — название подписи к изображению
%  Третий параметр — ширина
\newcommand{\фотомасштаб}[3]
{
    \begin{figure}[H]
        \center{\includegraphics[width=#3]{foto/#1}}
        \caption{\small{#2}}
        \label{рис:#1}
    \end{figure}
}



%  Её вызов — \фотоповорот{45.25mm}{название файла}{подпись рисунка}{угол поворота}
%  Первый параметр — ширина
%  Второй параметр — название файла
%  Третий параметр — название подписи к изображению
\newcommand{\фотоповорот}[4]
{
    \begin{figure}[hpt!]
        \center{\includegraphics[angle=#4,width=#1]{foto/#2}}
        \caption{\small {#3}}
        \label{рис:#2}
    \end{figure}
}

%%% ИЗМЕРИТЬ ШИРИНУ СТРАНИЦЫ
\newcommand{\ширина}{\the\textwidth\\
    \printinunitsof{mm}\prntlen{\textwidth}}



\Large{Фототаблица  повреждений }


\end{document}