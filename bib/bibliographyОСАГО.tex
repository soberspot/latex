\begin {thebibliography}{00} 

%\bibitem{fz:73} 
%Федеральный закон от 31 мая 2001 г. N 73-ФЗ "О государственной судебно-экспертной деятельности в Российской Федерации".

\bibitem{em:432}
Положение Банка России от 19 сентября 2014 года № 432-П \emph{О единой методике определения размера расходов на восстановительный ремонт в отношении поврежденного транспортного средства} // Вестник банка России, № 93 (1571). Нормативные акты и оперативная информация
Центрального банка Российской Федерации. Москва, 2014
\bibitem{em:433}
Положение ЦБ РФ от 19 сентября 2014 года №433-П \emph{О правилах проведения независимой технической экспертизы транспортного средства} // Вестник банка России, № 93 (1571). Нормативные акты и оперативная информация Центрального банка Российской Федерации. Москва, 2014
\bibitem{mu:2013} 
Исследование автомототранспортных средств в целях определения стоимости восстановительного ремонта и оценки // Методические рекомендации для судебных экспертов, утверждено  научно-методическим советом РФЦСЭ 13 марта 2013 г., протокол №35., Минюст России, 2013г.
\bibitem{ca:1} 
Судебная автотехническая экспертиза. Институт повышения квалификации Российского Федерального Центра Судебной Экспертизы, М. 2007.
\bibitem{koruh}
Корухов\,Ю.\,Г., Замиховский\, М.\,И. \emph {Криминалистическая фотография и видеозапись для экспертов-автотехников.}// Практическое пособие М.: ИПК РФЦСЭ при МЮ РФ, 2006г.
\bibitem{chava}
Чава\,И.\,И. \emph {Судебная автотехническая экспертиза} // Учебно-методическое пособие для  экспертов,    судей, следователей, дознавателей и адвокатов. НП «Судэкс», Москва, 2014.
\bibitem{fotometr}
Краснопевцев\,Б.,В. \emph{Фотограмметрия} // Учебное пособие. МИИГАиК, 2008.
\bibitem{suv}
Суворов\,Ю.\,Б. \emph{Судебная дорожно-транспортная экспертиза}. «Экзамен»
\bibitem{chalk}
Чалкина\,А.\,В.  \emph{Осмотр, фиксация и моделирование механизма образования внешних повреждений автомобилей с использованием их масштабных изображений} / А.\,В. Чалкин, А.\,Л. Пушнов, В.\,В. Чубченко // Учебное пособие.  М.:ВНКЦ МВД СССР 1991г.
\bibitem{torm}
Булгаков \,Н.\,А.  \emph{Исследование динамики торможения автомобиля} / Н.А. Булгаков, А.Б. Гредескул, С.И Ломака // Научное сообщение № 18.- Харьков: Изд-во Харьковского госуниверситета, 1962. - 36 с.
\bibitem{prich}
Бутырин\, А.\,Ю. \emph{Характеристика обстоятельств, имеющих значение при проведении казуальных автотехнических и строительно-технических судебно-экспертных исследований} / А.Ю. Бутырин, Д.С. Дубровский, Е.А. Холина, И.И. Чава // Юридический журнал № 4,- М.:  2010
\bibitem{mot}
ПОТ РМ-027-2003. Межотраслевые правила по охране труда на автомобильном транспорте
\bibitem{vin}
Сервис по автоматической расшифровке VIN номеров – AudaVIN
\bibitem{azt}
Методика окраски и расчета стоимости лакокрасочных материалов для проведения окраски транспортных средств   AZT, http://azt-automotive.com,   http://www.schwacke.ru/ereonline.htm
\bibitem{autodata}
Информационно-справочная система по ремонту и сервисному обслуживанию автомобилей, лицензионный код GA0288799. Autodata online,  Autodata Group Ltd, Великобритания. Дистрибьютор в России ЗАО «Легион-Автодата», вэб-сайты www.autodata.ru и www.autodata-online.ru
\bibitem{licauda}
CRASH 3 Technical Manual. National Highway Traffic Safety Administration, www-nass.nhtsa.gov

\end{thebibliography}
