%\left( \addcontentsline{toc}{section}{Использованные нормативы и источники информации}

\subsection{Использованные нормативы и источники информации}
%
\begin{enumerate}
%	
%\vspace{-10mm}
%
%\item                                 
%Федеральный закон от 31 мая 2001 г. N 73-ФЗ \emph{О государственной судебно-экспертной деятельности в Российской Федерации} 
%
%\item
%Федеральный закон от 25.04.2002 N 40-ФЗ \emph{Об обязательном страховании гражданской ответственности владельцев транспортных средств} // Собрание законодательства РФ  06.05.2002, N 18, ст. 1720
%
%\item
%Федеральный закон от 28 марта 2017 г. N 49-ФЗ \emph{О внесении изменений в Федеральный закон "Об обязательном страховании гражданской ответственности владельцев транспортных средств} // Российская газета - Федеральный выпуск №7234 (68)
%
%
%
%\item 
%Махнин\,Е.\,Л., Новоселецкий\, И.\,Н., Федотов\, С.\,В. \emph{Методические рекомендации по проведению судебных автотехнических экспертиз и исследований колесных транспортных средств в целях определения размера ущерба, стоимости восстановительного ремонта и оценки} // --М.: ФБУ РФЦСЭ при Минюсте России, 2018.-326 с.  ISBN 978-5-91133-185-6
%
%\item  «Исследование автомототранспортных средств в целях определения стоимости восстановительного ремонта и оценки» (Методические рекомендации для судебных экспертов, утверждено  научно-методическим советом РФЦСЭ 13 марта 2013 г., протокол №35.), Минюст России, 20013г.
%%
%\item  Судебная автотехническая экспертиза. Институт повышения квалификации Российского Федерального Центра Судебной Экспертизы, М. 2007.
%
%\item
%Положение Банка России от 19 сентября 2014 года № 432-П \emph{О единой методике определения размера расходов на восстановительный ремонт в отношении повреждённого транспортного средства} // Вестник банка России, № 93 (1571). Нормативные акты и оперативная информация
%Центрального банка Российской Федерации. Москва, 2014
%
%\item
%Положение ЦБ РФ от 19 сентября 2014 года №433-П \emph{О правилах проведения независимой технической экспертизы транспортного средства} // Вестник банка России, № 93 (1571). Нормативные акты и оперативная информация Центрального банка Российской Федерации. Москва, 2014
%
%\item 
%Технический регламент Таможенного союза \emph{О безопасности машин и оборудования} (ТР ТС 010/2011), утвержденный решением Комиссии Таможенного союза от 18.10.2011 № 823
%
\item 
Постановление Правительства РФ от 11.04.2001 N 290 (ред. от 31.01.2017) \emph{Об утверждении Правил оказания услуг (выполнения работ) по техническому обслуживанию и ремонту автомототранспортных средств.}

\item 
Технический регламент Таможенного союза ТР ТС 018/2011 \emph{О безопасности колёсных транспортных средств},  утвержденный решением Комиссии Таможенного союза от 9 декабря 2011 года N 877.
\item 
ГОСТ 27.002-2015  \emph{«Надежность в технике. Термины и определения».}
\item 
ГОСТ 58197-2018 \emph{Порядок проведения экспертизы качества автомототранспортных средств.}
%\item
%Рекомендации. Технологическое содержание услуг по техническому обслуживанию и ремонту автомототранспортных средств    Р 3112199-0395-00. Система сертификации ГОСТ Р.
%\item 
%ГОСТ 16093-81 \emph{Основные нормы взаимозаменяемости. Резьба метрическая. Допуски. Посадки с зазором.}
%\item 
%ГОСТ 19281-89 \emph{Прокат из стали повышенной прочности. Общие технические
%условия.}
%\item 
%ГОСТ ISO 15071-2014 \emph{Болты с шестигранной уменьшенной головкой с фланцем. Класс точности А.}
%\item 
%ГОСТ ISO 16047-2015 \emph{Испытания крутящего момента и усилия предварительной затяжки.}
%\item 
%ГОСТ 12856-96 \emph{Листы асбостальные и прокладки из них}
%%\item 
%СП 16.13330.2012 \emph{Стальные конструкции.}
%\item 
%ПНАЭ Г-7-002-86. Нормы расчёта на прочность оборудования и трубопроводов атомных
%энергетических установок  / Госатомэнергонадзор СССР.– М.:
%Энергоатомиздат, 1989.
%\item 
%Руководящий документ РД 37.009.026-92. \emph{Положение о техническом обслуживании и ремонте автотранспортных средств, принадлежащих гражданам (легковые и грузовые автомобили, автобусы, минитрактора)}, утв. приказом по Департаменту автомобильной промышленности Минпрома РФ от 1 ноября 1992 г. N 43.
%
\item \emph
{Правила оказания услуг (выполнения работ) по техническому обслуживанию и ремонту автомототранспортных средств}// утв. постановлением Правительства РФ от 11 апреля 2001 г. N 290.

%\item ПОТ РМ-027-2003  \emph{Межотраслевые правила по охране труда на автомобильном транспорте} // Спб.: ЦОТПБСП, 2003. - 200 с.  ISBN 5-326-00140-3
%
%\item ГОСТ 2.610-2006 \emph{"Единая система конструкторской документации (ЕСКД). Правила выполнения эксплуатационных документов"}
%
%\item Хусаинов\, А.\,Ш. \emph{Теория автомобиля: конспект лекций } //
%А.\,Ш.\,Хусаинов, В.\,В.\,Селифонов. – Ульяновсk: УлГТУ, 2008. – 121 с.
% 
%\item \emph
%{Судебная автотехническая экспертиза} Институт повышения квалификации Российского Федерального Центра Судебной Экспертизы, М. 2007.
%
%\item
%\emph{Марочник сталей и сплавов} Под редакцией Зубченко А.С. — М.:
%Машиностроение, 2003 г.
%
\item 
Шестопалова Л.П., Лихачева Т.Е. \emph{Методы исследования материалов и деталей машин при проведении автотехнической экспертизы}// учеб. пособие /
. – М.: МАДИ, 2017. – 180 с.
%
%\item  
%Кутателадзе С.С. \emph{Основы теории теплообмена}. – М.: Атомиздат, 1979. – 416с.
%
%\item \emph{Двигатели внутреннего сгорания. Теория поршневых и комбинированных двигателей}. – Под ред. С. Орлина,  М .Круглова. М.: Машиностроение, 1983.- 372с.
%
%\item Корухов\,Ю.\,Г., Замиховский\, М.\,И. \emph {Криминалистическая фотография и видеозапись для экспертов-автотехников} // Практическое пособие М.: ИПК РФЦСЭ при МЮ РФ, 2006г.
%
%\item
%Чава\,И.\,И. \emph {Судебная автотехническая экспертиза} // Учебно-методическое пособие для  экспертов,    судей, следователей, дознавателей и адвокатов. НП «Судэкс», Москва, 2014.
%%
\item 
Краснопевцев\,Б.,В. \emph{Фотограмметрия} // Учебное пособие. МИИГАиК, 2008.
%
\item 
Хрулев\,А.\,Э. \emph{Ремонт двигателей зарубежных автомобилей} // Производственно-практ. издание --М.: Издательство "За рулем", 1999. --440 с., ил., табл.  ISBN 5-85907-084-5.
%
%\item  
%Хрулев А.Э., Кротов М.В., \emph{Применение инженерных методов при экспертном исследовании и определении причины перегрева ДВС} // Двигатели внутреннего сгорания 2018 № 1, ISSN 0419-8719.
%
%\item 
%Захаров Ю.А., Булатов Р.Р. \emph{Восстановление рабочей поверхности гильз цилиндров двигателей внутреннего сгорания автомобилей} // Молодой ученый. — 2015. — №5. — С. 145-148. — URL \url{https://moluch.ru/archive/85/15983/} (дата обращения: 23.03.2020).
%
\item
Информационно-справочная система по ремонту и сервисному обслуживанию автомобилей. Autodata online,  Autodata Group Ltd, Великобритания. Дистрибьютор в России ЗАО «Легион-Автодата», , лицензионный код GA0288799, вэб-сайты www.autodata.ru и www.autodata-online.ru

\item 
Технология электромеханической обработки материалов [Электронный ресурс]. — Режим доступа: URL \url{http://www.vstu.ru/razrabotka/tekhnologiya-elektromekhanichesk.html} (дата обращения: 23.03.2020).

%\item \emph{Исследование транспортных средств в целях определения стоимости восстановительного ремонта и оценки: курс лекций} // под общ. ред. д-ра юрид. наук, профессора С.А. Смирновой; Министерство юстиции Российской Федерации, Федеральное бюджетное учреждение Рос. Федер. центр судеб экспертизы. - М.: ФБУ РФЦСЭ при Минюсте России, 2017. - 286 с.
%
%\item  \emph{Технологическое руководство «Приемка, ремонт и выпуск из ремонта кузовов легковых автомобилей предприятиями автотехобслуживания»} РД 37.009.024-92
%
%\item Карасев А.\,И. \emph{Теория вероятностей и математическая статистика} : учебник. М.: Статистика, 1979. 279 с.
%
%\item А.\,П. Ковалев, А.\,А. Кушель, И.\,В. Королев, П.\,В. Фадеев	\emph{Практика оценки стоимости машин и оборудования} : учебник //  ; под ред. М. А. Федотовой. М. : Финансы и статистика, 2005. 272 с.
%\item
%Утакаева И.Х. \emph{Применение пакета статистического анализа Python для анализа данных автомобильного рынка}// Вестник Алтайской академии экономики и права. – 2019. – № 2-2. – С. 346-351; 
%
%\item Макушев\,Ю.П., Иванов\,А.Л. \emph{Упрощенный расчет турбокомпрессора для двигателя внутреннего сгорания} // Сибирская государственная автомобильно-дорожная академия, г. Омск, Омский научный вестник № 4 (73), 2008
%
%\item Герт Хак \emph{Турбодвигатели и компрессоры} // Справ. пособие : Пер. с нем., Лангкабель. - М. : АСТ : Астрель, 2003. - 350 с. : ил., 25 см. ISBN 5170193777
%
%\item Повреждения поршней // Группа Motorservice, Kolbenschmidt Pierburg, www.ms-motorservice.com
%
\item
\emph{Повреждения подшипников качения и их причины}. SKF АВ, Санкт-Петербург, 2002, 47
\item 	
\emph{Повреждения подшипников скольжения}. Kolbenschmidt / MS Motorservice International Gmbh
%
%\item Хрулев А.Э., Кротов М.В. \emph{Влияние неисправностей в системе смазки на характер повреждения подшипников ДВС}. Научно-технический журнал "Двигатели внутреннего сгорания" №1/2018, DOI: 10.20998/0419-8719.2018.1.13
%
%\item  \emph{Анализ и предотвращение отказов подшипников}. https://access.cummins.com
%
%\item 	\emph{Руководство по замене вкладышей и устранению повреждений вкладышей}. Federal-Mogul Corportion/ General Distribution Centre, Belgium
%
\item
Автомобильный справочник //  Пер. с англ.  -- 2-е изд., перераб. и доп. --М.:ЗАО "КЖИ "За рулем", 2004. --992 с.: ил.  ISBN 5-85907-327-5.
%
%\item 
%Информация по радиаторам. Документ 0908-0107-13 (RU), Выпуск 4-11-2011, Cummins Power Generation, 2011, \url{cumminspower.com} (дата обращения: 23.03.2020).
%
\item 
Справочник физических свойств материалов //  www.matweb.com   by MatWeb, LLC
%\item
%Суворов\,Ю.\,Б. \emph{Судебная дорожно-транспортная экспертиза} // М.\,«Экзамен»
%
%\item
%Чалкин\,А.\,В.  \emph{Осмотр, фиксация и моделирование механизма образования внешних повреждений автомобилей с использованием их масштабных изображений} / А.\,В. Чалкин, А.\,Л. Пушнов, В.\,В. Чубченко // Учебное пособие -- М.:ВНКЦ МВД СССР 1991г.
%
%\item Шпатлевки полиэфирные «Novol». Технические карты. \url{http://professional.novol.pl/ru/}
%\item
%\emph{Сервис по автоматической расшифровке VIN номеров – AudaVIN} // Audatex
%
%\item \emph{Методика окраски и расчета стоимости лакокрасочных материалов для проведения окраски транспортных средств   AZT}// http://azt-automotive.com,   http://www.schwacke.ru/\-ereonline.htm

%\item
%\emph{Информационно-справочная система по ремонту и сервисному обслуживанию автомобилей Autodata}// Autodata online, лицензионный код GA0288799, Autodata Group Ltd, Великобритания. Дистрибьютор в России ЗАО «Легион-Автодата», www.autodata.ru, www.autodata-online.ru
%
%\item %\emph{Специализированное программное обеспечение для расчета стоимости  восстановительного ремонта, содержащее нормативы трудоемкости работ, регламентируемые изготовителями транспортного средства}//   AudaPadWeb, лицензионное соглашение № AS/APW-658  RU-P-409-409435

%\item
%\emph{Электронный каталог деталей автомобилей}// www.elcats.ru
%
%\item \emph{Электронный каталог деталей автомобилей}// \url{http://zavod-nn.ru/}
%\item
%\emph{Электронный каталог деталей автомобилей}//http://lada-original.ru/catalog/
%%
%\item  \emph{Электронная база данных РСА  средней стоимости запасных частей и нормо-часа работ} //http://prices.autoins.ru/spares/
%
%\item  
%Руководство по эксплуатации и техническому обслуживанию. FG(D)10 - 18-20, FG(D)20 - 35-16 ВИЛОЧНЫЙ АВТОПОГРУЗЧИК. 
%
%\item 
%Заводская инструкция «NISSAN» К15, К21, К25 бензиновый двигатель. KOMATSU FORK-LIFT».
%
%\item 
%Диагностика неисправностей прокладок головки блока цилиндров двигателей внутреннего сгорания и методы их устранения. Сервисное руководство. Federal Mogul,  \url{fmcampus.eu}.
%
%\item 
%Прокладки ГБЦ. ElringKlinger AG.
%
%\item 
%Уплотнения двигателя. Power Technologies Group, 2016.
%
%\item 
%Gasket Kit Failures. Technical bulletin. KB-15007, KMP Brand.
%\item
%\emph{Цены оригинальных деталей на автомобили Мерседес Бенц}https://partsprice.mercedes-benz.ru/?owda=misc
%%
\item 
Ssang Yong New Action, Korando с 2011 года. Руководство по ремонту и техническому обслуживанию в фотографиях. 
\item 
Руководство по эксплуатации и техническому обслуживанию автомобилей Ssang Yong Actyon. // EXPORT SERVICE TEAM, SSANGYONG MOTOR CO., LTD. 2013

\item Интернет-ресурсы:\\
%\url{https://www.warehouseiq.com/nissan-forklift-service-manuals-by-model-number} (дата обращения: 23.03.2020)\\
%\url{https://www.victorreinz.ru}  (дата обращения: 23.03.2020)\\
%\url{https://www.elring.ru/ru/produkcija/prokladki-gbc/} (дата обращения: 23.03.2020)\\
%\url{https://av-gk.ru/spare/komatsu/}(дата обращения: 23.03.2020)\\
%\url{autostels.ru}\\
%\url{autokorea22.ru}\\
%\url{chevrolet-daewoo-parts.ru}
%%%%
%                    НОВОЕ,
%  
%1. Автоматические коробки передач и раздаточные коробки. Диагностика и ремонт./Джек Гордон. – СПб.: Алфамер Паблишинг, 2004. – 392с.
%
%2. Устройство, обслуживание, диагностика и ремонт автоматических трансмиссий. Учебное пособие. Руководство №179. – СПб.: Издательство "РОКО", 2006. – 332с.: с ил. – (Серия «Арус»).
%
%3. Automatic Transmission and Transaxles by Tom Birch, Chuck Rockwood Prentice Hall, 576 pages 2nd edition (August 2, 2001)
%
%4. Автоматические коробки передач./ С. А. Харитонов. – М.: ООО «Издательство Астрель» : ООО «Издательство АСТ», 2003. – 335с.: с ил.
%
%5. Автоматические коробки передач: руководство по ремонту и техническому обслуживанию. – М.: «Технобук», 2000. – 224с.: с ил.
%
%6. Workshop Manual – Transmission, Aisin AW Corporation, 216 pages, 2006.
%
%7. Проектирование трансмиссий автомобилей: Справочник / Под общ. ред. А. И. Гришкевича. – М.: Машиностроение, 1984, - 272 с., ил.
%
%
%9. Федеральный закон "О защите прав потребителей" N 234-ФЗ.
%
%10. Автомобильные гидротрансформаторы. / С. М. Трусов. – М.: Машиностроение, 1977, 272 с.
%
%11. Машиностроительная гидравлика./ Т. М. Башта. – М.: Машиностроение, 1971, 672 с.
%
%12. Автомобильный справочник./ Б. С. Васильев, М. С. Высоцкий, К. Л. Гаврилов и др. Под общ. ред. В. М. Приходько. – М.: ОАО «Машиностроение», 2004, 704 с., ил.
%
%13. Справочник конструктора-машиностроителя: в 3 т. / В. И. Анурьев. Под ред. И. Н. Жестковой. – М.: Машиностроение, 2001.- 920 с., ил.
%
%
%15. Допуски и посадки. Выбор и расчет, указание на чертежах: / В. И. Анухин. – СПб.: Изд-во СПбГТУ, 2001. – 219 с.
%
%

%Бабков В.Ф. Автомобильные дороги. – М.: Транспорт, 1983.
%Бабков В.Ф. Дорожные условия и безопасность движения. – М.: Транспорт, 1993.
%Байэтт Р., Уоттс Р. Расследование дорожно-транспортных происшествий / Пер. с англ. – М.: Транспорт, 1983.
%Белкин Р.С. Курс криминалистики. – М.: Закон и право, 2001.
%Бирюков Б.М.. Интернет-справочник автомобилиста 2001: Автосправка; Маркировка автомобилей; Автомобильные каталоги и базы данных; Продажа и покупка автомобилей; Запчасти к автомобилям; Тюнинг, ремонт и ТО автомоб. – М.:ТИД КОНТИНЕНТ-Пресс, 2001.
%Боровских Ю.И. Техническое обслуживание и ремонт автомобилей. – М.: Высшая школа, 1988.
%Винберг А.И., Мирский Д.Я., Ростов М.Н. Гносеологический, информационный и процессуальный аспекты учения об объекте судебной экспертизы // Вопросы теории и практики судебной экспертизы: Сб. науч. тр. ВНИИСЭ. – М., 1983.
%Возможности производства судебной экспертизы в государственных судебно-экспертных учреждениях Минюста России – М., Антидор, 2004.
%Гаврилов К.Л. Диагностика электрооборудования автомобилей: Практическое руководство /Константин Гаврилов./ М.:НЦ ЭНАС Изд-во ЗАО, 2001.
%Гражданский процессуальный кодекс Российской Федерации.
%Грановский Г.Л., Поляков В.З., Майлис Н.П. Математическое моделирование в производстве трасологических экспертиз // Моделирование при производстве трасологических экспертиз: Сб. науч. тр. ВНИИСЭ. – М., вып. 49, 1981.
%Грановский Г.Л., Поляков В.З., Майлис Н.П. Математическое моделирование в производстве трасологических экспертиз // Моделирование при производстве трасологических экспертиз: Сб. науч. тр. ВНИИСЭ. – М., вып. 49, 1981.
%Григорян В.Г. Определение наличия (отсутствия) у водителя ТС технической возможности предотвратить наезд на пешехода // Проблемы судебной автотехнической экспертизы. – М. ВНИИСЭ, 1988.
%Григорян В.Г. Применение в экспертной практике параметров торможения автотранспортных средств: Методические рекомендации. – М.: РФЦСЭ, 1995.
%Дефекты автомобильных шин. Каталог. – М.: НИИШП, 2000.
%Дорожная терминология: Справочник / Под ред. М.И. Вейцмана. – М.: Транспорт, 1985.
%Жилинский Г.В., Суворов Ю.Б. Особенности исследования технического состояния транспортных средств, участвовавших в ДТП // Автомобильный транспорт. – 1986. – № 9.
%Жулев В.И. Предупреждение дорожно-транспортных происшествий. – М.: Юридическая литература, 1989.
%Замиховский М.И. и др. Определение характера движения ТС по следам колес: Методическое письмо. – М.: ВНИИСЭ, 1993.
%Замиховский М.И. и др. Экспертное исследование следов на ТС, возникших при ДТП: Метод. письмо. – М.: ВНИИСЭ, 1994.
%Замиховский М.И. Экспертная реконструкция механизма ДТП по его следам: Автореф. канд. дис. – М.: ВНИИСЭ, 1991.
%Замиховский М.И., Самарина Т.М. Комплексное изучение следов в целях идентификации человека, находившегося на месте водителя в момент ДТП // Экспертная техника. – М.: ВНИИСЭ, 1990. – Вып.114.
%Зинин А.М., Майлис Н.П. Судебная экспертиза (учебник для вузов). – М., 2002.
%Идентификация легковых автомобилей. Европа, Азия, Америка. — М., 1999 — 2000.
%Иларионов В.А., Чернов В.И., Дадашев Ф.А. Расчет параметров маневра транспортных средств: Метод. письмо для экспертов. – М.: ВНИИСЭ, 1988.
%Исаев А.А., Иванова Н.Ю., Замиховский М.И. Трасологическое определение механизма наезда ТС на пешехода: Метод реком. – М.: ВННИСЭ, 1990.
%Исследование обстоятельств дорожно-транспортного происшествия», Чава И.И., Москва, 2007 г.
%Каталоги запасных частей автомобилей ЗАЗ, ВАЗ, АЗЛК и Иж, ГАЗ (легковые), ГАЗ (грузовые), УАЗ, ЗиЛ (грузовые), МАЗ, КамАЗ, КРАЗ, «Урал», автобусов РАФ, ЛИАЗ, ПАЗ, тракторов. — М.: 2000 «Прайс - Н».
%Классификация основных методов судебной экспертизы. – М.: ВНИИСЭ, 1982.
%Кодекс Российской Федерации об административных правонарушениях.
%Колдин В.Я., Полевой Н.С. Информационные процессы и структуры в криминалистике. – М.: МГУ, 1985.
%Колеса и шины: Краткий справочник: Выпуск 2/Сост. А.М.Ладыгина./ М.:Катера и Яхты Журнал ЗАО КП, 2003.
%Комментарий к законодательству о судебной экспертизе – М., Норма, 2004.
%Комментарий к Федеральному закону «О государственной судебно-экспертной деятельности в Российской Федерации». – М., Проспект, 2002.
%Корухов Ю.Г. Криминалистическая диагностика при расследовании преступлений. – М.: Норма, 1998.
%Корухов Ю.Г. Трасологическая диагностика: Метод пособ. – М.: ВНИИСЭ, 1983.
%Косенков А.А. Диагностика неисправностей автоматических коробок передач и трансмиссий – Ростов-на-Дону: Изд-во ЗАО НЦ ЭНАС, 2003.
%Косенков А.А. Устройство тормозных систем иномарок и отечественных автомобилей – Ростов-на-Дону: Изд-во ООО Аист Принт, 2003.
%Кренцель Р.Б. Применение в экспертной практике экспериментально-расчетных значений параметров легкости рулевого управления автомобилей: Метод. реком. – М.: ВНИИСЭ, 1993.
%Курзуков Н.И. Аккумуляторные батареи: Краткий справочник/Н.И.Курзуков, В.М.Ягнятинский./ 2003.
%Майлис Н.П. Судебная трасология. М.: Право и закон, 2003.
%Методические рекомендации по применению нормативных документов (актов) в автотехнической экспертизе. РФЦСЭ, 2004.
%Методические рекомендации: Установление признаков дифференциации и характеристик покрытий автомобильных дорог на месте дорожно-транспортных происшествий. – М.: РФЦСЭ при Минюсте России, 1995.
%Методическое руководство Определение стоимости, затрат на восстановление и утраты товарной стоимости автотранспортных средств. - СПб, СЗРЦСЭ Минюста России, РФЦСЭ при Минюсте России, 2001.
%Назначение и производство судебных экспертиз: Пособ. для следователей, судей и экспертов. – М.: Юридическая литература, 1988
%Назначение и производство судебных экспертиз: Пособие для следователей. /Под ред. Аринушкина Г.П., Шляхова А.Р. – М.: Юридическая литература, 1988.
%Немчинов М.В. Сцепные качества дорожных покрытий и безопасность движения автомобиля. – М.: Транспорт, 1985.
%Орлов Ю.К. Судебная экспертиза как средство доказывания в уголовном судопроизводстве – М., ИПК РФЦСЭ, 2005.
%Основные положения по допуску транспортных средств к эксплуатации и обязанности должностных лиц по обеспечению безопасности дорожного движения. – М.: Транспорт, 1995.
%Руководства изготовителей по эксплуатации транспортных средств.
%Сильянов В.В. Транспортно-эксплуатационные качества автомобильных дорог. – М.: Транспорт, 1984.
%Синельников Р.А., Лосавио С.К. Ремонт аварийных кузовов легковых автомобилей отечественного и иностранного производства. – М., Транспорт. 2001
%Сова Ф.П. Определение типов и моделей автотранспортных средств по следам шин: Учеб. пособ. – М.: ВШ МВД СССР, 1973.
%Современные возможности судебных экспертиз. Под ред. Ю.Г. Корухова – М., 2000.
%Солохин А.А. Судебно-медицинская экспертиза в случаях автомобильной травмы. – М., 1968.
%Соснин Д.А. Автотроника: Электрооборудование и системы бортовой автоматики современных легковых автомобилей: Учебное пособие - специалисту по ремонту и владельцам автомобилей/Дмитрий Соснин./ 2001.
%Справочник по месторасположению идентификационных номеров на легковых автомобилях. — М., 1996. — Вып. 2.
%Строительная, дорожная и специальная техника: Краткий справочник. А.А. Глазов и др. — М., 1998.
%Суворов Ю.Б. Анализ влияния эксплуатационных факторов системы ВАД для экспертного исследования причин ДТП // Теоретические и методические вопросы судебной экспертизы: Сб. науч. тр. ВНИИСЭ. – М., 1988.
%Суворов Ю.Б. Анализ влияния эксплуатационных факторов системы ВАД для экспертного исследования причин ДТП В сб. Теоретические и методические вопросы судебной экспертизы. – М.: ВНИИСЭ, 1988.
%Суворов Ю.Б. и др. Диагностическое исследование элементов автомобильных дорог, влияющих на безопасность дорожного движения (дорожных условий), на участках ДТП: Метод. пособ. для экспертов, следователей и судей. – М.: ВНИИСЭ, 1990.
%Суворов Ю.Б. Интерпретация и использование выводов экспертов по результатам производства экспертиз водителя и дороги. В сб. Вопросы теории и практики судебной экспертизы. – М.: ВНИИСЭ, 1990.
%Суворов Ю.Б. Комплексное экспертное исследование причин ДТП. Учет факторов системы ВАД при установлении непосредственных причин ДТП экспертом // Экспертная практика и новые методы исследования. – М.: ВНИИСЭ, 1993. – Вып. 9.
%Суворов Ю.Б. Новые виды, состояние и перспективы развития САТЭ // Проблемы судебной автотехнической экспертизы: Сб. науч. тр. ВНИИСЭ. – М., 1988.
%Суворов Ю.Б. Судебная дорожно-транспортная экспертиза. Судебно-экспертная оценка действий водителей и других лиц, ответственных за обеспечение безопасности дорожного движения на участках ДТП. Учебное пособие. – М.: Экзамен, 2003.
%Суворов Ю.Б. Судебная дорожно-транспортная экспертиза. Технико-юридический анализ причин дорожно-транспортных происшествий и причинно-действующих факторов. Учебное пособие. – М.: Приор, 1998.
%Суворов Ю.Б. Установление признаков дифференциации покрытий и характеристик автомобильных дорог на месте дорожно-транспортного происшествия: Метод. реком. – М.: РФЦСЭ, 1995.
%Суворов Ю.Б., Кочнев В.А. Методика и результаты экспериментального исследования влияния неравномерности снижения сцепных свойств дороги на параметры движения автомобиля при торможении. Инф. сборник. Экспертная практика и новые методы исследования. – М.: ВНИИСЭ, 1993. вып. 2.
%Суворов Ю.Б., Решетников Б.М., Кочнев В.А. Результаты экспериментального определения коэффициентов сцепления дорожных покрытий. В сб. «Экспертная техника», № 117. – М.: ВНИИСЭ, 1990.
%Судебная автотехническая экспертиза: Методическое пособие для экспертов-автотехников, следователей и судей /Под ред. В.А. Иларионова. – М.: ВНИИСЭ, 1980. – Ч. 2.
%Судебная дорожно-транспортная экспертиза». Экспертное исследование технического состояния дорог, дорожных условий на месте дорожно-транспортного происшествия, Суворов Ю.Б., Панина А.С., Москва, 2007 г.
%Транспортно-трасологическая экспертиза о дорожно-транспортных происшествиях», выпуск 1, методическое руководство для экспертов. Москва 2006 г.
%Транспортно-трасологическая экспертиза о дорожно-транспортных происшествиях», выпуск 2, методическое руководство для экспертов. Москва 2006 г.
%Транспортно-трасологическая экспертиза по делам о дорожно-транспортных происшествиях (Диагностические исследования): Метод. пособ. для экспертов, следователей и судей /Под редакцией Ю.Г. Корухова. – М.: ВНИИСЭ, 1988. – Вып. 1 и 2.
%Уголовно-процессуальный кодекс Российской Федерации.
%Федеральный закон от 31 мая 2001 г. № 73-ФЗ «О государственной судебно-экспертной деятельности в Российской Федерации».
%Фотография и видеозапись для экспертов-автотехников». Ю.Г.Корухов, Москва 2006 г.
%Чубченко А.П. и др. Отпечатки протекторов автотранспортных средств: Учеб. пособ. – М.: ВНИИ МВД СССР, 1987.
%Шляхов А.Р. Судебная экспертиза: организация и проведение. – М.: Юридическая литература, 1979.
%Эксперт // Руководство для экспертов органов внутренних дел и юстиции. Под ред. Т.В. Аверьяновой, В.Ф. Статкуса. – М., 2002.
%%
\end{enumerate}
