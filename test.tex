\begin{table}[h]
	\centering
	\begin{tabular}{lllll}
		\toprule
		\multirow{2}{*}[-1em]{Models} & \multicolumn{3}{c}{Metric 1} & Metric 2\\
		\addlinespace[10pt]
		\cmidrule(lr){2-4} \cmidrule(lr){5-5} \\
		{} & precision & recall & F-score  & R@10 \\
		\midrule
		model 1 & 0.67  & 0.8 & 0.729  & 0.75 \\
		model 2 & 0.8 & 0.9 & 0.847 & 0.85 \\
		\bottomrule
	\end{tabular}
\end{table}





\begin{landscape}
{\tiny	
\csvautolongtable[separator=semicolon]{csv/avto.csv}
}
\end{landscape}




%



\begin{tabular}{|P{50mm}|P{50mm}|}\hline%
	\bfseries Персоны  & \bfseries Индекс%
	\csvreader[head to column names]{test.csv}{}%
%	
	{\\\givenname\  \name & \matriculation \,  \matriculation}%
	

%	
	\\\hline
%	
\end{tabular}

Или


\begin{longtable}{|p{10mm}|p{55mm}|p{40mm}|}\hline%
п/п	& Заголовок любой & Любой\\\hline\hline
	%

	\csvreader[%separator=semicolon, 
	late after line=\\\hline]%
	%
	{test.csv}{name=\name,givenname=\firstname,matriculation=\matnumber}%
	%
	{\thecsvrow & \firstname~\name & \matnumber}%

\end{longtable}



Еще более удобный и предпочтительный способ создания таблицы - это установка соответствующих клавиш параметров. Обратите внимание, что это дает вам возможность создать стиль pgfkeys, который содержит создание всей таблицы.

\csvreader[tabular=|r|l|c|,
table head=\hline & Таже таблица & но записана по другму\\\hline\hline,
late after line=\\\hline]%
%
{test.csv}{name=\name,givenname=\firstname,matriculation=\matnumber}%
%
{\thecsvrow & \firstname~\name & \matnumber}%


Еще один способ

\csvstyle{myTableStyle}{tabular=|r|l|c|,
	table head=\hline & Person & Matr.~No.\\\hline\hline,
	late after line=\\\hline,
	head to column names}

\csvreader[myTableStyle]{test.csv}{}%
{\thecsvrow & \givenname~\name & \matriculation}%




Другой способ адресации столбцов - использовать их римские номера. Прямая адресация осуществляется с помощью \ csvcoli, \ csvcolii, \ csvcoliii,. , , :

\csvreader[tabular=|r|l|c|,
table head=\hline & Person & Matr.~No.\\\hline\hline,
late after line=\\\hline]%
{test.csv}{}%
{\thecsvrow & \csvcolii~\csvcoli & \csvcoliii}%


И еще один способ присвоения макросов столбцам - использовать арабские числа для назначения:

\csvreader[tabular=|r|l|c|,
table head=\hline & Person & Matr.~No.\\\hline\hline,
late after line=\\\hline]%
{test.csv}{1=\name,2=\firstname,3=\matnumber}%
{\thecsvrow & \firstname~\name & \matnumber}%

Для повторяющихся приложений синтаксис pgfkeys позволяет создавать собственные стили для согласованного и централизованного дизайна. Следующий пример легко модифицируется для получения более или менее настроек параметров.

\csvset{myStudentList/.style={%
		tabular=|r|l|c|,
		table head=\hline & Person & #1\\\hline\hline,
		late after line=\\\hline,
		column names={name=\name,givenname=\firstname}
}}
\csvreader[myStudentList={Matr.~No.}]{test.csv}{matriculation=\matnumber}%
{\thecsvrow & \firstname~\name & \matnumber}%
\hfill%
\csvreader[myStudentList={Grade}]{test.csv}{grade=\grade}%
{\thecsvrow & \firstname~\name & \grade}%


Кроме того, имена столбцов могут быть установлены с помощью \ csvnames → стр. 11, а определения стилей - с помощью \ csvstyle → стр. 11. При этом последний пример переписывается следующим образом:

\csvnames{myNames}{1=\name,2=\firstname,3=\matnumber,5=\grade}
\csvstyle{myStudentList}{tabular=|r|l|c|,
	table head=\hline & Person & #1\\\hline\hline,
	late after line=\\\hline, myNames}
\csvreader[myStudentList={Matr.~No.}]{test.csv}{}%
{\thecsvrow & \firstname~\name & \matnumber}%
\hfill%
\csvreader[myStudentList={Grade}]{test.csv}{}%
{\thecsvrow & \firstname~\name & \grade}%

Строки данных файла CSV также могут быть отфильтрованы. В следующем примере сертификат печатается только для учащихся с оценкой, не равной 5,0.

\csvreader[filter not strcmp={\grade}{5.0}]%
{test.csv}{1=\name,2=\firstname,3=\matnumber,4=\gender,5=\grade}%
{\begin{center}\Large\bfseries Certificate in Mathematics\end{center}
	\large\ifcsvstrcmp{\gender}{f}{Ms.}{Mr.}
	\firstname~\name, matriculation number \matnumber, has passed the test
	in mathematics with grade \grade.\par\ldots\par
}%

