%% !TEX TS-program = xelatex
%
%%%%%%%%%%%%%%%%%%%%%%%%%%%%
\documentclass[a4paper,12pt,english,notitlepage,twoside]{report}
%
\usepackage{style/stylefile}  % СТИЛЕВОЙ ФАЙЛ
%
\begin{document}
%	
%\pagestyle{plain}
\pagestyle{fancy}
\fancyhf{}
\fancyhead{}
\fancyfoot{} 

%\fancyhead[LE,RO]{\thepage} номер страницы слева сверху на четных и справа на нечетных
%\fancyhead[CO]{текст-центр-нечетные}%текст-центр-нечетные
\fancyhead[LE]{ } %текст-слева-нечетные
\fancyhead{}
%\fancyhead[CE]{текст-центр-четные} %текст-центр-четные
\fancyhead[LO]{\small \NomerDoc} %текст-справа-четные
\fancyhead[RE]{\small \NomerDoc} %текст-справа-четные

\fancyfoot[CE]{\thepage}% номер страницы слева снизу на четных и справа на нечетных
\fancyfoot[CO]{\thepage}
%\fancyfoot[RO]{}%текст-центр-нечетные
%\fancyfoot[LO]{текст2-слева-четные} %текст2-слева-нечетные
%\fancyfoot[CE]{\thepage} %текст-центр-четные
%\fancyfoot[CO]{\thepage} %текст-справа-четные

\renewcommand{\headrulewidth}{0.1 mm}% толщина отделяющей полоски сверху
\renewcommand{\footrulewidth}{0.1 mm}% толщина отделяющей полоски снизу

%L, C, R — лево, центр и право.
%E, O — четные и нечетные страницы.



\thispagestyle{empty}  %  Не нумеровать первую страницу
%
%%%%%%%%%%%%%%%%%%%%%%%%%%%%%%%%%%%%%%%%%%%%%%%%
%
%    Титульные данные
%
%%%%%%%%%%%%%%%%%%%%%%%%%%%%%%%%%%%%%%%%%%%%%%%%
\newcommand{\NomerDoc}{27-03/2020}  % номер заключения
%
\newcommand{\dog}{25.02.2020} % Дата договора
%
\newcommand{\datastart}{25.02.2020} % Дата начала исследования
%	
%\newcommand{\datadtp}{\ldots}  % Дата ДТП        
%         
\newcommand{\osm}{25.02.2020} % ОСМОТР % Дата осмотра
%
\newcommand{\dataend}{25.03.2020} % Дата окончания
%
%\newcommand{\datastop}{\ldots}  % Дата приостановления
%\newcommand{\datarestart}{\ldots}
%
%%%%%%%%%%%   ДОКУМЕНТЫ  
%  
% Свидетельство о регистрации ТС
\newcommand{\свид}{ВК № 384784 }     
% Паспорт транспорного средства
\newcommand{\птс}{ТС № 001742 }     
\newcommand{\владелец}{\ldots}
\newcommand{\адресвладельца}{\ldots}
\newcommand{\заказчик}{ООО «Логистик-Юг», в лице генерального 
    директора Тарасова Андрея Ивановича}
\newcommand{\адресзаказчика}{г. Краснодар, ул. Стасова, д.178/1, офис 81}
% Страховой полис
\newcommand{\polis}{\ldots}  
%  Протокол
\newcommand{\prt}{\ldots}                        
 % Постановление
\newcommand{\постановление}{\ldots}
 % Определение
\newcommand{\определение}{\ldots}
% Повреждения
\newcommand{\pov}{\ldots} %Перечень повреждений  
% Присутствовали   
\newcommand{\присутствовали}{\ldots}
% Место осмотра
\newcommand{\местоосмотра}{г. Краснодар, хутор Ленина, МФТ-1, отделение 4, литер Г- 46}
%
%%%%%%%%%%%  ТРАНСПОРТНОЕ СРЕДСТВО
%
\newcommand{\tc}{KOMATSU FG15T-20}   % Транспортное сердство
\newcommand{\grz}{\ldots} % Регистрационный знак
\newcommand{\vin}{\ldots}  % VIN
%
\newcommand{\типкузова}{\ldots}
\newcommand{\двигатель}{К15-019305Х} 
%
\newcommand{\colr}{\ldots}  % Какого  цвета кузов
\newcommand{\цвет}{\ldots}  
\newcommand{\типлкп}{\ldots} 
%
\newcommand{\пробег}{\ldots}
\newcommand{\год}{2007}          % Год выпуска
\newcommand{\началоэкспл}{\ldots}
\newcommand{\датаизготовления}{\ldots}


%%%%%%%%% Другие транспортные средства

%
%\newcommand{\tca}{\ldots}
%\newcommand{\тса}{\ldots}
%%
\newcommand{\tcb}{\ldots}
%\newcommand{\тсв}{\cdod}



 
%
%%%%%%%%%%%%%%%%%%%%%                   % Если судебка 
%
\newcommand{\delosud}{гражданское дело 2-1033/19} % 

\newcommand{\delonum}{2-1033/19}  % номер дела

\newcommand{\opr}{дополнительной судебной автотехнической экспертизы}

\newcommand{\sud}{мирового судьи судебного участка № 11 г. Белореченска Краснодарского края Мозер Г.Л.}

\newcommand{\dataopr}{14.10.2019}

\newcommand{\isk}{Ильенко Р.В. к Шамояну Р.О. и САО "ВСК" о взыскании страхового возмещения}

%
\newcommand{\hod}{\ldots} % № ходатайство
%\newcommand{\hod2}{№ 3567 от 01.01.2008} % № ходатайство
%%%%%%%%%%%%%%%%%%%%%%%%%
% РЕЦЕНЗИЯ
\newcommand{\чел}{\ldots}
\newcommand{\назакл}{\ldots}

%
% example-image

\renewcommand{\chaptername}{Заключение эксперта}
\renewcommand{\refname}{Список}
\renewcommand{\bibname}{\large {Использованные нормативы и источники информации}}

\renewcommand{\epsilon}{\ensuremath{\varepsilon}}
\renewcommand{\phi}{\ensuremath{\varphi}}
\renewcommand{\kappa}{\ensuremath{\varkappa}}
\renewcommand{\le}{\ensuremath{\leqslant}}
\renewcommand{\leq}{\ensuremath{\leqslant}}
\renewcommand{\ge}{\ensuremath{\geqslant}}
\renewcommand{\geq}{\ensuremath{\geqslant}}
\renewcommand{\emptyset}{\varnothing}

%%%%%%%%%%%%%%%%%%%%%%%%%%%%%%%%%   ПРОИЗВОЛЬНЫЙ СЧЕТЧИК

\newcounter{@nnn}  % задаём имя счёчика 
\setcounter{@nnn}{0}  % устанавливаем его первое значение

\newcommand{\z}[2]{\par\addtocounter{@nnn}{1}  % формируем комманду 
	{\bf \arabic{@nnn}.   Работы по заказ-наряду  #1, произведеные на автомобиле #2:}}


%%%%%%%%%%%%%%%%%%%%%%%  Подсчет строк в таблице
\newcounter{rownum}
\setcounter{rownum}{0}
\newcommand{\Rownum}{\stepcounter{rownum}%
\arabic{rownum}}

%\def\contentsname{Содержание}
%Аннотация  \abstractname
%Часть       \partname
%Глава        \chaptername
%Список литературы  \refname
%Рис.                \figurename
%Таблица           \tablename
%Литература       \bibname

%Предметный указатель  \indexname
%Приложение                \appendixname
%Содержание          \contentsname
%Список иллюстраций \listfigurename
%Список таблиц        \listtablename
%\addto\captionsrussian{\def\refname{Список используемой литературы}}

%%%%%%%%%%%%%%   Размещение изображений
%\textfloatsep — расстояние между флоатс (в верхней или нижней части страницы) и текстом (по умолчанию, около 20pt)
%\floatsep — вертикальное расстояние между двумя флоатс (около 12pt)
%\intextsep — расстояние между флоатс вставленным "прямо здесь" (параметр h) и текстом (около 12pt)
%\abovecaptionskip и \belowcaptionskip — расстояние над и под подписью к флоат
\setcounter{totalnumber}{10}
\setcounter{topnumber}{10}
\renewcommand{\topfraction}{1}
\renewcommand{\textfraction}{0}
%%%%%%  Больше плавающих объектов на страницу
 \setlength{\textfloatsep}{10pt plus 1.0pt minus 2.0pt}
 \setlength{\floatsep}{5pt plus 1.0pt minus 1.0pt}
 \setlength{\intextsep}{5pt plus 1.0pt minus 1.0pt}

%%%%%%%%%%%%%%%%%%%%%%%%%%%%%%%%%%%%%%%%%%%%%%%%%%%%%%%%%%%%%%%
%
%   Заметка на полях  (ремарка)
%
%%%%%%%%%%%%%%%%%%%%%%%%%%%%%%%%%%%%%%%%%%%%%%%%%%%%%%%%%%%%%%%%
\newcommand{\rem}[1]
{
\marginpar{\scriptsize\textcolor{red}{#1}}
}
\newcommand{\рем}[1]
{
	\marginpar{\scriptsize\textcolor{red}{#1}}
}

%%%%%%%%%%%%%%%%%%%%%%%%%%%%%%%%%%%%%%%%%%%%%%%%%%%% ПЕРЕОПРЕДЕЛЕНИЕ ФОРМАТИРОВАНИЯ ЯЧЕЕК ТАБЛИЦЫ%%%%%%%%%%
%
\newcolumntype{P}[1]{>{\centering\arraybackslash}p{#1}}   %  \centering   \raggedleft  \raggedright
\newcolumntype{M}[1]{>{\raggedright\arraybackslash}m{#1}} %
\newcolumntype{G}[1]{>{\centering\arraybackslash}m{#1}} %

%%%%%%%%%%%% ВСАВКА с масштабированием ИЗОБРАЖЕНИЯ 2х3  В ТАБЛИЦУ
\newcommand{\imt}[1]
{\includegraphics[width=35mm, height=23mm, keepaspectratio=false]{#1}}

%%%% Переопределение команды для
%  Её вызов — \imgh{45.25mm}{zb}{Пример}
%  Первый параметр — ширина
%  Второй параметр — название файла
%  Третий параметр — название подписи к изображению
\newcommand{\imgh}[3]
{
	\begin{figure}[hpt!]
		\center{\includegraphics[width=#1]{foto/#2}}
		\caption{\small {#3}}
		\label{ris:#2}
	\end{figure}
}


\newcommand{\imgroot}[4]
{
	\begin{figure}[hpt!]
		\center{\includegraphics[angle=#4,width=#1]{foto/#2}}
		\caption{\small {#3}}
		\label{ris:#2}
	\end{figure}
}

%%Собственный  простейший список без нумерации и с обычными межстрочными интервалами
\newenvironment{compactlist}{
    \begin{list}{{$\bullet$}}{
            \setlength\partopsep{0pt}
            \setlength\parskip{0pt}
            \setlength\parsep{0pt}
            \setlength\topsep{0pt}
            \setlength\itemsep{0pt}
            \setlength{\itemindent}{\leftmargin}
            \setlength{\leftmargin}{0pt}
        }
    }{
    \end{list}
}
%%%%%%%%%%%%%%%%%%%%%%%%%%%%%%%%%%%%
%%
%% ПЕРЕОПРЕДЕЛЕНИЕ ДЛЯ ЗАПИСИ СТРОКИ АКТА ОСМОТРА
%%

\newcommand{\акт}[4]{\Rownum  & {\small #1}& #2  & #3 & #4\\  \toprule}

%%%%%%%%%%%%%%%
%% Переопределение для ЗАКЛЮЧЕНИЯ. Таблица ввода повреждений  с фото

\newcommand{\пов}[2]{\Rownum  & {\small #1 }&  \imt{foto/#2}\\ \hline \toprule}

%%%%%%%%%%%%%%%%
%%%  Переопределение длятаблицы ИСТОРИИ РЕМОНТА и сервисного обслуживания

\newcommand{\ист}[5]{#1 & #2  & #2 & #4   & #5 \\ \hline}

%%%%%%%%%%%%%%%%%%%%%%%%%%%%%%%%%%%
%%% ПЕРЕОПРЕДЕЛЕНИЕ ДЛЯ ТАБЛИЦЫ с Игдексом и Двумя Столбцами

\newcommand{\два}[2]{\small \Rownum  & {\small #1 }&  \small #2\\ \hline \toprule}

%%% ПЕРЕОПРЕДЕЛЕНИЕ ДЛЯ ТАБЛИЦЫ с Индексом и Пятью Столбцами

\newcommand{\пять}[5]{\small \Rownum  & \small #1 &  \small #2&\small #3&\small #4&\small #5\\ \hline \toprule}

%%%%%%%%%%%%%%%%%%%%%%%%%%%%%%%%%%%
\newcommand{\dee}{
	% вертикальные промежутки:
	\topsep=0pt % вокруг списка
	\parsep=0pt % между абзацами
	\itemsep=0pt % между пунктами % горизонтальные промежутки: \itemindent=0pt % абзацный выступ
	\labelsep=1ex % расстояние до метки
	\leftmargin=\parindent % отступ слева
	\rightmargin=0pt} % отступ справа
%%

%%%%%%%%%%%% Нумерованный список
\newcommand{\be}{\begin{enumerate}}
\newcommand{\en}{\end{enumerate}}

%%%% Вставить цитату
\newcommand{\цитата}[1]
{
	\begin{quote}
		\textcolor{gray}{#1}
	\end{quote}
}

\newcommand{\блеклый}[1]
{\textcolor{gray}{#1}[0.7]}

\newcommand{\сноска}[1]{\footnote{#1}}

\newcommand{\икс}{$x$}
\newcommand{\игрек}{$y$}
\newcommand{\зет}{$z$}
\newcommand{\audaОСАГО}{Audatex AudaWeb, в модуле ОСАГО ПРО}
\newcommand{\auda}{Audatex AudaWeb}

%%%%%%%%%%%%%%%%%%%%%%%%%%% ЧЕК БОКСЫ
\newcommand{\cmark}{\ding{51}}%$\checkmark $
\newcommand{\xmark}{\ding{55}}%
\newcommand{\done}{{$\square$}{\hspace{-6.5pt}\cmark}}
\newcommand{\wontfix}{{$\square$}{\hspace{-6.5pt}\xmark}}
%%%%%%%%%%%%%%%%%%%%%%%%%%%%%%%%%%%%%%%%%%

\newcommand{\г}{$\checkmark $}
\newcommand{\7}{$\checkmark $}
\newcommand{\галка}{\ding{51}}
\newcommand{\х}{\ding{55}}
\newcommand{\градус}{\circ}
\newcommand{\чек}{$\square$}

\newcommand{\чекг}{\done}
\newcommand{\чекх}{\wontfix}


% Площадь пореждений, М2
\newcommand{\s}[1]{$S_{\text{повр}} \approx#1\, m^2$}

\newcommand{\угол}[1]{$ #1^\circ $}
%%%% Стиль для колонтитулов

\newcommand{\грз}{\grz}
\newcommand{\вин}{\vin}
\newcommand{\датадтп}{\datadtp}
\newcommand{\датадоговора}{\dog}
\newcommand{\начато}{\datastart}
\newcommand{\датаосмотра}{\osm}
\newcommand{\датазаключения}{\zkl}
\newcommand{\страховойполис}{\polis}
\newcommand{\протокол}{\pr}
\newcommand{\повреждения}{\pov}
\newcommand{\иск}{\isk}
\newcommand{\тс}{\tc}
\newcommand{\окончено}{\dataend}
\newcommand{\прибл}{$ \approx $}
\newcommand{\тса}{\tca}
\newcommand{\тсб}{\tcb}
\newcommand{\ссылка}{\ref}
%\newcommand{}{}

%%%%%%%%%%%%%%%% ПЕРЕОПРЕДЕЛЕНИЕ  "По вопросу"     \повопросу{вопрос}
\newcommand{\повопросу}[1]{\,{\renewcommand\baselinestretch{0.86}\small\normalsize
\subsection{\underline{По  вопросу}\,\,\textbf{\small{<<#1>>}}}}
\renewcommand\baselinestretch{1.2}\small\normalsize}
%%%%%%%%%%%%%%%%%%%%%%%%%%%%%%%%%%%%%%%%%%%%%


\newcommand{\фото}[2]
{
    \begin{figure}[H]
        \center{\includegraphics[width=0.99\textwidth]{#1}}
        \caption{\small {#2}}
        \label{рис:#1}
    \end{figure}
}



\newcommand{\фотоб}[2]
{
	\begin{figure}[H]
		\center{\includegraphics[width=0.99\textwidth]{#1}}
		\caption*{\small {#2}}
		\label{рис:#1}
	\end{figure}
}



\newcommand{\фот}[2]
{
	\begin{figure}[H]
		\center{\includegraphics[width=0.99\textwidth]{#1}}
		\caption{\small {#2}}
	%	\label{рис:#1}
	\end{figure}
}

%%%%%%%%%%%%%%% ДВА РИСУНКА РЯДОМ            \дварядом{файл1}{подпись1}{файл2}{подпись2}
\newcommand{\дварядом}[4]{\begin{figure}[H]\centering
        \parbox[t]{0.49\textwidth}
        {\centering
            \includegraphics[width=.49\textwidth,  height=.32\textwidth]{foto/#1}
            \caption{\footnotesize {#2}}
            \label{рис:#1}}
        \hfil \hfil
        \parbox[t]{0.49\textwidth}
        {\centering
            \includegraphics[width=.49\textwidth, height=.32\textwidth]{foto/#3}
            \caption{\footnotesize {#4}}
            \label{рис:#3}}

\end{figure}}

%%%%%%%%%%%%%%%%%%%%%%%%%%%%%%%%%%%
% Два рядом с одной общей подписью
%%%%%%%%%%%%%%%%%%%%%%%%%%%%%%%%%%%

\newcommand{\дварисунка}[5]{\begin{figure}[H]
	\begin{minipage}{0.49\textwidth}
		\includegraphics[width=\linewidth,  height=.64\linewidth]{foto/#1}
		\subcaption{#2}
	\end{minipage}
	\hfill
	\begin{minipage}{0.49\textwidth}
		\includegraphics[width=\linewidth,  height=.64\linewidth]{foto/#3}
		\subcaption{#4}
	\end{minipage}

	\caption{#5}
	\label{рис:#1}
:\end{figure}}




%%%% СТС две стороны рядом
\newcommand{\стс}[4]{\begin{figure}[H]
        \centering
        \parbox[t]{0.49\textwidth}
        {\centering
            \includegraphics[width=.49\textwidth]{foto/#1}
            \caption{\footnotesize {#2}}
            \label{рис:#1}}
        \hfil \hfil
        \parbox[t]{0.49\textwidth}
        {\centering
            \includegraphics[width=.49\textwidth]{foto/#3}
            \caption{\footnotesize {#4}}
            \label{рис:#3}}

\end{figure}}


%%%%% ФОТО РЯДОМ С ТЕКСТОМ
%
%\newcommand{\фотосправа}[2]{
%    \begin{SCfigure}
%        \centering {\footnotesize \caption{#2}
%            \includegraphics[width = 0.6 \textwidth]{foto/#1}
%            \label{рис:#1}
%    \end{SCfigure}}



%%%% Переопределение команды для
%  Её вызов — \фотомасштаб{45.25mm}{название файла}{подпись рисунка}

%  Первый параметр — название файла
%  Второй параметр — название подписи к изображению
%  Третий параметр — ширина
\newcommand{\фотомасштаб}[3]
{
    \begin{figure}[H]
        \center{\includegraphics[width=#3]{foto/#1}}
        \caption{\small{#2}}
        \label{рис:#1}
    \end{figure}
}



%  Её вызов — \фотоповорот{45.25mm}{название файла}{подпись рисунка}{угол поворота}
%  Первый параметр — ширина
%  Второй параметр — название файла
%  Третий параметр — название подписи к изображению
\newcommand{\фотоповорот}[4]
{
    \begin{figure}[hpt!]
        \center{\includegraphics[angle=#4,width=#1]{foto/#2}}
        \caption{\small {#3}}
        \label{рис:#2}
    \end{figure}
}

%%% ИЗМЕРИТЬ ШИРИНУ СТРАНИЦЫ
\newcommand{\ширина}{\the\textwidth\\
    \printinunitsof{mm}\prntlen{\textwidth}}

%
%\setcounter{page}{1}
\clubpenalty=10000 
\widowpenalty=10000

%%%%%%%%%%%%%%%%%%%%%%%%%%%%%%%%%%%%%%%%
%      Шапка экспертной организации  
%%%%%%%%%%%%%%%%%%%%%%%%%%%%%%%%%%%%%%%%
%
%%%%%%%%%%%%%%%%%%%%%%%%%%%%%%%%%%%%%%%%%%
%
%   Экспертная организация ООО Южнорегиональная экспертная группа
%
%%%%%%%%%%%%%%%%%%%%%%%%%%%%%%%%%%%%%%%%%
\noindent %\qrcode[height=21mm]{\NomerDoc от \окончено }  %%% Добавлен QR-Code
\begin{pspicture}(21mm,21mm)
\obeylines
\psbarcode{%
	%\NomerDoc от \окончено
	BEGIN:VCARD^^J
	VERSION:4.0^^J
	%N:Мраморнов; Александр; Вчеславович^^J
	FN:Александр Мраморнов^^J
%	ORG:IP Alexandr Mramornov^^J
	TITLE: эксперт
	ORG: ИП
	URL:http://www.yourexp.ru^^J
	EMAIL:4516611@gmail.com^^J
	TEL:+7-918-451-6611^^J
	ADR:г. Краснодар, с/т № 2 А/О «Югтекс», ул. Зеленая, 472^^J
	END:VCARD
}{width=1.0 height=1.0}{qrcode}%
\end{pspicture}
\begin{center}
	\normalsize\textbf{ОБЩЕСТВО С ОГРАНИЧЕННОЙ ОТВЕТСТВЕННОСТЬЮ \\[-1.5mm] <<ЮЖНО-РЕГИОНАЛЬНАЯ\quad ЭКСПЕРТНАЯ\quad ГРУППА>> \\[-5mm]}
	%  
	\noindent\rule{\textwidth}{1pt}\\[-6mm]  % Горизонтальная линия
	% \line(1,0){460}% (1,0) -горизонтальная линия, и (0,1) - вертикальная 
\end{center}

\begin{center}
	\begin{footnotesize}\setstretch{0.3}
		%	\small\textbf\setlength   	%\raisebox{5mm}
		\vspace{-3.5mm}350072, Россия, Краснодарский край, г. Краснодар, Ростовское шоссе, 14/2, оф. 67\\[0mm]
		Телефон: \quad 8-918-451-66-11, e-mail:\quad 4516611@gmail.com\\ [-2mm]{ИНН 2311213020\quad КПП 231101001 ОГРН 1162375014560}
	\end{footnotesize}	\\[10mm]
\end{center}


\begin{flushright}
	Краснодар, 2020    \\[8mm]
\end{flushright}
\begin{center}
	\LARGE\textbf{ ЗАКЛЮЧЕНИЕ ЭКСПЕРТА}
	\bigskip\\[0mm]
	%	{\normnumxtbf{\NomerDoc}}	}{den}
\end{center}
\par
\vspace{-3mm}\noindent по гражданскому делу \delonum \, \isk \\[0mm]

%\raggedright 
%\def\hrf#1{\hbox to#1{\hrulefill}}
\noindent \textbf{№ 22-2019}\hfill           \textbf{\dataend}\\%[2mm]
%Приостановлено\hfill      \datastop\\
%Возобновлено\hfill          \datarestart\\
%Окончено\hfill                \dataend\\%[4mm]

\noindent\parbox[l][16mm]{16.5cm}
{\def\hrf#1{\hbox to#1{\hrulefill}}
	\noindent Начато\hfill            \datastart\\%[2mm]
	%	Приостановлено\hfill      \datastop\\
	%	Возобновлено\hfill          \datarestart\\
	Окончено\hfill                \dataend\\%[4mm]
}
\relax

\datastart г. ~в {\small ООО~ "ЮЖНО-РЕГИОНАЛЬНАЯ ЭКСПЕРТНАЯ ГРУППА"} \,  при определении  \, \sud  \,  от \, \dataopr \, о назначении \opr \, по гражданскому делу \delonum \, поступили:

\begin{enumerate}\setlist{nolistsep}\item  Материалы гражданского дела \delonum \, в двух томах, том 1 на 276 листах, том 2  на 143 листах.\\[-2mm]
	%	\item  
\end{enumerate}Экспертиза произведена экспертом {\small ООО "ЮЖНО-РЕГИОНАЛЬНАЯ ЭКСПЕРТНАЯ ГРУППА"} \,  Мраморновым Александром Вячеславовичем, имеющим высшее техническое образование по специальности «техническая физика», диплом РВ №311964 от 28.02.1989, квалификация -- инженер-физик, специальное образование в области оценки: Диплом ПП-1 № 037211 Российской экономической академии им. Г.В. Плеханова, квалификация -- оценка и экспертиза объектов и прав собственности, специальное образование в области независимой технической экспертизы транспортных средств: Диплом ПП-I № 424167, квалификация: эксперт-техник (специализация 150210 специальности 190601.65 – Автомобили и автомобильное хозяйство), состоящий в Государственном реестре экспертов-техников (№ в реестре 256, https://data.gov.ru/opendata/7707211418-experts,  общий трудовой  стаж 30 лет, стаж  экспертной работы  12 лет.  % Шапка организации ООО ЮРЕКСГРУП
%
%%   вопросы экспертизы
\subsection{Вопросы экспертизы}
Заказчик поручает, а Исполнитель принимает на себя обязательство выполнить Заказчику  комплекс работ в виде автотехнических исследований автомобиля Mazda 6, VIN RUMGJ52\-6802007133 (дата начала гарантии 07.05.2018 г.), по следующим вопросам:
\begin{enumerate}\item  <<Что послужило причиной выхода двигателя автомобиля из строя?>>	
	\item  <<Является ли данная причина:\begin{itemize}
		\item производственной, т.е. недостатком сборки и/или материала;
	\item связанной с некачественным/несвоевременным обслуживанием автомобиля, включая ежедневный осмотр;
\item связанной с неразрешенными/недопустимыми переделками агрегата и/или его систем;
\item связанной с предыдущим ремонтом (если применимо);
\item эксплуатационной, т.е. возникшей по причине неправильной/ненормальной эксплуатации;
\item  естественным износом в соответствии с пробегом автомобиля?>> 
	\end{itemize}
%	
\end{enumerate}
\subsection{Для производства исследования представлено} %Название по шаблону минюста
\begin{enumerate}
	\item Акт проверки качества от 03 сентября 2019г. транспортного средства Mazda 6 VIN: RUMGJ526802007133 
	\item  Заказ-наряд 480271247- от 23.01.2019
	\item Отчет по диагностике (ОД) к предварительному запросу на гарантийный ремонт. Код дилера 00420
	\end{enumerate}
%
%
%\vspace{-275mm}
\addcontentsline{toc}{section}{Использованные нормативы и источники информации}
%
%\left( \addcontentsline{toc}{section}{Использованные нормативы и источники информации}

\subsection{Использованные нормативы и источники информации}
%
\begin{enumerate}
\item 
Махнин\,Е.\,Л., Новоселецкий\, И.\,Н., Федотов\, С.\,В. \emph{Методические рекомендации по проведению судебных автотехнических экспертиз и исследований колёсных транспортных средств в целях определения размера ущерба, стоимости восстановительного ремонта и оценки} // -- М.: ФБУ РФЦСЭ при Минюсте России, 2018.-326 с.  ISBN 978-5-91133-185-6.
%
%
%
%
\item ТУ 017207-255-00232934-2014 \emph{Кузова автомобилей LADA. Технические требования при приёмке в ремонт, ремонте и выпуске из ремонта предприятиями дилерской сети ОАО "АВТОВАЗ"}//  ОАО НВП "ИТЦ АВТО", 2014
%
\item Смирнов  В.Л., Прохоров  Ю.С., Боюр В.С.  и др. \emph{Автомобили ВАЗ. Кузова. Технология ремонта, окраски и  антикоррозионной защиты. Часть II}// - Н.Новгород: АТИС, 2001.- 241с.
%
\item 
Савич Е.Л. \emph{Техническое  обслуживание  и  ремонт  легковых  автомобилей} : учеб. пособие / Е.Л. Савич, М.М. Болбас, В.К. Ярошевич ; под общ. ред. Е.Л. Савича. -Мн. : Вышэйшая школа,  2001. - 479 с. - ISBN985-06-0502-2.
%
\item 
Автомобили ВАЗ-2121, 21213, 21214, 2131 и их модификации: <<Трудоемкости работ (услуг) по техническому обслуживанию и ремонту>> /Куликов А.В., Христов П.Н., Климов В.Е.,  Боюр В.С., Рева В.В., Зимин В.А., Завьялова Н.Н., Хлыненкова Г.А. -- ИТЦТ "АвтоВАЗтехобслуживание", Тольяти -- 2005. 
%
\item
Автомобили LADA SAMARA и их модификации: <<Трудоемкости работ (услуг) по техническому обслуживанию и ремонту>> /Куликов А.В., Христов П.Н., Климов В.Е., Рева В.В., Боюр В.С., Васильев М.В., Фахрутдинов Р.В.,  Прудских Д.А., Гирко В.Б., Шмелева В.А., Зимин В.А. --  ОАО НВП "ИТЦ АВТО",  -- 2006. - 252 стр.
%
\item 
Автомобили LADA PRIORA. Трудоемкости работ (услуг) по техническому обслуживанию и ремонту /Куликов А.В., Христов П.Н., Климов В.Е., Рева В.В., Козлов П.Л., Боюр В.С., Прудских Д.А., Шмелева В.А., Зимин В.А. -- ООО "ИТЦТ АВОСФЕРА", Тольяти -- 2009. -- 344 с.
%
\item 
{Трудоемкости работ по техническому обслуживанию и ремонту автомобилей автомобилей Lada  Granta}/   \url{https://docplayer.ru/30250248-Trudoemkosti-rabot-po-teh\-nicheskomu-obsluzhivaniyu-i-remontu-avtomobiley-lada- granta.html}.
%
%
\item
{Специализированное программное обеспечение для расчёта стоимости  восстановительного ремонта, содержащее нормативы трудоёмкости работ, регламентируемые изготовителями транспортного средства}//   AudaPadWeb, лицензионное соглашение № AS/APW-658  RU-P-409-409435.
%
%
%
\item
{Специализированное программное обеспечение для расчёта стоимости  восстановительного ремонта, содержащее нормативы трудоёмкости работ, регламентируемые изготовителями транспортного средства ОАО «АвтоВАЗ», ЗАО «Джи-Эм-АвтоВАЗ», ОАО «СеАЗ» и ОАО «ЗМА»}//   Автосфера АС:Смета, v.3.9.11// ООО "ИТЦ «ИнтегроМаш», \url{https://autosmeta.pro}.
%
%
%
\item Информационный портал по техническому обслуживанию и ремонту автомобилей	 ВАЗ:\\ \url{www.autosphere.ru}.

%%
\end{enumerate}

%\bibliographystyle{utf8gost705u}  %% стилевой файл для оформления по ГОСТу
%\bibliography{biblio}     %% имя библиографической базы
%%%%%%%%%%%%%%%%%%%%%%%%%%%%%%%%%%%%%%%%%%%%%%%%%%%%%%%%%%%%%%%%%%%%%%%%%%%%%%%%%
\subsection{Технические средства}  %% Список не удалять!!!
\begin{itemize}
%
%%
%%\item   Диагностический сканер SDconnect   с программным обеспечением Xentry Diagnostics v19.11.3.1
%
\item   Линейка масштабная магнитная с цветографической шкалой, 100мм
%
%%\item   Рулетка измерительная металлическая, 5м
%%\item  Универсальный стенд для измерения углов установки колес Hunter Engineering %ProAlign с программным инструментом регулировки схождения колес без блокировки руля %автомобиля WinToe
\item 	Цифровой фотоаппарат Canon 760D s/n 143032001327 с объективом Canon EF-S 18-135, тип используемой памяти: Transcend,  32Gb
%
%\item  Специализированное программное обеспечение для расчёта стоимости  восстановительного ремонта, содержащее нормативы трудоёмкости работ, регламентируемые изготовителями транспортного средства     AudaPadWeb, лицензионное соглашение № AS/\- APW-658  RU-P-409-409435
%
\item  Программа обработки фото-видео изображений ImageJ, разработчик  Wayne Rasband (wa-yne@codon.nih.gov),
свободная лицензия GPL
%
\item  ПЭВМ под управлением операционной системы Windows 10 с установленным набором макрорасширений LaTeX системы компьютерной вёрстки TeX, cвободная лицензия LaTeX Project Public License (LPPL)
%	
\end{itemize}
%%%%%%%%%%%%%%%%%%%%%%%%%%%%%%%%%%%%%%%%%%%%%%%%%%%%%%%%%%%%%%%%%%%%%%%%%%%%%%%%%%%%%%%%%%%%%%%%%%%%%%
\subsection{Условные обозначения}
\begin{description}
%	 
%%\item[АВС] --Антиблокировочная система
\item[АМТС] --Автомототранспортное средство
\item[ДВС] --Двигатель внутреннего сгорания
\item[ДТП] --Дорожно--транспортное происшествие
\item[гос.\,рег.\,знак] --Государственный регистрационный знак
\item[КТС] --Колесно транспортное средство 
\item[ЛКП] --Лакокраочное покрытие
%\item[л.д.] --Лист дела
%%\item[Колесо турбины]  -- крыльчатка турбины
\item[ТС] --Транспортное средство
%\item[ТK, ТКР] -- Турбокомпрессор. Состоит из двух частей: турбины и компрессора, объединенных общим валом. Вал вращается в подшипниках, размещенных в центральном корпусе ТК
\item[ЭБУ] --Электронный блок управления
%%\item[FRAME] "--*Номер кузова транспортного средства, выпущенного для продажи на внутреннем рынке Японии и содержащий информацию производителя о транспортном средстве
\item[DTC] --Diagnostic Trouble Codes, диагностические коды неисправностей
\item[OBDII] -- On-board diagnostics. Протокол бортовой диагностики автомобиля
%%\item[SRS] -- Cистема пассивной защиты водителя и пассажиров
\item[VIN] --vehicle identification number, 17--значный идентификационный номер транспортного средства, соответствующий стандарту ISO 3779--2012.
%
\end{description}
\subsection{Методы исследования}
\begin{itemize}
\item  Органолептический метод – исследование и оценка качества объектов с помощью %органов чувств
\item 	Прямой измерительный метод – путем измерения размеров деталей специальными %измерительными приборами
\item Расчётный метод (косвенный измерительный метод) – путём расчётов различных параметров на основе результатов измерений и других данных
\item Экспертный метод (метод экспертной оценки) — совокупности операций по выбору комплекса или единичных характеристик объекта, определению их действительных значений и оценкой экспертом соответствия их установленным требованиям и/или технической информации
%%	\item Метод натурной реконструкции??
\end{itemize}
\subsection{Исходные данные}
Транспортное средство \тс в 
\begin{enumerate}
	\item[-]	
%	
%	
\end{enumerate}
%\subsection{Ранее по материалам дела выполнено}
%\subparagraph*{} Определением Советского районного суда   27.03.2018   № 2-2459/18 экспертам ИП Купринова Виктора Александровича было поручено производство судебной автотехнической экспертизы. \\
%\indent На разрешение экспертизы поставлены вопросы:
%%	
%\begin{enumerate}\setlist{nolistsep}\item  Возможно ли достоверно идентифицировать принадлежность систем и агрегатов, предоставленных к осмотру эксперту, именно двигателю № 8J6092, автомобиля \тс, регистрационный знак У 393 ОТ 123 с учетом их демонтажа и  разбора, произведенных до проведения осмотра. Если да, то какие повреждения имеются на двигателе № 8J16092, автомобиля \тс, и являются ли они причиной выхода его из строя?\item Какова причина образования установленных повреждений двигателя № 8J6092, \тс, регистрационный знак У 393 ОТ 123,  и могли ли они образоваться в результате естественного износа, неквалифицированного проведения ремонтных работ или технического обслуживания, а так же не предусмотренного заводом-изготовителем изменений конструкции двигателя и его систем?\item Находятся ли повреждения двигателя № 8J6092, автомобиля \тс, регистрационный знак У 393 ОТ 123 в прямой причинно-следственной связи с ремонтом и работами при замене и чистке форсунок, замене ТНВД, проведенными ООО "Регинтехцентр", а так же указанными в заказ-наряде 0000-000194 от 03.08.2017г.?\item Какова стоимость восстановительного ремонта двигателя № 8J6092, автомобиля \tc, регистрационный знак У 393 ОТ 123?
%%
%	\end{enumerate}
%%
%\vspace{5mm}
%
%06.07.2018г. по результатам исследований  экспертами  ИП Куприянова Виктора Александровича  Вольченко Н.А. и Куприяновым В.А. составлено  "Заключение экспертов" (Т1, л.д. 162-202), содержащее следующие выводы судебной автотехнической экспертизы:
%
%\begin{enumerate}
%\item "Поскольку представленные детали хранились в открытом доступе к ним, а эксперт не присутствовал при демонтаже и разборке двигателя и не может утверждать, что все указанные детали принадлежат исследуемому двигателю IVD 0050201, то достоверно идентифицировать принадлежность систем и агрегатов, представленных к осмотру эксперту, именно двигателю IVD 0050201, автомобиля \тс, регистрационный знак У 393 ОТ 123, не представляется возможным.
%	
%При проверке каталожных номеров, присутствующих на деталях, эксперт подтверждает, что представленные компоненты (детали) двигателя принадлежат автомобилю "TOYOTA LAND CRUISER 200". При этом блок двигателя имеет заводской номер, который соответствует данным ПТС 82ОЕ920989 (паспорт технического средства):п.6 модель, № двигателя - 1VD 0050201. 
%%	
%    Двигатель 1VD 0050201 автомобиля \тс, имеет следующие повреждения:
%	\begin{enumerate}
%		\item[1.] Цилиндр № 6 имеет вертикальную риску (царапину);
%		\item[2.] Повреждена (разрушена) крыльчатка тракта воздушного впуска левого турбокомпрессора.
%		\item[3.] Отложения нагара на стержнях впускных и выпускных клапанов.
%	\end{enumerate}
%	Других дефектов не обнаружено, компоненты двигателя имеют эксплуатационный износ. 
%	
%Двигатель имеет эксплуатационный износ, характерный при пробеге более 100 тыс. км и они не связаны с разрушением  крыльчатки турбокомпрессора. 
% 
% Эксперт отмечает, что при проведении диагностики двигателя на первоначальном этапе обращения можно было избежать полной разборки двигателя безо всяких на то оснований."
%%	\textbf{По второму вопросу} 
%	\item  "Причиной выхода из строя турбокомпрессора (разрушения крыльчатки) является попадание постороннего предмета в тракт воздушного впуска в результате неквалифицированного проведения ремонтных работ, проводимых ранее, и не соответствует повреждениям, образованным в результате попадания болта клапанной крышки распредвала. \\ Повреждения цилиндра № 6 и значительный нагар на стержнях  впускных и выпускных клапанов являются следствием естественного износа, причина возникновения носит эксплуатационный характер. \\ Поскольку в современной литературе, методических указаниях и рекомендациях, разработанных Российским Центром Судебных Экспертиз и утвержденных Министерством Юстиции РФ, а так же других нормативно-технических источниках и документах (ЭКЦ при МВД РФ) отсутствует экспертная методика, позволяющая определить срок давности  возникновения недостатков (дефектов, повреждений, неисправностей и др.), то определить момент возникновения повреждений, зафиксированных в представленном на исследование автомобиле \тс, регистрационный знак У 393 ОТ 123, не представляется возможным. Однако эксперт отмечает, что зафиксированный эксплуатационный износ на компонентах представленных на исследование, носит накопительный характер образования."
%%	\textbf{По третьему вопросу}   
%	\item "Причинно-следственная связь повреждений двигателя, турбокомпрессора автомобиля \тс, регистрационный знак У 393 ОТ 123 с ремонтом и работами при замене и чистке форсунок, замене ТНВД, проведенными ООО "Регионтехцентр", отсутствует."
%%	%\textbf{По четвертому вопросу} 
%	\item "Стоимость устранения выявленных недостатков складывается из стоимости ремонтных воздействий и стоимости запасных частей.   Таким образом, стоимость выявленных в процессе производства экспертизы недостатков двигателя автомобиля марки \тс, регистрационный знак У 393 ОT 123, составляет: \\ 407 806.50 (Четыреста семь тысяч восемьсот шесть рублей 50 коп.) рублей 50 копеек." 
%\end{enumerate}
%%           
\subsection{Обстоятельства дела}
\begin{itemize}
\item 03.09.2019 автомобиль \тс на эвакуаторе был доставлен в сервисный центр. При работе двигателя при увеличении оборотов до 2000 об/мин слышен стук в ДВС. 
%
\end{itemize}
% 
%
%
\section{Исследование}
%
\subsection{История ремонта и сервисного обслуживания \тс}
\begin{itemize}
	\item  Заказ-наряд 480271247- от 23.01.2019
\item 03.09.2019 автомобиль на эвакуаторе доставлен в сервисный центр. Посторонний стук в районе ДВС.  При увеличении оборотов до 2000 об/мин слышен стук в ДВС. При приеме ТС выявлено, что уровень масла ниже минимальной отметки, уровень охлаждающей жидкости на минимальном уровне, сигнализаторы или контрольные лампы на панели приборов не горят. Произвели замену масла, слили 3л масла, цвет масла темный, присутствовал запах бензина. Был выставлен уровень масла до максимального, стук не прошел. При считывании ошибок зафиксирована ошибка Р0524 (слишком низкое давление масла) на пробеге 32 674 км. Выполнили проверку согласно MESI по симптому № 21 <<Шум в двигателе>>. По итогам проверки, так как источник звука находится внутри ДВС, принято решение произвести частичную разборку для определения источника звука. Дополнительно выполнили проверку давления масла: нижний предел при 1500 об/мин - 2.4 бар; при 4500 об/мин - 4.4 бар. \рем[А какое должно быть?]. Проверили компрессию для данного двигателя (степень сатия 14) 1ц -6.5 кг/см2; 2ц-6.5 кг/см2; 3ц-6.5 кг/см2; 4ц-6.0 кг/см2. Выполнили снятие поддона ДВС и нижних головок шатуна. Вкладыш 4-го цилиндра имеет задиры, шатунная шейка коленвала 4го цилиндра имеет задиры, вкладыши 2 и 3 цилиндров имеют задиры. В маслозаборнике присутствуют металлические частицы.  На основании вышеизложенного, специалистами сервисного центра причиной возникновения неисправности названа эксплуатация автомобиля  с уровнем масла ние рекомендованного заводом изготовителем.
\end{itemize}
\subsection{Исследование предоставленных на экспертизу документов}
%
 \subparagraph*{}Из свидетельства о регистрации транспортного средства 23 48 № 257194,  (Т 1, л.д. 8)  экспертом установлена следующая общая информация об автомобиле, имеющая значение для дачи заключения:
	\begin{itemize}
		\item[ ] 
			\begin{description}
			\item[Марка, модель] --Тойота ЛЭНД КРУИЗЕР 200
			\item[VIN] -- \vin
			\item[Год выпуска] --2008
			\item[Шасси] --JTMHV05J004031859
			\item[Цвет ЛКП] --Серебристый
			\item[Двигатель] --172,8/235 кВт/л.с.
			\item[ПТС] --82ОЕ № 920989
%						
		\end{description}
		\end{itemize}
	\subparagraph*{} Идентификационный код автомобиля (VIN)  JTMHV05J004031859 содержит следующую информацию о транспортном средстве, имеющую значение для 	дачи заключения:
%
\begin{itemize}
	\item[ ] 
	    \begin{description}
%		
	\item[Дата изготовления] --10.2008
	\item[Двигатель] --1VDFTV  4500CC DISEL TURBO
	\item [Расположенние руля] --LEFT
	\item[VDS] --HV05J004
%	
		\end{description}
\end{itemize}
%
%
\textit{Источник: https://ru.vindecoder.pl/JTMHV05J004031859}

Пробег автомобиля  расчетный, согласно [1]  составляет 214 000км.
%
  \begin{figure}[!h]
	\centering
	\includegraphics[width=0.65\linewidth]{images/cm1}
	\caption{{\footnotesize {Компоновка \тс. Иллюстрация Audatex}}}
	\label{ris:images/cm1}
\end{figure}
%
\subparagraph*{}
%	
	Согласно заказ-наряду 0000-000194 от 03.08.17, (Т1, л.д. 9) на автомобиле \тс \, в течении одного дня 03.08.2017 специалистами ООО "РЕГИОНТЕХЦЕНТР" выполнены следующие работы:
%	
\begin{table}[H]
			\centering
				\caption{{\footnotesize Выполненные работы и услуги}}
			\label{tab:1}
	\begin{tabular}{|l|l|l|l|}
		\hline
		\rowcolor[HTML]{C0C0C0} 
		\multicolumn{1}{|c|}{\cellcolor[HTML]{C0C0C0}N п/п} & Наименование работ и услуг & Кол-во & Цена, руб \\ \hline
		1                                                   & Диагностика компьютерная   & 1      & 1000      \\ \hline
		\rowcolor[HTML]{EFEFEF} 
		2                                                   & Замер компрессии FSA       & 1      & 1000      \\ \hline
		3                                                   & Снятие/установка форсунок  & 8      & 8000      \\ \hline
		\rowcolor[HTML]{EFEFEF} 
	
	\end{tabular}
\end{table}

Итого на сумму:  37 300 (Тридцать семь тысяч триста) рублей
%
%

\begin{table}[H]
	\centering
	\caption{{\footnotesize Расходы по накладной к заказ-наряду 0000-000194 от 03.08.17}}
	\label{tab:2}
\begin{tabular}{|l|l|l|l|}
			\hline
			\rowcolor[HTML]{C0C0C0} 
			\multicolumn{1}{|c|}{\cellcolor[HTML]{C0C0C0}N п/п} & Наименование запчасти (материала) & Кол-во & Цена, руб \\ \hline
			1                                                   & Шайба под форсунку TOYOTA   & 10      & 2500      \\ \hline
			\rowcolor[HTML]{EFEFEF} 
			2                                                   & Очиститель       & 5     & 1000      \\ \hline
\end{tabular}
\end{table}

Итого на сумму:  3 500 (Три тысячи пятьсот) рублей 

Общая стоимость работ: 40 800 (Сорок тысяч восемьсот) рублей. 
%~\
\vspace{\baselineskip}  % вставка пустой строки

В исковом заявлении (Т1, л.д. 1--6) указано, что после ремонта автомобиля специалистами ООО "РЕГИОНТЕХЦЕНТР"  через непродолжительное время в двигателе автомобиля  \тс \, образовался посторонний, нефункциональный  шум.  Согласно заключения ИП Шаманского С.Н. для устранения неисправности ДВС необходимо было произвести следующие ремонтные работы:

\begin{table}[H]
	\centering
	\caption{{\footnotesize Работы по заказ-наряду № Ш000011955 от 10.08.2017}}
	\label{tab:3}
	\begin{tabular}{|l|l|l|l|}
		\hline
		\rowcolor[HTML]{C0C0C0} 
		\multicolumn{1}{|c|}{\cellcolor[HTML]{C0C0C0}N п/п} & Наименование запчасти (материала) & Кол-во н/ч & Цена, руб \\ \hline
		1    & Материалы, использованные при подготовке а/м к ремонту   & 0,2      & 260      \\ \hline
		\rowcolor[HTML]{EFEFEF} 
		2    & Слесарные работы (ремонт освещения багажного оделения)       & 0,5     & 650    \\ \hline
		3    & Интеркулер снять/установить      & 1     & 1300      \\ \hline
		\rowcolor[HTML]{EFEFEF} 
	
\end{tabular}
\end{table}
Итого на сумму  126 178 (Сто двадцать шесть тысяч сто семьдесят восемь) рублей
% 
 
 \begin{table}[H]
 	\centering
  	\caption{{\footnotesize Запчасти и материалы к заказ-наряду № Ш000011955 от 10.08.2017}}
 	\label{tab:4}
 	\begin{tabular}{|l|ll|l|l|}
 		\hline
 		\rowcolor[HTML]{C0C0C0} 
 		\multicolumn{1}{|c|}{\cellcolor[HTML]{C0C0C0}N кат} & Наименование запчасти (материала) & & Цена за шт. & Всего цена, руб \\ \hline
		7109113821    & Хомут пластиковый самозатяжной  & & 5      & 50      \\ \hline
	    \rowcolor[HTML]{EFEFEF} 
		0411151042    & Ремкомплект ДВС 1VDFTV (1 шт.)      & & 16000     & 16000    \\ \hline
		1111551030С0    & Прогкладка ГБЦ правая  (1 шт.)    & & 3250     & 3250      \\ \hline
		\rowcolor[HTML]{EFEFEF} 
		
 		\end{tabular}
\end{table}
 Итого на сумму  327 203 (Триста двадцать семь тысяч двести три) рубля. \\
 Всего ремонта по справке ИП Шаманского С.Н. на сумму:  397 203 (Триста девяносто семь тысяч двести три ) рубля.
% 
\vspace{\baselineskip}  % вставка пустой строки

Из заключения специалиста  

\vspace{\baselineskip}
%
%
Из заключения экспертов  

\vspace{\baselineskip}
\renewcommand\baselinestretch{0.86}\small\normalsize 
\subsection{\underline{По  вопросу}\, \, \,	\textbf{\small{1. "Опр"?}}}
\renewcommand\baselinestretch{1.2}\small\normalsize
На момент 

 \begin{figure}[H]\centering
	\parbox[t]{0.49\textwidth}
	{\centering
		\includegraphics[width=.49\textwidth]{images/k1}
		\caption{\footnotesize {Поврежденное компрессорное колесо (крыльчатка турбины) и его  гайка  }}
		\label{ris:images/k1}}
	\hfil \hfil
	\parbox[t]{0.49\textwidth}
	{\centering
		\includegraphics[width=.49\textwidth]{images/k2}
				\caption{\footnotesize {Поврежденное компрессорное колесо (крыльчатка турбины) и стенки корпуса турбины
				рис. 11, 13, Т 1, л.д. 258, 259}}
		\label{ris:images/k2}}
	
\end{figure}

В качестве постороннего предмета истцом заявлен болт клапанной крышки ДВС, извлеченный специалистом ООО "ЭКСПЕРТ",  Рис.\ref{ris:images/b1}
%%
\vspace{\baselineskip}  % вставка пустой строки

  \begin{figure}[!h]
	\centering
	\includegraphics[width=0.95\linewidth]{images/b1}
	\caption{{\footnotesize {Поврежденный болт клапанной крышки, основан на Рис. 12, Т 1, л.д. 258}}}
	\label{ris:images/b1}
\end{figure}

%   \begin{figure}[!h]
%	\centering
%	\includegraphics[width=0.85\linewidth]{images/b2}
%	\caption{{\footnotesize {Поврежденный болт клапанной крышки, вид с торца. Левая стрелка указывает на наклеп шляпки болта, правая- на повреждения торца резьбы детали}}}
%	\label{ris:images/b2}
%\end{figure}


\begin{figure}[!h]\centering
	\parbox[t]{0.49\textwidth}
	{\centering
		\includegraphics[width=.49\textwidth]{images/b2}
		\caption{\footnotesize {Поврежденный болт клапанной крышки, вид с торца. Левая стрелка указывает на наклеп шляпки болта, правая- на повреждения торца резьбы детали }}
		\label{ris:images/b2}}
	\hfil \hfil
	\parbox[t]{0.49\textwidth}
	{\centering
		\includegraphics[width=.49\textwidth]{images/g1}
		\caption{\footnotesize {Поврежденная гайка  компрессорного колеса (крыльчатки турбины)
				рис. 11, 13, Т 1, л.д. 258, 259}}
		\label{ris:images/g1}}
	
\end{figure}
%\begin{SCfigure}
%	\centering {\footnotesize \caption{Болт клапанной крышки, извлеченный из впускного тракта }} 
%	\includegraphics[]{images/b1}
%	\label{ris:images/b1}
%\end{SCfigure}


Эксперты ИП 
%

Из материалов дела следует, что .


\relax
\begin{figure}[h!]\centering
	\parbox[t]{0.49\textwidth}
	{\centering
		\includegraphics[width=.49\textwidth]{images/b3}
		\caption{\footnotesize {Сформированная фаска галтели поврежденного болта}}
		\label{ris:images/b3}}
	\hfil \hfil
	\parbox[t]{0.49\textwidth}
	{\centering
		\includegraphics[width=.49\textwidth]{images/b4}
		\caption{\footnotesize {Галтели неповрежденных болтов выраженных 
				фасок не имеют}}
		\label{ris:images/b4}}
\end{figure}

По совокупности результатов
\subparagraph*{}Таким образом, по совокупности признаков, эксперт приходит вероятностному  выводу о том, что повреждения колеса турбины (крыльчатки компрессора турбокомпрессора) могли быть образованы вследствие попадания постороннего предмета, а именно болта крышки клапанов, извлеченного специалистом ООО "ЭКСПЕРТ". 

\subparagraph*{}  Экспертами 


% 
%\begin{flushleft} 
% 	\hbox{% 
% 		\vrule\hspace{.8em}\parbox{1\textwidth}% 
% 		{ Согласно руководству по сервисному обслуживанию и ремонту TOYOTA LAND CRUISER 200 для проведения работ в объёме, указанном в заказ-наряде №0000-000194 от 03.08.2017 г. (л.д.9), составленном специалистом ООО «РЕГИОНЦЕНТР» (г. Краснодар) необходимо демонтировать агрегаты, находящиеся в пространстве между двигателем и передней панелью, в том числе патрубок подвода воздуха к компрессору левого турбокомпрессора исследуемого двигателя (где в дальнейшем был обнаружен такой посторонний предмет как крепёжный болт).}} 
%\end{flushleft}
% 
% 
\subsection{Исследование транспортного средства}
%
Осмотр автомобиля производился экспертом в условиях дилерского сервисного центра в г. Краснодаре, ул. Аэропортовская, 4/1. При проведении осмотра присутствовали: \присутствовали. \\
Автомобиль предоставлен частично разобранным: демонтирован поддон двигателя, вкладыши коленчатого вала, шатунные катушки зажигания, свечи зажигания.  Отдельно представлена пластиковая емкость, содержащая 3 литра масла из двигателя ТС \тс.
На момент осмотра на автомобиле имеются повреждения переднего бампера снизу слева в виде задиров, крыло заднее правое  имеет царапины ЛКП, бампер задний справа имеет царапины ЛКП, имеется повреждение лобового стекла. Давление в шинах колес передней и задней оси 2.3 бар, шины BRIDGESTONE TURA NZA 225/55R17 97V
%%      
%% \textbf{  Повреждения автомобиля \tcm,\, имеющиеся на момент осмотра 02.07.2018:} (рис. \ref{fig:merclz}, \ref{fig:51})  %\rem{Описания повреждений автомобилей}
%   \begin{itemize}{}{}
% 	 \vspace{-2mm}
%\subparagraph{title}
%\item Дверь передняя левая - компрессионная деформация  поверхности панели, деформация каракаса детали. В средней части лицевой панели, на площади $ \approx 1\, \text{дм}^2 $ глубокая вмятина с участками разрыва металла,   образованная в направлении снизу вверх и слева направо, ниже молдинга статический след  предмета прямоугольной формы  $ \approx 10$ x $20\, \text{см} $; 
%\item Молдинг двери передней левой - деформирован; 
%
% \end{itemize}   
%
%\vfill
%\begin{figure}[!h]
%	\centering
%	\includegraphics[width=0.85\linewidth]{images/51}
%	\caption{{\footnotesize Левая передняя часть автомобиля \tcm. Стрелки указывают на повреждения передней левой  части автомобиля}} \vspace{10mm} 
%	\label{fig:51}
%\end{figure}
%\vspace{5mm} 
%
%\textbf{Исследование повреждений ходовой части} проводилось с использованием ручного измерительного инструмента и стенда для измерения углов установки колес. 
%\begin{SCfigure}
%	\centering {\footnotesize \caption{ Автодата. Схема рычагов задней подвески }} 
%	\includegraphics[width = 0.4 \textwidth] % 
%	{images/p1} % picture filename 
%\end{SCfigure}
%\begin{SCfigure}
%	\centering {\footnotesize \caption{ Подвеска автомобиля \tcm\, левая повреждённая сторона. Стрелками показаны  деформированные рычаги }} 
%	\includegraphics[width = 0.4 \textwidth] % 
%	{images/50} % picture filename 
%\end{SCfigure}
%\relax
%\begin{figure}[h!]\centering
%	\parbox[t]{0.49\textwidth}
%	{\centering
%		\includegraphics[width=.49\textwidth]{images/p1}
%		\caption{\footnotesize {Автодата. Схема рычагов задней подвески}}
%		\label{fig:p1}}
%	\hfil \hfil
%	\parbox[t]{0.49\textwidth}
%	{\centering
%		\includegraphics[width=.49\textwidth]{images/50}
%		\caption{\footnotesize { Подвеска автомобиля \tcm\, левая повреждённая сторона. Стрелками показаны  деформированные рычаги }}
%		\label{fig:50}}
%\end{figure}
%\begin{figure}[h!]\centering
%	\parbox[t]{0.49\textwidth}
%	{\centering
%		\includegraphics[width=.49\textwidth]{images/47}
%		\caption{\footnotesize {Измерение углов установки колес}}
%		\label{fig:47}}
%	\hfil \hfil
%	\parbox[t]{0.49\textwidth}
%	{\centering
%		\includegraphics[width=.49\textwidth]{images/48}
%		\caption{\footnotesize { Экран стенда измерения углов установки колес }}
%		\label{fig:48}}
%\end{figure}
%%
%\begin{figure}[h!]
%	\centering
%	\includegraphics[width=0.9\linewidth]{images/50}
%	\caption[]{{\footnotesize Подвеска автомобиля \tcm\, левая повреждённая сторона. Стрелками показаны  деформированные рычаги}}
%		\label{fig:50}
%\end{figure}
%
%
%
%\begin{figure}[h!]
%	\centering
%	\includegraphics[width=0.85\linewidth]{images/50}
%	\caption[]{{\footnotesize Подвеска автомобиля \tcm\, левая повреждённая сторона. Стрелками показаны  деформированные рычаги}}
%	\label{fig:50}
%\end{figure}
%
%\vspace{10mm}
%   
% \subparagraph*{}\textbf{На предоставленном автомобиле} \tcm\  на момент осмотра раскрыты левая передняя боковая подушка безопасности и левая головная подушка безопасности, Рис. \ref{fig:52}. Система оконных подушек безопасности входит в базовую
% комплектацию модельного ряда W211. Оконные и боковые подушки безопасности срабатывают в том  случае, если центральный электронный блок управления ARMADA регистрирует боковое столкновение. Для определения поперечного  ускорения поступающая от центрального датчика столкновения  информация дополняется информацией от боковых датчиков,
% расположенных в зонах боковых поперечин соответствующих  сторон.
% 
% В результате произведенной проверки электрических цепей системы SRS на наличие повреждений, коррозии, нарушения контактов в   разъемных соединениях  неисправности проводки электрических цепей системы SRS не выявлено. Блок управления системы безопасности в реальном времени нормально реагирует на внешние тестовые воздействия.
% \vspace{3mm}
%\begin{figure}[!h]
%	\centering
%	\includegraphics[width=0.85\linewidth]{images/52}
%	\caption{{\footnotesize Автомобиль \tcm с раскрытыми надувными элементами подушек безопасности}}
%	\label{fig:52}
%\end{figure}
%
На автомобиле имеются характерные повреждения 

% \begin{figure}
% 	\centering
% 	\includegraphics[width=0.95\linewidth]{images/screenshot001}
% 	\caption{{\footnotesize Данные блока управления панели комбинации приборов, зафиксированные в процессе исследования автомобиля}}
% 	\label{fig:screenshot001}
% \end{figure}
% \vspace{5mm}
% \begin{figure}[!h]
% 	\centering
% 	\includegraphics[width=0.95\linewidth]{images/d1}
% 	\caption{\footnotesize Снимок экрана компьютера в процессе диагностики системы управления} 	\label{fig:d1}
% \end{figure}
%{\small \begin{enumerate}{\label{en:enum}}
%		\item [] Ошибки системы SRS:
%	\item 92A3 - высокое сопротивление запального контура левой оконной подушки безопасности. Начало ошибки -- ошибка снята: 19 109 126 мотосекунда -- 19 138 962 мотосекунда или 5 308,09 моточасов -- 5 316,37 моточасов);
%	\item 9223 - высокое сопротивление запального контура левой боковой подушки безопасности: 19 109 126 --33 554 430 мотосекунд или 5 308,09 -- 9 320,67 моточасов;
%	\item 92А0 - замыкание (или утечка)на массу в цепи левой оконной подушки безопасности:  19 138 964-- 33 554 430 мотосекунд или  5 316,37 --9 320,67 моточасов.
%\end{enumerate}}
%
%\vspace{-14mm}
%\begin{center}
%	%\renewcommand{\arraystretch}{0.8}ommand{\arraystretch}{0.8}
%%\begin{tabular}{|p{20mm}|p{7cm}|p{25mm}|p{25mm}|}
%%	\hline 
%%	{\footnotesize Код ошибки} & {\footnotesize Описание ошибки} & {\footnotesize Время возникновения, сек} &{\footnotesize  Время окончания, сек} \tabularnewline
%%	\hline 
%%	92A3 & {\small Высокое сопротивление запального контура оконной подушки безопасности} & 19109126 & 19138962 \tabularnewline
%%	\hline 
%%	9233 & {\small Высокое сопротивление запального контура боковой подушки безопасности} & 19109126 & 3355430 \tabularnewline
%%	\hline 
%%	92A0 & {\small Замыкание  в цепи левой оконной подушке безопасности} & 19138964 & 33554430 \tabularnewline
%%	\hline 
%%%	\label{t:pb}
%%\end{tabular}
%%\renewcommand{\arraystretch}{1.2}
%%\begin{tabular}{|p{20mm}|p{7cm}|p{25mm}|p{25mm}|}
%%	\hline 
%%	{\footnotesize Код ошибки} & {\footnotesize Описание ошибки} & {\footnotesize Время возникновения, сек} &{\footnotesize  Время окончания, сек} \tabularnewline
%%	\hline 
%%	92A3 & {\small Высокое сопротивление запального контура оконной подушки безопасности} & 19109126 & 19138962 \tabularnewline
%%	\hline 
%%	9233 & {\small Высокое сопротивление запального контура боковой подушки безопасности} & 19109126 & 3355430 \tabularnewline
%%	\hline 
%%	92A0 & {\small Замыкание  в цепи левой оконной подушке безопасности} & 19138964 & 33554430 \tabularnewline
%%	\hline 
%%%	\label{t:pb}
%%\end{tabular}
%%\renewcommand{\arraystretch}{1.2}}
%\end{center}
%\vspace{3mm}
%Высокое электрическое сопротивление цепи запального контура указывает на активацию газогенератора (пиропатрона). По данным блока SRS, активация газогенераторов обеих подушек безопасности произошла одновременно на 19109126 мотосекунде.  Снятие ошибки 92А3 совпадает (разница 2 секунды) с моментом возникновения ошибки 92А0 (одновременно с образованием замыкания на массу в цепи левой оконной подушки). Т.е. неисправность электрической цепи  системы управления головной подушкой безопасности в виде короткого замыкания или утечки на массу зафиксировано на 8 часов позднее, чем произошла активация системы SRS и, согласно этим данным, не может являться причиной нештатного раскрытия надувного элемента головной подушки.    
%Считанные показания счетчика моточасов работы из блока управления панели приборов -- 53 19,02 ч. или $ \approx 19 148 472 $ мотосекунды, при том, что в блоке SRS содержатся сведения о снятии ошибки на 33 554 430 мотосекунде или 9 320,67 моточасе  работы автомобиля, что есть существенное расхождение во времени работы автомобиля  в различных блоках управления.  Показания одометра на 10.11.2017г. составляли 89 855 км, (акт осмотра ИП Резенькова, л.~д. 30), показания одометра на момент настоящего исследования так же 
%89 855км. Данный факт указывает либо на вмешательство в  блоки управления автомобиля, либо на наличие неисправных блоков управления.  Таким образом, техническое состояние электрических систем исследуемого автомобиля на момент экспертизы не позволяет достоверно определить соответствие срабатывания боковых подушек безопасности заявленному ДТП \dtp.  
%\subsection*{}
%%\vspace{-10mm}
%\vspace{-10mm}\textbf{  Повреждения автомобиля \tca,\, имеющиеся на момент осмотра 02.07.2018:}   
%
%\begin{itemize}
%	
%	\item Капот - остаточная деформация передней части,\, деталь частично восстановлена; 
%	\item Фара левая - заменена;
%	\item Фара правая - разрушен корпус;
%	\item Бампер передний - расколот в левой части, остаточные признаки пластической деформации правой части детали, отслоение ЛКП; 
%	\item Фара правая противотуманная - разбита;
%	\item Облицовка передняя - деформирована;
%	\item Панель рамки радиатора - частично восстановлена геометрия, имеются остаточные деформации;
%	\item Лонжероны передние левые - имеют остаточные деформации передней части деталей;
%	\item Пластина переднего регистрационного знака - деформирована;
%	\item Крыло переднее левое - остаточная деформация, разрывы металла передней угловой части;
%	\item Крыло переднее правое - остаточная деформация, разрывы металла передней угловой части.\\
%	
%\end{itemize}
%\imgh{150mm}{images/38}{Автомобиль \tca\, вид спереди. Расстояние между передними крыльями 1310 мм. Передний верхний угол левого и правого крыла расположены на высоте  720 мм от опорной поверхности.}
%
%\imgh{165mm}{images/mercl}{Автомобиль \tcm\, левая боковина. Расстояние между повреждениями 1310 мм. Расположены на высоте 620...680 мм и 610...670 мм} 
%
%  	Следуя логической последовательности разрешения вопросов настоящей экспертизы,     первоначально необходимо установить механизм взаимодействия автомобилей при заявленных обстоятельствах,  исследовать повреждения транспортных средств,   произвести сопоставление имеющихся повреждений с  механизмом их образования, определить возможность образования повреждений при заявленном событии и только затем определить стоимость восстановительного ремонта повреждений автомобиля \tcm.  Таким образом, эксперт первоначально должен провести  исследование по второму вопросу.
%  
%  \renewcommand\baselinestretch{0.85}\small\normalsize 
% \subsection{\underline{По  вопросу}\, \, \,	\textbf{\small{2. Состоят ли в причинно-следственной связи повреждения транспортного средства Мерседес Бенц регистрационный знак Р781ЕХ93 2003 года выпуска, заявленного истцом с ДТП, имевшем место 31.10.2017г.?}}}  
% \renewcommand\baselinestretch{1.2}\small\normalsize
% 
% %\vspace{-10mm}
%\subparagraph*{}	\parindent=2em  В соответствии с теорией автомобиля и законами механики, взаимодействие  транспортных средств возможно  определёнными зонами	 контактирования на площади перекрытия в вертикальных и горизонтальных плоскостях, исходя из конструктивных особенностей компоновки кузова при геометрическом ориентировании на проезжей части. Соответственно,  образование повреждений на транспортных средствах возможны  в  закономерных направлениях – в продольном направлении: спереди назад или сзади вперёд; в поперечном направлении: слева направо или справа налево;  в комбинированных направлениях при определённых условиях взаимодействия, исходя из особенностей динамики перемещения: спереди назад и сверху вниз, спереди назад и снизу вверх; сзади вперед и сверху вниз; сзади вперед и снизу вверх. Определение механизма столкновения транспортных средств\footnote{С.Я. Евтюков, Я.В. Васильев\,/ Экспертиза ДТП: Методы и технологии/ СПбГАСУ.-Спб., 2012.-310с. ISBN 978-5-9227-0426-7} включает в себя установление траекторий схождения и расхождения транспортных средств; угла между продольными осями транспортных средств в момент их первичного контактного взаимодействия; частей транспортных средств, которыми они впервые вступили в контактное взаимодействие; площади перекрытия контактирующих при ДТП частей транспортных средств; факта состояния покоя или движения транспортных средств в момент первичного контактного взаимодействия; координат места столкновения и расположения транспортных средств относительно неподвижных элементов дороги.% Механизм столкновения устанавливается по следам на транспортных средствах и месте ДТП. Взаимное положение транспортных средств в момент первичного контактного взаимодействия определяется методом натурной реконструкции события ДТП (совмещение и сопоставление пар повреждений на транспортных средствах участвовавших в ДТП) либо при отсутствии такой возможности, по протоколам осмотра транспортных средств и фотографиям их повреждений, приобщенным к материалам дела. По удовлетворению ходатайства \hod транспортные средства на исследование представлены, что позволило эксперту провести натурную реконструкцию взаимодействия транспортных средств, в соответствии заявленному механизму  ДТП. %Установленный механизм взаимодействия автомобилей \tca и \tcm  не противоречит заявленным обстоятельствам ДТП и соответствует условиям перекрестного поперечного столкновения, обусловленного пересечением траекторий движения автомобилей.\,Проведённой в процеcсе исследования натурной реконструкцией столкновения представленных автомобилей   установлено, что правая боковая сторона автомобиля мерседес--бенц Е220  гос.рег.знак Р781ЕХ93 содержит  следы--отображения внешнего строения деталей передней части автомобиля ауди--80 регистрационный знак В393НУ12.
%\vspace{-10mm}
% \subsection*{}
%      \textbf{В результате детального изучения следов на предоставленных ТС  эксперт выделяет следующие контактные пары:}
%\vspace{2mm}
%%\begin{itemize}
%	\item рамка переднего номерного знака \tca и дверь задняя левая \tcm, рис. \ref{ris:images/11}, \ref{ris:images/13}.
%	\item  левая сторона переднего бампера с отверстием  для противотуманной фары  \tca --- дверь передня левая \tcm, рис. \ref{ris:images/46}, \ref{ris:images/32}.
%	\item повреждение левой передней двери \tcm, полученное от контактного взаимодействия с левым передним крылом \tca, рис. \ref{ris:images/35}, \ref{ris:images/36}.
%	\item повреждение левого заднего крыла \tcm, полученное от контактного взаимодействия с правым передним крылом \tca, рис. \ref{ris:images/33}, \ref{ris:images/34}.
%	\item повреждения левой передней и левой задней двери \tcm компрессионного характера - передний бампер ТС, капот, левая и правая фары \tca, рис. \ref{ris:images/37}, \ref{ris:images/38}. 
%	
%\end{itemize}
%\relax
%\begin{figure}[h!]\centering
%	\parbox[t]{0.49\textwidth}
%	{\centering
%		\includegraphics[width=.49\textwidth]{images/1}
%		\caption{\footnotesize {Этапы реконструкции ДТП}}
%		\label{ris:images/1}}
%	\hfil \hfil%раздвигаем боксы по горизонтали 
%	\parbox[t]{0.49\textwidth}
%	{\centering
%		\includegraphics[width=.49\textwidth]{images/39}
%		\caption{\footnotesize {Этапы реконструкции ДТП}}
%		\label{ris:images/39}}
%\end{figure}
%\relax
%\begin{figure}[h!]\centering
%	\parbox[t]{0.49\textwidth}
%	{\centering
%		\includegraphics[width=.49\textwidth]{images/11}
%		\caption{\footnotesize {След рамки номерного знака}}
%		\label{ris:images/11}}
%	\hfil \hfil%раздвигаем боксы по горизонтали 
%	\parbox[t]{0.49\textwidth}
%	{\centering
%		\includegraphics[width=.49\textwidth]{images/13}
%		\caption{\footnotesize {След воздействия правого переднего крыла автомобиля \tca на поверхность крыла заднего левого автомобиля \tcm}}
%		\label{ris:images/13}}
%\end{figure}
%\relax
%\begin{figure}[h!]\centering
%	\parbox[t]{0.49\textwidth}
%	{\centering
%		\includegraphics[width=.49\textwidth]{images/46}
%		\caption{\footnotesize {левая сторона переднего бампера с отверстием  для противотуманной фары  \tca}}
%		\label{ris:images/46}}
%	\hfil \hfil%раздвигаем боксы по горизонтали 
%	\parbox[t]{0.49\textwidth}
%	{\centering
%		\includegraphics[width=.49\textwidth]{images/32}
%		\caption{\footnotesize {дверь передня левая \tcm}}
%		\label{ris:images/32}}
%\end{figure}
%\relax
%\begin{figure}[h!]\centering
%	\parbox[t]{0.49\textwidth}
%	{\centering
%		\includegraphics[width=.49\textwidth]{images/35}
%		\caption{\footnotesize {Левая передняя дверь автомобиля  \tcm}}
%		\label{ris:images/35}}
%	\hfil \hfil%раздвигаем боксы по горизонтали 
%	\parbox[t]{0.49\textwidth}
%	{\centering
%		\includegraphics[width=.49\textwidth]{images/36}
%		\caption{\footnotesize {Левое переднее крыло автомобиля \tca}}
%		\label{ris:images/36}}
%\end{figure}
%
%\relax
%\begin{figure}[h!]\centering
%	\parbox[t]{0.49\textwidth}
%	{\centering
%		\includegraphics[width=.49\textwidth]{images/33}
%		\caption{\footnotesize {Левое заднее крыло автомобиля \tcm}}
%		\label{ris:images/33}}
%	\hfil \hfil%раздвигаем боксы по горизонтали 
%	\parbox[t]{0.49\textwidth}
%	{\centering
%		\includegraphics[width=.49\textwidth]{images/34}
%		\caption{\footnotesize {Левое переднее крыло автомобиля \tca}}
%		\label{ris:images/34}}
%\end{figure}
%\relax
%\begin{figure}[h!]\centering
%	\parbox[t]{0.49\textwidth}
%	{\centering
%		\includegraphics[width=.49\textwidth]{images/37}
%		\caption{\footnotesize {Левые двери автомобиля \tcm. Следовоспринимающая поверхность}}
%		\label{ris:images/37}}
%	\hfil \hfil%раздвигаем боксы по горизонтали 
%	\parbox[t]{0.49\textwidth}
%	{\centering
%		\includegraphics[width=.49\textwidth]{images/23}
%		\caption{\footnotesize {Передний бампер, капот, левая и правая фары \tca. Следообразующая поверхность}}
%		\label{ris:images/23}}
%\end{figure}
%%\rem{Возможно, необходимо перенести в другой раздел}
%\subparagraph*{} Имеющиеся следы комбинированные, статические, направленные перпендикулярно продольной оси автомобиля \tcm с незначительной динамической составляющей  следов внедрения, направленные снизу вверх  и немного слева направо (под углом $ \approx 30^{\circ} $) к вертикальной оси автомобиля.%  рис. \ref{ris:images/1}, \ref{ris:images/39}. \rem{фото характерных следов}
%
%Совокупность индивидуализирующих признаков групп следов и повреждений, имеющихся  на транспортных средствах, совпадающих между собой по уровню расположения, форме и  локализации указывают на   возможность их образования при взаимном контактном взаимодействии по механизму перекрёстного взаимодействия  с центральным ударом.  Различие по высоте ($ \approx 9  \cdots 10 $ см ), в данном случае,  не противоречит  механизму следообразования, так как  закономерность направления деформирующего воздействия со стороны следообразующего объекта на части кузова транспортного средства обусловлены  геометрическими размерами  рассматриваемых объектов, траекторией их перемещения, а также созданием определенных моментов сил,  в том числе вызывающих перераспределение массы по осям, рис.27\ref{ris:images/tormoz}.   Полагаем, что водитель автомобиля \tca, \, перед столкновением с автомобилем \tcm, применил торможение.  Тогда в  момент первичного контакта передняя часть автомобиля  \tca \ была расположена на несколько сантиметров ниже относительно расстояния при нормальных условиях движения или неподвижного состояния, так как согласно законам механики,  в момент торможения происходит перераспределение веса под действием силы
%  \begin{equation}\label{eq:f}
%  \vec{F_t} = \dfrac{\vec{a}*m*H}{L}, \,\,\,\,  \text{где:}
%  \end{equation}
%  \begin{itemize}
%  \item[ ] $\vec{a} $ ---величина замедления;
%  \item[ ] $ L $ --- длина базы автомобиля;  
%  \item[ ] $ H $ --- высота центра тяжести; 
%   \item[ ] $ m $ --- масса автомобиля
%  \end{itemize}
% Действие силы $ \vec{F_t} $   на переднюю ось автомобиля  приводит к дополнительному сжатию пружин передней подвески и, как следствие, уменьшению расстояния от деталей передней части кузова до опорной поверхности. Величина уменьшения расстояния зависит от значения силы $ \vec{F_t} $ и жёсткости  подвески автомобиля.
%В свою очередь,  в начальный момент контакта, положение кузова автомобиля \tcm \, должно было  соответствовать  положению при нормальном  условии движения, далее, под действием силы $ \vec{F_t} = \dfrac{mv^2}{2} $ , где $ m $ -- масса автомобиля \tcm, v -- его скорость, пространственное положение кузова \tcm измениться.
%\imgh{100mm}{images/torm}{Схематичное изображение перераспределения веса при торможении автомобиля }
%\begin{SCfigure}
%\centering \caption{рлрл captioорпоп опрпоп опорпопр опопо  порпорп орор оро опрорп орп ор оорп орп оn text ... } 
%\includegraphics[width = 0.6 \textwidth] % 
%{images/torm} % picture filename 
%\end{SCfigure}
%Анализ характера деформаций и направлений действующих сил, вызвавших повреждения частей деталей правой боковины автомобиля \tcm \, и передней части автомобиля \tca, а именно:
%\begin{itemize}
%	\item[ ] --  обширные площади деформации на ТС в местах, которыми они вошли в контактное взаимодействие с преградой;
%	\item[ ] --  оттиски отдельных участков, деталей одного ТС на поверхности частей другого;
%	\item[ ] --  следы внедрения в виде потёртостей на ТС -- снятия слоя лакокрасочного покрытия, локальных разрывов поверхности;
%	\item[ ] -- трассы (следы скольжения, давления, царапания), возникшие от контакта с другим ТС
%\end{itemize} 
%позволяют заключить, что данные повреждения могли быть образованны в результате  взаимного контактного взаимодействия указанных автомобилей.\\
%Данный механизм следообразования является характерным для \rem{вставить механизм следообразвания}
%Таким образом, сопоставлением характерных групп следов, местом и направлением их нанесения, пространственным совмещением выделенных следовоспринимающих и следообразующих объектов можно сделать общий вывод о наличии пространственно-следового изоморфизма, и соответственно о наличии контактно-следового взаимодействия автомобилей \tca и \tcm.
%
%	Учитывая заявленные обстоятельства ДТП \dtp, исходя из взаимного ориентирования обоих транспортных средств на проезжей части  соответствующими сторонами кузова, а также направления траектории сближения обоих транспортных средств по принципу организации дорожного движения, повреждения левой боковой части автомобиля \tcm могут состоят в причинно-следственной связи с указанным  ДТП. 
%Повреждения деталей передней части автомобиля \тсм, согласно предоставленным административных материалов по данному ДТП, получены в результате наезда автомобиля на дерево, расположенное  на расстоянии 4,6 м от края проезжей части и на удалении $ \approx 10 $ м от места столкновения автомобилей, рис.2. 
%На момент исследования, на автомобиле \тсм имеются остаточные признаки упругой деформации деталей передней левой части автомобиля, по совокупности морфологических признаков не противоречащие заявленному механизму образования. Положение автомобиля \тсм, указанное на схеме ДТП, должно быть обусловлено изменением траектории движения автомобиля после столкновения на угол, составляющий $ \approx 40^\circ $ вправо от направления  движения до столкновения автомобилей. При этом, схема места дорожно-транспортного происшествия не содержит следов потери устойчивости автомобиля после столкновения. Из  анализа механизма взаимодействия автомобилей, эксперт приходит к заключению, что положение автомобиля \тсм, указанное  на схеме ДТП не является результатом изменения траектории его  движения  вследствие удара, а вероятно, связано с управляющими действиями водителя, его ответной реакцией на столкновение.
%Следовательно, повреждения деталей передней части автомобиля \tcm могут находится в причинно-следственной связи с указанным  ДТП.% так как  характер повреждений  не противоречит заявленным обстоятельствам. 
%\subparagraph*{}Таким образом, по совокупности признаков, эксперт приходит к выводу о том, что  повреждения транспортного средства Мерседес Бенц регистрационный знак Р781ЕХ93 2003 года выпуска  состоят  в причинно-следственной связи с  ДТП, имевшем место 31.10.2017г.
%{\begin{enumerate}
%		
%	\item Наличие чётких отпечатков частей одного ТС на другом в местах их первичного контакта при отсутствии трасс местах образования отпечатков или при наличии трасс, возникших после образования отпечатков \rem{еще ремарка по ударв неподвижное тс}
%	\item  Совпадение направления первоначальных трасс и деформаций на ТС, по которому был нанесен удар при перекрёстном столкновении,  направлением движения другого ТС
%	\item Расположение трасс тангенциальной направленности на боковой поверхности колес
%	\item Разворот ТС в направлении момента, который мог возникнуть при столкновении только в случае движения тс, по которому был нанесён удар \\
%	
%	 
%\end{enumerate}}
%Необходимо отметить \rem{опять про изоморфизм}, что данный вид трасологических исследований (установление пространственно-следового изоморфизма, а
%именно установление факта контактно-следового взаимодействия,
%сопоставление обстоятельств ДТП, заявленных страхователем с механизмом нанесения повреждений и т.д.) в настоящее время становится все более и более востребованным и актуальным, т.к. в современных условиях складывается ситуация, когда мошенничество на
%транспорте, с целью получения страхового возмещения, приобретает объёмы снежного кома.
%%%%%%%%%%%%%%%%%%%%%%%%%%%%%%%%%%%%%%%%%%%%%%%%%%%%%%%%%%%%%%%%%%%%%%%%%%%%%%%%%%%%%%%%%%%%%%%%%%%%%%%%%%%%%%%%%%%%%%%%%%%%%
\renewcommand\baselinestretch{0.86}\small\normalsize 
\subsection{\underline{По  вопросу}\, \, \,	\textbf{\small{"Определить стоимость восстановительного ремонта с учетом износа стоимости запасных частей"?}}}
\renewcommand\baselinestretch{1.2}\small\normalsize
Колесное транспортное средство сроком эксплуатации более 7 лет относится к категории транспортных средств с граничным сроком эксплуатации [1], для которой возможно применение ремонтных операций при условии экономической целесообразности и  технической возможности.  
                                         
В соответствии с принятой экспертной методикой [1], стоимость восстановительного ремонта АМТС  $ C_p $ определяется по формуле:
%
\begin{equation}\label{eq:r}
C_\text{вp} =C_p + C_\text{м} + C_\text{зч}\cdot\left( 1-\frac{\text{И}}{100}\right)  \,\,\,\, \text{где:}
\end{equation}
%
%
\begin{itemize}
%	
\item[ ]$C_\text {р} $ --  стоимость ремонтных работ по восстановлению КТС, руб.;
\item[ ]$ C_\text{м} $ --  стоимость необходимых ремонтных материалов, руб.;
\item[ ]$ C_\text{зч} $ --  стоимость новых запасных частей, руб;
\item[ ] $ \text{И} $ -- коэффициент износа составной части, подлежащей замене, \%.
\end{itemize}
%
%
Коэффициент износа составных частей (И) КТС (кроме автобусов и грузовых автомобилей) при определении стоимости восстановительного ремонта расчитывается по формуле:

\begin{equation}\label{eqsnos}
\text{И} =\text{И1}\cdot\text{П}+\text{И2}\cdot \text{Д}, \%  \,\,\,\, \text{где:}
\end{equation}

\begin{itemize}
	\item [] $ \text{И1} $ --усредненный показатель износа на 1000 км пробега, \%; 
	\item [] $ \text{П} $ -- общий пробег (фактический или расчетный) за срок эксплуатации КТС, тыс.км;
	\item [] $ \text{И2} $ -- усредненный показатель старения за 1 год эксплуатации, \%;
	\item [] $ \text{Д} $ -- срок эксплуатации КТС (от даты изготовления КТС до момента, на который определяется износ), лет. 
\end{itemize}

Для исследуемого автомобиля \тс, согласно справочным таблицам [1]:
\begin{equation}\label{eqsnosr}
\text{И} =\text{И1}\cdot\text{П}+\text{И2}\cdot \text{Д} = 0.23\cdot 214  + 0.85\cdot 9 = 49 \, \%
\end{equation}
%%%%%%%%%%%%%%%%%%%%%%%%%%%%%%%%%%%%%%%%%%%%%%%%%%%%%%%%%%%%%%%%%%%%%%%%%%%%%%%%%%%%%%%%%%%%%%%%%%%%%%%%%%%%%%%%
%Стоимость восстановительных работ $ C_{\text{вр}} $ определяется на основании норм трудоёмкостей $ T_i $, \,предусмотренных заводом-изготовителем, и стоимостных параметров $ C_{i\text{нч}} $ (стоимости нормо-часа) работ по техническому обслуживанию и ремонту АМТС. 
%\begin{equation}\label{eq:cr}
%C_{\text{вр}}  =\sum{C_{ip}}= \sum\left({T_{ij}}\cdot {C_{i\text{нч}}}\right) + \sum{C_{ip^{\text{\,\,\,руб}}}} , \,\,\,\text{где:} 
%\end{equation}
%\vspace{2mm}
%\begin{itemize}
%	\item[ ]$ C_{ip} $ -- стоимость работ i-го вида: $C_\text {зам} $, $ C_\text{восст} $, $ C_\text{рег} $, $C_\text{контр} $, $ C_\text{антикор} $, $ C_\text{зч} $, $ C_\text{ом} $,$ C_\text{соп} $, $ C_\text{вм} $, руб;
%	\item[ ]$ T_{ij} $ -- трудоёмкость j-й операции(комплекса) по i-му виду работ, руб;
%%	\item[ ]$ C_{i\text{нч}} $ -- стоимость нормо-часа по i-му виду работ, руб;
%	\item[ ]$ C_{ip^\text{\,\,руб}} $ -- стоимость работ $ C_{ip} $, принятая непосредственно в денежном выражении, руб.
%\end{itemize}
%
%\vspace{5mm}
%При определении стоимости восстановительного ремонта АМТС с учётом износа под износом следует понимать количественную меру физического старения АМТС и его элементов, достигнутого в результате эксплуатации, т.е. эксплуатационный износ.\\
%%
%Расчёт износа производится в  соответствии с Положением Банка России от «19» сентября 2014 года № 432-П «О единой методике определения размера расходов на восстановительный ремонт в отношении повреждённого транспортного средства» [3].
%Износ комплектующих изделий (деталей, узлов, агрегатов) рассчитывается по следующей формуле:
%
%\begin{equation}\label{eq:I}
%\text{И}_{\text{ки}} 
%= 100\cdot\left( 1-e^ {-\left( \Delta_{T} \cdot T_{\text{КИ}} + \Delta_{L} \cdot L_{\text{КИ}} \right)}\right), \,\,\,\,\text{где:}   
%\end{equation}
%
%\begin{itemize}
%	\item[ ]$ \text{И}_{\text{ки}} $ -- износ комплектующего изделия (детали, узла, агрегата) (процентов); 
%	\item[ ]$ e $ -- основание натуральных логарифмов (e =  2,72);;
%	\item[ ]$ \Delta_{T}$ --  срок эксплуатации комплектующего изделия (детали, узла, агрегата) (лет);
%	\item[ ]$ T_{\text{КИ}} $ -- стоимость работ $ C_{ip} $, принятая непосредственно в денежном выражении, руб
%	\item[ ]$ \Delta_{L} $ --коэффициент, учитывающий влияние на износ комплектующего (детали, узла, агрегата) величины пробега транспортного средства с этим комплектующим изделием;
%	\item[ ]$ L_{\text{КИ}} $ --пробег транспортного средства на дату дорожно-транспортного происшествия (тысяч километров).  
%		\end{itemize}
%\vspace{5mm}
%Значения коэффициентов $ \Delta_{T}$  и $ \Delta_{L} $  для различных категорий и марок транспортных средств приведены в п.5. исп. лит~[3]. При этом, на комплектующие изделия (детали, узлы, агрегаты), которые находятся в заведомо худшем состоянии, чем общее состояние транспортного средства в целом, и его основные части, вследствие влияния факторов, не учтённых при расчете износа (например, проведение ремонта с нарушением технологии, не устранение значительных повреждений лакокрасочного покрытия), может быть начислен дополнительный индивидуальный износ. 
%Износ шины транспортного средства рассчитывается по следующей формуле:
%\begin{equation}\label{eq:sh}
%\text{И}_{\text{ш}} = \frac{\text{Н}_{\text{н}}-\text{Н}_{\text{ф}}}{\text{Н}_{\text{н}}-\text{Н}_{\text{доп}}} \cdot{100}\%  \,\,\,\,\text{где:} 
%\end{equation}
%
%\begin{itemize}
%	\item[ ] $ \text{И}_{\text{ш}} $ -- износ шины, \%;
%	\item[ ] $ \text{Н}_{\text{н}} $ -- высота рисунка протектора новой шины, мм;
%	\item[ ] $\text{Н}_{\text{ф}} $ -- фактическая высота рисунка протектора шины, мм;
%	\item[ ] $ \text{Н}_{\text{доп}} $ --минимально допустимая высота рисунка протектора шины в соответствии с требованиями законодательства Российской Федерации, мм.
%\end{itemize}
%
%\renewcommand\baselinestretch{1}\small\normalsize
%
%Износ шины дополнительно увеличивается для шин с возрастом от 3 до 5 лет - на 15 процентов, свыше 5 лет - на 25 процентов.
%
В результате исследования   экспертом установлено, что для устранения повреждений \тс \, необходимо  выполнить следующие  работы:
\begin{center}
	\begin{tabulary}{\textwidth}{LCL}
\hline 
\textbf{Наименование детали}      &   & \textbf{Ремонтное воздействие}\\
\hline    
Турбина левая     &   &    Заменить\\
Блок ДВС    &   &    Отремонтировать гильзованием, заменой колец и прокладок \\
Головка блока цилиндров & & Восстановить седла клапанов пятого цилиндра \\
Впускной тракт & &    Разобрать, прочистить\\
Интеркулер   & &     прочистить\\
 

	\end{tabulary}  
\end{center}
\renewcommand\baselinestretch{1.2}\small\normalsize
%
\textbf{Произвести  необходимые для выполнения  ремонта разборочно-сборочные, подготовительные и вспомогательные работы в соответствии с требованиями завода–изготовителя транспортного средства.}\\
%
Расчет стоимости восстановительных расходов выполнен в программе \auda. \\ Ниже представлены результаты расчета, полная калькуляция стоимости ремонта включена в раздел Приложение.
%%\smallskip
\begin{figure}[H]
	\centering
	\includegraphics[width=0.8\linewidth]{images/Screenshot_2}
%%	\caption{}
%%	\label{fig:screenshot001}
\end{figure}
%\begin{figure}[H]
%	\centering
%	\includegraphics[width=0.9\linewidth]{images/screenshot002}
%%%	\caption{}
%	\label{aud}
%\end{figure}
\medskip
Стоимость коммерческого нормо-часа работ применена  с учетом условий регионального рынка услуг и сложившихся средних расценок по видам работ, типу ТС, а также по маркам и моделям ТС  и   составляет  1300 р/ч для данного транспортного средства. \\
Трудоёмкость работ по разборке/сборке/замене  соответствует трудоемкости работ, рекомендованной заводом изготовителем ТС. 
Расчет стоимости ремонта, согласно положениям [1] производится с учетом  применения оригинальных запасных частей, которые поставляются изготовителем КТС авторизованным ремонтным организациям. Техническое состояние запасных частей учитывается коэффициентом износа, что в совокупности с установкой оригинальных запасных частей в максимальной степени отвечает понятию «восстановительный ремонт», то есть восстановления состояния КТС, при котором используются установленные изготовителем составные части, но с использованным частично ресурсом. 

%
В результате проведенных расчетов (см. Приложение, калькуляция № 22-2019) определена стоимость восстановительного ремонта транспортного средства  \тс, которая составляет 441 295 (Четыреста сорок одна тысяча двести девяносто пять) рублей без учета износа,
стоимость восстановительных расходов, с учетом уменьшения стоимости запасных частей вследствие их износа,  составляет 281 990 (двести восемьдесят одна тысяча девятьсот девяносто) рублей.



\section{Выводы}


\begin{enumerate}
	\item \textbf{"Причина поломки или неисправности ДВС на автомобиле Toyota Land Cruizer 200, 2008 года выпуска VIN: JTMHV05J004031859, после проведения ремонта ответчиком при изложенных в исковом заявлении обстоятельствах является попадание постороннего предмета, а именно болта крышки клапанов в левый впускной тракт, что привело к разрушению колеса турбины и повреждению рабочих поверхностей пятого цилиндра ДВС}
	
	\vspace{5mm}
	
	\item \textbf{ Стоимость восстановительного ремонта с учетом износа стоимости запасных частей составляет 281 990 (двести восемьдесят одна тысяча девятьсот девяносто) рублей}
\end{enumerate}
\vspace{15mm}
\relax
Приложение к заключению:\\
\textit{
	Расшифровка комплектации ТС по VIN \\
	Листинг калькуляции стоимости восстановительных расходов\\
	   }

\vspace{20mm}

{Эксперт}\hfill           {Мраморнов А.В.}

\includepdf[pages=-]{myfile.pdf}
%\includepdf[pages=-]{calc.pdf}

%\setcounter{page}{1}
\clubpenalty=10000 
\widowpenalty=10000
%%%%%%%%%%%%%%%%%%%%%%%%%%%%%%%%%%%%%%%%
%      Шапка экспертной организации  
%%%%%%%%%%%%%%%%%%%%%%%%%%%%%%%%%%%%%%%%
%
%%%%%%%%%%%%%%%%%%%%%%%%%%%%%%%%%%%%%%%%%
%
%   Экспертная организация ИП
%
%%%%%%%%%%%%%%%%%%%%%%%%%%%%%%%%%%%%%%%%%
\noindent
\begin{pspicture}(21mm,21mm)
\obeylines
\psbarcode{%
	%\NomerDoc от \окончено
	BEGIN:VCARD^^J
	VERSION:4.0^^J
	N:Мраморнов; Александр; Вчеславович^^J
	FN:Александр Мраморнов^^J
%	ORG:IP Alexandr Mramornov^^J
	TITLE: эксперт
	ORG: ИП
	URL:http://www.yourexp.ru^^J
	EMAIL:yourexpert.mramornov@gmail.com^^J
	TEL:+7-918-451-6611^^J
%	ADR:г. Краснодар, с/т № 2 А/О «Югтекс», ул. Зеленая, 472^^J
	END:VCARD
}{width=1.0 height=1.0}{qrcode}%
\end{pspicture}

 %%% Добавлен QR-Code
\vspace{-4mm}
\begin{center}
	\large\textbf{ИНДИВИДУАЛЬНЫЙ\quad ПРЕДПРИНИМАТЕЛЬ  \\[-1.5mm] МРАМОРНОВ  АЛЕКСАНДР ВЯЧЕСЛАВОВИЧ \\[-5.5mm]}
	%  
	\noindent\rule{\textwidth}{2pt}\\[-6mm]  % Горизонтальная линия
	% \line(1,0){460}% (1,0) -горизонтальная линия, и (0,1) - вертикальная 
\end{center}

\begin{center}
	\begin{footnotesize}\setstretch{0.3}
		%	\small\textbf\setlength   	%\raisebox{5mm}
		\vspace{-2.5mm}г. Краснодар, с/т № 2 А/О «Югтекс», ул. Зеленая, 472, 
		Телефон: 8-918-451-66-11, e-mail: 4516611@gmail.com\\ [-2mm]{ИНН\quad 231200665168\quad ОГРНИП \quad 310231220400043}
	\end{footnotesize}	\\[10mm]
\end{center}


\begin{flushright}
% 
	 \hfill	Краснодар, 2020    \\[8mm]
\end{flushright}  
\begin{center}
	\LARGE\textbf{АКТ ЭКСПЕРТНОГО ИССЛЕДОВАНИЯ}
	\bigskip\\[-5mm] 
	\textbf{  {\normalsize № \NomerDoc\,  от \dataend}}
\end{center}
\par
\vspace{4mm}

%\noindent на <<заключение эксперта № 00182/18 по гражданскому делу № 2-802/19 по иску \иск>>\\


%%%%%%%%%%%%  Если независимая
\vspace{2mm}
\noindent %\dog\, владелец транспортного средства \тс\, регистрационный знак \грз \,  \владелец, \, проживающий: \адресвладельца \, обратился с заявление об определении износа принадлежащего ему транспортного средства на момент дорожно-транспортного происшествия, имевшего место \датадтп. %  составлен на основании	обращения владельца транспортного средства 40ОС277336%договора № \NomerDoc\,  \dog \,  возмездного оказания услуг.
Составлен на основании	договора № \NomerDoc\,от \датадоговора\, возмездного оказания услуг   по проведению независимой технической экспертизы (далее экспертиза)  транспортного средства \тс\, регистрационный знак \грз\, и письменного заявления заказчика о проведении экспертизы.

\noindent Заказчик: \заказчик, \адресзаказчика. 

%Полис ОСАГО: \polis.

% Документ, удостоверяющий личность заказчика: водительское удостоверение    03\ 16\ 422344\ выдан 09.06.2011

%Транспортное средство виновника ДТП:  не предоставлялось.

\paragraph*{}
%Исследование произвели  специалисты:
Исследование произвёл  специалист
%{\small ООО "ЮЖНО-РЕГИОНАЛЬНАЯ ЭКСПЕРТНАЯ ГРУППА"}
\,  Мраморнов Александр Вячеславович, имеющий высшее  образование по специальности «техническая физика», диплом РВ №311964 от 28.02.1989, квалификация -- инженер-физик, специальное образование в области независимой технической экспертизы транспортных средств: Диплом ПП-I № 424167, квалификация: эксперт-техник (специализация 150210 специальности 190601.65 – Автомобили и автомобильное хозяйство), состоящий в Государственном реестре экспертов-техников (№ в реестре 256, https://data.gov.ru/opendata/7707211418-experts, специальное образование в области оценки: Диплом ПП-1 № 037211 Российской экономической академии им. Г.В. Плеханова, квалификация -- оценка и экспертиза объектов и прав собственности,   общий трудовой  стаж 30 лет, стаж  экспертной работы  12 лет. 

%%%%%%%% ВВС
%Вовчук Валерий Сергеевич, эксперт-автотехник,  имеющий высшее техническое образование, квалификацию «Инженер по эксплуатации автомобильной техники»,  квалификацию государственного судебного эксперта по специальностям: 2.1«Исследование обстоятельств дорожно-транспортного происшествия», 2.2«Исследование технического состояния узлов и деталей транспортного средства», 2.3«Исследование следов столкновения на транспортных средствах и месте дорожно-транспортного происшествия» (транспортно-трасологическая диагностика), 2.4«Исследование маркировочных обозначений транспортных средств», стаж экспертной работы 23 года (из них 15 лет государственный эксперт МВД РФ).

\par Заключение подготовлено по месту фактического расположения ИП по адресу: г. Краснодар, с/т № 2 А/О «Югтекс», ул. Зеленая, 472.
\vspace{4mm}  % Шапка организации ООО ЮРЕКСГРУП
%
%%   вопросы экспертизы
\subsection{Вопросы экспертизы}
%Заказчик поручает, а Исполнитель принимает на себя обязательство выполнить Заказчику  комплекс работ в виде автотехнических исследований автомобиля Mazda 6, VIN RUMGJ52\-6802007133 (дата начала гарантии 07.05.2018 г.), по следующим вопросам:
\begin{enumerate}
	\item  <<Связано ли повреждение панели рамки радиатора слева и брызговика с лонжероном переднего левого автомобиля ВАЗ 21099 с указанным ДТП?>>	
\end{enumerate}

\addcontentsline{toc}{section}{Использованные нормативы и источники информации}
%
%\left( \addcontentsline{toc}{section}{Использованные нормативы и источники информации}

\subsection{Использованные нормативы и источники информации}
%
\begin{enumerate}
\item 
Махнин\,Е.\,Л., Новоселецкий\, И.\,Н., Федотов\, С.\,В. \emph{Методические рекомендации по проведению судебных автотехнических экспертиз и исследований колёсных транспортных средств в целях определения размера ущерба, стоимости восстановительного ремонта и оценки} // -- М.: ФБУ РФЦСЭ при Минюсте России, 2018.-326 с.  ISBN 978-5-91133-185-6.
%
%
%
%
\item ТУ 017207-255-00232934-2014 \emph{Кузова автомобилей LADA. Технические требования при приёмке в ремонт, ремонте и выпуске из ремонта предприятиями дилерской сети ОАО "АВТОВАЗ"}//  ОАО НВП "ИТЦ АВТО", 2014
%
\item Смирнов  В.Л., Прохоров  Ю.С., Боюр В.С.  и др. \emph{Автомобили ВАЗ. Кузова. Технология ремонта, окраски и  антикоррозионной защиты. Часть II}// - Н.Новгород: АТИС, 2001.- 241с.
%
\item 
Савич Е.Л. \emph{Техническое  обслуживание  и  ремонт  легковых  автомобилей} : учеб. пособие / Е.Л. Савич, М.М. Болбас, В.К. Ярошевич ; под общ. ред. Е.Л. Савича. -Мн. : Вышэйшая школа,  2001. - 479 с. - ISBN985-06-0502-2.
%
\item 
Автомобили ВАЗ-2121, 21213, 21214, 2131 и их модификации: <<Трудоемкости работ (услуг) по техническому обслуживанию и ремонту>> /Куликов А.В., Христов П.Н., Климов В.Е.,  Боюр В.С., Рева В.В., Зимин В.А., Завьялова Н.Н., Хлыненкова Г.А. -- ИТЦТ "АвтоВАЗтехобслуживание", Тольяти -- 2005. 
%
\item
Автомобили LADA SAMARA и их модификации: <<Трудоемкости работ (услуг) по техническому обслуживанию и ремонту>> /Куликов А.В., Христов П.Н., Климов В.Е., Рева В.В., Боюр В.С., Васильев М.В., Фахрутдинов Р.В.,  Прудских Д.А., Гирко В.Б., Шмелева В.А., Зимин В.А. --  ОАО НВП "ИТЦ АВТО",  -- 2006. - 252 стр.
%
\item 
Автомобили LADA PRIORA. Трудоемкости работ (услуг) по техническому обслуживанию и ремонту /Куликов А.В., Христов П.Н., Климов В.Е., Рева В.В., Козлов П.Л., Боюр В.С., Прудских Д.А., Шмелева В.А., Зимин В.А. -- ООО "ИТЦТ АВОСФЕРА", Тольяти -- 2009. -- 344 с.
%
\item 
{Трудоемкости работ по техническому обслуживанию и ремонту автомобилей автомобилей Lada  Granta}/   \url{https://docplayer.ru/30250248-Trudoemkosti-rabot-po-teh\-nicheskomu-obsluzhivaniyu-i-remontu-avtomobiley-lada- granta.html}.
%
%
\item
{Специализированное программное обеспечение для расчёта стоимости  восстановительного ремонта, содержащее нормативы трудоёмкости работ, регламентируемые изготовителями транспортного средства}//   AudaPadWeb, лицензионное соглашение № AS/APW-658  RU-P-409-409435.
%
%
%
\item
{Специализированное программное обеспечение для расчёта стоимости  восстановительного ремонта, содержащее нормативы трудоёмкости работ, регламентируемые изготовителями транспортного средства ОАО «АвтоВАЗ», ЗАО «Джи-Эм-АвтоВАЗ», ОАО «СеАЗ» и ОАО «ЗМА»}//   Автосфера АС:Смета, v.3.9.11// ООО "ИТЦ «ИнтегроМаш», \url{https://autosmeta.pro}.
%
%
%
\item Информационный портал по техническому обслуживанию и ремонту автомобилей	 ВАЗ:\\ \url{www.autosphere.ru}.

%%
\end{enumerate}

%\bibliographystyle{utf8gost705u}  %% стилевой файл для оформления по ГОСТу
%\bibliography{biblio}     %% имя библиографической базы
%%%%%%%%%%%%%%%%%%%%%%%%%%%%%%%%%%%%%%%%%%%%%%%%%%%%%%%%%%%%%%%%%%%%%%%%%%%%%%%%%
\subsection{Технические средства}  %% Список не удалять!!!
\begin{itemize}
%
%%
%\item Диагностический сканер BOSH VCM II S/N 1324-88682639 c програмным обеспечением Mazda IDS - 115.02
%%\item   Диагностический сканер SDconnect   с программным обеспечением Xentry Diagnostics v19.11.3.1
%\item   Линейка масштабная магнитная с цветографической шкалой, 100мм
%%\item   Рулетка измерительная металлическая, 5м
%%\item  Универсальный стенд для измерения углов установки колес Hunter Engineering %ProAlign с программным инструментом регулировки схождения колес без блокировки руля %автомобиля WinToe
%\item 	Цифровой фотоаппарат Canon 760D s/n 143032001327 с объективом Canon EF-S 18-135, тип используемой памяти: Transcend,  32Gb
%\item  Специализированное программное обеспечение для расчёта стоимости  восстановительного ремонта, содержащее нормативы трудоёмкости работ, регламентируемые изготовителями транспортного средства     AudaPadWeb, лицензионное соглашение № AS/\- APW-658  RU-P-409-409435
\item Он-лайн программа моделирования кинематики подвески автомобиля // \url{http://www.vsusp.com/}
\item  Программа обработки фото-видео изображений ImageJ, разработчик  Wayne Rasband (wa-yne@codon.nih.gov),
свободная лицензия GPL.
\item  ПЭВМ под управлением операционной системы Windows 10 с установленным набором макрорасширений LaTeX системы компьютерной вёрстки TeX, cвободная лицензия LaTeX Project Public License (LPPL). 
%	
\end{itemize}
%%%%%%%%%%%%%%%%%%%%%%%%%%%%%%%%%%%%%%%%%%%%%%%%%%%%%%%%%%%%%%%%%%%%%%%%%%%%%%%%%%%%%%%%%%%%%%%%%%%%%%
\subsection{Условные обозначения}

\begin{description}
%	 
%%\item[АВС] --антиблокировочная система
\item[АМТС] --автомототранспортное средство
%\item[ГРМ] -- газораспределительный механизм
\item[ДВС] --двигатель внутреннего сгорания
\item[ДТП] --дорожно--транспортное происшествие
\item[гос.\,рег.\,знак] --государственный регистрационный знак
\item[КТС] --колесное транспортное средство 
\item[ЛКП] --лакокрасочное покрытие
\item[л.д.] --лист дела
%%\item[Колесо турбины]  -- крыльчатка турбины
\item[ТС] --транспортное средство
%\item[ТK, ТКР] -- турбокомпрессор. Состоит из двух частей: турбины и компрессора, объединенных общим валом. Вал вращается в подшипниках, размещенных в центральном корпусе ТК
%\item[ЦПГ] -- цилиндро-поршневая группа
%\item[ЭБУ] --электронный блок управления
%%\item[FRAME] -- номер кузова транспортного средства, выпущенного для продажи на внутреннем рынке Японии и содержащий информацию производителя о транспортном средстве
%%\item[OBDII] -- On-board diagnostics. Протокол бортовой диагностики автомобиля
%%\item[SRS] -- Cистема пассивной защиты водителя и пассажиров
\item[VIN] --vehicle identification number, 17--значный идентификационный номер транспортного средства, соответствующий стандарту ISO 3779--2012.
%
\end{description}
%%%%%%%%%%%%%%%%%%%%%%%%%%%%%%%%%%%%%%%%%%%%%

\subsection{Методы исследования}

\begin{itemize}
\item  Органолептический метод – исследование и оценка качества объектов с помощью %органов чувств
%\item 	Прямой измерительный метод – путем измерения размеров деталей специальными %измерительными приборами
\item Расчётный метод (косвенный измерительный метод) – путём расчётов различных параметров на основе результатов измерений и других данных
\item Экспертный метод (метод экспертной оценки) — совокупности операций по выбору комплекса или единичных характеристик объекта, определению их действительных значений и оценкой экспертом соответствия их установленным требованиям и/или технической информации
\item Графоаналитический метод
%\item Метод масштабной  реконструкции
\end{itemize}
%%%%%%%%%%%%%%%%%%%%%%%%%%%%%%%%%%%%%%%

\subsection{Исходные данные и объекты исследования}

19.07.2016 г.в 13 час.  42 мин. в г. Белореченске по ул. Красная, 66 водитель Терехов А.В., управляя автомобилем Renault, г/н Е431НЕ33 допустил столкновение с автомобилем \тс регистрационный знак \грз под управлением водителя Шамояна Р.О.  
\par Для производства исследования представлено:
\begin{enumerate}
\item материалы гражданского дела № \delonum \, в том числе:
\begin{itemize}
	\item Акт осмотра № 6164 транспортного средства \тс\, составленный 29.07.2016 г.  специалистом  компании "РАНЭ" 
	\item Акт осмотра № 295 от 19 июля 2016 г., составленный индивидуальным предпринимателем Новиковым Олегом Николаевичем
	\end{itemize}
\item Электронные копии цифровых фотоснимков поврежденного автомобиля \тс\, VIN \vin\, предоставленные в  количестве 16 файлов формата .jpg с сохраненными техническими данными EXIF, выполненные 19.07.2016 г. цифровым фотоаппаратом Сanon PowerShot SX150 IS 
\item Электронная копия свидетельства о регистрации ТС 23 38 № 793117
\end{enumerate}


\subsection{Ранее по материалам дела выполнено}
\noindent Судебная автотехническая экспертиза, выполненная  экспертом Дереберя Н.В.\\
Повторная судебная автотехническая экспертиза, выполненная экспертом Алифриенко В.В.
%\subparagraph*{} Определением $\cdots$
%\subsection{Обстоятельства дела}
%\begin{itemize}
%\item $\cdots$
%
%\end{itemize}
%
\section{Исследование}
%
\subsection{История ремонта и сервисного обслуживания}

На основании предоставленных материалов составлена история ремонта и сервисного обслуживания транспортного средства \тс \, по датам и пробегу, Таблица \ref*{tab:hist}:

{\small 
\begin{longtable}{|p{16mm}|p{12mm}|p{29mm}|p{50mm}|p{41mm}|}
	\caption[]{\footnotesize {\textbf{История ремонта и сервисного обслуживания по дате и пробегу}}} \label{tab:hist}\\
	\hline
	%\rowcolor[HTML]{C0C0C0} 
	% Заголовки столбцов
	\textbf{Дата} &\textbf{Пробег, км} &\textbf{№\,Акта,Заказ-наряда, накладной}& \textbf{Вид работы}& \textbf{Примечание} \\ \hline \endhead % повторение заголовка 
	% Строки
	%
27.04.2018 & 5  & Заказ-наряд № 480261860-1 & Предпродажная подготовка   &  \\ \hline
%
07.05.2018 &  &  &   & Начало гарантии \\ \hline
%
23.01.2019 & 17\,467 & Акт приема № 480271247-  & ТО-15000, С 3 на 4 коробка зависает  & Перепробег 2\,460 км, низкий уровень масла и охлаждающей жидкости \\ \hline
%
23.01.2019 & 17\,467 &Заказ-наряд № 480271247-1  & ТО-1, адаптация АКПП  & Следующее ТО на пробеге 32 467 км, передние колодки заменить \\ \hline
%

%
03.09.2019 &33\,300  & Акт приема 
		№ 480279303- от 03.09.2019& & 02.09.2010 при движении в пробке на скорости 5-10 км/ч водитель услышал посторонний звук в районе ДВС. Уровень масла в ДВС меньше минимума, перепробег 833 км после предыдущего ТО \\ \hline
%		
03.09.2019 &33\,300  &Заказ-наряд № 480279303-1 от 03.09.2019& Диагностика ДВС  & Проверка согласно MESI по симптому № 21 <<Шум в двигателе>>\\ \hline
%
03.09.2019 &33\,300  &Акт проверки качества&Осмотр ТС с целью проверки его качества& п.5 ст. 18 ФЗ РФ "О защите прав потребителей \\ \hline
%
    %\rowcolor[HTML]{EFEFEF} 
10.09.2019 & 33\,300 & Заказ-наряд № 480279278-1  & Замена тормозной жидкости, замена воздушного фильтра салона, замена моторного масла и фильтра & Плановое ТО \\ \hline
	%%% ..............& 
\end{longtable}}%\setcounter{rownum}{0} % Обнуляем счетчик строк для следующей таблицы

\par 03.09.2019 автомобиль с посторонним стуком в ДВС на эвакуаторе доставлен  в сервисный центр ООО "Формула-МК" по адресу: г. Краснодар, ул. Аэропортовская, 4/1.  Первичная диагностика показала, что при увеличении оборотов до 2000 об/мин слышен стук в ДВС. При приеме ТС выявлено, что уровень масла ниже минимальной отметки, уровень охлаждающей жидкости на минимальном уровне, сигнализаторы или контрольные лампы на панели приборов не горят. Специалистами сервисного центра произведена замена масла, слито 3л масла, цвет масла темный, по субъективной оценке специалиста, выполнявшего замену масла, в слитом масле присутствовал запах бензина. Залито новое масло  до максимального уровня. После замены масла стук в ДВС не прошел. При считывании ошибок зафиксирована ошибка Р0524 (слишком низкое давление масла) на пробеге 32 674 км. Выполнена проверка согласно MESI по симптому № 21 <<Шум в двигателе>>. По итогам проверки, так как источник звука находится внутри ДВС, принято решение произвести частичную разборку для определения источника звука. Дополнительно выполнили проверку давления масла: нижний предел при 1500 об/мин - 2.4 бар; при 4500 об/мин - 4.4 бар. %\rem{ Какое давление масла должно быть по техдоку?} 
Проверили компрессию для данного двигателя (степень сжатия 14) 1ц -6.5 кг/см2; 2ц-6.5 кг/см2; 3ц-6.5 кг/см2; 4ц-6.0 кг/см2. Выполнили снятие поддона ДВС и нижних головок шатуна. Вкладыш 4-го цилиндра имеет задиры, шатунная шейка коленвала 4го цилиндра имеет задиры, вкладыши 2 и 3 цилиндров имеют задиры. В маслозаборнике присутствуют металлические частицы.  На основании вышеизложенного, специалистами сервисного центра причиной возникновения неисправности названа эксплуатация автомобиля  с уровнем масла ниже рекомендованного заводом изготовителем.

\subsection{Исследование предоставленных на экспертизу документов}

 \subparagraph*{}Из Электронной сервисной книжки  известна следующая информация об автомобиле, имеющая значение для дачи заключения:
	\begin{itemize}
		\item[ ] 
			\begin{description}
			\item[Марка, модель] -- Mazda 6-2.0 L
			\item[VIN] -- \vin
			\item[Год выпуска] --2018
			\item[Шасси] --отсутствует
			\item[Цвет ЛКП] --Deep Crystal Blue
			\item[Двигатель] --21096953 (модель) PE
			\item[Трансмиссия] -- 6EAT
			\item[ПТС] --
%						
		\end{description}
		\end{itemize}
	
	\subparagraph*{} Идентификационный код автомобиля (VIN) \vin\, содержит следующую информацию о транспортном средстве, имеющую значение для 	дачи заключения:
%
\begin{description}
%		
	\item[Дата изготовления] --
	\item[Двигатель] --
	\item [Расположенние руля] --
	\item[VDS] --
	
		\end{description}

%
\textit{Источник: https://ru.vindecoder.pl/\vin}
%
%Пробег автомобиля  расчетный, согласно [1]  составляет 214 000км.
  \begin{figure}[!h]
	\centering
	\includegraphics[width=0.65\linewidth]{images/cm1}
	\caption{{\footnotesize {Компоновка \тс. Иллюстрация Audatex}}}
	\label{ris:images/cm1}
\end{figure}



\subsection{Исследование транспортного средства}
%
Исследование автомобиля производилось экспертом с использованием производственных мощностей автообслуживающего предприятия дилерского сервисного центра в г. Краснодаре, ул. Аэропортовская, 4/1. При проведении осмотра присутствовали: \присутствовали. \\
Автомобиль предоставлен частично разобранным: демонтирован поддон двигателя, вкладыши коленчатого вала, шатунные катушки зажигания, свечи зажигания.  Отдельно представлена пластиковая емкость, содержащая 3 литра масла из двигателя ТС \тс.
На момент осмотра на автомобиле имеются повреждения переднего бампера снизу слева в виде задиров, крыло заднее правое  имеет царапины ЛКП, бампер задний справа имеет царапины ЛКП, имеется повреждение лобового стекла. Давление в шинах колес передней и задней оси 2.3 бар, шины BRIDGESTONE TURA NZA 225/55R17 97V



\subsection{Цель анализа эксплуатационных разрушений – установление}
характера и причин эксплуатационных разрушений, вызвавших разрушение узла или детали, поскольку разрушения могут возникать по многим причинам, например в результате износа или эрозии поверхности,
искажения формы, низкой твёрдости, превышения нагрузки.


{Эксперт}\hfill           {Фефелов С. Л.}

%\includepdf[pages=-]{myfile.pdf}
%\includepdf[pages=-]{calc.pdf}






%	
\begin{table}[H]
	\centering
	\caption{{\footnotesize Выполненные работы и услуги}}
	\label{tab:1}
	\begin{tabular}{|l|l|l|l|}
		\hline
		\rowcolor[HTML]{C0C0C0} 
		\multicolumn{1}{|c|}{\cellcolor[HTML]{C0C0C0}N п/п} & Наименование работ и услуг & Кол-во & Цена, руб \\ \hline
		\Rownum                                                & Диагностика компьютерная   & 1      & 1000      \\ \hline
		\rowcolor[HTML]{EFEFEF} 
		\Rownum                                                 & Замер компрессии FSA       & 1      & 1000      \\ \hline
		\Rownum                                                 & Снятие/установка форсунок  & 8      & 8000      \\ \hline
		\rowcolor[HTML]{EFEFEF} 
		
	\end{tabular}
\end{table}\setcounter{rownum}{0}

Итого на сумму:  37 300 (Тридцать семь тысяч триста) рублей
%
%

\begin{table}[H]
	\centering
	\caption{{\footnotesize Расходы по накладной к заказ-наряду 0000-000194 от 03.08.17}}
	\label{tab:2}
	\begin{tabular}{|l|l|l|l|}
		\hline
		\rowcolor[HTML]{C0C0C0} 
		\multicolumn{1}{|c|}{\cellcolor[HTML]{C0C0C0}N п/п} & Наименование запчасти (материала) & Кол-во & Цена, руб \\ \hline
		1                                                   & Шайба под форсунку TOYOTA   & 10      & 2500      \\ \hline
		\rowcolor[HTML]{EFEFEF} 
		2                                                   & Очиститель       & 5     & 1000      \\ \hline
	\end{tabular}
\end{table}

Итого на сумму:  3 500 (Три тысячи пятьсот) рублей 

Общая стоимость работ: 40 800 (Сорок тысяч восемьсот) рублей. 

\vspace{\baselineskip}  % вставка пустой строки

В исковом заявлении (Т1, л.д. 1--6) указано, что после ремонта автомобиля специалистами ООО "РЕГИОНТЕХЦЕНТР"  через непродолжительное время в двигателе автомобиля  \тс \, образовался посторонний, нефункциональный  шум.  Согласно заключения ИП Шаманского С.Н. для устранения неисправности ДВС необходимо было произвести следующие ремонтные работы:

\newpage


\begin{table}[H]
	\centering
	\caption{{\footnotesize Работы по заказ-наряду № Ш000011955 от 10.08.2017}}
	\label{tab:3}
	\begin{tabular}{|l|l|l|l|}
		\hline
		\rowcolor[HTML]{C0C0C0} 
		\multicolumn{1}{|c|}{\cellcolor[HTML]{C0C0C0}N п/п} & Наименование запчасти (материала) & Кол-во н/ч & Цена, руб \\ \hline
		1    & Материалы, использованные при подготовке а/м к ремонту   & 0,2      & 260      \\ \hline
		\rowcolor[HTML]{EFEFEF} 
		2    & Слесарные работы (ремонт освещения багажного оделения)       & 0,5     & 650    \\ \hline
		3    & Интеркулер снять/установить      & 1     & 1300      \\ \hline
		\rowcolor[HTML]{EFEFEF} 
		
	\end{tabular}
\end{table}
Итого на сумму  126 178 (Сто двадцать шесть тысяч сто семьдесят восемь) рублей
% 

\begin{table}[H]
	\centering
	\caption{{\footnotesize Запчасти и материалы к заказ-наряду № Ш000011955 от 10.08.2017}}
	\label{tab:4}
	\begin{tabular}{|l|ll|l|l|}
		\hline
		\rowcolor[HTML]{C0C0C0} 
		\multicolumn{1}{|c|}{\cellcolor[HTML]{C0C0C0}N кат} & Наименование запчасти (материала) & & Цена за шт. & Всего цена, руб \\ \hline
		7109113821    & Хомут пластиковый самозатяжной  & & 5      & 50      \\ \hline
		\rowcolor[HTML]{EFEFEF} 
		0411151042    & Ремкомплект ДВС 1VDFTV (1 шт.)      & & 16000     & 16000    \\ \hline
		1111551030С0    & Прогкладка ГБЦ правая  (1 шт.)    & & 3250     & 3250      \\ \hline
		\rowcolor[HTML]{EFEFEF} 
		
	\end{tabular}
\end{table}
Итого на сумму  327 203 (Триста двадцать семь тысяч двести три) рубля. \\
Всего ремонта по справке ИП Шаманского С.Н. на сумму:  397 203 (Триста девяносто семь тысяч двести три ) рубля.
% 
\vspace{\baselineskip}  % вставка пустой строки

Из заключения специалиста  

\vspace{\baselineskip}
%
%
Из заключения экспертов  

\vspace{\baselineskip}
\renewcommand\baselinestretch{0.86}\small\normalsize 
\subsection{\underline{По  вопросу}\, \, \,	\textbf{\small{1. "Опр"?}}}
\renewcommand\baselinestretch{1.2}\small\normalsize
На момент 

\begin{figure}[H]\centering
	\parbox[t]{0.49\textwidth}
	{\centering
		\includegraphics[width=.49\textwidth]{images/k1}
		\caption{\footnotesize {Поврежденное компрессорное колесо (крыльчатка турбины) и его  гайка  }}
		\label{ris:images/k1}}
	\hfil \hfil
	\parbox[t]{0.49\textwidth}
	{\centering
		\includegraphics[width=.49\textwidth]{images/k2}
		\caption{\footnotesize {Поврежденное компрессорное колесо (крыльчатка турбины) и стенки корпуса турбины
				рис. 11, 13, Т 1, л.д. 258, 259}}
		\label{ris:images/k2}}
\end{figure}

В качестве постороннего предмета истцом заявлен болт клапанной крышки ДВС, извлеченный специалистом ООО "ЭКСПЕРТ",  Рис.\ref{ris:images/b1}
%%
\vspace{\baselineskip}  % вставка пустой строки

\begin{figure}[!h]
	\centering
	\includegraphics[width=0.95\linewidth]{images/b1}
	\caption{{\footnotesize {Поврежденный болт клапанной крышки, основан на Рис. 12, Т 1, л.д. 258}}}
	\label{ris:images/b1}
\end{figure}

%   \begin{figure}[!h]
%	\centering
%	\includegraphics[width=0.85\linewidth]{images/b2}
%	\caption{{\footnotesize {Поврежденный болт клапанной крышки, вид с торца. Левая стрелка указывает на наклеп шляпки болта, правая- на повреждения торца резьбы детали}}}
%	\label{ris:images/b2}
%\end{figure}


\begin{figure}[!h]\centering
	\parbox[t]{0.49\textwidth}
	{\centering
		\includegraphics[width=.49\textwidth]{images/b2}
		\caption{\footnotesize {Поврежденный болт клапанной крышки, вид с торца. Левая стрелка указывает на наклеп шляпки болта, правая- на повреждения торца резьбы детали }}
		\label{ris:images/b2}}
	\hfil \hfil
	\parbox[t]{0.49\textwidth}
	{\centering
		\includegraphics[width=.49\textwidth]{images/g1}
		\caption{\footnotesize {Поврежденная гайка  компрессорного колеса (крыльчатки турбины)
				рис. 11, 13, Т 1, л.д. 258, 259}}
		\label{ris:images/g1}}
	
\end{figure}
%\begin{SCfigure}
%	\centering {\footnotesize \caption{Болт клапанной крышки, извлеченный из впускного тракта }} 
%	\includegraphics[]{images/b1}
%	\label{ris:images/b1}
%\end{SCfigure}


Эксперты ИП 
%

Из материалов дела следует, что .


\relax
\begin{figure}[h!]\centering
	\parbox[t]{0.49\textwidth}
	{\centering
		\includegraphics[width=.49\textwidth]{images/b3}
		\caption{\footnotesize {Сформированная фаска галтели поврежденного болта}}
		\label{ris:images/b3}}
	\hfil \hfil
	\parbox[t]{0.49\textwidth}
	{\centering
		\includegraphics[width=.49\textwidth]{images/b4}
		\caption{\footnotesize {Галтели неповрежденных болтов выраженных 
				фасок не имеют}}
		\label{ris:images/b4}}
\end{figure}

По совокупности результатов
\subparagraph*{}Таким образом, по совокупности признаков, эксперт приходит вероятностному  выводу о том, что повреждения колеса турбины (крыльчатки компрессора турбокомпрессора) могли быть образованы вследствие попадания постороннего предмета, а именно болта крышки клапанов, извлеченного специалистом ООО "ЭКСПЕРТ". 

\subparagraph*{}  Экспертами 


% 
%\begin{flushleft} 
% 	\hbox{% 
% 		\vrule\hspace{.8em}\parbox{1\textwidth}% 
% 		{ Согласно руководству по сервисному обслуживанию и ремонту TOYOTA LAND CRUISER 200 для проведения работ в объёме, указанном в заказ-наряде №0000-000194 от 03.08.2017 г. (л.д.9), составленном специалистом ООО «РЕГИОНЦЕНТР» (г. Краснодар) необходимо демонтировать агрегаты, находящиеся в пространстве между двигателем и передней панелью, в том числе патрубок подвода воздуха к компрессору левого турбокомпрессора исследуемого двигателя (где в дальнейшем был обнаружен такой посторонний предмет как крепёжный болт).}} 
%\end{flushleft}
% 
% 

%%      
%% \textbf{  Повреждения автомобиля \tcm,\, имеющиеся на момент осмотра 02.07.2018:} (рис. \ref{fig:merclz}, \ref{fig:51})  %\rem{Описания повреждений автомобилей}
%   \begin{itemize}{}{}
% 	 \vspace{-2mm}
%\subparagraph{title}
%\item Дверь передняя левая - компрессионная деформация  поверхности панели, деформация каракаса детали. В средней части лицевой панели, на площади $ \approx 1\, \text{дм}^2 $ глубокая вмятина с участками разрыва металла,   образованная в направлении снизу вверх и слева направо, ниже молдинга статический след  предмета прямоугольной формы  $ \approx 10$ x $20\, \text{см} $; 
%\item Молдинг двери передней левой - деформирован; 
%
% \end{itemize}   
%
%\vfill
%\begin{figure}[!h]
%	\centering
%	\includegraphics[width=0.85\linewidth]{images/51}
%	\caption{{\footnotesize Левая передняя часть автомобиля \tcm. Стрелки указывают на повреждения передней левой  части автомобиля}} \vspace{10mm} 
%	\label{fig:51}
%\end{figure}
%\vspace{5mm} 
%
%\textbf{Исследование повреждений ходовой части} проводилось с использованием ручного измерительного инструмента и стенда для измерения углов установки колес. 
%\begin{SCfigure}
%	\centering {\footnotesize \caption{ Автодата. Схема рычагов задней подвески }} 
%	\includegraphics[width = 0.4 \textwidth] % 
%	{images/p1} % picture filename 
%\end{SCfigure}
%\begin{SCfigure}
%	\centering {\footnotesize \caption{ Подвеска автомобиля \tcm\, левая повреждённая сторона. Стрелками показаны  деформированные рычаги }} 
%	\includegraphics[width = 0.4 \textwidth] % 
%	{images/50} % picture filename 
%\end{SCfigure}
%\relax
%\begin{figure}[h!]\centering
%	\parbox[t]{0.49\textwidth}
%	{\centering
%		\includegraphics[width=.49\textwidth]{images/p1}
%		\caption{\footnotesize {Автодата. Схема рычагов задней подвески}}
%		\label{fig:p1}}
%	\hfil \hfil
%	\parbox[t]{0.49\textwidth}
%	{\centering
%		\includegraphics[width=.49\textwidth]{images/50}
%		\caption{\footnotesize { Подвеска автомобиля \tcm\, левая повреждённая сторона. Стрелками показаны  деформированные рычаги }}
%		\label{fig:50}}
%\end{figure}
%\begin{figure}[h!]\centering
%	\parbox[t]{0.49\textwidth}
%	{\centering
%		\includegraphics[width=.49\textwidth]{images/47}
%		\caption{\footnotesize {Измерение углов установки колес}}
%		\label{fig:47}}
%	\hfil \hfil
%	\parbox[t]{0.49\textwidth}
%	{\centering
%		\includegraphics[width=.49\textwidth]{images/48}
%		\caption{\footnotesize { Экран стенда измерения углов установки колес }}
%		\label{fig:48}}
%\end{figure}
%%
%\begin{figure}[h!]
%	\centering
%	\includegraphics[width=0.9\linewidth]{images/50}
%	\caption[]{{\footnotesize Подвеска автомобиля \tcm\, левая повреждённая сторона. Стрелками показаны  деформированные рычаги}}
%		\label{fig:50}
%\end{figure}
%
%
%
%\begin{figure}[h!]
%	\centering
%	\includegraphics[width=0.85\linewidth]{images/50}
%	\caption[]{{\footnotesize Подвеска автомобиля \tcm\, левая повреждённая сторона. Стрелками показаны  деформированные рычаги}}
%	\label{fig:50}
%\end{figure}
%
%\vspace{10mm}
%   
% \subparagraph*{}\textbf{На предоставленном автомобиле} \tcm\  на момент осмотра раскрыты левая передняя боковая подушка безопасности и левая головная подушка безопасности, Рис. \ref{fig:52}. Система оконных подушек безопасности входит в базовую
% комплектацию модельного ряда W211. Оконные и боковые подушки безопасности срабатывают в том  случае, если центральный электронный блок управления ARMADA регистрирует боковое столкновение. Для определения поперечного  ускорения поступающая от центрального датчика столкновения  информация дополняется информацией от боковых датчиков,
% расположенных в зонах боковых поперечин соответствующих  сторон.
% 
% В результате произведенной проверки электрических цепей системы SRS на наличие повреждений, коррозии, нарушения контактов в   разъемных соединениях  неисправности проводки электрических цепей системы SRS не выявлено. Блок управления системы безопасности в реальном времени нормально реагирует на внешние тестовые воздействия.
% \vspace{3mm}
%\begin{figure}[!h]
%	\centering
%	\includegraphics[width=0.85\linewidth]{images/52}
%	\caption{{\footnotesize Автомобиль \tcm с раскрытыми надувными элементами подушек безопасности}}
%	\label{fig:52}
%\end{figure}
%
На автомобиле имеются характерные повреждения 

% \begin{figure}
% 	\centering
% 	\includegraphics[width=0.95\linewidth]{images/screenshot001}
% 	\caption{{\footnotesize Данные блока управления панели комбинации приборов, зафиксированные в процессе исследования автомобиля}}
% 	\label{fig:screenshot001}
% \end{figure}
% \vspace{5mm}
% \begin{figure}[!h]
% 	\centering
% 	\includegraphics[width=0.95\linewidth]{images/d1}
% 	\caption{\footnotesize Снимок экрана компьютера в процессе диагностики системы управления} 	\label{fig:d1}
% \end{figure}
%{\small \begin{enumerate}{\label{en:enum}}
%		\item [] Ошибки системы SRS:
%	\item 92A3 - высокое сопротивление запального контура левой оконной подушки безопасности. Начало ошибки -- ошибка снята: 19 109 126 мотосекунда -- 19 138 962 мотосекунда или 5 308,09 моточасов -- 5 316,37 моточасов);
%	\item 9223 - высокое сопротивление запального контура левой боковой подушки безопасности: 19 109 126 --33 554 430 мотосекунд или 5 308,09 -- 9 320,67 моточасов;
%	\item 92А0 - замыкание (или утечка)на массу в цепи левой оконной подушки безопасности:  19 138 964-- 33 554 430 мотосекунд или  5 316,37 --9 320,67 моточасов.
%\end{enumerate}}
%
%\vspace{-14mm}
%\begin{center}
%	%\renewcommand{\arraystretch}{0.8}ommand{\arraystretch}{0.8}
%%\begin{tabular}{|p{20mm}|p{7cm}|p{25mm}|p{25mm}|}
%%	\hline 
%%	{\footnotesize Код ошибки} & {\footnotesize Описание ошибки} & {\footnotesize Время возникновения, сек} &{\footnotesize  Время окончания, сек} \tabularnewline
%%	\hline 
%%	92A3 & {\small Высокое сопротивление запального контура оконной подушки безопасности} & 19109126 & 19138962 \tabularnewline
%%	\hline 
%%	9233 & {\small Высокое сопротивление запального контура боковой подушки безопасности} & 19109126 & 3355430 \tabularnewline
%%	\hline 
%%	92A0 & {\small Замыкание  в цепи левой оконной подушке безопасности} & 19138964 & 33554430 \tabularnewline
%%	\hline 
%%%	\label{t:pb}
%%\end{tabular}
%%\renewcommand{\arraystretch}{1.2}
%%\begin{tabular}{|p{20mm}|p{7cm}|p{25mm}|p{25mm}|}
%%	\hline 
%%	{\footnotesize Код ошибки} & {\footnotesize Описание ошибки} & {\footnotesize Время возникновения, сек} &{\footnotesize  Время окончания, сек} \tabularnewline
%%	\hline 
%%	92A3 & {\small Высокое сопротивление запального контура оконной подушки безопасности} & 19109126 & 19138962 \tabularnewline
%%	\hline 
%%	9233 & {\small Высокое сопротивление запального контура боковой подушки безопасности} & 19109126 & 3355430 \tabularnewline
%%	\hline 
%%	92A0 & {\small Замыкание  в цепи левой оконной подушке безопасности} & 19138964 & 33554430 \tabularnewline
%%	\hline 
%%%	\label{t:pb}
%%\end{tabular}
%%\renewcommand{\arraystretch}{1.2}}
%\end{center}
%\vspace{3mm}
%Высокое электрическое сопротивление цепи запального контура указывает на активацию газогенератора (пиропатрона). По данным блока SRS, активация газогенераторов обеих подушек безопасности произошла одновременно на 19109126 мотосекунде.  Снятие ошибки 92А3 совпадает (разница 2 секунды) с моментом возникновения ошибки 92А0 (одновременно с образованием замыкания на массу в цепи левой оконной подушки). Т.е. неисправность электрической цепи  системы управления головной подушкой безопасности в виде короткого замыкания или утечки на массу зафиксировано на 8 часов позднее, чем произошла активация системы SRS и, согласно этим данным, не может являться причиной нештатного раскрытия надувного элемента головной подушки.    
%Считанные показания счетчика моточасов работы из блока управления панели приборов -- 53 19,02 ч. или $ \approx 19 148 472 $ мотосекунды, при том, что в блоке SRS содержатся сведения о снятии ошибки на 33 554 430 мотосекунде или 9 320,67 моточасе  работы автомобиля, что есть существенное расхождение во времени работы автомобиля  в различных блоках управления.  Показания одометра на 10.11.2017г. составляли 89 855 км, (акт осмотра ИП Резенькова, л.~д. 30), показания одометра на момент настоящего исследования так же 
%89 855км. Данный факт указывает либо на вмешательство в  блоки управления автомобиля, либо на наличие неисправных блоков управления.  Таким образом, техническое состояние электрических систем исследуемого автомобиля на момент экспертизы не позволяет достоверно определить соответствие срабатывания боковых подушек безопасности заявленному ДТП \dtp.  
%\subsection*{}
%%\vspace{-10mm}
%\vspace{-10mm}\textbf{  Повреждения автомобиля \tca,\, имеющиеся на момент осмотра 02.07.2018:}   
%
%\begin{itemize}
%	
%	\item Капот - остаточная деформация передней части,\, деталь частично восстановлена; 
%	\item Фара левая - заменена;
%	\item Фара правая - разрушен корпус;
%	\item Бампер передний - расколот в левой части, остаточные признаки пластической деформации правой части детали, отслоение ЛКП; 
%	\item Фара правая противотуманная - разбита;
%	\item Облицовка передняя - деформирована;
%	\item Панель рамки радиатора - частично восстановлена геометрия, имеются остаточные деформации;
%	\item Лонжероны передние левые - имеют остаточные деформации передней части деталей;
%	\item Пластина переднего регистрационного знака - деформирована;
%	\item Крыло переднее левое - остаточная деформация, разрывы металла передней угловой части;
%	\item Крыло переднее правое - остаточная деформация, разрывы металла передней угловой части.\\
%	
%\end{itemize}
%\imgh{150mm}{images/38}{Автомобиль \tca\, вид спереди. Расстояние между передними крыльями 1310 мм. Передний верхний угол левого и правого крыла расположены на высоте  720 мм от опорной поверхности.}
%
%\imgh{165mm}{images/mercl}{Автомобиль \tcm\, левая боковина. Расстояние между повреждениями 1310 мм. Расположены на высоте 620...680 мм и 610...670 мм} 
%
%  	Следуя логической последовательности разрешения вопросов настоящей экспертизы,     первоначально необходимо установить механизм взаимодействия автомобилей при заявленных обстоятельствах,  исследовать повреждения транспортных средств,   произвести сопоставление имеющихся повреждений с  механизмом их образования, определить возможность образования повреждений при заявленном событии и только затем определить стоимость восстановительного ремонта повреждений автомобиля \tcm.  Таким образом, эксперт первоначально должен провести  исследование по второму вопросу.
%  
%  \renewcommand\baselinestretch{0.85}\small\normalsize 
% \subsection{\underline{По  вопросу}\, \, \,	\textbf{\small{2. Состоят ли в причинно-следственной связи повреждения транспортного средства Мерседес Бенц регистрационный знак Р781ЕХ93 2003 года выпуска, заявленного истцом с ДТП, имевшем место 31.10.2017г.?}}}  
% \renewcommand\baselinestretch{1.2}\small\normalsize
% 
% %\vspace{-10mm}
%\subparagraph*{}	\parindent=2em  В соответствии с теорией автомобиля и законами механики, взаимодействие  транспортных средств возможно  определёнными зонами	 контактирования на площади перекрытия в вертикальных и горизонтальных плоскостях, исходя из конструктивных особенностей компоновки кузова при геометрическом ориентировании на проезжей части. Соответственно,  образование повреждений на транспортных средствах возможны  в  закономерных направлениях – в продольном направлении: спереди назад или сзади вперёд; в поперечном направлении: слева направо или справа налево;  в комбинированных направлениях при определённых условиях взаимодействия, исходя из особенностей динамики перемещения: спереди назад и сверху вниз, спереди назад и снизу вверх; сзади вперед и сверху вниз; сзади вперед и снизу вверх. Определение механизма столкновения транспортных средств\footnote{С.Я. Евтюков, Я.В. Васильев\,/ Экспертиза ДТП: Методы и технологии/ СПбГАСУ.-Спб., 2012.-310с. ISBN 978-5-9227-0426-7} включает в себя установление траекторий схождения и расхождения транспортных средств; угла между продольными осями транспортных средств в момент их первичного контактного взаимодействия; частей транспортных средств, которыми они впервые вступили в контактное взаимодействие; площади перекрытия контактирующих при ДТП частей транспортных средств; факта состояния покоя или движения транспортных средств в момент первичного контактного взаимодействия; координат места столкновения и расположения транспортных средств относительно неподвижных элементов дороги.% Механизм столкновения устанавливается по следам на транспортных средствах и месте ДТП. Взаимное положение транспортных средств в момент первичного контактного взаимодействия определяется методом натурной реконструкции события ДТП (совмещение и сопоставление пар повреждений на транспортных средствах участвовавших в ДТП) либо при отсутствии такой возможности, по протоколам осмотра транспортных средств и фотографиям их повреждений, приобщенным к материалам дела. По удовлетворению ходатайства \hod транспортные средства на исследование представлены, что позволило эксперту провести натурную реконструкцию взаимодействия транспортных средств, в соответствии заявленному механизму  ДТП. %Установленный механизм взаимодействия автомобилей \tca и \tcm  не противоречит заявленным обстоятельствам ДТП и соответствует условиям перекрестного поперечного столкновения, обусловленного пересечением траекторий движения автомобилей.\,Проведённой в процеcсе исследования натурной реконструкцией столкновения представленных автомобилей   установлено, что правая боковая сторона автомобиля мерседес--бенц Е220  гос.рег.знак Р781ЕХ93 содержит  следы--отображения внешнего строения деталей передней части автомобиля ауди--80 регистрационный знак В393НУ12.
%\vspace{-10mm}
% \subsection*{}
%      \textbf{В результате детального изучения следов на предоставленных ТС  эксперт выделяет следующие контактные пары:}
%\vspace{2mm}
%%\begin{itemize}
%	\item рамка переднего номерного знака \tca и дверь задняя левая \tcm, рис. \ref{ris:images/11}, \ref{ris:images/13}.
%	\item  левая сторона переднего бампера с отверстием  для противотуманной фары  \tca --- дверь передня левая \tcm, рис. \ref{ris:images/46}, \ref{ris:images/32}.
%	\item повреждение левой передней двери \tcm, полученное от контактного взаимодействия с левым передним крылом \tca, рис. \ref{ris:images/35}, \ref{ris:images/36}.
%	\item повреждение левого заднего крыла \tcm, полученное от контактного взаимодействия с правым передним крылом \tca, рис. \ref{ris:images/33}, \ref{ris:images/34}.
%	\item повреждения левой передней и левой задней двери \tcm компрессионного характера - передний бампер ТС, капот, левая и правая фары \tca, рис. \ref{ris:images/37}, \ref{ris:images/38}. 
%	
%\end{itemize}
%\relax
%\begin{figure}[h!]\centering
%	\parbox[t]{0.49\textwidth}
%	{\centering
%		\includegraphics[width=.49\textwidth]{images/1}
%		\caption{\footnotesize {Этапы реконструкции ДТП}}
%		\label{ris:images/1}}
%	\hfil \hfil%раздвигаем боксы по горизонтали 
%	\parbox[t]{0.49\textwidth}
%	{\centering
%		\includegraphics[width=.49\textwidth]{images/39}
%		\caption{\footnotesize {Этапы реконструкции ДТП}}
%		\label{ris:images/39}}
%\end{figure}
%\relax
%\begin{figure}[h!]\centering
%	\parbox[t]{0.49\textwidth}
%	{\centering
%		\includegraphics[width=.49\textwidth]{images/11}
%		\caption{\footnotesize {След рамки номерного знака}}
%		\label{ris:images/11}}
%	\hfil \hfil%раздвигаем боксы по горизонтали 
%	\parbox[t]{0.49\textwidth}
%	{\centering
%		\includegraphics[width=.49\textwidth]{images/13}
%		\caption{\footnotesize {След воздействия правого переднего крыла автомобиля \tca на поверхность крыла заднего левого автомобиля \tcm}}
%		\label{ris:images/13}}
%\end{figure}
%\relax
%\begin{figure}[h!]\centering
%	\parbox[t]{0.49\textwidth}
%	{\centering
%		\includegraphics[width=.49\textwidth]{images/46}
%		\caption{\footnotesize {левая сторона переднего бампера с отверстием  для противотуманной фары  \tca}}
%		\label{ris:images/46}}
%	\hfil \hfil%раздвигаем боксы по горизонтали 
%	\parbox[t]{0.49\textwidth}
%	{\centering
%		\includegraphics[width=.49\textwidth]{images/32}
%		\caption{\footnotesize {дверь передня левая \tcm}}
%		\label{ris:images/32}}
%\end{figure}
%\relax
%\begin{figure}[h!]\centering
%	\parbox[t]{0.49\textwidth}
%	{\centering
%		\includegraphics[width=.49\textwidth]{images/35}
%		\caption{\footnotesize {Левая передняя дверь автомобиля  \tcm}}
%		\label{ris:images/35}}
%	\hfil \hfil%раздвигаем боксы по горизонтали 
%	\parbox[t]{0.49\textwidth}
%	{\centering
%		\includegraphics[width=.49\textwidth]{images/36}
%		\caption{\footnotesize {Левое переднее крыло автомобиля \tca}}
%		\label{ris:images/36}}
%\end{figure}
%
%\relax
%\begin{figure}[h!]\centering
%	\parbox[t]{0.49\textwidth}
%	{\centering
%		\includegraphics[width=.49\textwidth]{images/33}
%		\caption{\footnotesize {Левое заднее крыло автомобиля \tcm}}
%		\label{ris:images/33}}
%	\hfil \hfil%раздвигаем боксы по горизонтали 
%	\parbox[t]{0.49\textwidth}
%	{\centering
%		\includegraphics[width=.49\textwidth]{images/34}
%		\caption{\footnotesize {Левое переднее крыло автомобиля \tca}}
%		\label{ris:images/34}}
%\end{figure}
%\relax
%\begin{figure}[h!]\centering
%	\parbox[t]{0.49\textwidth}
%	{\centering
%		\includegraphics[width=.49\textwidth]{images/37}
%		\caption{\footnotesize {Левые двери автомобиля \tcm. Следовоспринимающая поверхность}}
%		\label{ris:images/37}}
%	\hfil \hfil%раздвигаем боксы по горизонтали 
%	\parbox[t]{0.49\textwidth}
%	{\centering
%		\includegraphics[width=.49\textwidth]{images/23}
%		\caption{\footnotesize {Передний бампер, капот, левая и правая фары \tca. Следообразующая поверхность}}
%		\label{ris:images/23}}
%\end{figure}
%%\rem{Возможно, необходимо перенести в другой раздел}
%\subparagraph*{} Имеющиеся следы комбинированные, статические, направленные перпендикулярно продольной оси автомобиля \tcm с незначительной динамической составляющей  следов внедрения, направленные снизу вверх  и немного слева направо (под углом $ \approx 30^{\circ} $) к вертикальной оси автомобиля.%  рис. \ref{ris:images/1}, \ref{ris:images/39}. \rem{фото характерных следов}
%
%Совокупность индивидуализирующих признаков групп следов и повреждений, имеющихся  на транспортных средствах, совпадающих между собой по уровню расположения, форме и  локализации указывают на   возможность их образования при взаимном контактном взаимодействии по механизму перекрёстного взаимодействия  с центральным ударом.  Различие по высоте ($ \approx 9  \cdots 10 $ см ), в данном случае,  не противоречит  механизму следообразования, так как  закономерность направления деформирующего воздействия со стороны следообразующего объекта на части кузова транспортного средства обусловлены  геометрическими размерами  рассматриваемых объектов, траекторией их перемещения, а также созданием определенных моментов сил,  в том числе вызывающих перераспределение массы по осям, рис.27\ref{ris:images/tormoz}.   Полагаем, что водитель автомобиля \tca, \, перед столкновением с автомобилем \tcm, применил торможение.  Тогда в  момент первичного контакта передняя часть автомобиля  \tca \ была расположена на несколько сантиметров ниже относительно расстояния при нормальных условиях движения или неподвижного состояния, так как согласно законам механики,  в момент торможения происходит перераспределение веса под действием силы
%  \begin{equation}\label{eq:f}
%  \vec{F_t} = \dfrac{\vec{a}*m*H}{L}, \,\,\,\,  \text{где:}
%  \end{equation}
%  \begin{itemize}
%  \item[ ] $\vec{a} $ ---величина замедления;
%  \item[ ] $ L $ --- длина базы автомобиля;  
%  \item[ ] $ H $ --- высота центра тяжести; 
%   \item[ ] $ m $ --- масса автомобиля
%  \end{itemize}
% Действие силы $ \vec{F_t} $   на переднюю ось автомобиля  приводит к дополнительному сжатию пружин передней подвески и, как следствие, уменьшению расстояния от деталей передней части кузова до опорной поверхности. Величина уменьшения расстояния зависит от значения силы $ \vec{F_t} $ и жёсткости  подвески автомобиля.
%В свою очередь,  в начальный момент контакта, положение кузова автомобиля \tcm \, должно было  соответствовать  положению при нормальном  условии движения, далее, под действием силы $ \vec{F_t} = \dfrac{mv^2}{2} $ , где $ m $ -- масса автомобиля \tcm, v -- его скорость, пространственное положение кузова \tcm измениться.
%\imgh{100mm}{images/torm}{Схематичное изображение перераспределения веса при торможении автомобиля }
%\begin{SCfigure}
%\centering \caption{рлрл captioорпоп опрпоп опорпопр опопо  порпорп орор оро опрорп орп ор оорп орп оn text ... } 
%\includegraphics[width = 0.6 \textwidth] % 
%{images/torm} % picture filename 
%\end{SCfigure}
%Анализ характера деформаций и направлений действующих сил, вызвавших повреждения частей деталей правой боковины автомобиля \tcm \, и передней части автомобиля \tca, а именно:
%\begin{itemize}
%	\item[ ] --  обширные площади деформации на ТС в местах, которыми они вошли в контактное взаимодействие с преградой;
%	\item[ ] --  оттиски отдельных участков, деталей одного ТС на поверхности частей другого;
%	\item[ ] --  следы внедрения в виде потёртостей на ТС -- снятия слоя лакокрасочного покрытия, локальных разрывов поверхности;
%	\item[ ] -- трассы (следы скольжения, давления, царапания), возникшие от контакта с другим ТС
%\end{itemize} 
%позволяют заключить, что данные повреждения могли быть образованны в результате  взаимного контактного взаимодействия указанных автомобилей.\\
%Данный механизм следообразования является характерным для \rem{вставить механизм следообразвания}
%Таким образом, сопоставлением характерных групп следов, местом и направлением их нанесения, пространственным совмещением выделенных следовоспринимающих и следообразующих объектов можно сделать общий вывод о наличии пространственно-следового изоморфизма, и соответственно о наличии контактно-следового взаимодействия автомобилей \tca и \tcm.
%
%	Учитывая заявленные обстоятельства ДТП \dtp, исходя из взаимного ориентирования обоих транспортных средств на проезжей части  соответствующими сторонами кузова, а также направления траектории сближения обоих транспортных средств по принципу организации дорожного движения, повреждения левой боковой части автомобиля \tcm могут состоят в причинно-следственной связи с указанным  ДТП. 
%Повреждения деталей передней части автомобиля \тсм, согласно предоставленным административных материалов по данному ДТП, получены в результате наезда автомобиля на дерево, расположенное  на расстоянии 4,6 м от края проезжей части и на удалении $ \approx 10 $ м от места столкновения автомобилей, рис.2. 
%На момент исследования, на автомобиле \тсм имеются остаточные признаки упругой деформации деталей передней левой части автомобиля, по совокупности морфологических признаков не противоречащие заявленному механизму образования. Положение автомобиля \тсм, указанное на схеме ДТП, должно быть обусловлено изменением траектории движения автомобиля после столкновения на угол, составляющий $ \approx 40^\circ $ вправо от направления  движения до столкновения автомобилей. При этом, схема места дорожно-транспортного происшествия не содержит следов потери устойчивости автомобиля после столкновения. Из  анализа механизма взаимодействия автомобилей, эксперт приходит к заключению, что положение автомобиля \тсм, указанное  на схеме ДТП не является результатом изменения траектории его  движения  вследствие удара, а вероятно, связано с управляющими действиями водителя, его ответной реакцией на столкновение.
%Следовательно, повреждения деталей передней части автомобиля \tcm могут находится в причинно-следственной связи с указанным  ДТП.% так как  характер повреждений  не противоречит заявленным обстоятельствам. 
%\subparagraph*{}Таким образом, по совокупности признаков, эксперт приходит к выводу о том, что  повреждения транспортного средства Мерседес Бенц регистрационный знак Р781ЕХ93 2003 года выпуска  состоят  в причинно-следственной связи с  ДТП, имевшем место 31.10.2017г.
%{\begin{enumerate}
%		
%	\item Наличие чётких отпечатков частей одного ТС на другом в местах их первичного контакта при отсутствии трасс местах образования отпечатков или при наличии трасс, возникших после образования отпечатков \rem{еще ремарка по ударв неподвижное тс}
%	\item  Совпадение направления первоначальных трасс и деформаций на ТС, по которому был нанесен удар при перекрёстном столкновении,  направлением движения другого ТС
%	\item Расположение трасс тангенциальной направленности на боковой поверхности колес
%	\item Разворот ТС в направлении момента, который мог возникнуть при столкновении только в случае движения тс, по которому был нанесён удар \\
%	
%	 
%\end{enumerate}}
%Необходимо отметить \rem{опять про изоморфизм}, что данный вид трасологических исследований (установление пространственно-следового изоморфизма, а
%именно установление факта контактно-следового взаимодействия,
%сопоставление обстоятельств ДТП, заявленных страхователем с механизмом нанесения повреждений и т.д.) в настоящее время становится все более и более востребованным и актуальным, т.к. в современных условиях складывается ситуация, когда мошенничество на
%транспорте, с целью получения страхового возмещения, приобретает объёмы снежного кома.
%%%%%%%%%%%%%%%%%%%%%%%%%%%%%%%%%%%%%%%%%%%%%%%%%%%%%%%%%%%%%%%%%%%%%%%%%%%%%%%%%%%%%%%%%%%%%%%%%%%%%%%%%%%%%%%%%%%%%%%%%%%%%
\renewcommand\baselinestretch{0.86}\small\normalsize 
\subsection{\underline{По  вопросу}\, \, \,	\textbf{\small{"Определить стоимость восстановительного ремонта с учетом износа стоимости запасных частей"?}}}
\renewcommand\baselinestretch{1.2}\small\normalsize
Колесное транспортное средство сроком эксплуатации более 7 лет относится к категории транспортных средств с граничным сроком эксплуатации [1], для которой возможно применение ремонтных операций при условии экономической целесообразности и  технической возможности.  

В соответствии с принятой экспертной методикой [1], стоимость восстановительного ремонта АМТС  $ C_p $ определяется по формуле:
%
\begin{equation}\label{eq:r}
C_\text{вp} =C_p + C_\text{м} + C_\text{зч}\cdot\left( 1-\frac{\text{И}}{100}\right)  \,\,\,\, \text{где:}
\end{equation}
%
%
\begin{itemize}
	%	
	\item[ ]$C_\text {р} $ --  стоимость ремонтных работ по восстановлению КТС, руб.;
	\item[ ]$ C_\text{м} $ --  стоимость необходимых ремонтных материалов, руб.;
	\item[ ]$ C_\text{зч} $ --  стоимость новых запасных частей, руб;
	\item[ ] $ \text{И} $ -- коэффициент износа составной части, подлежащей замене, \%.
\end{itemize}
%
%
Коэффициент износа составных частей (И) КТС (кроме автобусов и грузовых автомобилей) при определении стоимости восстановительного ремонта расчитывается по формуле:

\begin{equation}\label{eqsnos}
\text{И} =\text{И1}\cdot\text{П}+\text{И2}\cdot \text{Д}, \%  \,\,\,\, \text{где:}
\end{equation}

\begin{itemize}
	\item [] $ \text{И1} $ --усредненный показатель износа на 1000 км пробега, \%; 
	\item [] $ \text{П} $ -- общий пробег (фактический или расчетный) за срок эксплуатации КТС, тыс.км;
	\item [] $ \text{И2} $ -- усредненный показатель старения за 1 год эксплуатации, \%;
	\item [] $ \text{Д} $ -- срок эксплуатации КТС (от даты изготовления КТС до момента, на который определяется износ), лет. 
\end{itemize}

Для исследуемого автомобиля \тс, согласно справочным таблицам [1]:
\begin{equation}\label{eqsnosr}
\text{И} =\text{И1}\cdot\text{П}+\text{И2}\cdot \text{Д} = 0.23\cdot 214  + 0.85\cdot 9 = 49 \, \%
\end{equation}
%%%%%%%%%%%%%%%%%%%%%%%%%%%%%%%%%%%%%%%%%%%%%%%%%%%%%%%%%%%%%%%%%%%%%%%%%%%%%%%%%%%%%%%%%%%%%%%%%%%%%%%%%%%%%%%%
%Стоимость восстановительных работ $ C_{\text{вр}} $ определяется на основании норм трудоёмкостей $ T_i $, \,предусмотренных заводом-изготовителем, и стоимостных параметров $ C_{i\text{нч}} $ (стоимости нормо-часа) работ по техническому обслуживанию и ремонту АМТС. 
%\begin{equation}\label{eq:cr}
%C_{\text{вр}}  =\sum{C_{ip}}= \sum\left({T_{ij}}\cdot {C_{i\text{нч}}}\right) + \sum{C_{ip^{\text{\,\,\,руб}}}} , \,\,\,\text{где:} 
%\end{equation}
%\vspace{2mm}
%\begin{itemize}
%	\item[ ]$ C_{ip} $ -- стоимость работ i-го вида: $C_\text {зам} $, $ C_\text{восст} $, $ C_\text{рег} $, $C_\text{контр} $, $ C_\text{антикор} $, $ C_\text{зч} $, $ C_\text{ом} $,$ C_\text{соп} $, $ C_\text{вм} $, руб;
%	\item[ ]$ T_{ij} $ -- трудоёмкость j-й операции(комплекса) по i-му виду работ, руб;
%%	\item[ ]$ C_{i\text{нч}} $ -- стоимость нормо-часа по i-му виду работ, руб;
%	\item[ ]$ C_{ip^\text{\,\,руб}} $ -- стоимость работ $ C_{ip} $, принятая непосредственно в денежном выражении, руб.
%\end{itemize}
%
%\vspace{5mm}
%При определении стоимости восстановительного ремонта АМТС с учётом износа под износом следует понимать количественную меру физического старения АМТС и его элементов, достигнутого в результате эксплуатации, т.е. эксплуатационный износ.\\
%%
%Расчёт износа производится в  соответствии с Положением Банка России от «19» сентября 2014 года № 432-П «О единой методике определения размера расходов на восстановительный ремонт в отношении повреждённого транспортного средства» [3].
%Износ комплектующих изделий (деталей, узлов, агрегатов) рассчитывается по следующей формуле:
%
%\begin{equation}\label{eq:I}
%\text{И}_{\text{ки}} 
%= 100\cdot\left( 1-e^ {-\left( \Delta_{T} \cdot T_{\text{КИ}} + \Delta_{L} \cdot L_{\text{КИ}} \right)}\right), \,\,\,\,\text{где:}   
%\end{equation}
%
%\begin{itemize}
%	\item[ ]$ \text{И}_{\text{ки}} $ -- износ комплектующего изделия (детали, узла, агрегата) (процентов); 
%	\item[ ]$ e $ -- основание натуральных логарифмов (e =  2,72);;
%	\item[ ]$ \Delta_{T}$ --  срок эксплуатации комплектующего изделия (детали, узла, агрегата) (лет);
%	\item[ ]$ T_{\text{КИ}} $ -- стоимость работ $ C_{ip} $, принятая непосредственно в денежном выражении, руб
%	\item[ ]$ \Delta_{L} $ --коэффициент, учитывающий влияние на износ комплектующего (детали, узла, агрегата) величины пробега транспортного средства с этим комплектующим изделием;
%	\item[ ]$ L_{\text{КИ}} $ --пробег транспортного средства на дату дорожно-транспортного происшествия (тысяч километров).  
%		\end{itemize}
%\vspace{5mm}
%Значения коэффициентов $ \Delta_{T}$  и $ \Delta_{L} $  для различных категорий и марок транспортных средств приведены в п.5. исп. лит~[3]. При этом, на комплектующие изделия (детали, узлы, агрегаты), которые находятся в заведомо худшем состоянии, чем общее состояние транспортного средства в целом, и его основные части, вследствие влияния факторов, не учтённых при расчете износа (например, проведение ремонта с нарушением технологии, не устранение значительных повреждений лакокрасочного покрытия), может быть начислен дополнительный индивидуальный износ. 
%Износ шины транспортного средства рассчитывается по следующей формуле:
%\begin{equation}\label{eq:sh}
%\text{И}_{\text{ш}} = \frac{\text{Н}_{\text{н}}-\text{Н}_{\text{ф}}}{\text{Н}_{\text{н}}-\text{Н}_{\text{доп}}} \cdot{100}\%  \,\,\,\,\text{где:} 
%\end{equation}
%
%\begin{itemize}
%	\item[ ] $ \text{И}_{\text{ш}} $ -- износ шины, \%;
%	\item[ ] $ \text{Н}_{\text{н}} $ -- высота рисунка протектора новой шины, мм;
%	\item[ ] $\text{Н}_{\text{ф}} $ -- фактическая высота рисунка протектора шины, мм;
%	\item[ ] $ \text{Н}_{\text{доп}} $ --минимально допустимая высота рисунка протектора шины в соответствии с требованиями законодательства Российской Федерации, мм.
%\end{itemize}
%
%\renewcommand\baselinestretch{1}\small\normalsize
%
%Износ шины дополнительно увеличивается для шин с возрастом от 3 до 5 лет - на 15 процентов, свыше 5 лет - на 25 процентов.
%
В результате исследования   экспертом установлено, что для устранения повреждений \тс \, необходимо  выполнить следующие  работы:
\begin{center}
	\begin{tabulary}{\textwidth}{LCL}
		\hline 
		\textbf{Наименование детали}      &   & \textbf{Ремонтное воздействие}\\
		\hline    
		Турбина левая     &   &    Заменить\\
		Блок ДВС    &   &    Отремонтировать гильзованием, заменой колец и прокладок \\
		Головка блока цилиндров & & Восстановить седла клапанов пятого цилиндра \\
		Впускной тракт & &    Разобрать, прочистить\\
		Интеркулер   & &     прочистить\\
		
		
	\end{tabulary}  
\end{center}
\renewcommand\baselinestretch{1.2}\small\normalsize
%
\textbf{Произвести  необходимые для выполнения  ремонта разборочно-сборочные, подготовительные и вспомогательные работы в соответствии с требованиями завода–изготовителя транспортного средства.}\\
%
Расчет стоимости восстановительных расходов выполнен в программе \auda. \\ Ниже представлены результаты расчета, полная калькуляция стоимости ремонта включена в раздел Приложение.
%%\smallskip
\begin{figure}[H]
	\centering
	\includegraphics[width=0.8\linewidth]{images/Screenshot_2}
	%%	\caption{}
	%%	\label{fig:screenshot001}
\end{figure}
%\begin{figure}[H]
%	\centering
%	\includegraphics[width=0.9\linewidth]{images/screenshot002}
%%%	\caption{}
%	\label{aud}
%\end{figure}
\medskip
Стоимость коммерческого нормо-часа работ применена  с учетом условий регионального рынка услуг и сложившихся средних расценок по видам работ, типу ТС, а также по маркам и моделям ТС  и   составляет  1300 р/ч для данного транспортного средства. \\
Трудоёмкость работ по разборке/сборке/замене  соответствует трудоемкости работ, рекомендованной заводом изготовителем ТС. 
Расчет стоимости ремонта, согласно положениям [1] производится с учетом  применения оригинальных запасных частей, которые поставляются изготовителем КТС авторизованным ремонтным организациям. Техническое состояние запасных частей учитывается коэффициентом износа, что в совокупности с установкой оригинальных запасных частей в максимальной степени отвечает понятию «восстановительный ремонт», то есть восстановления состояния КТС, при котором используются установленные изготовителем составные части, но с использованным частично ресурсом. 

%
В результате проведенных расчетов (см. Приложение, калькуляция № 22-2019) определена стоимость восстановительного ремонта транспортного средства  \тс, которая составляет 441 295 (Четыреста сорок одна тысяча двести девяносто пять) рублей без учета износа,
стоимость восстановительных расходов, с учетом уменьшения стоимости запасных частей вследствие их износа,  составляет 281 990 (двести восемьдесят одна тысяча девятьсот девяносто) рублей.

\section{Результаты исследования}

\section{Выводы}

\subparagraph*{}
%	

{\small 
	\begin{longtable}{|p{4mm}|p{10mm}|p{16mm}|p{5cm}|p{6cm}|}
		\caption[]{\footnotesize {\textbf{$\cdots$}}} \label{tab:5}\\
		\hline
		%\rowcolor[HTML]{C0C0C0} 
		% Заголовки столбцов
		\text{n/n} &\textbf{Дата} &\textbf{Пробег,\~км}& \textbf{Причина обращения }& \textbf{Выполнено } \\ \hline \endhead % повторение заголовка 
		% Строки
		\Rownum &  & n& Панель задка  & Замена, окраска \\ \hline
		%\rowcolor[HTML]{EFEFEF} 
		\Rownum & &n & Боковина задняя левая   & Замена, окраска \\ \hline
		%%% ..............& 
		
\end{longtable}}\setcounter{rownum}{0}


\begin{enumerate}
	\item \textbf{"Причина поломки или неисправности ДВС на автомобиле Toyota Land Cruizer 200, 2008 года выпуска VIN: JTMHV05J004031859, после проведения ремонта ответчиком при изложенных в исковом заявлении обстоятельствах является попадание постороннего предмета, а именно болта крышки клапанов в левый впускной тракт, что привело к разрушению колеса турбины и повреждению рабочих поверхностей пятого цилиндра ДВС}
	
	\vspace{5mm}
	
	\item \textbf{ Стоимость восстановительного ремонта с учетом износа стоимости запасных частей составляет 281 990 (двести восемьдесят одна тысяча девятьсот девяносто) рублей}
\end{enumerate}
\vspace{15mm}
\relax
Приложение к заключению:\\
\textit{
	Расшифровка комплектации ТС по VIN \\
	Листинг калькуляции стоимости восстановительных расходов\\
}

\vspace{20mm}


%\setcounter{page}{1}
\clubpenalty=10000 
\widowpenalty=10000
%%%%%%%%%%%%%%%%%%%%%%%%%%%%%%%%%%%%%%%%
%      Шапка экспертной организации  
%%%%%%%%%%%%%%%%%%%%%%%%%%%%%%%%%%%%%%%%
%
%%%%%%%%%%%%%%%%%%%%%%%%%%%%%%%%%%%%%%%%%
%
%   Экспертная организация ООО Южнорегиональная экспертная группа
%
%%%%%%%%%%%%%%%%%%%%%%%%%%%%%%%%%%%%%%%%%
\noindent %\qrcode[height=21mm]{\NomerDoc от \окончено }  %%% Добавлен QR-Code
\begin{pspicture}(21mm,21mm)
\obeylines
\psbarcode{%
	%\NomerDoc от \окончено
	BEGIN:VCARD^^J
	VERSION:4.0^^J
	%N:Мраморнов; Александр; Вчеславович^^J
	FN:Александр Мраморнов^^J
%	ORG:IP Alexandr Mramornov^^J
	TITLE: эксперт
	ORG: ИП
	URL:http://www.yourexp.ru^^J
	EMAIL:4516611@gmail.com^^J
	TEL:+7-918-451-6611^^J
	ADR:г. Краснодар, с/т № 2 А/О «Югтекс», ул. Зеленая, 472^^J
	END:VCARD
}{width=1.0 height=1.0}{qrcode}%
\end{pspicture}
\begin{center}
	\normalsize\textbf{$\cdots$\\[-1.5mm] <<$\cdots$>> \\[-5mm]}
	%  
	\noindent\rule{\textwidth}{1pt}\\[-6mm]  % Горизонтальная линия
	% \line(1,0){460}% (1,0) -горизонтальная линия, и (0,1) - вертикальная 
\end{center}

\begin{center}
	\begin{footnotesize}\setstretch{0.3}
		%	\small\textbf\setlength   	%\raisebox{5mm}
		\vspace{-3.5mm}$\cdots$\\[0mm]
		Телефон: \quad $\cdots$, e-mail:\quad $\cdots$\\ [-2mm]{$\cdots$\quad$\cdots$}
	\end{footnotesize}	\\[10mm]
\end{center}


\begin{flushright}
	%Краснодар,
	$\cdots$, 2020    \\[8mm]
\end{flushright}
\begin{center}
	\LARGE\textbf{ ЗАКЛЮЧЕНИЕ ЭКСПЕРТА}
	\bigskip\\[0mm]
	%	{\normnumxtbf{\NomerDoc}}	}{den}
\end{center}
\par
\vspace{-3mm}\noindent по гражданскому делу \delonum \, по иску \isk \\[0mm]

%\raggedright 
%\def\hrf#1{\hbox to#1{\hrulefill}}
\noindent \textbf{№ $\cdots$}\hfill           \textbf{\окончено}\\%[2mm]
%Приостановлено\hfill      \datastop\\
%Возобновлено\hfill          \datarestart\\
%Окончено\hfill                \dataend\\%[4mm]

\noindent\parbox[l][16mm]{16.5cm}
{\def\hrf#1{\hbox to#1{\hrulefill}}
	\noindent Начато\hfill            \datastart\\%[2mm]
	%	Приостановлено\hfill      \datastop\\
	%	Возобновлено\hfill          \datarestart\\
	Окончено\hfill                \окончено\\%[4mm]
}
\relax

\datastart г. ~в {\small $\cdots$} \,  при определении  \, \sud  \,  от \, \dataopr \, о назначении \opr \, по гражданскому делу \delonum \, поступили:

\begin{enumerate}\setlist{nolistsep}\item  Материалы гражданского дела \delonum \\[-2mm]
	%	\item  
\end{enumerate}Экспертиза произведена экспертом  $\cdots$  % Шапка организации ООО ЮРЕКСГРУП
%
%%   вопросы экспертизы
\subsection{Вопросы экспертизы}
%Заказчик поручает, а Исполнитель принимает на себя обязательство выполнить Заказчику  комплекс работ в виде автотехнических исследований автомобиля Mazda 6, VIN RUMGJ52\-6802007133 (дата начала гарантии 07.05.2018 г.), по следующим вопросам:
\begin{enumerate}
	\item  <<Связано ли повреждение панели рамки радиатора слева и брызговика с лонжероном переднего левого автомобиля ВАЗ 21099 с указанным ДТП?>>	
\end{enumerate}

\addcontentsline{toc}{section}{Использованные нормативы и источники информации}
%
%\left( \addcontentsline{toc}{section}{Использованные нормативы и источники информации}

\subsection{Использованные нормативы и источники информации}
%
\begin{enumerate}
\item 
Махнин\,Е.\,Л., Новоселецкий\, И.\,Н., Федотов\, С.\,В. \emph{Методические рекомендации по проведению судебных автотехнических экспертиз и исследований колёсных транспортных средств в целях определения размера ущерба, стоимости восстановительного ремонта и оценки} // -- М.: ФБУ РФЦСЭ при Минюсте России, 2018.-326 с.  ISBN 978-5-91133-185-6.
%
%
%
%
\item ТУ 017207-255-00232934-2014 \emph{Кузова автомобилей LADA. Технические требования при приёмке в ремонт, ремонте и выпуске из ремонта предприятиями дилерской сети ОАО "АВТОВАЗ"}//  ОАО НВП "ИТЦ АВТО", 2014
%
\item Смирнов  В.Л., Прохоров  Ю.С., Боюр В.С.  и др. \emph{Автомобили ВАЗ. Кузова. Технология ремонта, окраски и  антикоррозионной защиты. Часть II}// - Н.Новгород: АТИС, 2001.- 241с.
%
\item 
Савич Е.Л. \emph{Техническое  обслуживание  и  ремонт  легковых  автомобилей} : учеб. пособие / Е.Л. Савич, М.М. Болбас, В.К. Ярошевич ; под общ. ред. Е.Л. Савича. -Мн. : Вышэйшая школа,  2001. - 479 с. - ISBN985-06-0502-2.
%
\item 
Автомобили ВАЗ-2121, 21213, 21214, 2131 и их модификации: <<Трудоемкости работ (услуг) по техническому обслуживанию и ремонту>> /Куликов А.В., Христов П.Н., Климов В.Е.,  Боюр В.С., Рева В.В., Зимин В.А., Завьялова Н.Н., Хлыненкова Г.А. -- ИТЦТ "АвтоВАЗтехобслуживание", Тольяти -- 2005. 
%
\item
Автомобили LADA SAMARA и их модификации: <<Трудоемкости работ (услуг) по техническому обслуживанию и ремонту>> /Куликов А.В., Христов П.Н., Климов В.Е., Рева В.В., Боюр В.С., Васильев М.В., Фахрутдинов Р.В.,  Прудских Д.А., Гирко В.Б., Шмелева В.А., Зимин В.А. --  ОАО НВП "ИТЦ АВТО",  -- 2006. - 252 стр.
%
\item 
Автомобили LADA PRIORA. Трудоемкости работ (услуг) по техническому обслуживанию и ремонту /Куликов А.В., Христов П.Н., Климов В.Е., Рева В.В., Козлов П.Л., Боюр В.С., Прудских Д.А., Шмелева В.А., Зимин В.А. -- ООО "ИТЦТ АВОСФЕРА", Тольяти -- 2009. -- 344 с.
%
\item 
{Трудоемкости работ по техническому обслуживанию и ремонту автомобилей автомобилей Lada  Granta}/   \url{https://docplayer.ru/30250248-Trudoemkosti-rabot-po-teh\-nicheskomu-obsluzhivaniyu-i-remontu-avtomobiley-lada- granta.html}.
%
%
\item
{Специализированное программное обеспечение для расчёта стоимости  восстановительного ремонта, содержащее нормативы трудоёмкости работ, регламентируемые изготовителями транспортного средства}//   AudaPadWeb, лицензионное соглашение № AS/APW-658  RU-P-409-409435.
%
%
%
\item
{Специализированное программное обеспечение для расчёта стоимости  восстановительного ремонта, содержащее нормативы трудоёмкости работ, регламентируемые изготовителями транспортного средства ОАО «АвтоВАЗ», ЗАО «Джи-Эм-АвтоВАЗ», ОАО «СеАЗ» и ОАО «ЗМА»}//   Автосфера АС:Смета, v.3.9.11// ООО "ИТЦ «ИнтегроМаш», \url{https://autosmeta.pro}.
%
%
%
\item Информационный портал по техническому обслуживанию и ремонту автомобилей	 ВАЗ:\\ \url{www.autosphere.ru}.

%%
\end{enumerate}

%%%%%%%%%%%%%%%%%%%%%%%%%%%%%%%%%%%%%%%%%%%%%%%%%%%%%%%%%%%%%%%%%%%%%%%%%%%%%%%%%
\subsection{Технические средства}  %% Список не удалять!!!
\begin{itemize}
%
%%
%\item Диагностический сканер BOSH VCM II S/N 1324-88682639 c програмным обеспечением Mazda IDS - 115.02
%%\item   Диагностический сканер SDconnect   с программным обеспечением Xentry Diagnostics v19.11.3.1
%\item   Линейка масштабная магнитная с цветографической шкалой, 100мм
%%\item   Рулетка измерительная металлическая, 5м
%%\item  Универсальный стенд для измерения углов установки колес Hunter Engineering %ProAlign с программным инструментом регулировки схождения колес без блокировки руля %автомобиля WinToe
%\item 	Цифровой фотоаппарат Canon 760D s/n 143032001327 с объективом Canon EF-S 18-135, тип используемой памяти: Transcend,  32Gb
%\item  Специализированное программное обеспечение для расчёта стоимости  восстановительного ремонта, содержащее нормативы трудоёмкости работ, регламентируемые изготовителями транспортного средства     AudaPadWeb, лицензионное соглашение № AS/\- APW-658  RU-P-409-409435
\item Он-лайн программа моделирования кинематики подвески автомобиля // \url{http://www.vsusp.com/}
\item Он-лайн ресурс проверки метаданных цифровых изображений \url{http://exif.regex.info/exif.cgi}
\item  Программа обработки изображений с открытым исходным кодом ImageJ, разработанная для научных многомерных изображений.  Wayne Rasband (wa-yne@codon.nih.gov),
свободная лицензия GPL.
\item  ПЭВМ под управлением операционной системы Windows 10 с установленным набором макрорасширений LaTeX системы компьютерной вёрстки TeX, cвободная лицензия LaTeX Project Public License (LPPL). 
%	
\end{itemize}
%%%%%%%%%%%%%%%%%%%%%%%%%%%%%%%%%%%%%%%%%%%%%%%%%%%%%%%%%%%%%%%%%%%%%%%%%%%%%%%%%%%%%%%%%%%%%%%%%%%%%%
\subsection{Условные обозначения}

\begin{description}
%	 
%%\item[АВС] --антиблокировочная система
\item[АМТС] --автомототранспортное средство
%\item[ГРМ] -- газораспределительный механизм
\item[ДВС] --двигатель внутреннего сгорания
\item[ДТП] --дорожно--транспортное происшествие
\item[гос.\,рег.\,знак] --государственный регистрационный знак
\item[КТС] --колесное транспортное средство 
\item[ЛКП] --лакокрасочное покрытие
\item[л.д.] --лист дела
%%\item[Колесо турбины]  -- крыльчатка турбины
\item[ТС] --транспортное средство
%\item[ТK, ТКР] -- турбокомпрессор. Состоит из двух частей: турбины и компрессора, объединенных общим валом. Вал вращается в подшипниках, размещенных в центральном корпусе ТК
%\item[ЦПГ] -- цилиндро-поршневая группа
%\item[ЭБУ] --электронный блок управления
%%\item[FRAME] -- номер кузова транспортного средства, выпущенного для продажи на внутреннем рынке Японии и содержащий информацию производителя о транспортном средстве
%%\item[OBDII] -- On-board diagnostics. Протокол бортовой диагностики автомобиля
%%\item[SRS] -- Cистема пассивной защиты водителя и пассажиров
\item[VIN] --vehicle identification number, 17--значный идентификационный номер транспортного средства, соответствующий стандарту ISO 3779--2012.
%
\end{description}
%%%%%%%%%%%%%%%%%%%%%%%%%%%%%%%%%%%%%%%%%%%%%

\subsection{Методы исследования}

\begin{itemize}
\item  Органолептический метод – исследование и оценка качества объектов с помощью органов чувств
%\item 	Прямой измерительный метод – путем измерения размеров деталей специальными %измерительными приборами
\item Расчётный метод (косвенный измерительный метод) – путём расчётов различных параметров на основе результатов измерений и других данных
\item Экспертный метод (метод экспертной оценки) — совокупности операций по выбору комплекса или единичных характеристик объекта, определению их действительных значений и оценкой экспертом соответствия их установленным требованиям и/или технической информации
\item Графоаналитический метод  
\item Метод масштабного моделирования и проецирования
\end{itemize}
%%%%%%%%%%%%%%%%
%
\subsection{Термины и определения}
\begin{description}
	\item[Аварийные повреждения] --- повреждения, механизм образования которых определяется контактом с посторонними объектами, что привело к деформации или разрушению и к необходимости ремонта или замены составной части, или контактам с агрессивной средой, которая привела к необходимости ремонта (замены) составной части [1, часть II, п. 1.5].
	%	\item[Восстановительный ремонт]--- один из способов возмещения ущерба, состоящий в выполнении технологических операций ремонта КТС, действующий в сети торгово-сервисного обслуживания, созданной изготовителем этого КТС [1, часть II, п. 1.4].
%	\item[Годные остатки] --- работоспособные, имеющие остаточную стоимость детали (агрегаты, узлы) поврежденного автотранспортного средства, годные к дальнейшей эксплуатации, которые можно демонтировать с поврежденного автотранспортного средства и реализовать.
	\item[Дата исследования]--- дата, на которую проводятся расчеты и используются стоимостные данные КТС, запасных частей, материалов, нормо-часа ремонтных работ [1, часть II, п. 1.5].
	\item[Линия удара]--- линия, определяемая направлением вектора равнодействующего импульса сил, возникающих при контакте ТС при столкновении до прекращения взаимного внедрения деформирующихся при ударе частей. Положением линии удара на ТС определяются направление и величина момента импульса сил, возникающих при ударе, и, следовательно, направлением и интенсивность разворота ТС относительно центра масс после столкновения.  
	\item[Моделирование]--- исследование каких-либо явлений, процессов или систем объектов путем построения и изучения их моделей.
	\item[Морфологические признаки]--- признаки, отображающие внешнее и внутреннее строение объекта
	\item[Срок эксплуатации КТС]--- период времени от даты изготовления (даты выпуска) КТС, до даты оценки (исследования), определяемой условиями задачи исследования (независимо от даты его регистрации и начала использования по назначению (эксплуатации)).
\end{description}
%%%%%%%%%%%%%%%%%%%%%%%
\subsection{Исходные данные и объекты исследования}

19.07.2016 г.в 13 час.  42 мин. в г. Белореченске по ул. Красная, 66 водитель Терехов А.В., управляя автомобилем Renault, г/н Е431НЕ33 перед началом движения не предоставил преимущество и допустил столкновение с автомобилем \тс\, регистрационный знак \грз\, под управлением водителя Шамояна Р.О.  
\par Для производства исследования представлено:
\begin{enumerate}
\item материалы гражданского дела № \delonum, \, в том числе:
\begin{itemize}
	\item Копия акта осмотра № 6164 транспортного средства \тс\, составленного 29.07.2016 г.  специалистом  компании "РАНЭ" 
	\item Копия акта осмотра № 295 от 19 июля 2016 г., составленного индивидуальным предпринимателем Новиковым Олегом Николаевичем
	\item Копия свидетельства о регистрации ТС 23 38 № 793117
	\end{itemize}
\item Электронные копии цифровых фотоснимков поврежденного автомобиля \тс\, VIN \vin\, предоставленные в  количестве 16 файлов формата .jpg с сохраненными техническими данными EXIF, выполненные 19.07.2016 г. цифровым фотоаппаратом Сanon PowerShot SX150 IS 
\end{enumerate}

\subsection{Ранее по материалам дела выполнено}

\noindent Судебная автотехническая экспертиза, выполненная  экспертом Дереберя Н.В.\\
Повторная судебная автотехническая экспертиза, выполненная экспертом Алифиренко В.В.

\section{Исследование}
%
Установление связи между повреждениями автомобиля и заявленными событиями доро\-жно-транспортного происшествия есть вопрос  исследования   взаимосвязи между повреждением и механизмом, вызвавшим это повреждение. Разрешение поставленного вопроса предполагает исследование механизма происшествия с точки зрения анализа характера деформаций и причин образования повреждений компонентов транспортного средства.	В транспортной трасологии различают первичные следы -- следы, возникшие в процессе первичного, начального контакта транспортных средств между собой или транспортных средств с различными преградами, и вторичные следы-- следы, появившиеся в  процессе дальнейшего смещения и деформации вступивших в следовое взаимодействие объектов. Таким образом, если рассматривать объемный след в виде деформации, то различаются первичные деформации, характеризующиеся наличием признаков непосредственного контакта деталей и частей транспортных средств,  и вторичные деформации, характеризующиеся отсутствием признаков непосредственного контакта деталей и  частей транспортных средств. Вторичные деформации являются следствием первичных, контактных деформаций. Детали изменяют свою форму под воздействием сил и моментов, возникающих в случае контактных деформаций по законам механики и сопротивления материалов. Такие деформации могут располагаться на удалении от места непосредственного контакта воздействием силы удара, распространяющейся по деталям несущей конструкции кузова автомобиля.
 

 Из постановления об административном правонарушении и справки о дорожно-транс\-портном происшествии следует, что в результате столкновения левая часть автомобиля \тс\, взаимодействовала с передним бампером автомобиля 
 \tcb.
 
   В процессе следового взаимодействия контактирующие поверхности \тс \, в составе левого зеркала заднего вида, переднего левого крыла, передней левой двери деформирующим воздействием бампера переднего автомобиля \tcb\, получили повреждения в направлении спереди назад и слева направо.  Графо-аналитическим методом исследования фотоизображений  повреждений автомобиля \тс \, установлено, что повреждения левой стороны кузова автомобиля  содержат признаки динамического и статического деформирующих воздействий, в совокупности не противоречащие заявленному механизму дорожно-транспортного происшествия.
   
\begin{figure}[h!]\centering
	\parbox[t]{0.49\textwidth}
	{\centering
		\includegraphics[width=.49\textwidth]{images/tc1}
		\caption{\footnotesize {Поврежденный в исследуемом ДТП автомобиль \тс,\, вид спереди слева}}
		\label{ris:images/b3}}
	\hfil \hfil
	\parbox[t]{0.49\textwidth}
	{\centering
		\includegraphics[width=.49\textwidth]{images/tb1}
		\caption{\footnotesize {Автомобиль, аналогичный автомобилю второго участника ДТП \tcb}}
		\label{ris:images/b4}}
\end{figure}



\subsection{Исследование транспортного средства}
%
С момента повреждения автомобиля до момента настоящего исследования прошел значительный период времени (более трех лет). Исходя из имеющихся в распоряжении эксперта изображений автомобиля, на момент ДТП, \датадтп\, на кузове автомобиля, преимущественно в нижней части, присутствовали  очаги коррозионных повреждений. Коррозия - физико-химическое или химическое взаимодействие между металлом и средой,  самопроизвольный окислительно-восстановительный процесс разрушения металлов и сплавов вследствие взаимодействия с окружающей средой,   необратимо усиливающихся с течением времени.  Ретроспективный характер исследования объективно не позволяет произвести натурное исследование автомобиля в том состоянии, в котором  автомобиль находился сразу  после заявленного ДТП.  На основании изложенного, исследование автомобиля \тс\, VIN \вин\, производилось экспертом по предоставленным электронным копиям цифровых фотоснимков. Представленные фотоснимки удовлетворительного качества и содержат необходимую и достаточную информации для производства экспертизы.


\begin{figure}[!h]\centering
	\parbox[t]{0.49\textwidth}
	{\centering
		\includegraphics[width=.49\textwidth]{images/tc2}
		\caption{\footnotesize {Обзорный снимок поврежденной области исследуемого автомобиля }}
		\label{ris:images/tc2}}
	\hfil \hfil
	\parbox[t]{0.49\textwidth}
	{\centering
		\includegraphics[width=.49\textwidth]{images/tc3}
		\caption{\footnotesize {Масштабное изображение  поврежденной области кузова исследуемого автомобиля}}
		\label{ris:images/tc3}}
	
\end{figure}



\begin{figure}[!h]\centering
	\parbox[t]{0.49\textwidth}
	{\centering
		\includegraphics[width=.49\textwidth]{images/tc8}
		\caption{\footnotesize {Повреждение левой передней двери }}
		\label{ris:images/tc8}}
	\hfil \hfil
	\parbox[t]{0.49\textwidth}
	{\centering
		\includegraphics[width=.49\textwidth]{images/tc9}
		\caption{\footnotesize {Повреждение левого переднего крыла}}
		\label{ris:images/tc9}}
	
\end{figure}

Представленные  изображения первичных повреждений левой передней стороны автомобиля \тс\,,  получены в результате заявленного ДТП \датадтп.  Из материалов дела известно, что оспариваемыми  страховой компанией повреждениями являются повреждения    панели рамки рамки радиатора слева и  брызговика с лонжероном переднего левого, Рис. \ref{ris:images/tc5}, Рис. \ref{ris:images/tc4}, Рис. \ref{ris:images/tc6}, Рис. \ref{ris:images/tc7}:


  \begin{figure}[h]\centering
  	\parbox[t]{0.49\textwidth}
  	{\centering
  		\includegraphics[width=.49\textwidth]{images/tc5}
  		\caption{\footnotesize {Повреждение левой нижней части панели рамки радиатора  }}
  		\label{ris:images/tc5}}
  	\hfil \hfil
  	\parbox[t]{0.49\textwidth}
  	{\centering
  		\includegraphics[width=.49\textwidth]{images/tc4}
  		\caption{\footnotesize {Повреждение передней нижней части левого лонжерона}}
  		\label{ris:images/tc4}}
  \end{figure}
%%%%%%%%%%%%%%%%%%%%%%%%%%%%%%%%%%%%%%%%%%%  

\begin{figure}[h]
	\centering
	\includegraphics[width=0.98\linewidth]{images/tc6}
	\caption{{\footnotesize {Разрыв панели рамки радиатора внизу слева. Вид спереди}}}
	\label{ris:images/tc6}
\end{figure}
\pagebreak

\begin{figure}[!h]
	\centering
	\includegraphics[width=0.98\linewidth]{images/tc7}
	\caption{{\footnotesize {Разрыв панели рамки радиатора внизу слева. Вид снизу слева. "1" - }}}
	\label{ris:images/tc7}
\end{figure}

\begin{figure}[!h]
	\centering
	\includegraphics[width=0.98\linewidth]{images/tc11}
	\caption{{\footnotesize {Разрыв панели рамки радиатора внизу слева. Вид снизу слева}}}
	\label{ris:images/tc11}
\end{figure}

\begin{figure}[H]
	\centering
	\includegraphics[width=0.98\linewidth]{images/tc12}
	\caption{{\footnotesize {Разрыв панели рамки радиатора. Слева усталостные трещины, справа - деформация и разрыв}}}
	\label{ris:images/tc12}
\end{figure}


Анализ поверхностных  деформаций крыла переднего левого  автомобиля \тс\, показывает, что в процессе столкновения  автомобиль \тс\, находился в движении по направлению "вперед", так как  относительно продольной оси автомобиля \тс\,, вектор удара направлен слева направо и спереди назад, автомобиль Renault Master мог двигаться в направлении слева направо, пересекая траекторию движения \тс, Таким образом, согласно имеющимся материалам, механизм столкновения,    вероятно, соответствует перекрестному, косому, скользящему, эксцентричному, левому столкновению.  Ударное воздействие локализовалось в области левой передней стойки автомобиля, при этом крыло переднее левое деформировалось на глубину не менее 5см, передняя  кромка двери левой передней получила загиб на глубину не менее 3 см в направлении действия сил, верхняя петля и примыкающая к ней часть левой передней стойки так же получили деформацию.  

Из технических данных  цифровых изображений EXIF известно, что 
фотографирование поврежденного автомобиля производилось  2016:07:19 в 14:44:54,  согласно справки о дорожно-транспортном происшествии, ДТП произошло 2016:07:19 в 13:42 то есть через 1 час после заявленного страхового события. Актом осмотра № 295, л.д. 19, составленным ИП Новиковым О.Н. 19.07.2016г. зафиксировано, в том числе, повреждение панели рамки радиатора слева и брызговик с лонжероном  слева.

На фотоизображениях Рис. \ref{ris:images/tc11} и Рис. \ref{ris:images/tc12} эксперт обращает внимание на два различающихся внешними признаками  вида повреждений ( области "1" и "2"). Повреждения, отмеченные на Рис. \ref{ris:images/tc11} выделенной областью "1" и левый снимок на Рис. \ref{ris:images/tc12} имеют характерные для усталостного разрушения трещины металла, возникающие под воздействием знакопеременной нагрузки или периодической динамической нагрузки. Коррозия поверхности излома металла в нижней части балки однозначно указывают на повреждение, образовавшееся задолго до дорожно-транспортного происшествия \датадтп.  Повреждения, отмеченные областью "2", правый снимок на Рис.  \ref{ris:images/tc12} и на Рис.\ref{ris:images/tc6}  имеют характер пластической деформации с разрывом металла, при этом, поверхности металла на разрыве имеют чистый металлический цвет без загрязнений и следов коррозии. Совокупность морфологических признаков этой части повреждений нижней поперечины рамки радиатора позволяют эксперту полагать, что указанные повреждения могут являться вторичными деформациями вследствие ДТП от \датадтп.

 \begin{figure}[H]
 	\centering
 	\includegraphics[width=0.75\linewidth]{images/s1}
 	\caption{{\footnotesize {Cхема деталей передней подвески ВАЗ 21099}}}
 	\label{ris:images/s1}
 \end{figure}

На Рис.\ref{ris:images/s1} приведена схема передней подвески автомобилей ВАЗ 21099, на которой стрелками показаны точки крепления подвески к кузову автомобиля.  Стрелка 6 указывает на область 1, где расположено крепление продольной растяжки к нижней поперечине рамки радиатора, стрелка 5 указывает на  область 2, место крепление поперечного рычага к основанию лонжерона и стрелка 4 указывает на точку 3 - место крепления стойки передней подвески к брызговику. Крепление рычага, растяжки и стойки выполнено через сайлентблоки. Практика ремонта автомобилей ВАЗ 2108-2109 показывает, что участок места крепления кронштейна рычага к панели рамки радиатора (на схеме отмечен "1"), в силу конструктивных особенностей, является  концентратором напряжений. 

 \begin{figure}[H]
	\centering
	\includegraphics[width=0.75\linewidth]{images/s2}
	\caption{{\footnotesize {Направление   ударного воздействия  и реакции кузова ВАЗ 2109}}}
	\label{ris:images/s2}
\end{figure}

На рисунке \ref{ris:images/s2}  стрелка 1 показывает направление  вектора удара, стрелки 2 и 3 направление реакции кузова.

\pagebreak

\begin{multicols}{3}[\columnsep=1cm]
\noindent	\includegraphics[scale=0.5]{m1}
	\columnbreak
	\includegraphics[scale=0.5]{m2}
	\columnbreak
	\includegraphics[scale=0.5]{m3}
\end{multicols}


\captionof{figure}{\footnotesize{Пример конечно-элементной модели реакции кузова автомобиля при боковом ударе (по материалам CompMechLab LLC, \url{http://fea.ru/compound/automotive}  ) }}

  \vspace{3mm}

На примере конечно-элементной модели автомобиля  наглядно видно, что при  ударном воздействии в боковую стойку в направлении слева направо левый передний лонжерон стремиться сместится в направлении, противоположном вектору удара. При этом, отклонение лонжерона от продольной оси может составлять нескольких сантиметров. Таким образом, именно в области левой нижней части панели рамки радиатора автомобиля \тс в месте сопряжения с передним левым лонжероном при боковом ударе создается дополнительная  концентрация напряжений.

 В случае исследуемого ДТП, имело место  сочетание ослабления места крепления кронштейна  рычага на панели рамки радиатора вследствие  усталости металла и образования дополнительной концентрации напряжений, возникшей в результате ударной нагрузки в область передней левой стойки автомобиля, что, в совокупности, привело к превышению предела прочности и  разрушению панели раки радиатора и передней нижней части левого лонжерона.  

\section{Вывод} 


\textbf{  <<Повреждение панели рамки радиатора слева и брызговика с лонжероном переднего левого автомобиля ВАЗ 21099 связано  с указанным ДТП.>>	}
  
  
  \vspace{20mm}
{Эксперт}\hfill           {Алифиренко В.В.}



%\setcounter{page}{1}
\clubpenalty=100000  % Недопуск висячей строки в начале страницы
\widowpenalty=100000 %Недопуск висячей строки в конце абзаца
%%%%%%%%%%%%%%%%%%%%   Шапка ИП рецензии
%
%%%%%%%%%%%%%%%%%%%%%%%%%%%%%%%%%%%%%%%%%
%
%   Экспертная организация ИП
%
%%%%%%%%%%%%%%%%%%%%%%%%%%%%%%%%%%%%%%%%%
\noindent %\qrcode[height=21mm]{\NomerDoc от \окончено }  %%% Добавлен QR-Code
\begin{pspicture}(21mm,21mm)
\obeylines
\psbarcode{%
%	\NomerDoc от \окончено
	BEGIN:VCARD^^J
	VERSION:4.0^^J
	%N:Мраморнов; Александр; Вчеславович^^J
	FN:Александр Мраморнов^^J
%	ORG:IP Alexandr Mramornov^^J
	TITLE: эксперт
	ORG: ИП
	URL:http://www.yourexp.ru^^J
	EMAIL:4516611@gmail.com^^J
	TEL:+7-918-451-6611^^J
	ADR:г. Краснодар, с/т № 2 А/О «Югтекс», ул. Зеленая, 472^^J
	END:VCARD
}{width=1.0 height=1.0}{qrcode}%
\end{pspicture} %%% Добавлен QR-Code
\vspace{-4mm}
\begin{center}
	\large\textbf{ИНДИВИДУАЛЬНЫЙ\quad ПРЕДПРИНИМАТЕЛЬ  \\[-1.5mm] МРАМОРНОВ  АЛЕКСАНДР ВЯЧЕСЛАВОВИЧ \\[-5.5mm]}
	%  
	\noindent\rule{\textwidth}{2pt}\\[-6mm]  % Горизонтальная линия
	% \line(1,0){460}% (1,0) -горизонтальная линия, и (0,1) - вертикальная 
\end{center}

\begin{center}
	\begin{footnotesize}\setstretch{0.3}
		%	\small\textbf\setlength   	%\raisebox{5mm}
		\vspace{-2.5mm}г. Краснодар, с/т № 2 А/О «Югтекс», ул. Зеленая, 472, 
		Телефон: 8-918-451-66-11, e-mail: 4516611@gmail.com\\ [-2mm]{ИНН\quad 231200665168\quad ОГРНИП \quad 310231220400043}
	\end{footnotesize}	\\[10mm]
\end{center}


\begin{flushright}
% 
	 \hfill	Краснодар, 2020    \\[8mm]
\end{flushright}  
\begin{center}
	\LARGE\textbf{РЕЦЕНЗИЯ}
	\bigskip\\[-5mm] 
	\textbf{  {\normalsize № \NomerDoc\,  от \dataend}}
\end{center}
\par
\vspace{4mm}
\noindent на <<заключение эксперта № 00182/18 по гражданскому делу № 2-802/19 по иску Хачатряна Андраника Гайковича к Мокину Игорю Витальевичу о возмещении материального ущерба, причиненного дорожно-транспортным происшествием>>\\

%Рецензия составлена на основании	договора № \NomerDoc\, от \dog \, возмездного оказания услуг.

%
%%%%%%%%%%%%%%%%%%%%
\subsection{ Сведения о рецензенте}

Рецензия составлена рецензентом Мраморновым Александром Вячеславовичем, имеющим высшее техническое образование по специальности «техническая физика», диплом РВ № 311964 от 28.02.1989г, квалификация -- инженер-физик, специальное образование в области оценки: Диплом ПП- 1 № 037211 Российской экономической академии им. Г.В. Плеханова, квалификация -- оценка и экспертиза объектов и прав собственности, специальное образование в области независимой технической экспертизы транспортных средств: Диплом ПП-I № 424167, квалификация: эксперт-техник (специализация 150210 специальности 190601.65 -- Автомобили и автомобильное хозяйство), состоящий в Государственном реестре экспертов-техников (№ в реестре 256, http://minjust.ru/ru/node/105988),  общий трудовой  стаж 29 лет, стаж  экспертной работы  12 лет.

%\textbf{Фефеловым Сергеем Леонтьевичем}, имеющим высшее техническое образование по специальности «Инженер механик» (диплом ИВ №091813), высшее юридическое образование по специальности «Юриспруденция»: диплом ВСВ  №1089728, прошедшему профессиональную переподготовку в ГОУ ВПО «Кубанский государственный технологический университет» по программе «Независимая техническая экспертиза транспортного средства»: Диплом ПП-I №705152, имеющим квалификацию судебного эксперта, прошедшим обучение в некоммерческом партнёрстве «Палата судебных экспертов» по программе повышения квалификации: экспертной специальности 13.2 «Исследование технического состояния транспортных средств»  и 13.4 «Исследование транспортных средств в целях определения стоимости восстановительного ремонта и остаточной стоимости»  в системе аттестации судебных экспертов Министерства Юстиции Российской Федерации, включенного в Государственный реестр экспертов-техников (http://minjust.{\textbackslash}{\textbackslash}/ru/node/105988 ), номер в реестре экспертов-техников - №244. Общий трудовой стаж 31 год. Стаж работы по экспертной специальности 10 лет;

\subsection{В распоряжение рецензента предоставлено}

\begin{enumerate}
\item Копия заключения эксперта  по гражданскому делу № 2-802/19 по иску Хачатряна Андраника Гайковича к Мокину Игорю Витальевичу о возмещении материального ущерба, причиненного дорожно-транспортным происшествием (на 17 листах);
\item  Цифровые копии изображений  поврежденного автомобиля \тс регистрационнный знак \грз \, в количестве 177 файлов формата JPG, с сохраненными данными EXIF,  фотосъемка выполнена 20.06.2019, фотокамерой Aple iPhone 7.  
\item Копия свидетельства 99 05 № 642746 о регистрации транспортного средства Мерседес-Бенц  AMG C63S VIN WDD2053871F497775, 2016 года выпуска,  принадлежащего Хачатряну Андранику Гайковичу. 
\item Копия постановления 18810223177771606065 по делу об административном правонарушении, вынесенного 18.06.2019г. в отношении гражданина Мокина Игоря Витальевича по факту совершенного им  17.06.2019 г. административного правонарушения, дорожно-транспортного происшествия, предусмотренное частью 1 статьи 12.15 КОАП РФ;
\item Копия заказ-наряда № ЗН19005605 от 22.06.2019 ООО "СБСВ-КЛЮЧАВТО СЕВЕР" на ремонт (дефектова а/м после ДТП) автомобиля 205 387 MERCEDES  BENZ C63 AMG S, VIN WDD2053871F497775, кузов цвет designo серый селенит mango, (на 2 листах);
\item Копия счета на оплату С0000055677 от 22.06.2019, предоставленного ООО "СБСВ-КЛЮЧАВТО СЕВЕР", по заказ-наряду № ЗН19005605 от 22.06.2019 с указанием перечня и дилерской цены деталей, подлежащих замене;  
\item Копия  акта осмотра № 0652 от 20.06.2019 г. транспортного средства Mersedes-Benz AMG C63S VIN WDD2053871F497775, произведенного специалистом ООО "Независимая экспертная компания "Фаворит" (на 2 листах);
\item  Копия  ремонт-калькуляции автомобиля Mersedes-Benz AMG C63S VIN WDD2053871F497775, произведенного специалистом ООО "Независимая экспертная компания "Фаворит" по акту осмотра № 0652 от 20.06.2019 г. (на 7 листах).
\end{enumerate}


\subsection{ Иcходные данные}

Рецензированию подлежит: заключение эксперта  на основании определения судьи Курганинского районного суда Бабенко А.А. о назначении судебной экспертизы по гражданскому делу № 2-802/19 по иску Хачатряна Андраника Гайковича к Мокину Игорю Витальевичу о возмещении материального ущерба, причиненного дорожно-транспортным происшествием.

Заключение составлено экспертом ООО "Глобал-Эксперт" Лаптиевым Алексеем Ивановичем, имеющем высшее образование, прошедшим профессиональную переподготовку по программам "Оценка стоимости предприятия (бизнеса)" и "Программе профессиональной переподготовки экспертов-техников", имеющим квалификацию оценщика, эксперта техника, стаж работы свыше 15 лет. 

 Заключение представлено на рецензирование в виде электронных копий.


\textbf{Вопросы, поставленные перед экспертом при производстве первичной судебной экспертизы:}

\begin{enumerate}
	\item <<Какова рыночная стоимость автомобиля Мерседес Бенц AMG C63S государственный регистрационный номер У022ХР на момент ДТП  в исправном техническом состоянии>>
	\item <<Какова стоимость восстановительного ремонта поврежденного транспортного средства Мерседес Бенц AMG C63S государственный регистрационный номер У022ХР93 с учетом износа заменяемых запасных частей на дату ДТП?>>
	\item <<Какова величина утраты товарной стоимости Мерседес Бенц AMG C63S государственный регистрационный номер У022ХР93.>>	\vspace{2mm}\\
	
	Экспертизу провести в соответствии с Единой Методикой определения размера расходов на восстановительный ремонт в отношении поврежденного транспортного средства, утвержденной Центральным Банком РФ с  декабря 2014 г.
\end{enumerate}

	\vspace{2mm}
 В качестве  \textbf{исходных данных}, представленных для производства исследования экспертом указываются  материалы гражданского дела № 2-1481/19, в том числе включающие акт осмотра и расчет, выполненный независимой технической экспертизой, проведенной специалистами ООО "Фаворит"  и цветные фотоизображения в количестве 181 шт. поврежденного автомобиля \тс\,(стр. 3, 4 заключения эксперта). 
	\vspace{2mm}
	
\par  \textbf{В результате  производства судебной экспертизы эксперт приходит к следующим выводам}: 

\begin{enumerate}
	\item Рыночная стоимость автомобиля Мерседес Бенц AMG C63S государственный регистрационный номер У022ХР93 на момент ДТП в исправном техническом состоянии равна  3\,619\,500  (Три миллиона шестьсот девятнадцать тысяч пятьсот) рублей.
	\item Стоимость восстановительного ремонта поврежденного транспортного средства Мерседес Бенц AMG C63S государственный регистрационный номер У022ХР93 с учетом износа заменяемых запасных частей на дату ДТП составляет: 857\,090  Восемьсот пятьдесят семь тысяч девяносто рублей;
	\item Величина утраты товарной стоимости Мерседес Бенц AMG C63S государственный регистрационный номер У022ХР93 составляет 92 297.25 руб (Девяносто две тысячи двести девяносто рублей,  двадцать пять копеек)
\end{enumerate}


\subsection{ Анализ заключения}

\paragraph{ Анализ оформления заключения }


\par Требования к содержанию заключения эксперта прописаны в статье 25 Федерального закона от 31 мая 2001 г. N 73-ФЗ «О государственной судебно-экспертной деятельности в Российской Федерации»
Согласно ст. 25 Федерального закона "О государственной судебно-экспертной деятельности в РФ" от 30.12.2001 № 73-ФЗ с изменениями, "Заключение эксперта или комиссии экспертов и его содержание", на основании проведенных исследований с учетом их результатов эксперт от своего имени или комиссия экспертов дают письменное заключение и подписывают его. Подписи эксперта или комиссии экспертов удостоверяются печатью судебно-экспертного учреждения.

\noindent В заключении эксперта или комиссии экспертов должны быть отражены:\\
 -время и место производства судебной экспертизы;

\noindent -основания производства судебной экспертизы;

\noindent -сведения об органе или о лице, назначивших судебную экспертизу;

\noindent -сведения о государственном судебно-экспертном учреждении, об эксперте (фамилия, имя, отчество, образование, специальность, стаж работы, ученая степень и ученое звание, занимаемая должность), которым поручено производство судебной экспертизы;

\noindent -предупреждение эксперта в соответствии с законодательством Российской Федерации об ответственности за дачу заведомо ложного заключения;

\noindent -вопросы, поставленные перед экспертом или комиссией экспертов;

\noindent -объекты исследований и материалы дела, представленные эксперту для производства судебной экспертизы;

\noindent -содержание и результаты исследований с указанием примененных методов;

\noindent -оценка результатов исследований, обоснование и формулировка выводов по поставленным вопросам  при производстве судебной экспертизы;

	\vspace{2mm}
	
Материалы, иллюстрирующие заключение эксперта или комиссии экспертов, прилагаются к заключению и служат его составной частью. Документы, фиксирующие ход, условия и результаты исследований, хранятся в государственном судебно-экспертном учреждении. По требованию органа или лица, назначивших судебную экспертизу, указанные документы предоставляются для приобщения к делу.

В действующем законодательстве РФ имеются различные определения наименования документа, который является оформленным результатом проведенного исследования, в рамках производства судебной экспертизы: «Заключение эксперта», «Заключение». Согласно внутренних нормативно-локальных документов государственных экспертных учреждений, а также имеющейся судебно-экспертной практики, оформленным результатом проведенного исследования (произведенной экспертизы), является документ с наименованием «Заключение эксперта». Представленное на рецензию "заключение эксперта  по гражданскому делу № 2-802/19 по иску Хачатряна Андраника Гайковича к Мокину Игорю Витальевичу о возмещении материального ущерба, причиненного дорожно-транспортным происшествием", имеет различные  порядковые номера, указанные на титульном и первом листе заключения эксперта. Титульный лист   содержит наименование "Заключение эксперта № 071019", первый лист содержит наименование "Заключение эксперта № 00182/18". Далее рецензентом заключение эксперта по гражданскому делу № 2-802/19, составленное экспертом \чел \, будет  нумероваться "заключение эксперта № 00182/18". 


\par Согласно  разделу заключения эксперта № 00182/18 <<Нормативное, методическое и другое обеспечение, использованное при проведении экспертизы>>, (поз.3) эксперт руководствовался устаревшим  и отмененным Постановление Правительства РФ от 24 мая 2010 г. N 361 "Об утверждении Правил установления размера расходов на материалы и запасные части при восстановительном ремонте транспортных средств" постановлением Правительства РФ от 7 октября 2014 г. N 1017  признанным утратившим силу
\url{ http://base.garant.ru/12176002/#ixzz659Hgtxwr}, а также (поз.4) несуществующим  "Методическим руководством для судебных экспертов "Исследование автомототранспортных средств, в целях определения стоимости   восстановительного ремонта и оценки", напечатанными по решению научно-методи\-ческого совета Российской Федерации Центра Судебной Экспертизы при Минюсте России 2018 г."  Информационное письмо исх. 23-2624 ФБУ РФЦСЭ при Минюсте России сообщает, что «Методические рекомендации по проведению судебных автотехнических экспертиз и исследований колесных транспортных средств в целях определения размера ущерба, стоимости восстановительного ремонта и оценки» (далее – «Методические рекомендации») утверждены и рекомендованы для использования в экспертной практике решением научно-методи\-ческого совета РФЦСЭ 20.12.2017 (Протокол №4). Решением секции по судебной автотехнической экспертизе РФЦСЭ, утвержденным научно-методическим советом РФЦСЭ 18.10.2018 (протокол №3), вышеуказанные Методические рекомендации введены в действие с 01.01.2019 г., и с этого момента применяются в судебно-экспертных учреждениях системы Минюста России при производстве судебных автотехнических экспертиз независимо от даты, на которую производятся расчеты.>> 

Методические рекомендации «Исследование автомототранспортных средств в целях определения стоимости восстановительного ремонта и оценки», изданные в 2013 году (с изменениями от 2015 года) и действовавшие до 01.01.2019 г., прекратили свое действие».\\

Таким образом, эксперт в своем заключении ссылается на несуществующие или устаревшие методики, неверно указывает наименование действующей на момент исследования методики, в тексте заключения дает ссылки на отмененные Методические рекомендации 2013 года.  Заключение  содержит формальные нарушения, которые могут  служить  признанием экспертизы недействительным доказательством.




\subsection{ Анализ проведенного экспертом исследования}

Заключении эксперта  № 00182/18 по гражданскому делу № 2-802/19 по иску Хачатряна Андраника Гайковича к Мокину Игорю Витальевичу о возмещении материального ущерба, причиненного дорожно-транспортным происшествием не  содержит описание методики проведенного исследования; в исследовательской части заключения эксперта отсутствуют анализ исходной документации, не указаны методы исследования.


В разделе «Нормативное, методическое и другое обеспечение, использованное при проведении экспертизы»   отсутствуют сведения об использовании технологических инструкций завода-изготовителя. Автомобиль \тс является сложным, высокотехнологичным изделием, содержащим большое количество решений, присущих только автомобилям данной марки и модели.  Для проведения всестороннего и объективного исследования эксперту  необходимо было использовать технические инструкции завода-изготовителя при проведении ремонтных работ, которую необходимо было эксперту официально запросить по соответствующей форме ходатайства через суд у официального представителя изготовителя транспортного средства. Указанная экспертом программа для расчета стоимости восстановительного ремонта автомобилей иностранного и отечественного производства «ПС: Комплекс 6» на момент исследования не содержала подробных сведений об автомобиле исследуемой марки и модели. \url{http://www.autoxp.ru/rus/pscmodels.aspx?marka=mercl#mercl}



В представленном заключении в разделе «Исследование» эксперт указывает, что "исследованию подлежит стоимость восстановительного ремонта автомобиля...".  В данном случае имеет место деятельностная операционная ошибка, допущенная экспертом на подготовительной стадии исследования, так как в заключении эксперта отражено неверное представление эксперта об объектах исследования. Объектами автотехнической экспертизы по специальности 13.4 являются колесно транспортное средств и его  составные части, а так же документы, свидетельствующие о повреждении КТС, выполненном ремонте, другие материалы, по которым возможно установить связь рассматриваемых фактов с событием происшествия"[2, часть I, п.1.]. Определение стоимости является целью исследования.



В экспертном заключении  отсутствует описание процесса  исследования.  После перечисления исходных данных экспертизы и  регистрационных данных автомобиля эксперт сразу переходит к расчетам. Из текста заключения абсолютно не ясно, как и чем руководствовался эксперт при назначении   ремонтных воздействий транспортного средства.\\

Так,  с одной стороны эксперт указывает, что "определяя стоимость восстановительного ремонта следует придерживаться "первичных" документов - справки об участии в ДТП и акта осмотра поврежденного ТС, наиболее полно отражающего причино-следственную связь   между обстоятельствами ДТП и характером повреждений", так же принимая во внимание "фотоматериалы, являющиеся приложением, которые могут с достоверной долей вероятности указать на характер повреждений, а также деформации, которые хотя и не указаны в актах осмотра, но явно имеют отношение к рассматриваему событию".   Но, с одной стороны, в материалах дела справка об участии в ДТП отсутствует (вступивший в силу 20 октября 2017г.приказ МВД № 664 об утверждении регламента работы дорожно-патрульной службы отменил выдачу справки о ДТП при оформлении дорожных аварий сотрудниками ГИБДД, в настоящее время необходимые для страховых компаний данные о повреждениях транспортных средств вносятся в протокол или постановления об административном правонарушении), с другой стороны - экспертом в заключении не приводятся сведения об исследовании и анализе  характера, степени и относимости повреждений \тс \, заявленному событию. \\
В соответствии с п.1.4. "Единой методики определения размера расходов на восстановительный ремонт в отношении поврежденного транспортного средства"[4],  для характеристики повреждений деталей каркаса кузова и оперения транспортного средства, используются следующие показатели, в зависимости от которых определяются методы и трудоемкость устранения повреждений:
	
	\begin{itemize}
	
	\item -площадь повреждения либо отношение площади повреждения к общей площади части, детали (в процентном соотношении или частях) и \item \item -глубина (объем) повреждения (количественные показатели);
	\item -вид деформации и первоначальные (установленные заводом-производителем) конструктивные характеристики части, детали транспортного средства в зоне повреждения (качественные показатели);
	\item -локализация (место расположения) повреждений для определения доступности ремонтного воздействия.
	
	\end{itemize}

 
В расчетах рыночной стоимости транспортного средства (стр.11,12 заключения) отсутствует  информация о дате, на которую были  действительны сведения о ценах предложений продажи транспортных средств-аналогов. Не ясна математическая модель расчета рыночной стоимости, отсутствуют значимые для поведения оценки сведения об аналогах, такие как комплектность и комплектация, техническое состояние, количество владельцев, пробег, регион нахождения автомобиля на дату размещения объявления о продаже.

  
  
Согласно таблице расчета рыночной стоимости автомобиля, приведенной на стр. 11 экспертного заключения № 00182/18, эксперт определил рыночную стоимость исследуемого автомобиля простым нахождением среднего арифметического пяти предложений продажи автомобилей   Mersedes Bens AMG C  2016-2017 годов выпуска, размещенных на сайте \url {avito.ru} без применения  корректировок  или методов  регрессионного анализа или иных методов оценки, за исключением корректировки 5\% "на торг". Так же не ясно, чем руководствовался эксперт при выборе коэффициента корректировки "на торг" равной 5\%. Никакие исследования, подтверждающие достоверность принятого коэффициента в заключении эксперта не предоставлено. 
  
  
   
\par По совокупности, заключение № 00182/18 по гражданскому делу № 2-802/19  содержит явные  ошибки в выборе методик исследования, в понимании экспертом объектов и предметов экспертизы, позволяющие  усомнится в точности и достоверности полученных результатов. 
  
  
\par Рецензентом выборочно выполнена проверка предложенных экспертом аналогов исследуемому автомобилю, см. таблицы \ref{tab:8}, \ref{tab:9}, \ref{tab:10} (объявление \url{https://www.avito.ru/novorossiysk/avtomobili/mercedes-benz_c-klass_amg_2016_1765307088}, объявление \url {https://www.avito.ru/sankt-peterburg/avtomo\-bili/mercedes-benz_c-klass_amg_2016_1809470870}, объявление \url {https://www.avito.ru/moskva/avtomobili/mercedes-benz_c-klass_amg_2016_1815634416})  
 
 
\begin{longtable}{|p{5cm}|p{5cm}|p{5cm}|}
	\caption[]{\footnotesize {Таблица сравнения аналогов}} \label{tab:8}\\ 
	\hline
	\rowcolor[HTML]{EFEFEF} 
	
	Компонент сравнения & Исследуемый автомобиль& Аналог (1)  \\ \hline \endhead % повторение заголовка 
	Объем двигателя  &4.0 & 4.0 \\ \hline
%	\rowcolor[HTML]{EFEFEF} 
	Модификация  &C 63 4.0 7G-Tronic (510 л.с.)  & C 63 4.0 7G-Tronic (510 л.с.) \\ \hline
	Регион  & Краснодар  & Новороссийск\\ \hline
	\rowcolor[HTML]{ FAEBD7} 
	Пробег & 12582   & 33000\\ \hline
	Год выпуска  & 2016  & 2016 \\ \hline
	\rowcolor[HTML]{ FAEBD7} 
	Комплектация  & Расширенный набор опций  & Base \\ \hline
	\rowcolor[HTML]{ FAEBD7} 
	Тип кузова  & Купе  & Седан, 4дв. \\ \hline
	Цена  & --  & 3400000 \\ \hline
	%%% ..............
\end{longtable}

  \begin{longtable}{|p{5cm}|p{5cm}|p{5cm}|}
	\caption[]{\footnotesize {Таблица сравнения аналогов}} \label{tab:9}\\ 
	\hline
	\rowcolor[HTML]{EFEFEF} 
	
	Компонент сравнения & Исследуемый автомобиль& Аналог (1)  \\ \hline \endhead % повторение заголовка 
	Объем двигателя  &4.0 & 4.0 \\ \hline
%	\rowcolor[HTML]{EFEFEF} 
	Модификация  &C 63 4.0 7G-Tronic (510 л.с.)  & C 63 4.0 7G-Tronic (510 л.с.)\\ \hline
	\rowcolor[HTML]{ FAEBD7}
	Регион  & Краснодар  & Петербург\\ \hline
%	\rowcolor[HTML]{ FAEBD7}
	Пробег, км & 12582  & 13386\\ \hline
	Год выпуска  & 2016  & 2016 \\ \hline
		\rowcolor[HTML]{ FAEBD7}
	Комплектация  & Расширенный набор опций  & Base \\ \hline
	\rowcolor[HTML]{ FAEBD7}
	Тип кузова  & Купе  & Седан, 4дв. \\ \hline
	Цена  & --  & 3349000 \\ \hline
	%%% ..............
\end{longtable}

  \begin{longtable}{|p{5cm}|p{5cm}|p{5cm}|}
	\caption[]{\footnotesize {Таблица сравнения аналогов}} \label{tab:10}\\ 
	\hline
	\rowcolor[HTML]{EFEFEF} 
	
	Компонент сравнения & Исследуемый автомобиль& Аналог (1)  \\ \hline \endhead % повторение заголовка 
	Объем двигателя  &4.0 & 4.0 \\ \hline
    \rowcolor[HTML]{ FAEBD7} 
	Модификация  &C 63 4.0 7G-Tronic (510 л.с.)  & C 63 4.0 7G-Tronic (467 л.с.)\\ \hline
		\rowcolor[HTML]{ FAEBD7}
	Регион  & Краснодар  & Москва\\ \hline
	\rowcolor[HTML]{ FAEBD7}
	Пробег, км & 12582  & 36359\\ \hline
	Год выпуска  & 2016  & 2016 \\ \hline
	\rowcolor[HTML]{ FAEBD7}
	Комплектация  & Расширенный набор опций  & Base \\ \hline
	Тип кузова  & Купе  & Купе \\ \hline
	Цена  &  -- & 3353000 \\ \hline
	%%% ..............
\end{longtable}

\noindent \tikz \fill [red, opacity=0.2] (1,0.5) rectangle (0.1,0.1); -- строки таблицы с несовпадающими опциями

\relax

\begin{figure}[h!]\centering
	\parbox[t]{0.49\textwidth}
	{\centering
		\includegraphics[width=.49\textwidth]{images/i}
		\caption{\footnotesize {Исследуемый автомобиль}}
		\label{ris:images/i}}
	\hfil \hfil%раздвигаем боксы по горизонтали 
	\parbox[t]{0.49\textwidth}
	{\centering
		\includegraphics[width=.49\textwidth]{images/an}
		\caption{\footnotesize {Аналог 1, выбранный для оценки}}
		\label{ris:images/an}}
\end{figure}
  
  \begin{figure}[h!]\centering
  	\parbox[t]{0.49\textwidth}
  	{\centering
  		\includegraphics[width=.49\textwidth]{images/i}
  		\caption{\footnotesize {Исследуемый автомобиль}}
  		\label{}}
  	\hfil \hfil%раздвигаем боксы по горизонтали 
  	\parbox[t]{0.49\textwidth}
  	{\centering
  		\includegraphics[width=.49\textwidth]{images/an2}
  		\caption{\footnotesize {Аналог 4, выбранный для оценки}}
  		\label{ris:images/an2}}
  \end{figure}
  
    \begin{figure}[h!]\centering
  	\parbox[t]{0.49\textwidth}
  	{\centering
  		\includegraphics[width=.49\textwidth]{images/i}
  		\caption{\footnotesize {Исследуемый автомобиль}}
  		\label{}}
  	\hfil \hfil%раздвигаем боксы по горизонтали 
  	\parbox[t]{0.49\textwidth}
  	{\centering
  		\includegraphics[width=.49\textwidth]{images/an3}
  		\caption{\footnotesize {Аналог 2, выбранный для оценки}}
  		\label{ris:images/an3}}
  \end{figure}


\par Результат выборочного анализа (3 из 5) сравнений принятых для оценки рыночной стоимости автомобилей однозначно указывает на существенное отличие выбранных аналогов от исследуемого автомобиля. Значимые различия в типе кузова, комплектации, мощности двигателя не позволяют достоверно определить рыночную стоимость исследуемого автомобиля \тс \, на основании выборки, сделанной экспертом в заключении № 00182/18.

  
  В качестве примера рецензент приводит новый, товарный, 2016 года выпуска, идентичный исследуемому автомобиль в идентичной комплектации, выставленный на продажу дилером г.  Краснодара 	«Мерседес-Бенц» Центр КЛЮЧАВТО на Красной Площади,  таблица \ref{tab:11}, по цене 8\,461\,554 (Восемь миллионов четыреста шестьдесят одна тысяча пятьсот пятьдесят четыре) рубля, \url {https://cars.mercedes-benz.ru/mobile/CarDetails/Index/443423}. Согласно расчетов эксперта \чел, уменьшение рыночной цены автомобиля \тс  более чем в два раза произошло за неполные два года эксплуатации, за безаварийный пробег менее 15000 км и находящегося на гарантийном обслуживании. Ликвидность   --50\% за два года эксплуатации вызывает сомнения в правильности расчетов. 
  
    \begin{longtable}{|p{5cm}|p{5cm}|p{5cm}|}
  	\caption[]{\footnotesize {Таблица сравнения аналогов}} \label{tab:11}\\ 
  	\hline
  	\rowcolor[HTML]{EFEFEF} 
  	Компонент сравнения & Исследуемый автомобиль& Аналог (1)  \\ \hline \endhead % повторение заголовка 
  	Объем двигателя  &4.0 & 4.0 \\ \hline
 	Модификация  &C 63 4.0 7G-Tronic (510 л.с.)  & C 63 4.0 7G-Tronic (510 л.с.)\\ \hline
  	Регион  & Краснодар  & Краснодар\\ \hline
  	Пробег, км & 12582  & 0\\ \hline
  	Год выпуска  & 2016  & 2016 \\ \hline
  	Комплектация  & Расширенный набор опций  & Расширенный набор опций \\ \hline
  	Тип кузова  & Купе  & Купе \\ \hline
  	Цена  &  -- & 8\,461\,554 \\ \hline
  	%%% ..............
  \end{longtable}
  
      \begin{figure}[h!]\centering
  	\parbox[t]{0.49\textwidth}
  	{\centering
  		\includegraphics[width=.49\textwidth]{images/i}
  		\caption{\footnotesize {Исследуемый автомобиль}}
  		\label{}}
  	\hfil \hfil%раздвигаем боксы по горизонтали 
  	\parbox[t]{0.49\textwidth}
  	{\centering
  		\includegraphics[width=.49\textwidth]{images/is}
  		\caption{\footnotesize {Товарный автомобиль}}
  		\label{ris:images/is}}
  \end{figure}
%  \pagebreak
  
\par Согласно заказ-наряду № ЗН19005605 от 22.06.2019 ООО "СБСВ-КЛЮЧАВТО СЕВЕР" на ремонт (дефектовка а/м после ДТП), стоимость устранения аварийных повреждений автомобиля \тс \, составляет 2\,436,843 (Два миллиона четыреста тридцать шесть тысяч восемьсот сорок три) рубля. 
\par Согласно расчетов независимой экспертизы  № 0652 от 20.06.2019 г. ООО "Фаворит", стоимость восстановительного ремонта автомобиля составляет 2\,428\,665 (Два миллиона четыреста двадцать восемь тысяч шестьсот шестьдесят пять) рублей;
\par Стоимость восстановительных расходов автомобиля, согласно рецензируемого  заключения эксперта № 00182/18  составляет 965 134 (Девятьсот шестьдесят пять тысяч сто тридцать четыре) рубля без учета износа и 857 093 (Восемьсот пятьдесят семь тысяч девяносто три) рубля с учетом износа.

Таким образом, если  разница в расчетах официального дилера изготовителя ТС и независимой экспертизой составляет менее 1\%, то разница в расчетах в сравнении с заключением эксперта составляет  более чем в 2.5 раза или 1 463 531 (Один миллион четыреста шестьдесят три тысячи пятьсот тридцать один) рубль. С целью установления причины образования столь существенной разницы, принимая во внимание вероятную разницу в стоимости детали  между розничными ценами изготовителя и указанной с справочниках РСА, рецензентом произведен сравнительный анализ повреждений автомобиля, принятых в вышеуказанных расчетах.   Результаты сравнения приведены ниже в \ref*{tab:sravnenie}.


{\footnotesize \
\begin{longtable}{|p{1.5in}|p{1.5in}|c|c|p{1.5in}|c|c|p{1.5in}|c|}
\caption{\footnotesize {Сводная сравнительная таблица поврежденных деталей}}
\label{tab:sravnenie}\\ \hline
\textbf{Наименование детали} & \textbf{Заключение эксперта} & \textbf{Акт осмотра ООО "Фаворит"} & \textbf{Дилер Мерседес} \\ \hline \endhead
Крыло переднее левое & \bullet & \bullet&\bullet \\ \hline 
Облицовка бампера переднего & \bullet& \bullet&\bullet \\ \hline 
Дверь левая & \bullet& \bullet& \bullet\\ \hline 
Шумоизоляция двери & \bullet &   &\bullet \\ \hline 
Фара левая & \bullet& \bullet&\bullet \\ \hline 
Капот & \bullet& \bullet& \bullet\\ \hline 
Диск переднего левого колеса & \bullet & \bullet&\bullet \\ \hline 
Шина переднего левого колеса & \bullet& \bullet&\bullet \\ \hline 
Облицовка арки переднего левого колеса & \bullet& \bullet& \bullet\\ \hline 
Облицовка переднего бампера левая & \bullet & \bullet&\bullet \\ \hline 
Накладка защитная переднего бампера левая& \bullet& \bullet&\bullet \\ \hline 
Усилитель передний & \bullet& \bullet& \bullet\\ \hline 
Балка переднего моста & \bullet & \bullet&\bullet \\ \hline 
Управление рулевое & \bullet& \bullet&\bullet \\ \hline 
Амортизатор передний левый & \bullet& \bullet& \bullet\\ \hline 
Стабилизатор передний & \bullet & \bullet&\bullet \\ \hline 
Насос омывателя & \bullet& \bullet&\bullet \\ \hline 
Шланг омывателя & \bullet& \bullet& \bullet\\ \hline 
Усилитель переднего бампера & \bullet & \bullet&\bullet \\ \hline 
Буфер передний левый & \bullet& \bullet&\bullet \\ \hline 
Радиатор левый & \bullet& \bullet& \bullet\\ \hline 
Желоб передний левый & \bullet & \bullet&\bullet \\ \hline 
Датчик ускорения передний левый & \bullet& \bullet&\bullet \\  \hline 
Датчик парковки передн. нар. лев. & \bullet & \bullet&\bullet \\ \hline 
Датчик парковки передн. вн. лев. & \bullet& \bullet&\bullet \\  \hline 
 Поперечная тяга левая &  --- & \bullet& \bullet\\ \hline 
 Панель передка  & --- & \bullet&\bullet \\ \hline 
 Шина передняя левая & ---& \bullet&\bullet \\ \hline 
 Лонжерон передний левый &--- & \bullet& ---\\ \hline 
 Тормозной диск передний левый& --- & \bullet&\bullet \\ \hline
 Тормозной диск передний правый&  ---& \bullet&\bullet \\ \hline  
 Стабилизатор  & --- & \bullet&\bullet \\ \hline 
Порог внутренний левый  & --- & \bullet& \bullet\\ \hline 
Уплотнитель передней левой двери  & --- & ---&\bullet \\ \hline 
 Молдинг крыши слева &---& ---&\bullet \\ \hline 
 Шильдик V8 Biturbo слева &---  & --- &\bullet \\ \hline 
 Стойка переднего стабилизатора слева&---  & --- &\bullet \\ \hline
 Уплотнитель между крылом и дверью слева&---  & --- &\bullet \\ \hline
Кронштейн переднего бампера левый &---  &--- &\bullet \\ \hline
Электропроводка бампера & --- & \bullet&\bullet \\ \hline

\end{longtable}}
\noindent \bullet -- деталь присутствует в перечне повреждений \\
--- деталь отсутствует в перечне повреждений.

Приведенная таблица \ref*{tab:sravnenie}  наглядно иллюстрирует то, что в заключении эксперта № 00182/18 отсутствует часть деталей, присутствующая в акте осмотра  автомобиля \тс \, № 0652 от 20.06.2019 г. ООО "Фаворит" и в заказ-наряде № ЗН19005605 от 22.06.2019 ООО "СБСВ-КЛЮЧАВТО СЕВЕР" на ремонт (дефектовка а/м после ДТП). При этом, эксперт \чел в своем заключении не представил обоснования исключения деталей из расчета.
\par С целью определения степени влияния на итоговые результаты исследования,  рецензентом был выполнен расчет стоимости замены некоторых  исключенных позиций: жгута бампера, надписи крыла переднего левого, уплотнителя двери, тяги рычага левого, шины передней левой, тормозного диска. Расчет производился   в специализированном программном продукте, содержащем нормативы трудоёмкости работ, регламентируемые изготовителями транспортного средства     AudaPadWeb, лицензионное соглашение № AS/\- APW-658  RU-P-409-409435. \\[3mm]
%
% Идентификационный код автомобиля (VIN)  \vin \, содержит следующую информацию о транспортном средстве, имеющую значение для 	дачи заключения:
% 
\begin{figure}[H]
	\centering
	\includegraphics[width=0.8\linewidth]{images/itog}
	\caption[]{Итоговый расчет 934 809 руб.}
	\label{fig:itog}
\end{figure}

\noindent Калькуляция выполнена при следующих допущениях:\\
-- возникшие правовые отношения  регулируются Гражданским кодексом РФ;\\
-- стоимость коммерческого нормо-часа работ применена  с учетом условий регионального рынка услуг и сложившихся средних расценок по видам работ, типу ТС, а также по маркам и моделям ТС  и   составляет 3900 р/ч для данного транспортного средства. Трудоёмкость работ по разборке/сборке/замене  соответствует трудоемкости работ, рекомендованной заводом изготовителем ТС. Расчет стоимости ремонта, согласно положениям Методики [2] производится с учетом  применения оригинальных запасных частей, которые поставляются изготовителем КТС авторизованным ремонтным организациям. Техническое состояние запасных частей учитывается коэффициентом износа \footnote{Согласно п. 7.8.\, Методики [2]  для случаев, не регулируемых законодательством об ОСАГО, для составных частей КТС значение износа принимается равным нулю если  срок эксплуатации КТС не превышает пяти лет, иные повреждения и следы ремонта отсутствуют, признаки интенсивной эксплуатации отсутствуют }, что в совокупности с установкой оригинальных запасных частей в максимальной степени отвечает понятию «восстановительный ремонт», то есть восстановления состояния КТС, при котором используются установленные изготовителем составные части, но с использованным частично ресурсом.\\
В результате произведенного расчета получено значение  стоимости ремонта исключенных экспертом деталей автомобиля \тс,\, составляющее  934 809 рублей и превышающее стоимость восстановительного ремонта с учетом износа (857 093 рубля) согласно заключению эксперта № 00182/18.\\
\indent Таким образом,  исключенные экспертом детали оказывают существенное влияния на итоговые результаты исследования. Добавление жгута бампера, надписи крыла переднего левого, уплотнителя двери, тяги рычага левого, шины передней левой, тормозного диска в совокупности с учетом требований изготовителя по замене парных деталей приводят к увеличению стоимости ремонта вдвое.


\section{ Анализ представленных экспертом выводов}

Вывод эксперта о стоимости восстановительного ремонта автомобиля \тс \, основывается на перечне поврежденных деталей, из которого экспертом \чел не\-обосновано  исключены поврежденные дорогостоящие детали, является неполным и недостоверным.

Вывод эксперта о рыночной стоимости транспортного средства   на момент ДТП в исправном техническом состоянии основывается на не полных аналогах, имеющих существенные отличия от исследуемого автомобиля \тс. \, Следовательно, определенная экспертом \чел величина рыночной  стоимости является недостоверной и ведет к ошибкам в определении зависимым от нее величинам. 

Вывод эксперта о величине утраты товарной стоимости автомобиля \тс \, сделан экспертом на основании произведенных расчетов, основанных на сведениях о виде, характере и объеме повреждений и ремонтных воздействий, а так же на значении рыночной стоимости автомобиля.   
\par Величина УТС ( $ C_\text{YTC} $)  определяется на дату оценки (исследования) по формуле: 

\begin{equation}\label{uts}
C_{YTC} = C_{KTC} \cdot \dfrac{\sum К_{УТСi}}{100\%}, \text{руб.},
\end{equation}

\noindent $ C_{KTC} $ -- стоимость КТС на дату оценки (исследования), руб;\\
$ K_{YTCi} $ -- коэффициент УТС по i-му элементу КТС, ремонтному воздействию, \%. 
\par Значения коэффициентов УТС ($ C_{KTC} $) определены по результатам экспертой практики и приведены в приложении [2,Приложение 2.9].

\par Так как величина утраты товарной стоимости прямо пропорциональна рыночной стоимости  автомобиля, количеству и степени повреждений автомобиля, то в случае неверного, ошибочного определения рыночной стоимости  транспортного средства расчитанная величина  УТС  так же  является недостоверной.
 


\subsection{ ОЦЕНКА ЗАКЛЮЧЕНИЯ}


Заключение эксперта должно основываться на положениях, дающих возможность проверить обоснованность и достоверность сделанных выводов на базе общепринятых научных и практических данных. Исследование, результаты которого изложены в представленном заключении, не является полным, всесторонним и объективным, что противоречит действующим требованиям о том, что заключение должно быть объективным, обоснованным и полным (то есть, содержать исчерпывающие ответы на поставленные вопросы), всесторонним, тщательным, проводиться в пределах специальности эксперта, на строго научной и практической основе с использованием современных достижений науки и техники. Ответы на поставленные вопросы не являются исчерпывающими, выводы эксперта не обоснованы и вызывают сомнение в правильности.


\subsection{ ВЫВОД}

Заключение эксперта № 00182/18 по гражданскому делу № 2-802/19 по иску Хачатряна Андраника Гайковича к Мокину Игорю Витальевичу о возмещении материального ущерба, причиненного дорожно-транспортным происшествием произведено с нарушениями действующего законодательства, методических рекомендаций проведения данного вида исследований и вызывают сомнения в компетентности эксперта,  что является основанием для назначения повторной судебной автотехнической экспертизы.\textbf{}
\vspace{7mm}
\relax
Приложение:\\
\vspace{3mm}
\textit{\small 
\noindent	Приложение № 1. Расшифровка модельных опций ТС \тс \\
%	Приложение № 1. Акт осмотра ТС \тс\\
%	Приложение № 2. Фототаблица повреждений ТС \тс\\
	Приложение № 3. Калькуляция стоимости восстановительного ремонта ТС \тс\\
%	Приложение № 4. Цифровые копии регистрационных документов ТС\\
%	Приложение № 5. Цифровая копия постановления по делу об административном правонарушении дорожно-транспортном происшествии\\
	Приложение № 6. Правоустанавливающие документы\\}


\vspace{15mm}


\noindent Рецензент \hfill  \underbar{ }Мраморнов А.В.

\vspace{20mm}

\input bibliography

%\includepdf[pages=-]{foto.pdf}
\includepdf[pages=-]{myfile.pdf}
%\includepdf[pages=-]{calc.pdf}

%\input{sections/pod}
%\begin{titlepage}
\includegraphics[width=\paperwidth,height=\paperheight]{777.pdf} 
\end{titlepage}


%\input{sections/nteOSAGOstandart}
%\setcounter{page}{1}
\clubpenalty=100000  % Недопуск Висячей строки в начале страницы
\widowpenalty=100000 %Недопуск висячей строки в конце абзаца
%%%%%%%%%%%%%%%%%%%%%%%%%%%%%%%%%%%%%%%%
%      Шапка экспертной организации  
%%%%%%%%%%%%%%%%%%%%%%%%%%%%%%%%%%%%%%%%
%%%%%%%%%%%%%%%%%%%%%%%%%%%%%%%%%%%%%%%%%
%
%   Экспертная организация ООО Южнорегиональная экспертная группа
%
%%%%%%%%%%%%%%%%%%%%%%%%%%%%%%%%%%%%%%%%%
\noindent %\qrcode[height=21mm]{\NomerDoc от \окончено }  %%% Добавлен QR-Code
\begin{pspicture}(21mm,21mm)
\obeylines
\psbarcode{%
%	\NomerDoc от \окончено
	BEGIN:VCARD^^J
	VERSION:4.0^^J
	%N:Мраморнов; Александр; Вчеславович^^J
	FN:Александр Мраморнов^^J
%	ORG:IP Alexandr Mramornov^^J
	TITLE: эксперт
	ORG: ИП
	URL:http://www.yourexp.ru^^J
	EMAIL:4516611@gmail.com^^J
	TEL:+7-918-451-6611^^J
	ADR:г. Краснодар, с/т № 2 А/О «Югтекс», ул. Зеленая, 472^^J
	END:VCARD
}{width=1.0 height=1.0}{qrcode}%
\end{pspicture}%%% Добавлен QR-Code
\vspace{-4mm}
\begin{center}
	\large\textbf{ИНДИВИДУАЛЬНЫЙ\quad ПРЕДПРИНИМАТЕЛЬ  \\[-1.5mm] МРАМОРНОВ  АЛЕКСАНДР ВЯЧЕСЛАВОВИЧ \\[-5.5mm]}
	%  
	\noindent\rule{\textwidth}{2pt}\\[-6mm]  % Горизонтальная линия
	% \line(1,0){460}% (1,0) -горизонтальная линия, и (0,1) - вертикальная 
\end{center}

\begin{center}
	\begin{footnotesize}\setstretch{0.3}
		%	\small\textbf\setlength   	%\raisebox{5mm}
		\vspace{-3.5mm}г. Краснодар, с/т № 2 А/О «Югтекс», ул. Зеленая, 472, 
		Телефон: 8-918-451-66-11, e-mail: 4516611@gmail.com\\ [-2mm]{ИНН\quad 231200665168\quad ОГРНИП \quad 310231220400043}
	\end{footnotesize}	\\[10mm]
\end{center}


\begin{flushright}
% 
	 \hfill	Краснодар, 2020    \\[8mm]
\end{flushright}
\begin{center}
	\LARGE\textbf{ЭКСПЕРТНОЕ ЗАКЛЮЧЕНИЕ}
	\bigskip\\[0mm]
	%	{\normnumxtbf{\NomerDoc}}	}{den}
\end{center}
\par
\vspace{-6mm}
\noindent независимой технической экспертизы по определению размера расходов на восстановительный ремонт транспортного средства   \тс  \\[2mm]

%\raggedright 
%\def\hrf#1{\hbox to#1{\hrulefill}}
\noindent \textbf{№ \NomerDoc}\hfill           \textbf{\окончено}\\%[2mm]
%Приостановлено\hfill      \datastop\\
%Возобновлено\hfill          \datarestart\\
%Окончено\hfill                \dataend\\%[4mm]

\noindent\parbox[l][16mm]{16.5cm}
{\def\hrf#1{\hbox to#1{\hrulefill}}
	\noindent Начато\hfill            \datastart\\%[2mm]
	%	Приостановлено\hfill      \datastop\\
	%	Возобновлено\hfill          \datarestart\\
	Окончено\hfill                \окончено
}
\relax

%%%%%%%%%%%% Если судебка
%
%\datastart г. ~в {\small ООО~ "ЮЖНО-РЕГИОНАЛЬНАЯ ЭКСПЕРТНАЯ ГРУППА"} \,  при определении  \, \sud  \,  от \, \dataopr \, о назначении \opr \, по гражданскому делу \delonum \, поступили:
%
%\begin{enumerate}\setlist{nolistsep}\item  Материалы гражданского дела \delonum \, в двух томах, том 1 на 276 листах, том 2  на 143 листах.\\[-2mm]
%	%	\item  
%\end{enumerate}
%
%%%%%%%%%%%%  Если независимая
\vspace{4mm}
Составлено на основании	договора № \NomerDoc\,  возмездного оказания услуг по проведению независимой технической экспертизы (далее экспертиза)  транспортного средства и письменного заявления заказчика о проведении экспертизы.

Заказчик  экспертизы: \заказчик, \адресзаказчика. 

Полис ОСАГО: \polis.

% Документ, удостоверяющий личность заказчика: водительское удостоверение    03\ 16\ 422344\ выдан 09.06.2011

%Транспортное средство виновника ДТП:  не предоставлялось.

\paragraph*{}
Экспертиза произведена  экспертом--техником
%{\small ООО "ЮЖНО-РЕГИОНАЛЬНАЯ ЭКСПЕРТНАЯ ГРУППА"}
\,  Мраморновым Александром Вячеславовичем, имеющим высшее техническое образование по специальности «техническая физика», диплом РВ № 311964 от 28.02.1989, квалификация -- инженер-физик, специальное образование в области оценки: Диплом ПП-1 № 037211 Российской экономической академии им. Г.В. Плеханова, квалификация -- оценка и экспертиза объектов и прав собственности, специальное образование в области независимой технической экспертизы транспортных средств: Диплом ПП-I № 424167, квалификация: эксперт-техник (специализация 150210 специальности 190601.65 – Автомобили и автомобильное хозяйство), состоящий в Государственном реестре экспертов-техников (№ в реестре 256, https://data.gov.ru/opendata/7707211418-experts,  общий трудовой  стаж 30 лет, стаж  экспертной работы  12 лет. \par Заключение подготовлено по месту фактического расположения ИП по адресу: г. Краснодар, с/т № 2 А/О «Югтекс», ул. Зелёная, 472.
  % Шапка организации ИП
%%   вопросы экспертизы
\subsection{Вопросы экспертизы}
\begin{enumerate}
\item  <<Установить наличие, характер и объем (степень) технических повреждений транспортного средства  \tc \,>>?\\[-2mm]
%\item  <<Установить причины возникновения технических повреждений транспортного средства \tc \,и возможность их отнесения к рассматриваемому дорожно-транспортному происшествию (далее ДТП)>>?\\[-2mm]
\item <<Установить технологию, объем восстановительного  ремонта транспортного средства \tc \,>>?\\[-2mm]
\item <<Установить размер затрат на восстановительный ремонт (с учётом износа) транспортного средства \tc \,>>?\\[-2mm]
%	
\end{enumerate}
\subsection{Исходные данные} %Название по шаблону 
Исходные  данные,  необходимые  для   исследования,  изложены   в  заявлении о проведения исследования  (оценки)  колесного  транспортного  средства (далее —  KTC):

\begin{enumerate}\item Цифровые фотоснимки
транспортного средства \тс \, в поврежденном состоянии.\\[-2mm]
	%\item Свидетельство о регистрации \свид транспортного средства \тс \\ [-2mm]
\item Копия экспертного заключения № 006/06/19 <<Об определении рыночной стоимости затрат на восстановление а/м МАЗДА 6 г/н О552 СА 123 после ДТП иутраты товарной стоимости (УТС) >>\\ [-2mm]
%	\item Постановление об административном правонарушении дорожно-транспортном %происшествии, имевшем место   \датадтп  с участием  ТС \тс,\, согласно которой %на ТС \тс \, в результате ДТП повреждено:\, "\повреждения".  \\[-2mm]
%	\item Полис страхования  ОСАГО \polis.
	\end{enumerate}
%
\addcontentsline{toc}{section}{Использованные нормативы и источники информации}
%\subsubsection*{}
%\left( \addcontentsline{toc}{section}{Использованные нормативы и источники информации}

\subsection{Использованные нормативы и источники информации}
%
\begin{enumerate}
\item 
Махнин\,Е.\,Л., Новоселецкий\, И.\,Н., Федотов\, С.\,В. \emph{Методические рекомендации по проведению судебных автотехнических экспертиз и исследований колёсных транспортных средств в целях определения размера ущерба, стоимости восстановительного ремонта и оценки} // -- М.: ФБУ РФЦСЭ при Минюсте России, 2018.-326 с.  ISBN 978-5-91133-185-6.
%
%
%
%
\item ТУ 017207-255-00232934-2014 \emph{Кузова автомобилей LADA. Технические требования при приёмке в ремонт, ремонте и выпуске из ремонта предприятиями дилерской сети ОАО "АВТОВАЗ"}//  ОАО НВП "ИТЦ АВТО", 2014
%
\item Смирнов  В.Л., Прохоров  Ю.С., Боюр В.С.  и др. \emph{Автомобили ВАЗ. Кузова. Технология ремонта, окраски и  антикоррозионной защиты. Часть II}// - Н.Новгород: АТИС, 2001.- 241с.
%
\item 
Савич Е.Л. \emph{Техническое  обслуживание  и  ремонт  легковых  автомобилей} : учеб. пособие / Е.Л. Савич, М.М. Болбас, В.К. Ярошевич ; под общ. ред. Е.Л. Савича. -Мн. : Вышэйшая школа,  2001. - 479 с. - ISBN985-06-0502-2.
%
\item 
Автомобили ВАЗ-2121, 21213, 21214, 2131 и их модификации: <<Трудоемкости работ (услуг) по техническому обслуживанию и ремонту>> /Куликов А.В., Христов П.Н., Климов В.Е.,  Боюр В.С., Рева В.В., Зимин В.А., Завьялова Н.Н., Хлыненкова Г.А. -- ИТЦТ "АвтоВАЗтехобслуживание", Тольяти -- 2005. 
%
\item
Автомобили LADA SAMARA и их модификации: <<Трудоемкости работ (услуг) по техническому обслуживанию и ремонту>> /Куликов А.В., Христов П.Н., Климов В.Е., Рева В.В., Боюр В.С., Васильев М.В., Фахрутдинов Р.В.,  Прудских Д.А., Гирко В.Б., Шмелева В.А., Зимин В.А. --  ОАО НВП "ИТЦ АВТО",  -- 2006. - 252 стр.
%
\item 
Автомобили LADA PRIORA. Трудоемкости работ (услуг) по техническому обслуживанию и ремонту /Куликов А.В., Христов П.Н., Климов В.Е., Рева В.В., Козлов П.Л., Боюр В.С., Прудских Д.А., Шмелева В.А., Зимин В.А. -- ООО "ИТЦТ АВОСФЕРА", Тольяти -- 2009. -- 344 с.
%
\item 
{Трудоемкости работ по техническому обслуживанию и ремонту автомобилей автомобилей Lada  Granta}/   \url{https://docplayer.ru/30250248-Trudoemkosti-rabot-po-teh\-nicheskomu-obsluzhivaniyu-i-remontu-avtomobiley-lada- granta.html}.
%
%
\item
{Специализированное программное обеспечение для расчёта стоимости  восстановительного ремонта, содержащее нормативы трудоёмкости работ, регламентируемые изготовителями транспортного средства}//   AudaPadWeb, лицензионное соглашение № AS/APW-658  RU-P-409-409435.
%
%
%
\item
{Специализированное программное обеспечение для расчёта стоимости  восстановительного ремонта, содержащее нормативы трудоёмкости работ, регламентируемые изготовителями транспортного средства ОАО «АвтоВАЗ», ЗАО «Джи-Эм-АвтоВАЗ», ОАО «СеАЗ» и ОАО «ЗМА»}//   Автосфера АС:Смета, v.3.9.11// ООО "ИТЦ «ИнтегроМаш», \url{https://autosmeta.pro}.
%
%
%
\item Информационный портал по техническому обслуживанию и ремонту автомобилей	 ВАЗ:\\ \url{www.autosphere.ru}.

%%
\end{enumerate}

%\bibliographystyle{utf8gost705u}  %% стилевой файл для оформления по ГОСТу
%\bibliography{biblio}     %% имя библиографической базы (bib-файла)
%%%%%%%%%%%%%%%%%%%%%%%%%%%%%%%%%%%%%%%%%%%%%%%%%%%%%%%%%%%%%%%%%%%%%%%%%%%%%%%%%
\subsection{Технические средства}  %% Список не удалять!!!
\begin{itemize}
%
%\item Диагностический сканер SDconnect   с программным обеспечением Xentry Diagnostics v19.11.3.1
%\item 	Линейка масштабная магнитная с цветографической шкалой, 100мм
%\item  Рулетка измерительная металлическая, 5м
%%\item Универсальный стенд для измерения углов установки колес Hunter Engineering %ProAlign с программным инструментом регулировки схождения колес без блокировки руля %автомобиля WinToe
%%\item Цифровой фотоаппарат Canon 760D s/n 143032001327 с объективом Canon EF-S %18-135, тип используемой памяти: Transcend,  32Gb
\item  Специализированное программное обеспечение для расчёта стоимости  восстановительного ремонта, содержащее нормативы трудоёмкости работ, регламентируемые изготовителями транспортного средства     AudaPadWeb, лицензионное соглашение № AS/\- APW-658  RU-P-409-409435
\item  Программа обработки фото-видео изображений ImageJ, разработчик  Wayne Rasband (wa-yne@codon.nih.gov),
свободная лицензия GPL.
\item  ПЭВМ под управлением операционной системы Windows 10 с установленным набором макрорасширений LaTeX системы компьютерной вёрстки TeX, cвободная лицензия LaTeX Project Public License (LPPL). 
%	
	\end{itemize}
%%%%%%%%%%%%%%%%%%%%%%%%%%%%%%%%%%%%%%%%%%%%%%%%%%%%%%%%%%%%%%%%%%%%%%%%%%%%%%%%%%%%%%%%%%%%%%%
\subsection{Условные обозначения}
\begin{description}
%	 
%%\item[АВС] --Антиблокировочная система
\item[АМТС] --автомототранспортное средство
\item[ДВС] --двигатель внутреннего сгорания
\item[ДТП] --дорожно--транспортное происшествие
\item[гос.\,рег.\,знак] -государственный регистрационный знак
\item[КТС] --колесно транспортное средство 
\item[ЛКП] --лакокрасочное покрытие
%\item[л.д.] --Лист дела
%%\item[Колесо турбины]  -- крыльчатка турбины
\item[ТС] --транспортное средство
%\item[ТK, ТКР] -- Турбокомпрессор. Состоит из двух частей: турбины и компрессора, объединенных общим валом. Вал вращается в подшипниках, размещенных в центральном корпусе ТК
%\item[ЭБУ] --Электронный блок управления
%\item[DTC] --Diagnostic Trouble Codes, диагностические коды неисправностей
%\item[FRAME] --номер кузова транспортного средства, выпущенного для продажи на внутреннем рынке Японии и содержащий информацию производителя о транспортном средстве
\item[OBDII] -- On-board diagnostics, протокол бортовой диагностики автомобиля
\item[SRS] -- система пассивной защиты водителя и пассажиров
\item[VIN] --vehicle identification number, 17--значный идентификационный номер транспортного средства, соответствующий стандарту ISO 3779--2012.
%
\end{description}

\subsection{Методы исследования}

\begin{itemize}
\item  Органолептический метод – исследование и оценка качества объектов с помощью органов чувств
%\item 	Прямой измерительный метод – путем измерения размеров деталей специальными %измерительными приборами
\item Расчётный метод (косвенный измерительный метод) – путём расчётов различных параметров на основе результатов измерений и других данных
\item Экспертный метод (метод экспертной оценки) — совокупности операций по выбору комплекса или единичных характеристик объекта, определению их действительных значений и оценкой экспертом соответствия их установленным требованиям и/или технической информации
%%	\item Метод натурной реконструкции??
\end{itemize}

\subsection{Ограничения и пределы применения полученных результатов}

Следующие допущения и условия, ограничивающие пределы применения полученных результатов, являются неотъемлемой частью данного экспертного заключения.
\begin{itemize}
\item  {Результаты, полученные экспертом-техником, носят рекомендательный консультационный характер и не являются обязательными. Исполнитель высказывает своё субъективное суждение о наиболее вероятных будущих (абстрактных) расходах, их предполагаемом размере и дает заключение в пределах своей компетенции.}
\item { Под компетенцией эксперта-техника понимают его знания и опыт в области теории и методов экспертных исследований ТС, а также круг полномочий, представленных ему законом, и вопросов, которые он может решать на основе своих специальных познаний.
В компетенцию эксперта-техника входит исследование технического состояния поврежденного ТС в целях установления характера повреждений ТС, установления причины возникновения технических повреждений технологии, методов, стоимости его ремонта.}
\item  {Исполнитель в рамках своих обязательств по заключенному договору об экспертном обслуживании признает свою ответственность перед заказчиком и настоящим утверждает, что экспертное заключение выполнено профессионально, тщательно и с должной заботливостью и внимаем, как это обычно принято для компетентного специалиста в области технической экспертизы ТС при ОСАГО, а полученная величина восстановительных расходов, разумна и реальна.}
\item  {Исполнитель считает, что поскольку, по общему правилу, оценка доказательств является прерогативой и компетенцией органа дознания, следствия или суда, а в досудебном порядке - страховщика, постольку после проверки результатов экспертизы последним, их признания и принятия решения о выплате страхового возмещения этап возможного оспаривания достоверности исследований между заказчиком и исполнителем завершен.
Соответственно, обязанности Исполнителя по договору являются надлежаще исполненными в полном объеме и от исполнителя не требуется свидетельствовать по поводу произведённого исследования перед третьими лицами.}
\item  {Отдельные части настоящего экспертного исследования не могут трактоваться раздельно, а только в связи с полным текстом о проведенных расчетах.}
\item  {Исходные данные, использованные исполнителем при подготовке экспертного заключения, получены из надежных источников и считаются достоверными. Тем не менее, исполнитель не может гарантировать абсолютную точность, поэтому там, где это, возможно, делаются ссылки на источники информации.}
\item  {В процессе экспертного исследования специальная юридическая экспертиза документов, касающихся прав собственности на ТС, не проводилась.}
\item  {Суждения, содержащиеся в экспертном заключении, основываются на текущей ситуации на дату аварии и, в будущем, могут быть подвержены изменениям.
Исполнитель не принимает на себя никакой ответственности за изменение экономических, юридических и иных факторов, которые могут возникнуть после даты исследования и повлиять на результаты технической экспертизы.
Данное заключение составлено на основании Правил Независимой Технической Экспертизы и может применяться только при решении вопроса о выплате страхового возмещения по ОСАГО.
}\end{itemize}
% 
\section{Исследование}
%
%%\subsection{Исследование предоставленных на экспертизу документов}
%
Настоящее исследование проводится на основании материалов, предоставленных Заказчиком, а также на основании данных, самостоятельно полученных экспертом-техником. Выводы, содержащиеся в настоящем Заключении, могут расцениваться как достоверные только в контексте того количества информации, на основании которого они были сделаны. При поступлении дополнительной или измененной информации данные выводы могут быть
скорректированы. %При анализе скрытых повреждений экспертом-техником не принимается во внимание наличие или отсутствие записей о них в документах компетентных органов, в связи с отсутствием у сотрудников компетентных органов объективной возможности  идентификации таких повреждений на месте происшествия.
 \par Из предоставленных материалов   экспертом-техником установлена следующая общая информация об автомобиле, имеющая значение для дачи заключения:\\
 \parbox[]{10cm}{}
	\begin{itemize}
		\item[ ] 
			\begin{description}
			\item[Марка, модель] -- \тс
		%	\item[VIN] -- \vin
		%	\item[Год выпуска] -- \год
			\item[Шасси] -- Отсутствует
			\item[Цвет ЛКП] -- Синий
			\item[Пробег] --  \пробег\,км%, считан с одометра
		%	\item[Двигатель] --
		%	\item[ПТС] --\птс
        %						
		\end{description}
		\end{itemize}
	\subparagraph*{} Идентификационный код автомобиля (VIN)  \vin \, содержит следующую информацию о транспортном средстве, имеющую значение для 	дачи заключения:\\[3mm]
        %
\noindent\parbox[]{10cm}{
\begin{itemize}
	\item[ ] 
	    \begin{description}
	\item[Дата изготовления] \hfill \началоэкспл
	\item[Расположенние руля] \hfill Left
	\item[Двигатель] \hfill бензиновый, 149,6 л.с.
	\item[Объем двигатель] \hfill 1998 $ \text{см}^3 $
	\item[КПП] \hfill АКПП
	\item[Тип кузова] \hfill  Седан
	\item[Количество дверей] \hfill 4 
	%	\item[VDS] --
    %	
		\end{description}
\end{itemize}}\\%[3mm]
    %
\vspace{5mm}
\textit{Источник: https://emex.ru/catalogs/original/?screen=units\&vin=/\вин}\\
    %Пробег автомобиля  расчетный, согласно [1]  составляет 
  \begin{figure}[!h]
	\centering
	\includegraphics[width=0.98\linewidth]{images/1}
	\caption{{\footnotesize {Общие виды поврежденного ТС \тс. Фототаблица предоставлена заказчиком}}}
	\label{ris:images/1}
\end{figure}
 
Согласно сведениям, содержащимся в предоставленных материалах, \osm \, экс\-пертом-техником Яковлевым С.В. проведен осмотр поврежденного транспортного средства \tc.\,
% Осмотр проводился в сухую, ясную погоду с 14 час. 00 мин. до 14 час. 30 мин. на открытой площадке по адресу:  г. Краснодар, ул. Кореновская, 24. При осмотре присутствовали: владелец ТС \tc \, \владелец. 
Соответствие маркировочных обозначений на кузове представленного ТС записям в регистрационных документах ТС экс\-пертом-техником установлено. Видимые изменения конструкции ТС отсутствуют.  Представленный на исследование автомобиль \tc\, имеет кузов типа "\типкузова". Кузов автомобиля окрашен рефлексной %(лессирующей) 
с "металлическим" эффектом эмалью %(краской) 
синего цвета.

\renewcommand\baselinestretch{0.86}\small\normalsize 
\subsection{\underline{По  вопросу}\, \, \,	\textbf{\small{<<Установить наличие, характер и объем (степень) технических повреждений транспортного средства  \tc \,>>?}}}
\renewcommand\baselinestretch{1.2}\small\normalsize
%
% 1.\ Наличие, характер и объем технических повреждений, а так же  планируемые (предполагаемые) ремонтные воздействия для восстановления поврежденного автомобиля  исследованы в присутствии заинтересованных лиц, зафиксированы в акте осмотра № 006/06/19  от \osm% \NomerDoc \, (Приложение №1)
% ,  и фотоматериалах по принадлежности (Приложение №2). 
%
Первичное установление наличия и характера повреждений транспортного средства, в отношении которых определяются расходы на восстановительный ремонт, в соответствии с  Единой методикой определения размера расходов на восстановительный ремонт 
%в отношении поврежденного транспортного средства  
утверждены Положением Банка России от «19» сентября 2014 года № 432-П (Единой методикой), должно производится во время осмотра транспортного средства и фиксироваться актом осмотра, в который  должны включаться сведения о повреждениях транспортного средства, с обязательной  характеристикой поврежденных элементов с указанием расположения, вида и объема повреждения.   
%
В акте № 006/06/19  от \osm\, повреждений транспортного средства (ТС)  не содержатся установленные Единой методикой обязательные количественные показатели (характеристики) повреждений транспортного средства.  Отсутствие количественных характеристик повреждений в представленном акте осмотра предполагает его применение в настоящем исследовании исключительно в виде списка повреждений и перечня ремонтных воздействий, определенных экспертом-техником Яковлевым С.В. для восстановительного ремонта транспортного средства. %  не позволяет эксперту использовать указанный акт осмотра для объективного выбора необходимого и достаточного комплекса работ по восстановительному ремонту транспортного средства.
%
%На момент определения о проведении повторной судебной экспертизы, объект исследования неоднократно подвергался видоизменениям как при исследовании специалистом ООО "ЭКСПЕРТ", так и при производстве первичной судебной экспертизы экспертами ИП Куприянова Виктора Александровича. 
%
\par
Согласно данному акту осмотра № 006/06/19  от \osm \, для устранения повреждений ТС необходимо было произвести следующие ремонтные работы:
%\vspace{\baselineskip}  % вставка пустой строки
% 
\begin{longtable}{|p{1cm}|p{11cm}|p{3cm}|}
\caption[]{\footnotesize {Ремонтные воздействия по акту осмотра № 006/06/19 от \, \osm \, ИП Яковлева С.В.}} \label{tab:4}\\ 
	 \hline
		\rowcolor[HTML]{C0C0C0} 
	%	\multicolumn{1}{|c|}
	%	{\cellcolor[HTML]{C0C0C0}N/N} & Наименование запчасти (материала) & Ремонтное воздействие 
    %	
 \text{N/N} & Наименование запчасти (материала) & Ремонтное воздействие  \\ \hline \endhead
		\Rownum  & Панель задка  & Замена, окраска \\ \hline
		\rowcolor[HTML]{EFEFEF} 
		\Rownum  & Боковина задняя левая   & Замена, окраска \\ \hline
	    \Rownum  & Боковина задняя правая  & Замена, окраска  \\ \hline
		
\end{longtable}
%
\par Таким образом, размер восстановительных расходов (затрат) \тс\,  может быть произведен в объеме сведений, содержащихся в представленной Таблице \ref{tab:4}.
%\vspace{\baselineskip}  % вставка пустой строки
%
%\renewcommand\baselinestretch{0.86}\small\normalsize 
%\subsection{\underline{По  вопросу}\, 	\textbf{\small{2. <<Установить причины возникновения технических повреждений транспортного средства \tc \, и возможность их отнесения к рассматриваемому дорожно-транспортному происшествию (ДТП)>>?}}}
%\renewcommand\baselinestretch{1.2}\small\normalsize
%%
%%2.\ Причины возникновения технических повреждений и возможность их отнесения к
%%рассматриваемому ДТП исследованы при осмотре ТС. 
%Для определения причины возникновения повреждений, указанных в Акте осмотра ТС
%% №   (Приложение № 1) 
%экспертом-техником изучены документы, представленные Заказчиком и сведения о ДТП, с участием ТС \вин% \, по данным открытых источников https://гибдд.рф. 
%По предоставленным документам экспертом установлена причина ДТП, установлены обстоятельства ДТП, выявлены повреждения ТС и установлены причины их образования. Проведено исследование характера выявленных повреждений, сопоставление повреждений ТС потерпевшего с повреждениями ТС иных участников ДТП в соответствии со сведениями, зафиксированными в документах о ДТП. Проведена проверка взаимосвязанности повреждений на ТС с заявленными обстоятельствами ДТП, определен объем восстановительных работ (Приложение №1 Акт осмотра ТС). 
%Из открытых банков данных полиции следует, что автомобиль с VIN  \вин\,  как минимум дважды становился участником ДТП.
%Первый раз 29.06.2018  06:40, извещение о ДТП № 030046913, в котором автомобиль получил повреждения задней правой двери, заднего правого порога, заднего правого колеса, подушки SRS справа, Рис. \ref{ris:images/d1} и второй раз 22.05.2019 06:50, извещение о ДТП № 030034947, в котором автомобиль получил повреждения деталей передней левой и задней частей кузова, Рис. \ref{ris:images/d2}.
%
%
%\begin{figure}[H]\centering
%	\parbox[t]{0.49\textwidth}
%	{\centering
%		\includegraphics[width=.49\textwidth]{images/d1}
%		\caption{\footnotesize {Повреждения в ДТП 29.06.2018 }}
%		\label{ris:images/d1}}
%	\hfil \hfil%раздвигаем боксы по горизонтали 
%	\parbox[t]{0.49\textwidth}
%	{\centering
%		\includegraphics[width=.49\textwidth]{images/d2}
%		\caption{\footnotesize {Повреждения в ДТП 22.05.2019}}
%		\label{ris:images/d2}}
%\end{figure}
%%
%{\noindent  \footnotesize \tikz \fill [red] (1,0.5) rectangle (0.1,0.1); --{\footnotesize  Вмятины, вырывы, заломы, перекосы, разрывы и другие повреждения с изменением геометрии элементов (деталей) кузова и эксплуатационных характеристик ТС.}\\
%\tikz \fill [yellow] (1,0.5) rectangle (0.1,0.1); --  {\footnotesize Повреждения колёс (шин), элементов ходовой части, стекол, фар, указателей поворота, стоп-сигналов и других стеклянных элементов (в т.ч. зеркал), а также царапины, сколы, потертости лакокрасочного покрытия или пластиковых конструктивных деталей и другие повреждения без изменения геометрии элементов (деталей) кузова и эксплуатационных характеристик ТС.}\\[1mm]
%%%%%%%%%%%%%%%%%%%%%%%%%%%%%%%%%%%%%%%%%%%%%%%%%%%%%%%%%%%%%%%%%%%%%%%%%%%}%%%%%%%%%%%%%%%%%%%%%%%%%%%%%%%%%%%%%%%%%%%%%%%%%%
%
%\renewcommand\baselinestretch{1.2}\small\normalsize
%\begin{spacing}{1.25}Таким образом, перечень повреждений, указанный в акте осмотра № 006/06/19 от \osm \, ИП Яковлева С.В. может соответствовать повреждениям автомобиля \тс\,, полученным в результате ДТП \датадтп. 
%\end{spacing}
%\renewcommand\baselinestretch{0.86}\small\normalsize 
%\subsection{\underline{По  вопросу}\, \, \,	\textbf{\small{3. <<Установить технологию, объем восстановительного  ремонта транспортного средства \tc \,>>?}}}
%\renewcommand\baselinestretch{1.2}\small\normalsize
%%%%%%%%%%%%%%%%%%%%%%%%% Для автомобилей, старше семи лет:
%Колесное транспортное средство сроком эксплуатации более 7 лет относится к категории транспортных средств с граничным сроком эксплуатации [1], для которой возможно применение ремонтных операций при условии экономической целесообразности и  технической возможности.  
%
\renewcommand\baselinestretch{0.86}\small\normalsize 
\subsection{\underline{По  вопросу}\, \, \,	\textbf{\small{<<Установить размер затрат на восстановительный ремонт (с учётом износа) транспортного средства \tc \,>>?}}}
\renewcommand\baselinestretch{1.2}\small\normalsize
%
В соответствии с существующей экспертной методикой размер расходов на восстановительный ремонт определяется исходя из стоимости ремонтных работ (работ по восстановлению, в том числе окраске, контролю, диагностике и регулировке, сопутствующих работ), стоимости используемых в процессе восстановления транспортного средства деталей (узлов, агрегатов) и материалов взамен поврежденных. 
%                                         
%В соответствии с принятой экспертной методикой [1], стоимость восстановительного ремонта АМТС  $ C_p $ определяется по формуле:
%%
%\begin{equation}\label{eq:r}
%C_\text{вp} =C_p + C_\text{м} + C_\text{зч}\cdot\left( 1-\frac{\text{И}}{100}\right)  \,\,\,\, \text{где:}
%\end{equation}
%%
%%
%\begin{itemize}
%%	
%\item[ ]$C_\text {р} $ --  стоимость ремонтных работ по восстановлению КТС, руб.;
%\item[ ]$ C_\text{м} $ --  стоимость необходимых ремонтных материалов, руб.;
%\item[ ]$ C_\text{зч} $ --  стоимость новых запасных частей, руб;
%\item[ ] $ \text{И} $ -- коэффициент износа составной части, подлежащей замене, \%.
%\end{itemize}
%%
%%
%Коэффициент износа составных частей (И) КТС (кроме автобусов и грузовых автомобилей) при определении стоимости восстановительного ремонта расчитывается по формуле:
%
%\begin{equation}\label{eqsnos}
%\text{И} =\text{И1}\cdot\text{П}+\text{И2}\cdot \text{Д}, \%  \,\,\,\, \text{где:}
%\end{equation}
%
%\begin{itemize}
%	\item [] $ \text{И1} $ --усредненный показатель износа на 1000 км пробега, \%; 
%	\item [] $ \text{П} $ -- общий пробег (фактический или расчетный) за срок эксплуатации КТС, тыс.км;
%	\item [] $ \text{И2} $ -- усредненный показатель старения за 1 год эксплуатации, \%;
%	\item [] $ \text{Д} $ -- срок эксплуатации КТС (от даты изготовления КТС до момента, на который определяется износ), лет. 
%\end{itemize}
%
%Для исследуемого автомобиля \тс, согласно справочным таблицам [1]:
%\begin{equation}\label{eqsnosr}
%\text{И} =\text{И1}\cdot\text{П}+\text{И2}\cdot \text{Д} = 0.23\cdot 214  + 0.85\cdot 9 = 49 \, \%
%\end{equation}
%%%%%%%%%%%%%%%%%%%%%%%%%%%%%%%%%%%%%%%%%%%%%%%%%%%%%%%%%%%%%%%%%%%%%%%%%%%%%%%%%%%%%%%%%%%%%%%%%%%%%%%%%%%%%%%%
Стоимость восстановительных работ $ C_{\text{вр}} $ определяется на основании норм трудоёмкостей $ T_i $, \, предусмотренных заводом-изготовителем, и стоимостных параметров $ C_{i\text{нч}} $ (стоимости нормо-часа) работ по техническому обслуживанию и ремонту АМТС.  Расчет размера расходов (в рублях) на восстановительный ремонт производится по формуле:
\begin{equation}\label{eq:cr}
C_{\text{вр}}  =\sum{C_{ip}}= \sum\left({T_{ij}}\cdot {C_{i\text{нч}}}\right) + \sum{C_{ip^{\text{\,\,\,руб}}}} , \,\,\,\text{где:} 
\end{equation}
%\vspace{2mm}
\begin{itemize}
	\item[ ]$ C_{ip} $ -- стоимость работ i-го вида: $C_\text {зам} $, $ C_\text{восст} $, $ C_\text{рег} $, $C_\text{контр} $, $ C_\text{антикор} $, $ C_\text{зч} $, $ C_\text{ом} $,$ C_\text{соп} $, $ C_\text{вм} $, руб;
	\item[ ]$ T_{ij} $ -- трудоёмкость j-й операции(комплекса) по i-му виду работ, руб;
	\item[ ]$ C_{i\text{нч}} $ -- стоимость нормо-часа по i-му виду работ, руб;
	\item[ ]$ C_{ip^\text{\,\,руб}} $ -- стоимость работ $ C_{ip} $, принятая непосредственно в денежном выражении, руб.
\end{itemize}
%
\par При определении стоимости восстановительного ремонта АМТС с учётом износа под износом следует понимать количественную меру физического старения АМТС и его элементов, достигнутого в результате эксплуатации, т.е. эксплуатационный износ.
%
Расчёт износа производится в  соответствии с Положением Банка России от «19» сентября 2014 года № 432-П «О единой методике определения размера расходов на восстановительный ремонт в отношении повреждённого транспортного средства» [2].
Износ комплектующих изделий (деталей, узлов, агрегатов) рассчитывается по следующей формуле:
%
\begin{equation}\label{eq:I}
\text{И}_{\text{ки}} 
= 100\cdot\left( 1-e^ {-\left( \Delta_{T} \cdot T_{\text{КИ}} + \Delta_{L} \cdot L_{\text{КИ}} \right)}\right), \,\,\,\,\text{где:}   
\end{equation}
%
\begin{itemize}
	\item[ ]$ \text{И}_{\text{ки}} $ -- износ комплектующего изделия (детали, узла, агрегата) (процентов); 
	\item[ ]$ e $ -- основание натуральных логарифмов (e =  2,72);
	\item[ ]$ \Delta_{T}$ --  срок эксплуатации комплектующего изделия (детали, узла, агрегата) (лет);
	\item[ ]$ T_{\text{КИ}} $ -- стоимость работ $ C_{ip} $, принятая непосредственно в денежном выражении, руб
	\item[ ]$ \Delta_{L} $ --коэффициент, учитывающий влияние на износ комплектующего (детали, узла, агрегата) величины пробега транспортного средства с этим комплектующим изделием;
	\item[ ]$ L_{\text{КИ}} $ --пробег транспортного средства на дату дорожно-транспортного происшествия (тысяч километров).  
		\end{itemize}
\vspace{5mm}
\par Значения коэффициентов $ \Delta_{T}$  и $ \Delta_{L} $  для различных категорий и марок транспортных средств приведены в п.5. сп. лит~[2]. При этом, на комплектующие изделия (детали, узлы, агрегаты), которые находятся в заведомо худшем состоянии, чем общее состояние транспортного средства в целом, и его основные части, вследствие влияния факторов, не учтённых при расчете износа (например, проведение ремонта с нарушением технологии, не устранение значительных повреждений лакокрасочного покрытия), может быть начислен дополнительный индивидуальный износ. 
Износ шины транспортного средства рассчитывается по следующей формуле:
\begin{equation}\label{eq:sh}
\text{И}_{\text{ш}} = \frac{\text{Н}_{\text{н}}-\text{Н}_{\text{ф}}}{\text{Н}_{\text{н}}-\text{Н}_{\text{доп}}} \cdot{100}\%  \,\,\,\,\text{где:} 
\end{equation}
%
\begin{itemize}
	\item[ ] $ \text{И}_{\text{ш}} $ -- износ шины, \%;
	\item[ ] $ \text{Н}_{\text{н}} $ -- высота рисунка протектора новой шины, мм;
	\item[ ] $\text{Н}_{\text{ф}} $ -- фактическая высота рисунка протектора шины, мм;
	\item[ ] $ \text{Н}_{\text{доп}} $ --минимально допустимая высота рисунка протектора шины в соответствии с требованиями законодательства Российской Федерации, мм.
\end{itemize}
%
\vspace{5mm}
\renewcommand\baselinestretch{1}\small\normalsize
%
%Износ шины дополнительно увеличивается для шин с возрастом от 3 до 5 лет - на 15 процентов, свыше 5 лет - на 25 процентов.
%%%%%%%%%%%%%%%%%%%%%%%%%%%%%%%%%%%%%%%%%%%%%%%%%%%%%%%%%%%%%%%%%%%%%%%%%%%%%%%%%%%5
%В результате исследования   экспертом-техником установлено, что для устранения повреждений \тс \, необходимо  выполнить следующие  работы:
%\begin{center}
%	\begin{tabulary}{\textwidth}{LCL}
%\hline 
%\textbf{Наименование детали}      &   & \textbf{Ремонтное воздействие}\\
%\hline Турбина левая              &   &    Заменить\\
%Блок ДВС                          &   &    Отремонтировать гильзованием, заменой колец и прокладок \\
%	\end{tabulary}  
%\end{center}
%%%%%%%%%%%%%%%%%%%%%%%%%%%%%%%%%%%%%%%%%%%%%%%%%%%%%%%%%%%%%%%%%%%%%%%%%   CSV    %%%%%%%%%%%%%
%\csvautobooklongtable[separator=comma]{test.csv} % semicolon- ;  comma- , pipe- | tab- 	
%%%%%%%%%%%%%%%%%%%%%%%%%%%%%%%%%%%%%%%%%%%%%%%%%%%
%%%%%%%%%%%%%%%%%%%%%%%%%%%%%%%%%%%%%%%%%%%%%%%%%%%%%%%%%%%%%%%%%%%%%%%%%%%%%%%%%
%
%%%%%%%%%%%%%%%%%%%%%%%%%%%%%%%%%%%%%%%%%%%%%%%%%%%%%%%%%%%%%%%%%%%%%%%%%%%%%%%%%
%
%%%%%%%%%%%%%%%%%%%%%%%%%%%%%%%%%%%%%%%%%%%%%%%%%%%%%%%%%%%%%%%%%%%%%%%%%%%%%%%%5
\renewcommand\baselinestretch{1.2}\small\normalsize
%%
%\textbf{Произвести  необходимые для выполнения  ремонта разборочно-сборочные, подготовительные и вспомогательные работы в соответствии с требованиями завода–изготовителя транспортного средства.}\\
%%
Расчет стоимости восстановительных расходов выполнен в программе \auda\, в модуле ОСАГО ПРО. Процент износа ТС \тс\, так же произведен в указанной программе, на момент  исследуемого страхового события составлял 22.96\%. \par Ниже представлены результаты расчета, полная калькуляция стоимости ремонта включена в раздел <<Приложение>> настоящего заключения.
%%\smallskip
\begin{figure}[H]
	\centering
	\includegraphics[width=0.8\linewidth]{images/Screenshot_2}
%%	\caption{}
%%	\label{fig:screenshot001}
\end{figure}
%\begin{figure}[H]
%	\centering
%	\includegraphics[width=0.9\linewidth]{images/screenshot002}
%%%	\caption{}
%	\label{aud}
%\end{figure}
\medskip
\renewcommand\baselinestretch{1.2}\small\normalsize
%%%%%%%%%%%%%%%%  Если не ОСАГО
%Стоимость коммерческого нормо-часа работ применена  с учетом условий регионального рынка услуг и сложившихся средних расценок по видам работ, типу ТС, а также по маркам и моделям ТС  и   составляет  1300 р/ч для данного транспортного средства. \\
%Трудоёмкость работ по разборке/сборке/замене  соответствует трудоемкости работ, рекомендованной заводом изготовителем ТС. 
%Расчет стоимости ремонта, согласно положениям [1] производится с учетом  применения оригинальных запасных частей, которые поставляются изготовителем КТС авторизованным ремонтным организациям. Техническое состояние запасных частей учитывается коэффициентом износа, что в совокупности с установкой оригинальных запасных частей в максимальной степени отвечает понятию «восстановительный ремонт», то есть восстановления состояния КТС, при котором используются установленные изготовителем составные части, но с использованным частично ресурсом. 
%
%%%%%%%%%%%%%%%%%% Если ОСАГО
Стоимость одного нормо-часа работ определена в соответствии с пунктом 3.8.1 Единой методики [2] путем применения электронных баз данных стоимостной информации.
Трудоемкость работ по разборке/сборке/замене  соответствует трудоемкостям работ, рекомендованным заводом изготовителем ТС. Трудоемкости окрасочных работ приняты в соответствии с технологией  AZT (http://www.schwacke.ru/down/azt \_reparaturlackierung\_ru.pdf). Расчет размера расходов на материалы произведен  согласно пункту 3.7.2 Приложения к Единой методике [2]. 
Стоимость запасных частей определена в соответствии с пунктом 3.6.3 Единой методики путем применения электронных баз данных стоимостной информации.
% (по утвержденному справочнику: http://prices.autoins.ru/priceAutoParts/repair\_parts.html.
%
\par Таким образом, в результате проведенных расчетов (см. Приложение, калькуляция \NomerDoc) определена стоимость восстановительного ремонта транспортного средства  \тс, которая составляет 1\,085\,696 (Один миллион восемьдесят пять тысяч шестьсот девяносто шесть) рублей без учета износа,
стоимость восстановительных расходов, с учетом уменьшения стоимости запасных частей вследствие их износа,  составляет 871\,841 (Восемьсот семьдесят одна тысяча восемьсот сорок один) рубль, что с учетом округления составляет 871\,800 (Восемьсот семьдесят одна тысяча восемьсот) рублей.
\nopagebreak
\include{rinok}
\include{go}
%
\section{Выводы}
%
%
\begin{enumerate}
 \item  Наличие, характер и объем (степень) технических повреждений, причиненных ТС, определены при осмотре и зафиксированы в Акте осмотра \NomerDoc.
 % \, и фототаблице повреждений, являющимися неотъемлемой частью настоящего экспертного заключения.
 \\[-2mm]

%\item  Направление, расположение и характер повреждений определены путем сопоставления полученных повреждений, изучения административных материалов по рассматриваемому событию, и  являются  следствиями рассматриваемого ДТП (события).\\[-2mm]

\item  Технология и объем необходимых ремонтных воздействий зафиксированы в калькуляции \NomerDoc \, по определению стоимости восстановительного ремонта транспортного средства \тс. Расчетная стоимость восстановительного ремонта составляет 1\,085\,696 (Один миллион восемьдесят пять тысяч шестьсот девяносто шесть) рублей.
\\[-2mm]
\item  Размер затрат на проведение восстановительного ремонта с учётом износа (восстановительные расходы) транспортного средства \тс \, составляет 871\,800 (Восемьсот семьдесят одна тысяча восемьсот) рублей  рублей.\\[-2mm]   
\item Рыночная стоимость транспортного средства ТС \тс\, на момент повреждения составляла  980000 (Девятьсот восемьдесят тысяч рублей). \\[-2mm]
\item Стоимость годных остатков ТС \тс \, \, составляет 370 000 (Триста семьдесят тысяч) рублей.
\end{enumerate}
\vspace{15mm}
\relax
Приложение к заключению:\\
\textit{
	Приложение № 1. Расшифровка модельных опций ТС \тс \\
	Приложение № 2. Калькуляция стоимости восстановительных расходов ТС \тс\\
%	Приложение № 3. Цифровые копии регистрационных документов ТС\\
%	Приложение № 4. Цифровая копия постановление по делу об административном правонарушении дорожно-транспортном происшествии\\
	Приложение № 3. Правоустанавливающие документы\\
	   }

\vspace{20mm}
%\newlength{\ML}
{Эксперт-техник}\hfill           {Мраморнов А.В.}


%\includepdf[pages=-]{myfile.pdf}

%\includepdf[pages=-]{calc.pdf}        %%% Экспертное заключение ОСАГО
\input{sections/noOSAGOnoKACKO}   %%% Экспертное заключение не ОСАГО и не КАСКО
%\setcounter{page}{1}
\clubpenalty=100000  % Недопуск Висячей строки в начале страницы
\widowpenalty=100000 %Недопуск висячей строки в конце абзаца

%%%%%%%%%%%%%%%%%%%%%%%%%%%%%%%%%%%%%%%%%
%
%   Экспертная организация ООО Южнорегиональная экспертная группа
%
%%%%%%%%%%%%%%%%%%%%%%%%%%%%%%%%%%%%%%%%%
%\noindent\qrcode[height=21mm]{\NomerDoc от \dataend }  %%% Добавлен QR-Code
\vspace{-4mm}
\begin{center}
	\large\textbf{ИНДИВИДУАЛЬНЫЙ\quad ПРЕДПРИНИМАТЕЛЬ  \\[-1.5mm] МРАМОРНОВ  АЛЕКСАНДР ВЯЧЕСЛАВОВИЧ \\[-5.5mm]}
	%  
	\noindent\rule{\textwidth}{2pt}\\[-6mm]  % Горизонтальная линия
	% \line(1,0){460}% (1,0) -горизонтальная линия, и (0,1) - вертикальная 
\end{center}

\begin{center}
	\begin{footnotesize}\setstretch{0.3}
		%	\small\textbf\setlength   	%\raisebox{5mm}
		\vspace{-3.5mm}г. Краснодар, с/т № 2 А/О «Югтекс», ул. Зеленая, 472, 
		Телефон: 8-918-451-66-11, e-mail: 4516611@gmail.com\\ [-2mm]{ИНН\quad 231200665168\quad ОГРНИП \quad 310231220400043}
	\end{footnotesize}	\\[10mm]
\end{center}


\begin{flushright}
% 
	 \hfill	Краснодар, 2019    \\[8mm]
\end{flushright}
\begin{center}
	\LARGE\textbf{ЭКСПЕРТНОЕ ЗАКЛЮЧЕНИЕ}
	\bigskip\\[0mm]
	%	{\normnumxtbf{\NomerDoc}}	}{den}
\end{center}
\par
\vspace{-6mm}
%%%%%%%%%%%%%%%%%  ОСАГО
%\noindent независимой технической экспертизы по определению размера расходов на восстановительный ремонт транспортного средства   \тс  \\[2mm]
%%%%%%%%%%%%%%%%% НЕ ОСАГО
\noindent независимой технической экспертизы по определению стоимости восстановительного ремонта транспортного средства   \тс\, регистрационный знак \грз  \\[2mm]
%%%%%%%%%%%%%%%%%%%%%%%%%
%\raggedright 
%\def\hrf#1{\hbox to#1{\hrulefill}}
\noindent \textbf{№ \NomerDoc}\hfill           \textbf{\dataend}\\%[2mm]
%Приостановлено\hfill      \datastop\\
%Возобновлено\hfill          \datarestart\\
%Окончено\hfill                \dataend\\%[4mm]

\noindent\parbox[l][16mm]{16.5cm}
{\def\hrf#1{\hbox to#1{\hrulefill}}
	\noindent Начато\hfill            \datastart\\%[2mm]
	%	Приостановлено\hfill      \datastop\\
	%	Возобновлено\hfill          \datarestart\\
	Окончено\hfill                \dataend
}
\relax

%%%%%%%%%%%% Если судебка
%
%\datastart г. ~в {\small ООО~ "ЮЖНО-РЕГИОНАЛЬНАЯ ЭКСПЕРТНАЯ ГРУППА"} \,  при определении  \, \sud  \,  от \, \dataopr \, о назначении \opr \, по гражданскому делу \delonum \, поступили:
%
%\begin{enumerate}\setlist{nolistsep}\item  Материалы гражданского дела \delonum \, в двух томах, том 1 на 276 листах, том 2  на 143 листах.\\[-2mm]
%	%	\item  
%\end{enumerate}
%
%%%%%%%%%%%%  Если независимая
\vspace{4mm}
Составлено на основании	договора № \NomerDoc\, от \dog\,  возмездного оказания услуг по проведению независимой технической экспертизы (далее экспертиза)  транспортного средства и письменного заявления заказчика о проведении экспертизы. 

Заказчик  экспертизы:  \заказчик, зарегистрированный по адресу: \адресзаказчика.

% Документ, удостоверяющий личность заказчика: водительское удостоверение    03\ 16\ 422344\ выдан 09.06.2011

%Транспортное средство виновника ДТП:  не предоставлялось.

\paragraph*{}
\noindent Экспертиза произведена  экспертом--техником
%{\small ООО "ЮЖНО-РЕГИОНАЛЬНАЯ ЭКСПЕРТНАЯ ГРУППА"}
\,  Мраморновым Александром Вячеславовичем, имеющим высшее образование по специальности «техническая физика», диплом РВ №311964 от 28.02.1989, квалификация -- инженер-физик, специальное образование в области оценки: Диплом ПП-1 № 037211 Российской экономической академии им. Г.В. Плеханова, квалификация -- оценка и экспертиза объектов и прав собственности, специальное образование в области независимой технической экспертизы транспортных средств: Диплом ПП-I № 424167, квалификация: эксперт-техник (специализация 150210 специальности 190601.65 – Автомобили и автомобильное хозяйство), состоящий в Государственном реестре экспертов-техников (№ в реестре 256, http://minjust.ru/ru-/node/105988),  общий трудовой  стаж 29 лет, стаж  экспертной работы  12 лет. \par Заключение подготовлено по месту фактического расположения ИП по адресу: г. Краснодар, с/т № 2 А/О «Югтекс», ул. Зеленая, 472.
  % Шапка организации ИП если НЕТ ОСАГО

%%%%%%%%%%%%%%%%%%%%%%%%%%%%%%%%%%%%%%%%   вопросы экспертизы
%\subsection{Вопросы экспертизы}
\subsection{Вопрос исследования}
\begin{enumerate}
%\item  <<Установить наличие, характер и объем (степень) технических повреждений транспортного средства  \tc?>>
%\item  <<Установить причины возникновения технических повреждений транспортного средства \tc \,и возможность их отнесения к рассматриваемому дорожно-транспортному происшествию (далее ДТП)?>>
%\item <<Установить технологию, объем восстановительного  ремонта \!транспортного средства \tc?>>
%\item <<Установить размер затрат на восстановительный ремонт (с учётом износа) транспортного средства \tc?>>
\item <<Определить размер ущерба, причиненного владельцу  транспортного средства \tc\,\грз\, \, в результате дорожно-транспортного происшествия, имевшего место \датадтп?>>
%\item <<Определить стоимость восстановительного ремонта  транспортного средства \tc\, регистрационный знак \грз,\, \, получившего механические повреждения в результате противоправных действий, имевших место \датадтп?>>
%\item <<Определить величину физического износа  транспортного средства \tc\,\грз\, \, получившего повреждения в результате дорожно-транспортного происшествия, имевшего место \датадтп?>>
%%	
\end{enumerate}
%%%%%%%%%%%%%%%%%%


\subsection{Исходные данные и объекты исследования} 
Исходные  данные,  необходимые  для   исследования,  изложены   заказчиком при подаче заявления о проведения исследования   колесного  транспортного  средства (далее —  KTC):\\
- на момент повреждения \датадтп\, автомобиль \тс\,\грз\, находился в рабочем состоянии;\\
- автомобиль полностью комплектный;\\
- кузов автомобиля  отремонтирован и окрашен;\\
- на стекла боковин левой и правой, стекло двери задка наклеена тонировочная пленка.   

	Для проведения исследования представлено:\\
\begin{enumerate}
\item Транспортное средство \тс\, \грз \, в поврежденном состоянии
\item Копия паспорта транспортного средства (ПТС) \птс
%\item Копия постановления \постановление об административном правонарушении дорожно-транспортном происшествии, имевшем место   \датадтп \, с участием  ТС \тс,\,\грз \, согласно которому  в результате дорожно-транспортного происшествия автомобиль \тс\, \грз\, получил механические повреждения. Повреждено:\, "\повреждения".
%\item Полис страхования  ОСАГО \polis.
	\end{enumerate}
%
\subsection*{Обстоятельства происшествия}
14.01.2020 г. автомобиль \тс \, г/н \грз\, под управлением водителя Бойкова Андрея Анатольевича  при выезде с придомовой территории был поврежден внешним воздействием металлическим предметом (металлическим ломом), лицом, находящимся в состоянии алкогольного опьянения.
%%%%%%%%%%%%%%%%%%%%%  
%
%\left( \addcontentsline{toc}{section}{Использованные нормативы и источники информации}

\subsection{Использованные нормативы и источники информации}
%
\begin{enumerate}
\item 
Махнин\,Е.\,Л., Новоселецкий\, И.\,Н., Федотов\, С.\,В. \emph{Методические рекомендации по проведению судебных автотехнических экспертиз и исследований колёсных транспортных средств в целях определения размера ущерба, стоимости восстановительного ремонта и оценки} // -- М.: ФБУ РФЦСЭ при Минюсте России, 2018.-326 с.  ISBN 978-5-91133-185-6.
%
%
%
%
\item ТУ 017207-255-00232934-2014 \emph{Кузова автомобилей LADA. Технические требования при приёмке в ремонт, ремонте и выпуске из ремонта предприятиями дилерской сети ОАО "АВТОВАЗ"}//  ОАО НВП "ИТЦ АВТО", 2014
%
\item Смирнов  В.Л., Прохоров  Ю.С., Боюр В.С.  и др. \emph{Автомобили ВАЗ. Кузова. Технология ремонта, окраски и  антикоррозионной защиты. Часть II}// - Н.Новгород: АТИС, 2001.- 241с.
%
\item 
Савич Е.Л. \emph{Техническое  обслуживание  и  ремонт  легковых  автомобилей} : учеб. пособие / Е.Л. Савич, М.М. Болбас, В.К. Ярошевич ; под общ. ред. Е.Л. Савича. -Мн. : Вышэйшая школа,  2001. - 479 с. - ISBN985-06-0502-2.
%
\item 
Автомобили ВАЗ-2121, 21213, 21214, 2131 и их модификации: <<Трудоемкости работ (услуг) по техническому обслуживанию и ремонту>> /Куликов А.В., Христов П.Н., Климов В.Е.,  Боюр В.С., Рева В.В., Зимин В.А., Завьялова Н.Н., Хлыненкова Г.А. -- ИТЦТ "АвтоВАЗтехобслуживание", Тольяти -- 2005. 
%
\item
Автомобили LADA SAMARA и их модификации: <<Трудоемкости работ (услуг) по техническому обслуживанию и ремонту>> /Куликов А.В., Христов П.Н., Климов В.Е., Рева В.В., Боюр В.С., Васильев М.В., Фахрутдинов Р.В.,  Прудских Д.А., Гирко В.Б., Шмелева В.А., Зимин В.А. --  ОАО НВП "ИТЦ АВТО",  -- 2006. - 252 стр.
%
\item 
Автомобили LADA PRIORA. Трудоемкости работ (услуг) по техническому обслуживанию и ремонту /Куликов А.В., Христов П.Н., Климов В.Е., Рева В.В., Козлов П.Л., Боюр В.С., Прудских Д.А., Шмелева В.А., Зимин В.А. -- ООО "ИТЦТ АВОСФЕРА", Тольяти -- 2009. -- 344 с.
%
\item 
{Трудоемкости работ по техническому обслуживанию и ремонту автомобилей автомобилей Lada  Granta}/   \url{https://docplayer.ru/30250248-Trudoemkosti-rabot-po-teh\-nicheskomu-obsluzhivaniyu-i-remontu-avtomobiley-lada- granta.html}.
%
%
\item
{Специализированное программное обеспечение для расчёта стоимости  восстановительного ремонта, содержащее нормативы трудоёмкости работ, регламентируемые изготовителями транспортного средства}//   AudaPadWeb, лицензионное соглашение № AS/APW-658  RU-P-409-409435.
%
%
%
\item
{Специализированное программное обеспечение для расчёта стоимости  восстановительного ремонта, содержащее нормативы трудоёмкости работ, регламентируемые изготовителями транспортного средства ОАО «АвтоВАЗ», ЗАО «Джи-Эм-АвтоВАЗ», ОАО «СеАЗ» и ОАО «ЗМА»}//   Автосфера АС:Смета, v.3.9.11// ООО "ИТЦ «ИнтегроМаш», \url{https://autosmeta.pro}.
%
%
%
\item Информационный портал по техническому обслуживанию и ремонту автомобилей	 ВАЗ:\\ \url{www.autosphere.ru}.

%%
\end{enumerate}
 %%%%   БИБЛИОГРАФИЯ
%
%%%%%%%%%%%%%%%%%%%%%%%%%%%%%%%%%%%%%%%%%%%%%%%%%%%%%%%%%%%%%%%%%%%%%%%%%%%%%%%%%
\subsection{Технические средства}  %% Список не удалять!!!
%
\begin{itemize}
%
%\item Диагностический сканер SDconnect   с программным обеспечением Xentry Diagnostics v19.11.3.1
\item  Линейка масштабная магнитная с цветографической шкалой, 100мм
\item  Рулетка измерительная металлическая, 5м.
%%\item Универсальный стенд для измерения углов установки колес Hunter Engineering %ProAlign с программным инструментом регулировки схождения колес без блокировки руля %автомобиля WinToe
\item Цифровой фотоаппарат Canon 760D s/n 143032001327 с объективом Canon EF-S 18-135, тип используемой памяти: Transcend,  32Gb.
%
\item Специализированное программное обеспечение для расчёта стоимости  восстановительного ремонта, содержащее нормативы трудоёмкости работ, регламентируемые изготовителями транспортного средства     AudaPadWeb, лицензионное соглашение № AS/\- APW-658  RU-P-409-409435.
%
\item  Программа обработки фото-видео изображений ImageJ, разработчик  Wayne Rasband (wa-yne@codon.nih.gov),
свободная лицензия GPL.
%
\item  ПЭВМ под управлением операционной системы Windows 10 с установленным набором макрорасширений LaTeX системы компьютерной вёрстки TeX, cвободная лицензия LaTeX Project Public License (LPPL). 
%	
	\end{itemize}
%%%%%%%%%%%%%%%%%%%%%%%%%%%%%%%%%%%%%%%%%%%%%%%%%%%%%%%%%%%%%%%%%%%%%%%%%%%%%%%%%%%%%%%%%%%%%%%
\subsection{Условные обозначения}
\begin{description}
%	 
%%\item[АВС] --Антиблокировочная система
\item[АМТС] --автомототранспортное средство
\item[ДВС] --двигатель внутреннего сгорания
\item[ДТП] --дорожно--транспортное происшествие
\item[гос.\,рег.\,знак] -государственный регистрационный знак
\item[КТС] --колесно транспортное средство 
\item[ЛКП] --лакокрасочное покрытие
%\item[л.д.] --Лист дела
%%\item[Колесо турбины]  -- крыльчатка турбины
\item[ТС] --транспортное средство
%\item[ТK, ТКР] -- Турбокомпрессор. Состоит из двух частей: турбины и компрессора, объединенных общим валом. Вал вращается в подшипниках, размещенных в центральном корпусе ТК
%\item[УУК] -- Углы установки колес
%\item[ЭБУ] --Электронный блок управления
%\item[DTC] --Diagnostic Trouble Codes, диагностические коды неисправностей
%\item[FRAME] --номер кузова транспортного средства, выпущенного для продажи на внутреннем рынке Японии и содержащий информацию производителя о транспортном средстве
%\item[OBDII] -- On-board diagnostics, протокол бортовой диагностики автомобиля
%\item[SRS] -- система пассивной защиты водителя и пассажиров
\item[VIN] --vehicle identification number, 17--значный идентификационный номер транспортного средства, соответствующий стандарту ISO 3779--2012.
%
\end{description}
%
\subsection{Термины и определения}
\begin{description}
	\item[Аварийные повреждения] --- повреждения, механизм образования которых определяется контактом с посторонними объектами, что привело к деформации или разрушению и к необходимости ремонта или замены составной части, или контактам с агрессивной средой, которая привела к необходимости ремонта (замены) составной части [1, часть II, п. 1.5].
	\item[Восстановительный ремонт]--- один из способов возмещения ущерба, состоящий в выполнении технологических операций ремонта КТС, действующий в сети торгово-сервисного обслуживания, созданной изготовителем этого КТС [1, часть II, п. 1.4].
%	\item[Годные остатки] --- работоспособные, имеющие остаточную стоимость детали (агрегаты, узлы) поврежденного автотранспортного средства, годные к дальнейшей эксплуатации, которые можно демонтировать с поврежденного автотранспортного средства и реализовать.
	\item[Дата исследования (оценки)]--- дата, на которую проводятся расчеты и используются стоимостные данные КТС, запасных частей, материалов, нормо-часа ремонтных работ [1, часть II, п. 1.5].
	\item[Срок эксплуатации КТС]--- период времени от даты изготовления (даты выпуска) КТС, до даты оценки (исследования), определяемой условиями задачи исследования (независимо от даты его регистрации и начала использования по назначению (эксплуатации)).
%	\item[Утрата товарной стоимости (УТС)]--- условное снижение рыночной стоимости КТС, восстановленного в соответствии с нормативными требованиями после повреждения, по сравнению с рыночной стоимостью подобного неповрежденного КТС. 
	\item[Характер повреждения]--- качественная характеристика повреждения, определяющая сущность и природу изменения состояния ТС (составной части)

\end{description}
\subsection{Методы исследования}
\begin{itemize}
\item  Органолептический метод – исследование и оценка качества объектов с помощью органов чувств.
\item 	Прямой измерительный метод – путём измерения  размеров  %деталей специальными %
мерительным инструментом %и специальными приборами
\item Расчётный метод (косвенный измерительный метод) – путём расчётов различных параметров на основе результатов измерений и других данных.
\item Экспертный метод (метод экспертной оценки) — совокупности операций по выбору комплекса или единичных характеристик объекта, определению их действительных значений и оценкой экспертом соответствия их установленным требованиям и/или технической информации.
%%	\item Метод натурной реконструкции
\end{itemize}
%
%
\subsection{Ограничения и пределы применения полученных результатов}

Следующие допущения и условия, ограничивающие пределы применения полученных результатов, являются неотъемлемой частью данного экспертного заключения.
\begin{itemize}
\item  {Результаты, полученные экспертом-техником, носят рекомендательный консультационный характер и не являются обязательными. Исполнитель высказывает своё субъективное суждение о наиболее вероятных будущих (абстрактных) расходах, их предполагаемом размере и дает заключение в пределах своей компетенции.}
\item { Под компетенцией эксперта-техника понимают его знания и опыт в области теории и методов экспертных исследований ТС, а также круг полномочий, представленных ему законом, и вопросов, которые он может решать на основе своих специальных познаний.
В компетенцию эксперта-техника входит исследование технического состояния поврежденного ТС в целях установления характера повреждений ТС, установления причины возникновения технических повреждений технологии, методов, стоимости его ремонта.}
\item  {Исполнитель в рамках своих обязательств по заключенному договору об экспертном обслуживании признает свою ответственность перед заказчиком и настоящим утверждает, что экспертное заключение выполнено профессионально, тщательно и с должной заботливостью и внимаем, %как это обычно принято для компетентного специалиста в области технической экспертизы ТС при ОСАГО,
а полученная величина восстановительных расходов, разумна и реальна.}
\item  {Исполнитель считает, что поскольку, по общему правилу, оценка доказательств является прерогативой и компетенцией органа дознания, следствия или суда, а в досудебном порядке - страховщика, постольку после проверки результатов экспертизы последним, их признания и принятия решения о выплате страхового возмещения этап возможного оспаривания достоверности исследований между заказчиком и исполнителем завершен.
Соответственно, обязанности Исполнителя по договору являются надлежаще исполненными в полном объеме и от исполнителя не требуется свидетельствовать по поводу произведённого исследования перед третьими лицами.}
\item  {Отдельные части настоящего экспертного исследования не могут трактоваться раздельно, а только в связи с полным текстом о проведенных расчетах.}
\item  {Исходные данные, использованные исполнителем при подготовке экспертного заключения, получены из надежных источников и считаются достоверными. Тем не менее, исполнитель не может гарантировать абсолютную точность, поэтому там, где это, возможно, делаются ссылки на источники информации.}
\item  {В процессе экспертного исследования специальная юридическая экспертиза документов, касающихся прав собственности на ТС, не проводилась.}
\item  {Суждения, содержащиеся в экспертном заключении, основываются на текущей ситуации на дату аварии и, в будущем, могут быть подвержены изменениям.
Исполнитель не принимает на себя никакой ответственности за изменение экономических, юридических и иных факторов, которые могут возникнуть после даты исследования и повлиять на результаты технической экспертизы.
%Данное заключение составлено на основании Правил Независимой Технической Экспертизы и может применяться только при решении вопроса о выплате страхового возмещения по ОСАГО.
}\end{itemize}
%
%
\section{Исследование}
%
Настоящее исследование проводится на основании материалов, предоставленных Заказчиком, а также на основании данных, самостоятельно полученных экспертом-техником. Выводы, содержащиеся в настоящем Заключении, могут расцениваться как достоверные только в контексте того количества информации, на основании которого они были сделаны. При поступлении дополнительной или измененной информации данные выводы могут быть
скорректированы. При анализе скрытых повреждений экспертом-техником не принимается во внимание наличие или отсутствие записей о них в документах компетентных органов, в связи с отсутствием у сотрудников компетентных органов объективной возможности  идентификации таких повреждений на месте происшествия.

 \par Из предоставленных материалов   экспертом-техником установлена следующая общая информация об автомобиле, имеющая значение для дачи заключения:\\
 \parbox[]{10cm}{}
	\begin{itemize}
		\item[ ] 
			\begin{description}
			\item[Марка, модель] -- \тс
			\item[VIN] -- \vin
			\item[Год выпуска] -- \год
			\item[Шасси] -- Отсутствует
			\item[Цвет ЛКП] -- \цвет
			\item[Пробег] --  \пробег\,км%, считан с одометра
			\item[Двигатель] -- \двигатель
	    	\item[ПТС] --\птс	
		\end{description}
		\end{itemize}
%%%%%%%%   Информация, полученная при расшифровке ВИН
%	\subparagraph*{} Идентификационный код автомобиля (VIN)  \vin \, содержит следующую информацию о транспортном средстве, имеющую значение для 	дачи заключения:\\
%    Модель ТС: CHEVROLET LACETTI + NUBIRA + OPTRA (J200) [EUR]\\[3mm]
%    
%    
%\noindent\parbox[]{10cm}{
%\begin{itemize}
%	\item[ ] 
%	    \begin{description}
%%	\item{Модель} \hfill CHEVROLET J03 - LACETTI  (J200) [EUR]
%	\item[Дата изготовления] \hfill \началоэкспл
%	\item[Расположенние руля] \hfill Left
%	\item[Двигатель] \hfill \двигатель
%%	\item[Объем двигатель] \hfill 1998 $ \text{см}^3 $
%%	\item[КПП] \hfill АКПП
%	\item[Тип кузова] \hfill  Седан
%	\item[Количество дверей] \hfill 4 
%	%	\item[VDS] --
%    %	
%		\end{description}
%\end{itemize}}\\%[3mm]
%    
%\vspace{3mm}
%\textit{Источник: https://emex.ru/catalogs/original/?screen=units\&vin=/\вин}\\
Рыночная стоимость аналогичного автомобиля составляет 45000 (Сорок пять тысяч) рублей, Рис. \ref{ris:images/2}, 
\textit{Источник: https://auto.ru/krasnodar/cars/vaz/2108/all/; https://automama.ru/ocenka-avto}
\vspace{3mm}
%Пробег автомобиля  расчетный, согласно [1]  составляет 
%%%%%%%%%%%%%%%% Вставка изобржения общего вида авто
  \begin{figure}[!h]
	\centering
	\includegraphics[width=0.68\linewidth]{images/2}
	\caption{{\footnotesize {Диаграмма рыночной стоимости ТС \тс \, 1989 года выпуска в г. Краснодаре.  }}}
	\label{ris:images/2}
\end{figure}
\relax
%
%\par За дату изготовления принимается 01.01.1996 г., при этом срок эксплуатации автомобиля \тс \, на момент исследования \датадтп, составляет 23,78 лет.  %%%%%% расчет в jupiter "разность дат"
%Исследуемый автомобиль не на гарантийном периоде эксплуатации и старше семи лет. 
%Пробег автомобиля \пробег, с учетом срока эксплуатации фактический среднегодовой пробег составляет \пробег/23,78 = 8495 км.
\subparagraph{Осмотр транспортного средства }


\noindent \begin{spacing}{1.2} { Осмотр транспортного средства \тс\, регистрационный знак  \грз\, проводился \osm\,с 15 час. 00 мин. до 15 час. 30 мин. в сухую, ясную погоду  на открытой площадке по адресу: г. Краснодар, ул. Лесная, 28, в присутствии представителя собственника Бойкова Андрея Анатольевича. %Виновник ДТП уведомлен надлежащим образом, на осмотр не явился.
Соответствие маркировочных обозначений на кузове представленного ТС записям в регистрационных документах ТС экс\-пертом-техником установлено. Видимые изменения конструкции ТС отсутствуют.  Представленный на исследование автомобиль \tc\, имеет кузов типа "\типкузова". Кузов автомобиля окрашен  рефлексной (лессирующей) с "металлическим" эффектом 
эмалью (краской) \colr цвета. Внешнее состояние автомобиля удовлетворительное, повреждения  кузовных элементов отсутствуют, кузов автомобиля окрашен ремонтной эмалью.  Стекла боковин и двери задка  оклеены тонирующей плёнкой. % оснащен легкосплавными 15 дюймовыми пятиспицевыми колесами LS Wheels.  На наружных и внутренних поверхностях кузова автомобиля присутствуют признаки  эксплуатации, коррозионные повреждения отдельных кузовных элементов,  повреждения и следы ремонта.
\par Для определения причины возникновения повреждений %, указанных в Акте осмотра ТС 
%№ \NomerDoc \, (Приложение № 2) 
экспертом-техником изучены документы, представленные Заказчиком.  %и сведения о ДТП, с участием ТС \вин. % \, по данным открытых источников https://гибдд.рф. 
%По предоставленным документам экспертом-техником установлена причина ДТП, установлены обстоятельства ДТП, выявлены повреждения ТС и установлены причины их образования. 
Проведено исследование характера выявленных повреждений и сопоставление повреждений ТС  
%потерпевшего с повреждениями ТС иных участников ДТП в соответствии 
со сведениями, зафиксированными в представленных документах. %документах о ДТП. 
Проведена проверка взаимосвязанности повреждений на ТС с заявленными обстоятельствами, определён объем восстановительных работ.% (Приложение №1 Акт осмотра ТС). 
\par Наличие, характер и объем технических повреждений, а так же  планируемые (предполагаемые) ремонтные воздействия для восстановления поврежденного автомобиля  исследованы в присутствии заинтересованных лиц, зафиксированы в акте осмотра № \NomerDoc \,  от \osm  (Приложение №2),  и фотоматериалах по принадлежности (Приложение №3).  }
\end{spacing}
%
%%%%%%%%%%%%%%%%%% Если автомобиль новый, до пяти лет
%
%\par  настоящем исследовании износ транспортного средства принимается равным нулю по следующим основаниям: обстоятельства заявленного события не регулируются законодательством об ОСАГО,  срок эксплуатации КТС не превышает пяти лет, иные повреждения и следы ремонта отсутствуют, признаки интенсивной эксплуатации отсутствуют. 
%
%\relax
%\renewcommand\baselinestretch{0.86}\small\normalsize 
%\subsection{\underline{По  вопросу}\, \, \,	\textbf{\small{<<Установить наличие, характер и объем (степень) технических повреждений транспортного средства  \tc \,>>?}}}
%\renewcommand\baselinestretch{1.2}\small\normalsize
%%
%Первичное установление наличия и характера повреждений транспортного средства, в отношении которых определяются расходы на восстановительный ремонт, в соответствии с  Единой методикой определения размера расходов на восстановительный ремонт 
%в отношении поврежденного транспортного средства  
%утверждены Положением Банка России от «19» сентября 2014 года № 432-П (Единой методикой), должно производится во время осмотра транспортного средства и фиксироваться актом осмотра, в который  должны включаться сведения о повреждениях транспортного средства, с обязательной  характеристикой поврежденных элементов с указанием расположения, вида и объема повреждения.   
%
%В акте № 006/06/19  от \osm\, повреждений транспортного средства (ТС)  не содержатся установленные Единой методикой обязательные количественные показатели (характеристики) повреждений транспортного средства.  Отсутствие количественных характеристик повреждений в представленном акте осмотра предполагает его применение в настоящем исследовании исключительно в виде списка повреждений и перечня ремонтных воздействий, определенных экспертом-техником Яковлевым С.В. для восстановительного ремонта транспортного средства. %  не позволяет эксперту использовать указанный акт осмотра для объективного выбора необходимого и достаточного комплекса работ по восстановительному ремонту транспортного средства.
%
%На момент определения о проведении повторной судебной экспертизы, объект исследования неоднократно подвергался видоизменениям как при исследовании специалистом ООО "ЭКСПЕРТ", так и при производстве первичной судебной экспертизы экспертами ИП Куприянова Виктора Александровича. 
%
%\par
%Согласно данному акту осмотра № 1002  от \osm \, для устранения повреждений ТС необходимо было произвести следующие ремонтные работы:
%\vspace{\baselineskip}  % вставка пустой строки
% 
%\begin{longtable}{|p{1cm}|p{11cm}|p{3cm}|}
%\caption[]{\footnotesize {Ремонтные воздействия по акту осмотра № 006/06/19 от \, \osm \, ИП Яковлева С.В.}} \label{tab:4}\\ 
%	 \hline
%		\rowcolor[HTML]{C0C0C0} 
%	%	\multicolumn{1}{|c|}
%	%	{\cellcolor[HTML]{C0C0C0}N/N} & Наименование запчасти (материала) & Ремонтное воздействие 
%    %	
% \text{N/N} & Наименование запчасти (материала) & Ремонтное воздействие  \\ \hline \endhead
%		\Rownum  & Панель задка  & Замена, окраска \\ \hline
%		\rowcolor[HTML]{EFEFEF} 
%		\Rownum  & Боковина задняя левая   & Замена, окраска \\ \hline
%	    \Rownum  & Боковина задняя правая  & Замена, окраска  \\ \hline
%		
%\end{longtable}
%
%\setcounter{rownum}{0}
%%
%\par Таким образом, размер восстановительных расходов (затрат) \тс\,  может быть произведен в объеме сведений, содержащихся в представленной Таблице \ref{tab:4}.
%\vspace{\baselineskip}  % вставка пустой строки
%
%\renewcommand\baselinestretch{0.86}\small\normalsize 
%\subsection{\underline{По  вопросу}\, 	\textbf{\small{2. <<Установить причины возникновения технических повреждений транспортного средства \tc \, и возможность их отнесения к рассматриваемому дорожно-транспортному происшествию (ДТП)>>?}}}
%\renewcommand\baselinestretch{1.2}\small\normalsize
%
%
%Колесное транспортное средство сроком эксплуатации более 7 лет относится к категории транспортных средств с граничным сроком эксплуатации [1], для которой возможно применение ремонтных операций при условии экономической целесообразности и  технической возможности.
% 
%Из открытых банков данных полиции следует, что автомобиль с VIN  \вин\,  как минимум дважды становился участником ДТП.
%Первый раз 29.06.2018  06:40, извещение о ДТП № 030046913, в котором автомобиль получил повреждения задней правой двери, заднего правого порога, заднего правого колеса, подушки SRS справа, Рис. \ref{ris:images/d1} и второй раз 22.05.2019 06:50, извещение о ДТП № 030034947, в котором автомобиль получил повреждения деталей передней левой и задней частей кузова, Рис. \ref{ris:images/d2}.
%
%
%\begin{figure}[H]\centering
%	\parbox[t]{0.49\textwidth}
%	{\centering
%		\includegraphics[width=.49\textwidth]{images/d1}
%		\caption{\footnotesize {Повреждения в ДТП 29.06.2018 }}
%		\label{ris:images/d1}}
%	\hfil \hfil%раздвигаем боксы по горизонтали 
%	\parbox[t]{0.49\textwidth}
%	{\centering
%		\includegraphics[width=.49\textwidth]{images/d2}
%		\caption{\footnotesize {Повреждения в ДТП 22.05.2019}}
%		\label{ris:images/d2}}
%\end{figure}
%%
%{\noindent  \footnotesize \tikz \fill [red] (1,0.5) rectangle (0.1,0.1); --{\footnotesize  Вмятины, вырывы, заломы, перекосы, разрывы и другие повреждения с изменением геометрии элементов (деталей) кузова и эксплуатационных характеристик ТС.}\\
%\tikz \fill [yellow] (1,0.5) rectangle (0.1,0.1); --  {\footnotesize Повреждения колёс (шин), элементов ходовой части, стекол, фар, указателей поворота, стоп-сигналов и других стеклянных элементов (в т.ч. зеркал), а также царапины, сколы, потертости лакокрасочного покрытия или пластиковых конструктивных деталей и другие повреждения без изменения геометрии элементов (деталей) кузова и эксплуатационных характеристик ТС.}\\[1mm]
%%%%%%%%%%%%%%%%%%%%%%%%%%%%%%%%%%%%%%%%%%%%%%%%%%%%%%%%%%%%%%%%%%%%%%%%%%%}%%%%%%%%%%%%%%%%%%%%%%%%%%%%%%%%%%%%%%%%%%%%%%%%%%
%
%\renewcommand\baselinestretch{1.2}\small\normalsize
%\begin{spacing}{1.25}Таким образом, перечень повреждений, указанный в акте осмотра № 006/06/19 от \osm \, ИП Яковлева С.В. может соответствовать повреждениям автомобиля \тс\,, полученным в результате ДТП \датадтп. 
%\end{spacing}
%\renewcommand\baselinestretch{0.86}\small\normalsize 
%\subsection{\underline{По  вопросу}\, \, \,	\textbf{\small{3. <<Установить технологию, объем восстановительного  ремонта транспортного средства \tc \,>>?}}}
%%%%%%%%%%%%%%%%%%%%%%%%
%
%Для автомобилей, старше семи лет:
%\renewcommand\baselinestretch{1.2}\normalsize
%
%
%\renewcommand\baselinestretch{0.86}\small\normalsize 
%\subsection{\underline{По  вопросу}\, \, \,	\textbf{\small{<<Установить размер затрат на восстановительный ремонт (с учётом износа) транспортного средства \tc \,>>?}}}
%\renewcommand\baselinestretch{1.2}\small\normalsize
%
%
\par  Так как обстоятельства заявленного события не регулируются законодательством об ОСАГО, то в качестве экспертной методики принимаются \emph{Методические рекомендации по проведению судебных автотехнических экспертиз и исследований колесных транспортных средств в целях определения размера ущерба, стоимости восстановительного ремонта и оценки} [1].
В соответствии с принятой экспертной методикой размер расходов на восстановительный ремонт определяется исходя из стоимости ремонтных работ (работ по восстановлению, в том числе окраске, контролю, диагностике и регулировке, сопутствующих работ), стоимости используемых в процессе восстановления транспортного средства деталей (узлов, агрегатов) и материалов взамен поврежденных.\\
%                                         
Стоимость восстановительного ремонта АМТС ( $ C_\text{вp} $) определяется по формуле:
%
\begin{equation}\label{eq:r}
C_\text{вp} =C_p + C_\text{м} + C_\text{зч}
\end{equation}
%
\noindent где:
%
\begin{itemize}
%	
\item[ ]$C_\text {р} $ --  стоимость ремонтных работ по восстановлению КТС, руб.;
\item[ ]$ C_\text{м} $ --  стоимость необходимых ремонтных материалов, руб.;
\item[ ]$ C_\text{зч} $ --  стоимость новых запасных частей, руб;
%\item[ ] $ \text{И} $ -- коэффициент износа составной части, подлежащей замене, \%.
\end{itemize}
%
\vspace{5mm}
\renewcommand\baselinestretch{1.2}\small\normalsize
%
%Коэффициент износа составных частей (И) КТС (кроме автобусов и грузовых автомобилей) при определении стоимости восстановительного ремонта расчитывается по формуле:
%
%\begin{equation}\label{eqsnos}
%\text{И} =\text{И1}\cdot\text{П}+\text{И2}\cdot \text{Д}, \%  \,\,\,\, \text{где:}
%\end{equation}
%
%\begin{itemize}
%	\item [] $ \text{И1} $ --усредненный показатель износа на 1000 км пробега, \%; 
%	\item [] $ \text{П} $ -- общий пробег (фактический или расчетный) за срок эксплуатации КТС, тыс.км;
%	\item [] $ \text{И2} $ -- усредненный показатель старения за 1 год эксплуатации, \%;
%	\item [] $ \text{Д} $ -- срок эксплуатации КТС (от даты изготовления КТС до момента, на который определяется износ), лет. 
%\end{itemize}
%
%Для исследуемого автомобиля \тс, параметры для расчета коэффициента износа приняты согласно справочным таблицам [1, ч. II, Приложение 2.4]:
%\begin{equation}\label{eqsnosr}
%\text{И} =\text{И1}\cdot\text{П}+\text{И2}\cdot \text{Д} = 0.4\cdot 202  + 2.2\cdot 23 = 80^* \, \%
%\end{equation}
%%%%%%%%%%%%%%%%%%%%%%%%%%%%%%%%%%%%%%%%%%%%%%%%%%%%%%%%%%%%%%%%%%%%%%%%%%%%%%%%%%%%%%%%%%%%%%%%%%%%%%%%%%%%%%%%
%Стоимость восстановительных работ $ C_{\text{вр}} $ определяется на основании норм трудоёмкостей $ T_i $, \, предусмотренных заводом-изготовителем, и стоимостных параметров $ C_{i\text{нч}} $ (стоимости нормо-часа) работ по техническому обслуживанию и ремонту АМТС.  Расчет размера расходов (в рублях) на восстановительный ремонт производится по формуле:
%\begin{equation}\label{eq:cr}
%C_{\text{вр}}  =\sum{C_{ip}}= \sum\left({T_{ij}}\cdot {C_{i\text{нч}}}\right) + \sum{C_{ip^{\text{\,\,\,руб}}}} , \,\,\,\text{где:} 
%\end{equation}
%%\vspace{2mm}
%\begin{itemize}
%	\item[ ]$ C_{ip} $ -- стоимость работ i-го вида: $C_\text {зам} $, $ C_\text{восст} $, $ C_\text{рег} $, $C_\text{контр} $, $ C_\text{антикор} $, $ C_\text{зч} $, $ C_\text{ом} $,$ C_\text{соп} $, $ C_\text{вм} $, руб;
%	\item[ ]$ T_{ij} $ -- трудоёмкость j-й операции(комплекса) по i-му виду работ, руб;
%	\item[ ]$ C_{i\text{нч}} $ -- стоимость нормо-часа по i-му виду работ, руб;
%	\item[ ]$ C_{ip^\text{\,\,руб}} $ -- стоимость работ $ C_{ip} $, принятая непосредственно в денежном выражении, руб.
%\end{itemize}
%%
%\par При определении стоимости восстановительного ремонта АМТС с учётом износа под износом следует понимать количественную меру физического старения АМТС и его элементов, достигнутого в результате эксплуатации, т.е. эксплуатационный износ.
%%
%Расчёт износа производится в  соответствии с Положением Банка России от «19» сентября 2014 года № 432-П «О единой методике определения размера расходов на восстановительный ремонт в отношении повреждённого транспортного средства» [2].
%Износ комплектующих изделий (деталей, узлов, агрегатов) рассчитывается по следующей формуле:
%%
%\begin{equation}\label{eq:I}
%\text{И}_{\text{ки}} 
%= 100\cdot\left( 1-e^ {-\left( \Delta_{T} \cdot T_{\text{КИ}} + \Delta_{L} \cdot L_{\text{КИ}} \right)}\right), \,\,\,\,\text{где:}   
%\end{equation}
%%
%\begin{itemize}
%	\item[ ]$ \text{И}_{\text{ки}} $ -- износ комплектующего изделия (детали, узла, агрегата) (процентов); 
%	\item[ ]$ e $ -- основание натуральных логарифмов (e =  2,72);
%	\item[ ]$ \Delta_{T}$ --  срок эксплуатации комплектующего изделия (детали, узла, агрегата) (лет);
%	\item[ ]$ T_{\text{КИ}} $ -- стоимость работ $ C_{ip} $, принятая непосредственно в денежном выражении, руб
%	\item[ ]$ \Delta_{L} $ --коэффициент, учитывающий влияние на износ комплектующего (детали, узла, агрегата) величины пробега транспортного средства с этим комплектующим изделием;
%	\item[ ]$ L_{\text{КИ}} $ --пробег транспортного средства на дату дорожно-транспортного происшествия (тысяч километров).  
%		\end{itemize}
%\vspace{5mm}
%\par Значения коэффициентов $ \Delta_{T}$  и $ \Delta_{L} $  для различных категорий и марок транспортных средств приведены в п.5. сп. лит~[2]. При этом, на комплектующие изделия (детали, узлы, агрегаты), которые находятся в заведомо худшем состоянии, чем общее состояние транспортного средства в целом, и его основные части, вследствие влияния факторов, не учтённых при расчете износа (например, проведение ремонта с нарушением технологии, не устранение значительных повреждений лакокрасочного покрытия), может быть начислен дополнительный индивидуальный износ. 
%Износ шины транспортного средства рассчитывается по следующей формуле:
%\begin{equation}\label{eq:sh}
%\text{И}_{\text{ш}} = \frac{\text{Н}_{\text{н}}-\text{Н}_{\text{ф}}}{\text{Н}_{\text{н}}-\text{Н}_{\text{доп}}} \cdot{100}\%  \,\,\,\,\text{где:} 
%\end{equation}
%%
%\begin{itemize}
%	\item[ ] $ \text{И}_{\text{ш}} $ -- износ шины, \%;
%	\item[ ] $ \text{Н}_{\text{н}} $ -- высота рисунка протектора новой шины, мм;
%	\item[ ] $\text{Н}_{\text{ф}} $ -- фактическая высота рисунка протектора шины, мм;
%	\item[ ] $ \text{Н}_{\text{доп}} $ --минимально допустимая высота рисунка протектора шины в соответствии с требованиями законодательства Российской Федерации, мм.
%\end{itemize}
%%
%\renewcommand\baselinestretch{1}\small\normalsize
%Износ шины дополнительно увеличивается для шин с возрастом от 3 до 5 лет - на 15 процентов, свыше 5 лет - на 25 процентов. 
%%%%%%%%%%%%%%%%%%%%%%%  Нулевой и предельный износ %%%%%%%%%%%%%%%%%%%%%%%%%%%%%%
%\par $ ^*$Согласно п. 7.8.\, Методики [1]  для случаев, не регулируемых законодательством об ОСАГО, для составных частей КТС значение износа принимается равным нулю, срок эксплуатации которых не превышает 5 лет,  предельное значение износа комплектующих транспортного средства  не должно превышать 80\% стоимости запасных частей. Для составных частей, имеющих срок эксплуатации более 12 лет, при отсутствии факторов снижения износа (проведенный капитальный ремонт, замена составных частей  и т.д) рекомендуемое значение износа составляет 80\%.
%%%%%%%%%%%%%%%%%%%%%%%%%%%%%%%%%%%%%%%%%%%%%%%%%%%%%%%%%%%%%%%%%%%%%%%%%%%%%%%%%%%5
  \begin{figure}[!h]
	\centering
	\includegraphics[width=0.98\linewidth]{images/1}
	\caption{{\footnotesize {Кузов снаружи. Локализация повреждений ТС \тс}}}
	\label{ris:images/1}
\end{figure}


\indent Повреждения автомобиля \тс\, установленные по результатам осмотра, отражены в таблице \ref{tab:5}: 

\begin{longtable}{|M{125mm}|M{30mm}|}
\caption[]{\footnotesize {Повреждения автомобиля, установленные при его осмотре}} \label{tab:5}\\ \hline
\bf {\small Наименование  детали с описанием повреждения} & \bf {\small Изображение} \\ \hline \endhead
%
% {\small    } & \imt{fp1} \\ \hline  %фото повреждений
%
{\small Стекло ветровое - разбито   } & \imt{fp1} \\ \hline  %фото повреждений
{\small Стойка левая ветрового окна - сложная деформация на площади 2дм2  } & \imt{fp2} \\ \hline 
{\small Стекло левой двери - разбито  } & \imt{fp3} \\ \hline
{\small Облицовка левой средней стойки - расколота }&  \imt{fp4}\\ \hline 
{\small Фара левая - разбито стекло, корпус   } & \imt{fp5} \\ \hline
{\small Стойка боковая левая - сложная деформация в средней части  } & \imt{fp6} \\ \hline
{\small Стекло левой боковины - разбито   } & \imt{fp7} \\ \hline
{\small Стойка левая задняя - сложная деформация детали на площади 3дм2   } & \imt{fp8} \\ \hline
{\small Стекло двери задка - разбито  } & \imt{fp9} \\ \hline
{\small ЛКП двери задка - повреждение в смежной с поврежденной левой стойкой частью   } & \imt{fp10} \\ \hline
{\small Уплотнитель стекла двери левой верхний -разорван  } & \imt{fp11} \\ \hline
{\small Уплотнитель стекла двери левой нижний - разорван  } & \imt{fp12} \\ \hline
{\small Колесо рулевое - погнуто  } & \imt{fp13} \\ \hline
{\small Уплотнитель стекла ветрового - разорван  } & \imt{fp14} \\ \hline
{\small Тонировочная пленка двери задка - разорвана  } & \imt{fp15} \\ \hline
{\small Тонировочная пленка секла боковины - разорвана } & \imt{fp16} \\ \hline
{\small Дефлектор двери левой - разбит  } & \imt{fp17} \\ \hline
\end{longtable}

\par В результате исследования   экспертом-техником установлено, что для устранения повреждений \тс \, необходимо  выполнить следующие  работы:
%\begin{center}
	\begin{longtable}{M{65mm}M{85mm}}
\hline 
\textbf{Наименование детали}        & \textbf{Ремонтное воздействие}\\
\hline {\small Стекло ветровое }      &   {\small  Заменить}\\
{\small Стойка стекла ветрового левая }  &   {\small Ремонт, окраска} \\
{\small Стекло двери передней левой  } &  {\small  Заменить } \\
{\small Стекло левой боковины  } &  {\small Заменить, тонировочную пленку наклеить } \\
{\small Стекло двери задка  } &  {\small Заменить, тонировочную пленку наклеить   } \\
{\small Облицовка средней левой стойки   } &  {\small Заменить  } \\
{\small Фара левая   } &  {\small Заменить  } \\
{\small Стойка левой боковиины  } &  {\small Ремонт, окраска  } \\
{\small Дверь задка  } &  {\small Окраска поверхности  } \\
{\small Рулевое колесо  } &  {\small Заменить  } \\
{\small Уплотнитель стекла левой двери верхний  } &  {\small Заменить  } \\
{\small Уплотнитель стекла левой двери нижний  } &  {\small Заменить } \\
{\small Уплотнитель стекла ветрового } &  {\small Заменить } \\
{\small Стойка левая задняя   } &  {\small Ремонт, окраска  } \\
{\small Салон   } &  {\small Чистка, удаление остатков стекла  } \\
{\small Дефлектор двери левой   } &  {\small Заменить  } \\

%{\small   } &  {\small   } \\
%{\small   } &  {\small   } \\
%{\small   } &  {\small   } \\
	\end{longtable}  
%\end{center}


\renewcommand\baselinestretch{1.2}\small\normalsize
%
\textbf{Произвести  необходимые для выполнения  ремонта разборочно-сборочные, подготовительные и вспомогательные работы в соответствии с требованиями завода–изгото\-ви\-теля транспортного средства.}\\
%%
\par Расчет стоимости восстановительного ремонта выполнен в программе \auda.\\
Полный текст калькуляции представлен в Приложении  <<Калькуляция стоимости восстановительного ремонта>>.\\ 
\indent Результаты расчёта представлены ниже:
\vspace{3mm}
%
\begin{figure}[H]
	\centering
	\includegraphics[width=0.95\linewidth]{images/Screenshot_1}
	%%	\caption{}
	%%	\label{fig:screenshot001}
\end{figure}
\begin{figure}[H]
	\centering
	\includegraphics[width=0.95\linewidth]{images/Screenshot_2}
%%	\caption{}
%%	\label{fig:screenshot001}
\end{figure}
%\begin{figure}[H]
%	\centering
%	\includegraphics[width=0.9\linewidth]{images/screenshot002}
%%%	\caption{}
%	\label{aud}
%\end{figure}
\medskip
\renewcommand\baselinestretch{1.2}\small\normalsize
%%%%%%%%%%%%%%%%  Не ОСАГО
Стоимость коммерческого нормо-часа работ применена  с учётом условий регионального рынка услуг и сложившихся средних расценок по видам работ, типу ТС, а также по маркам и моделям ТС  и   составляет 800 р/ч для данного транспортного средства (\url{http://prices.autoins.ru/priceAutoParts/}). Трудоёмкость работ по разборке/сборке/замене  соответствует трудоёмкости работ, рекомендованной заводом изготовителем ТС[1, часть II, п. 7.32], а так же рекомендованные значения оценочной трудоёмкости ремонта кузовных составных частей [1, часть II, п. 7.33]. Расчёт стоимости ремонта, согласно положениям Методики [1] производится с учётом  применения оригинальных запасных частей. %, которые поставляются изготовителем КТС авторизованным ремонтным организациям. %Техническое состояние запасных частей учитывается коэффициентом износа, что в совокупности с установкой оригинальных запасных частей в максимальной степени отвечает понятию «восстановительный ремонт», то есть восстановления состояния КТС, при котором используются установленные изготовителем составные части, но с использованным частично ресурсом.  
%
\par Таким образом, в результате проведённых расчётов (см. Приложение, <<Калькуляция стоимости восстановительного ремонта № \NomerDoc>>) определена стоимость восстановительного ремонта транспортного средства  \тс, составляющая на дату  повреждения 37 650 (Тридцать семь тысяч шестьсот пятьдесят) рублей.%без учета износа, стоимость восстановительных расходов, с учетом уменьшения стоимости запасных частей вследствие их износа,  составляет 79 943 рубля, что с учетом округления составляет 80\,000 (Восемьдесят тысяч) рублей.
%%%%%%%%%%%%%%% УТС, годные,  рыночная
%\nopagebreak
% \input{rinok}   %% Расчет рыночной стоимости ТС
% \input{go}      %% Расчет стоимости годных остатков
%\subsection{Расчет стоимости годных остатков}

\par В случаях правовых отношений, регулируемых Гражданским кодексом РФ, расчет стоимости годных остатков и определение стоимости  реального ущерба с его уменьшением на стоимость годных остатков не предусмотрены [1, ч.II,п.9.6]. Вместе с тем, экспертная практика свидетельствует о возможности постановки перед экспертом задачи определения стоимости годных остатков вне  поля действия законодательства об ОСАГО.
\par Под годными остатками автотранспортного средства понимаются работоспособные, имеющие остаточную стоимость детали (агрегаты, узлы) поврежденного автотранспортного средства, как правило, годные к дальнейшей эксплуатации, которые можно демонтировать с поврежденного автотранспортного средства и реализовать. 
Годные остатки должны отвечать следующим условиям:

1) деталь (агрегат, узел) не должна иметь повреждений, нарушающих ее целостность и товарный вид, а агрегат (узел), кроме того, должен находиться в работоспособном состоянии;

2) деталь (агрегат, узел) не должна иметь изменений конструкции, формы, целостности и геометрии, не предусмотренных изготовителем автотранспортного средства (например, дополнительные отверстия и вырезы для крепления несерийного оборудования);

3) деталь не должна иметь следов предыдущих ремонтных воздействий (следов правки, рихтовки, следов шпатлевки, следов частичного ремонта и т.д.).

Под стоимостью годных остатков понимается наиболее вероятная стоимость, по которой они могут быть реализованы, учитывая затраты на их демонтаж, дефектовку, ремонт, хранение и продажу.
К годным остаткам не могут быть отнесены [1, ч.II,п.10.3] составные части:

- демонтаж которых требует работ, связанных с применением газосварочного и электродугового резания;

-имеющие изменения конструкции, формы, нарушения целостности, не предусмотренные изготовителем ТС;

- подвергшиеяся ранее ремонтным воздействиям (например, правке, рихтовке, шпатлеванию  и т.д.);

- влияющие на безопасность дорожного движения.Номенклатура таких составных частей приведена в приложении  2.6 методики [1];

- имеющие коррозионные повреждения;

-требующие ремонта.


Стоимость годных остатков автотранспортного средства может рассчитываться только при соблюдении следующего условия: 

- полная гибель автотранспортного средства в результате ДТП. Под полной гибелью понимается случай, когда стоимость восстановительного ремонта поврежденного ТС превышает его рыночную стоимость на момент повреждения, или проведение восстановительного ремонта технически невозможно.
 
Расчет стоимости годных остатков не производится в следующих случаях:

- когда автотранспортное средство не подлежит, с учетом технического состояния, разборке на запасные части;

- когда, в силу региональных особенностей вторичного  рынка запасных частей, годные остатки данного автотранспортного средства не пользуются спросом.

Учитывая срок эксплуатации ТС \тс \, (23 года), предельную величину физического износа (80\%), региональные особенности вторичного рынка запасных частей, общее техническое состояние и  степень повреждения исследуемого транспортного средства,  в данном случае,   под стоимостью годных остатков понимается стоимость металлической массы автомобиля (металлического лома).
 $  \text{Сго} = \text{Мтс*Сметаллолома}  $, (руб/кг).
 
Стоимость металлолома в городе Краснодаре по данным организаций, занимающихся  приемом лома цветных и черных металлов, составляет от 6,0 до 7,5 рублей за килограмм в зависимости от размеров и качественных характеристик сдаваемого лома (ниже сравнительная таблица цены лома категории 12А1, автомобильный и бытовой легковесный лом). 


\begin{table}[H]
	
\begin{tabular}{|p{8mm}|l|p{26mm}|l|p{32mm}|l|p{18mm}|l|p{10mm}|c|p{8mm}}
	\hline 
№п/п	& {\small Наименование  предприятия} &{\small Адресс } &{\small Телефон}  & 	{\small Стоимость 1 кг, прием лома (руб)} \\
	\hline 
{\small 1}	& {\small Булатов И.П.} &{\small г.Краснодар,  Восточно-Кругликовская, 38 } &	{\small +7 (918) 430-16-70}  & {\small 6,5} \\ 
	\hline 
{\small 2}	&{\small  ВТМ-ЮГПЛЮС} &	{\small г.Краснодар , ул.Текстильная, 3 } &{\small +7 (861) 227-57-07 }&{\small  7,3 }\\ 
	\hline 
{\small 3}	&{\small Евростандарт } &{\small г.Краснодар пос.Знаменский ул.Богатырская 17}  &{\small 89181535143}  & {\small 6,0} \\ 
	\hline 
{\small 4}	& {\small Метализам} & {\small г.Краснодар ул.Уралская, 141/1} &	 {\small 8(918) 467-11-68 } & {\small 	6,5} \\ 
	\hline 
{\small 5}	& {\small Промышленный ресурс} & {\small Краснодар, ул. Соколова, 54}	 &{\small (861)2701773 } & {\small 	7,2 }\\ 
	\hline
\end{tabular} 

\end{table}
		

Средняя цена приема  лома :	 		6.7 руб/кг.

%По информации, представленной организациями, занимающимися приемом и переработкой лома, данная цена не включает в себя стоимость разборки, прессовки и транспортные издержки. В дальнейших расчетах, расходы на проведение указанных выше мероприятий будут вычтены из среднерыночной стоимости приема «чистого» лома.  Поэтому в дальнейших расчетах стоимость  лома металла принята в размере 6,0 рублей за кг.
 
\noindent Если:
масса транспортного средства = 1049 руб. (по данным паспорта ТС);\\
\indent масса неметаллических изделий  ≈ 300 кг. \\
Тогда:  $ \text{Сго = (1049-300)Х6.7= 5 018} $ руб.

\par Таким образом, результаты проведенного  исследования позволяют сделать вывод о том, что стоимость годных остатков транспортного средства \тс\,, с учетом округления,   составляет 5000 (Пять тысяч) рублей.\\


   %% Расчет стоимости годных остатков
%\subsection{Расчет утраты товарной стоимости ТС}


\par Утрата товарной стоимости (УТС) обусловлена снижением товарной стоимости из-за ухудшения потребительских свойств вследствие наличия дефектов (повреждений), или следов их устранения либо наличия достоверной информации, что дефекты (повреждения) устранялись [1,п. 8].

	УТС может быть рассчитана для КТС, находящихся как в поврежденном, так и в отремонтированном состоянии (при возможности установить степень повреждения).

УТС может определяться при необходимости выполнения одного из нижеперечисленных видов ремонтных воздействий или если установлено их выполнение:

-	устранение перекоса кузова или рамы КТС;

-	замена несъемных элементов кузова КТС (полная или частичная); ремонт съемных или несъемных элементов кузова (включая оперение) КТС (в том числе пластиковых капота, крыльев, дверей, крышки багажника);

-	полная или частичная окраска наружных (лицевых) поверхностей кузова (включая оперение) КТС, бамперов;

-	полная или частичная разборка салона КТС, вызывающая нарушение качества заводской сборки.

УТС не рассчитывается:

а)	если срок эксплуатации легковых автомобилей превышает 5 лет;

б)	если легковые автомобили эксплуатируются в интенсивном режиме, а срок эксплуатации превышает 2,5 года;


в)	в случае замены кузова до оцениваемых повреждений (за исключением кузова грузового КТС, установленного на раме за кабиной);

г)	если КТС ранее подвергалось восстановительному ремонту (в том числе окраске - полной, наружной, частичной; «пятном с переходом») или имело аварийные повреждения, кроме повреждений, указанных в [1, п. 8.4];

д)	если КТС имело коррозионные повреждения кузова или кабины на момент происшествия.



Нижеприведенные повреждения не требуют расчета УТС вследствие исследуемого происшествия, а их наличие до исследуемого происшествия не обуславливает отказ от расчета УТС при таких повреждениях:

а)	эксплуатационных повреждениях ЛКП в виде меления, трещин, а также повреждений, вызванных механическими воздействиями - незначительных по площади сколов, рисок, не нарушающих защитных функций ЛКП составных частей оперения;

б)	одиночного эксплуатационного повреждения оперения кузова (кабины) в виде простой деформации, не требующего окраски, площадью не более 0,25 дм2;

в)	повреждения, которые приводят к замене отдельных составных частей, которые не нуждаются в окрашивании и не ухудшают внешний вид КТС (стекло, фары, бампера неокрашиваемые, пневматические шины, колесные диски, внешняя и внутренняя фурнитура и т. п.). Если, кроме указанных составных частей, повреждены составные части кузова, рамы, кабины или детали оперения - крылья съемные, капот, двери, крышка багажника, - то расчет величины УТС должен учитывать все повреждения составных частей в комплексе;

г)	в случае окраски молдингов, облицовок, накладок, ручек, корпусов зеркал и других мелких наружных элементов, колесных дисков.

В случае исследуемого события для автомобиля \тс\, VIN \vin\, все условия  при которых производится расчет УТС выполняются.\\


\par Величина УТС зависит от вида, характера и объема повреждений и ремонтных воздействий по их устранению.
\par Величина УТС ($ C_\text{YTC} $) определяется на дату оценки (исследования) по формуле: 

\begin{equation}\label{uts}
C_{YTC} = C_{KTC} \cdot \dfrac{\sum\limits_{i=1}^n K_{YTCi}}{100\%}, \text{руб.},
\end{equation}

\noindent где:\\
\noindent $ C_{KTC} $ -- стоимость КТС на дату оценки (исследования), руб;\\
$ K_{YTCi} $ -- коэффициент УТС по i-му элементу КТС, ремонтному воздействию, \%.
 


\par  При ремонте съемной составной части сумма стоимости ремонта (включая стоимость разборки для ремонта и при необходимости снятия детали для ремонта) и величины УТС (без учета УТС вследствие окраски) не должна превышать суммы стоимости этой составной части (с учетом коэффициента износа) и стоимости работ по ее замене.

\par   Значение коэффициента УТС $ K_{\text{утсокр}} $ при подетальной окраске наружных поверхностей кузова КТС рассчитывается с учетом количества окрашиваемых кузовных составных частей и бамперов по формуле:

\begin{equation}\label{f:yc}
K_{\text{утсокр}}=K_{\text{утсокр(1)}}+K_{\text{утсокр(N-1)}}\cdot(N-1), %/ 
\end{equation}
        
\noindent где:\\
\noindent $ \text{К}_{\text{утсокр(1)}} $ - коэффициент УТС по окраске первой кузовной составной части или бампера, \%;\\
$ \text{К}_{\text{утсокр(N-1)}} $ - коэффициент УТС по окраске второй и каждой следующей кузовной составной части или бампера, \%;\\
N - количество окрашиваемых составных частей, по которым рассчитывается УТС.\\
Значения коэффициентов УТС ($ K_{YTC} $) определены по результатам экспертой практики и приведены в приложении [1, Приложение 2.9].

\par Для исследуемого автомобиля \тс \, соответствующие ремонтным воздействиям  коэффициенты УТС приведены ниже в таблице:

\begin{table}[H]
		%\caption{}
	\begin{tabular}{|p{5mm}|p{80mm}|c|c|c|}
	\hline 
	\textbf{п/п} & \textbf{Наименование детали} &\textbf{ К-замена }& \textbf{К-ремонт }&\textbf{ К-окраска} \\ 
	\hline 
	1 & Наружная окраска кузова & -- & -- & 5 \\ 
	\hline 
	2 & Бампер задний & -- & -- & 0,35 \\ 
	\hline 
	3 & Дверь правая & -- & 0,2 & -- \\ 
	\hline 
	4 & Панель задка & 0,3 & -- & -- \\ 
	\hline 
	5 & Крыло заднее левое & 0,5 & -- & -- \\ 
	\hline 
	6 & Арка колеса наружная & 0,2 & -- & -- \\ 
	\hline 
	7 & Надставка левая задняя & 0,2 & -- & -- \\ 
	\hline 
	8 &  Устранить перекос проема левой двери & -- & 0,5 & -- \\ 
	\hline 
	8 & Нарушение целостности заводской сборки при полной разборке/сборке салона & -- & 1 & -- \\ 
	\hline 
	
\end{tabular} 

\end{table}

\vspace{7mm}

$  \sum\limits_{i=1}^n K_{YTCi} = 5+1+0.35+0.2+0.3+0.5+0.2+0.2+0.5 = 8.25 $\\
  
$   C_{KTC} = C_{KTC} \cdot \dfrac{\sum\limits_{i=1}^n K_{YTCi}}{100} = 4508000 \cdot 8.25/100 = 371910 $, или с учетом округления 372000 (Триста семьдесят две тысячи) рублей.\\

Таким образом, величина УТС автомобиля \тс\, составляет 372 000 (Триста семьдесят две тысячи) рублей.

%%%%%%%%%%%%%%%%%%%%%%%%%%%%%%%   Условие расчета ущерба
%\par Согласно п.7.2. Медодики [1] размер ущерба вследствие повреждения КТС принимается равным рыночной стоимости КТС, если соблюдается условие:
%\begin{equation}\label{2}
%C_{BP} + C_{YTC} \geqslant C_{KTC},
%\end{equation} 
%где: $C_{KTC} $ -- рыночная стоимость КТС, руб;\\
%\indent $C_{YTC} $ -- величина УТС, руб.\\
%В этом случае рыночная стоимость определяется на заданную дату оценки, с учетом срока его эксплуатации и технического состояния на момент происшествия. \\
%Согласно п. 9.6. Методики [1], в случаях правовых отношений, регулируемым Гражданским кодексом РФ, расчет стоимости годных остатков и определение стоимости реального ущерба с его уменьшением на стоимость годных остатков не предусмотрены.

%%%%%%%%%%%%%%%%%%%%%%%%%%%%%%%%

\section{Выводы}
%
%
%При условии достоверности предоставленных данных в части повреждений ТС \тс \, при заявленных обстоятельствах, 
%эксперт приходит к следующим выводам:\par
\begin{enumerate}
% \item  Наличие, характер и объем (степень) технических повреждений, причиненных ТС, определены при осмотре и зафиксированы в Акте осмотра \NomerDoc.
% \, и фототаблице повреждений, являющимися неотъемлемой частью настоящего экспертного заключения.
% \\[-2mm]
%\item  Направление, расположение и характер повреждений определены путем сопоставления полученных повреждений, изучения административных материалов по рассматриваемому событию, и  являются  следствиями рассматриваемого ДТП (события).\\[-2mm]
%\item  Технология и объем необходимых ремонтных воздействий зафиксированы в калькуляции \NomerDoc \, по определению стоимости восстановительного ремонта транспортного средства \тс. Расчетная стоимость восстановительного ремонта составляет 1\,085\,696 (Один миллион восемьдесят пять тысяч шестьсот девяносто шесть) рублей.
%\\[-2mm]
\item  Стоимость восстановительного ремонта  транспортного средства \tc\,\- \грз\, получившего повреждения вследствие противоправных действий, имевших место \- \датадтп, составляет 37 650 (Тридцать семь тысяч шестьсот пятьдесят) рублей.\\[-2mm]   
%\item Размер ущерба, причиненного владельцу  транспортного средства \tc \, в результате дорожно-транспортного происшествия, имевшего место \датадтп\,
%составляет 80\~000 (Восемьдесят тысяч) рублей.\\[-2mm]  
%\item Рыночная стоимость транспортного средства ТС \тс\, на момент повреждения составляет 35000 (Тридцать пять тысяч) рублей.\\[-2mm]
%\item Стоимость годных остатков ТС \тс \, \, составляет 5 000 (Пять тысяч) рублей.
\end{enumerate}
\vspace{10mm}
\noindent{Эксперт-техник}      \hfill                        {Мраморнов А.В.}
\vspace{7mm}
\relax

Приложения к заключению:\\

\noindent \textit{\small 
	%	Приложение № 1. Расшифровка модельных опций ТС \тс \\
    Приложение № 1. Калькуляция стоимости восстановительного ремонта ТС \тс\\
	Приложение № 2. Акт осмотра ТС \тс\\
	Приложение № 3. Фототаблица повреждений ТС \тс\\
%	Приложение № 4. Цифровые копии регистрационных документов ТС\\
%	Приложение № 5. Цифровая копия постановления по делу об административном правонарушении дорожно-транспортном происшествии\\
	Приложение № 4. Правоустанавливающие документы\\}

\vspace{20mm}


%\noindent{Эксперт-техник}      \hfill                        {Мраморнов А.В.}

%\includepdf[pages=-]{foto.pdf}
%\includepdf[pages=-]{calc.pdf}
%\includepdf[pages=-]{myfile.pdf}
%
\setcounter{page}{1}
\clubpenalty=100000  % Недопуск Висячей строки в начале страницы
\widowpenalty=100000 %Недопуск висячей строки в конце абзаца
%%%%%%%%%%%%%%%%%%%%%%%%%%%%%%%%%%%%%%%%
%      Шапка экспертной организации  
%%%%%%%%%%%%%%%%%%%%%%%%%%%%%%%%%%%%%%%%
%%%%%%%%%%%%%%%%%%%%%%%%%%%%%%%%%%%%%%%%%
%
%   Экспертная организация ООО Южнорегиональная экспертная группа
%
%%%%%%%%%%%%%%%%%%%%%%%%%%%%%%%%%%%%%%%%%
\noindent %\qrcode[height=21mm]{\NomerDoc от \окончено }  %%% Добавлен QR-Code
\begin{pspicture}(21mm,21mm)
\obeylines
\psbarcode{%
%	\NomerDoc от \окончено
	BEGIN:VCARD^^J
	VERSION:4.0^^J
	%N:Мраморнов; Александр; Вчеславович^^J
	FN:Александр Мраморнов^^J
%	ORG:IP Alexandr Mramornov^^J
	TITLE: эксперт
	ORG: ИП
	URL:http://www.yourexp.ru^^J
	EMAIL:4516611@gmail.com^^J
	TEL:+7-918-451-6611^^J
	ADR:г. Краснодар, с/т № 2 А/О «Югтекс», ул. Зеленая, 472^^J
	END:VCARD
}{width=1.0 height=1.0}{qrcode}%
\end{pspicture}%%% Добавлен QR-Code
\vspace{-4mm}
\begin{center}
	\large\textbf{ИНДИВИДУАЛЬНЫЙ\quad ПРЕДПРИНИМАТЕЛЬ  \\[-1.5mm] МРАМОРНОВ  АЛЕКСАНДР ВЯЧЕСЛАВОВИЧ \\[-5.5mm]}
	%  
	\noindent\rule{\textwidth}{2pt}\\[-6mm]  % Горизонтальная линия
	% \line(1,0){460}% (1,0) -горизонтальная линия, и (0,1) - вертикальная 
\end{center}

\begin{center}
	\begin{footnotesize}\setstretch{0.3}
		%	\small\textbf\setlength   	%\raisebox{5mm}
		\vspace{-3.5mm}г. Краснодар, с/т № 2 А/О «Югтекс», ул. Зеленая, 472, 
		Телефон: 8-918-451-66-11, e-mail: 4516611@gmail.com\\ [-2mm]{ИНН\quad 231200665168\quad ОГРНИП \quad 310231220400043}
	\end{footnotesize}	\\[10mm]
\end{center}


\begin{flushright}
% 
	 \hfill	Краснодар, 2020    \\[8mm]
\end{flushright}
\begin{center}
	\LARGE\textbf{ЭКСПЕРТНОЕ ЗАКЛЮЧЕНИЕ}
	\bigskip\\[0mm]
	%	{\normnumxtbf{\NomerDoc}}	}{den}
\end{center}
\par
\vspace{-6mm}
\noindent независимой технической экспертизы по определению размера расходов на восстановительный ремонт транспортного средства   \тс  \\[2mm]

%\raggedright 
%\def\hrf#1{\hbox to#1{\hrulefill}}
\noindent \textbf{№ \NomerDoc}\hfill           \textbf{\окончено}\\%[2mm]
%Приостановлено\hfill      \datastop\\
%Возобновлено\hfill          \datarestart\\
%Окончено\hfill                \dataend\\%[4mm]

\noindent\parbox[l][16mm]{16.5cm}
{\def\hrf#1{\hbox to#1{\hrulefill}}
	\noindent Начато\hfill            \datastart\\%[2mm]
	%	Приостановлено\hfill      \datastop\\
	%	Возобновлено\hfill          \datarestart\\
	Окончено\hfill                \окончено
}
\relax

%%%%%%%%%%%% Если судебка
%
%\datastart г. ~в {\small ООО~ "ЮЖНО-РЕГИОНАЛЬНАЯ ЭКСПЕРТНАЯ ГРУППА"} \,  при определении  \, \sud  \,  от \, \dataopr \, о назначении \opr \, по гражданскому делу \delonum \, поступили:
%
%\begin{enumerate}\setlist{nolistsep}\item  Материалы гражданского дела \delonum \, в двух томах, том 1 на 276 листах, том 2  на 143 листах.\\[-2mm]
%	%	\item  
%\end{enumerate}
%
%%%%%%%%%%%%  Если независимая
\vspace{4mm}
Составлено на основании	договора № \NomerDoc\,  возмездного оказания услуг по проведению независимой технической экспертизы (далее экспертиза)  транспортного средства и письменного заявления заказчика о проведении экспертизы.

Заказчик  экспертизы: \заказчик, \адресзаказчика. 

Полис ОСАГО: \polis.

% Документ, удостоверяющий личность заказчика: водительское удостоверение    03\ 16\ 422344\ выдан 09.06.2011

%Транспортное средство виновника ДТП:  не предоставлялось.

\paragraph*{}
Экспертиза произведена  экспертом--техником
%{\small ООО "ЮЖНО-РЕГИОНАЛЬНАЯ ЭКСПЕРТНАЯ ГРУППА"}
\,  Мраморновым Александром Вячеславовичем, имеющим высшее техническое образование по специальности «техническая физика», диплом РВ № 311964 от 28.02.1989, квалификация -- инженер-физик, специальное образование в области оценки: Диплом ПП-1 № 037211 Российской экономической академии им. Г.В. Плеханова, квалификация -- оценка и экспертиза объектов и прав собственности, специальное образование в области независимой технической экспертизы транспортных средств: Диплом ПП-I № 424167, квалификация: эксперт-техник (специализация 150210 специальности 190601.65 – Автомобили и автомобильное хозяйство), состоящий в Государственном реестре экспертов-техников (№ в реестре 256, https://data.gov.ru/opendata/7707211418-experts,  общий трудовой  стаж 30 лет, стаж  экспертной работы  12 лет. \par Заключение подготовлено по месту фактического расположения ИП по адресу: г. Краснодар, с/т № 2 А/О «Югтекс», ул. Зелёная, 472.
  % Шапка организации ИП
%%%%%%%%%%%%%%%%%%%%%%%%%%%%%%%%%%%%%%%%%%
%
%   Экспертная организация ООО Южнорегиональная экспертная группа
%
%%%%%%%%%%%%%%%%%%%%%%%%%%%%%%%%%%%%%%%%%
\noindent %\qrcode[height=21mm]{\NomerDoc от \окончено }  %%% Добавлен QR-Code
\begin{pspicture}(21mm,21mm)
\obeylines
\psbarcode{%
	%\NomerDoc от \окончено
	BEGIN:VCARD^^J
	VERSION:4.0^^J
	%N:Мраморнов; Александр; Вчеславович^^J
	FN:Александр Мраморнов^^J
%	ORG:IP Alexandr Mramornov^^J
	TITLE: эксперт
	ORG: ИП
	URL:http://www.yourexp.ru^^J
	EMAIL:4516611@gmail.com^^J
	TEL:+7-918-451-6611^^J
	ADR:г. Краснодар, с/т № 2 А/О «Югтекс», ул. Зеленая, 472^^J
	END:VCARD
}{width=1.0 height=1.0}{qrcode}%
\end{pspicture}
\begin{center}
	\normalsize\textbf{$\cdots$\\[-1.5mm] <<$\cdots$>> \\[-5mm]}
	%  
	\noindent\rule{\textwidth}{1pt}\\[-6mm]  % Горизонтальная линия
	% \line(1,0){460}% (1,0) -горизонтальная линия, и (0,1) - вертикальная 
\end{center}

\begin{center}
	\begin{footnotesize}\setstretch{0.3}
		%	\small\textbf\setlength   	%\raisebox{5mm}
		\vspace{-3.5mm}$\cdots$\\[0mm]
		Телефон: \quad $\cdots$, e-mail:\quad $\cdots$\\ [-2mm]{$\cdots$\quad$\cdots$}
	\end{footnotesize}	\\[10mm]
\end{center}


\begin{flushright}
	%Краснодар,
	$\cdots$, 2020    \\[8mm]
\end{flushright}
\begin{center}
	\LARGE\textbf{ ЗАКЛЮЧЕНИЕ ЭКСПЕРТА}
	\bigskip\\[0mm]
	%	{\normnumxtbf{\NomerDoc}}	}{den}
\end{center}
\par
\vspace{-3mm}\noindent по гражданскому делу \delonum \, по иску \isk \\[0mm]

%\raggedright 
%\def\hrf#1{\hbox to#1{\hrulefill}}
\noindent \textbf{№ $\cdots$}\hfill           \textbf{\окончено}\\%[2mm]
%Приостановлено\hfill      \datastop\\
%Возобновлено\hfill          \datarestart\\
%Окончено\hfill                \dataend\\%[4mm]

\noindent\parbox[l][16mm]{16.5cm}
{\def\hrf#1{\hbox to#1{\hrulefill}}
	\noindent Начато\hfill            \datastart\\%[2mm]
	%	Приостановлено\hfill      \datastop\\
	%	Возобновлено\hfill          \datarestart\\
	Окончено\hfill                \окончено\\%[4mm]
}
\relax

\datastart г. ~в {\small $\cdots$} \,  при определении  \, \sud  \,  от \, \dataopr \, о назначении \opr \, по гражданскому делу \delonum \, поступили:

\begin{enumerate}\setlist{nolistsep}\item  Материалы гражданского дела \delonum \\[-2mm]
	%	\item  
\end{enumerate}Экспертиза произведена экспертом  $\cdots$  % Шапка организации ИП


% Составлено на основании	договора № \NomerDoc\,  на оказание услуг по проведению независимой технической экспертизы (далее экспертиза)  транспортного средства и письменного заявления заказчика о проведении экспертизы.
%
%Заказчик  экспертизы: \заказчик, \адресзаказчика. 
%
%Полис ОСАГО: \polis.

%
%    \paragraph{}  Экспертиза проводится специалистом экспертом-техником  Мраморновым Александром Вячеславовичем, имеющим высшее техническое образование по специальности «техническая физика», диплом РВ №311964, квалификация – инженер-физик, специальное образование в в области независимой технической экспертизы транспортных средств: Диплом ПП-I № 424167, квалификация:  эксперт-техник (специализация 150210 специальности 190601.65 – Автомобили и автомобильное хозяйство), специальное образование в области оценки: Диплом ПП-1 № 037211 Российской экономической академии им. Г.В. Плеханова, квалификация – оценка и экспертиза объектов и прав собственности, , состоящий в Государственном реестре экспертов-техников (№ в реестре 256,  \url{http://minjust.ru/ru/node/105988}), стаж экспертной работы  12 лет.
%	
	                                                       
 \subsection{Исходные данные}

1. Транспортное средство \tc\, регистрационный знак \грз\,  в повреждённом состоянии.

2. Свидетельство \свид\, о регистрации  транспортного средства \тс \, регистрационный знак \грз.

3. Постановление \постановление\, по делу об административном правонарушении, дорожно-транспортном происшествии, имевшем место  \датадтп\, с участием  ТС \тс\, регистрационный знак \грз, согласно которому на ТС \тс\, регистрационный знак \грз\,  в результате ДТП повреждено:  <<\повреждения >>.  
 
4. Полис страхования  ОСАГО №  \polis
                                                              

\subsection{Вопросы  экспертизы}
      
1. Установить наличие, характер и объем (степень) технических повреждений транспортного средства  \tc\, регистрационный знак \грз.

2. Установить причины возникновения технических повреждений транспортного средства \тс\, регистрационный знак \грз\, и возможность их отнесения к рассматриваемому дорожно-транспортному происшествию (далее ДТП).

3. Установить технологию, объем восстановительного  ремонта транспортного средства \tc\, регистрационный знак \грз.


5. Определить стоимость восстановительного ремонта  транспортного средства \tc\, регистрационный знак \грз,\, \, получившего механические повреждения в результате дорожно-транспортного происшествия, имевшего место \датадтп\, с участием транспортных средств \тс\, регистрационный знак \грз\, и \tcb.


4. Установить размер затрат на восстановительный ремонт (с учётом износа) транспортного  \тс\, регистрационный знак \грз.


%5. В случае полной гибели ТС \тс\, регистрационный знак \грз\, определить стоимость годных остатков.

%5. Определить размер утраты товарной стоимости автомобиля \тс\, регистрационный знак \грз.
  
\subsubsection{Условные обозначения}

\noindent АМТС – автомототранспортное средство\\
ДТП – дорожно-транспортное происшествие\\
г.р.з. – государственный регистрационный знак транспортного средства\\
ТС – транспортное средство
ЛКП –лако-красочное покрытие\\
VIN – 17-значный идентификационный номер транспортного\\ средства, соответствующий стандарту ISO 3779-2012. 


\subsubsection{Использованные нормативы и источники информации}

1. Федеральный закон «Об обязательном страховании гражданской ответственности владельцев транспортных средств» от 25.04.2002 г. № 40-ФЗ.

2. Положения Банка России от «19» сентября 2014 года № 431-П «О правилах обязательного страхования гражданской ответственности владельцев транспортных средств».

3. Положение Банка России от «19» сентября 2014 года № 432-П «О единой методике определения размера расходов на восстановительный ремонт в отношении повреждённого транспортного средства».

4. Положение ЦБ РФ № 433-П «О правилах проведения независимой технической экспертизы транспортного средства» от 19 сентября 2014 г.

5. Технический регламент Таможенного союза <<О безопасности колёсных транспортных средств>> (ТР ТС - 018 - 2011).

6. Методические рекомендации по проведению судебных автотехнических экспертиз и исследований колёсных транспортных средств в целях определения размера ущерба, стоимости восстановительного ремонта и оценки / Е. Л. Махнин, И. Н. Новоселецкий, С. В. Федотов и [др.]. - М. : ФБУ РФЦСЭ при Минюсте  России, 2018. - 326 с.

7. Технологическое руководство «Приёмка, ремонт и выпуск из ремонта кузовов легковых автомобилей предприятиями автотехобслуживания» РД 37.009.024-92.

8. Предотвращение страхового мошенничества в автостраховании  (практическое  пособие)  М.  2005.

9. Исследование транспортных средств в целях определения стоимости восстановительного ремонта и оценки: курс лекций / под общ. ред. д-ра юрид. наук, профессора С.А. Смирновой; Министерство юстиции Российской Федерации, Федеральное бюджетное учреждение Рос. Федер. центр судеб экспертизы. - М.: ФБУ РФЦСЭ при Минюсте России, 2017. - 286 с.

10. Методика окраски и расчёта стоимости лакокрасочных материалов для проведения окраски ТС – AZT. 

11. Сервис по автоматической расшифровке VIN номеров – AudaVIN.

12. Сервис РСА для проверки текущего договора ОСАГО,  http://86.62.95.12:8080/dkbm-web-1.0/bsostate.htmhttp://prices.autoins.ru/spares/.

13. Онлайн сервис РСА средней стоимости запасной части и нормочаса в экономическом районе,    http://prices.autoins.ru/priceAutoParts/.

14.  Ресурсы Интернет.



\subsubsection{Технические средства}

1. Рулетка измерительная металлическая, 0-5000мм, «HORTZ» №451, отклонение от действительной длины ± 1,20мм;

2. Линейка измерительная металлическая, ГОСТ 427-75, заводской номер 51118, 0-500мм, цена деления 1мм, пг ± 0,15мм;

3. Линейка масштабная магнитная с цветографической шкалой, 100 мм;

4. Цифровой фотоаппарат  Canon 760D s/n 143032001327  с объективом Canon EF-S 18-135;

5. Специализированное программное обеспечение для расчёта стоимости  восстановительного ремонта, содержащее нормативы трудоёмкости работ, регламентируемые изготовителями транспортного средства     AudaPadWeb, лицензионное соглашение № AS/APW-658  RU-P-409-409435;

6. MotorData — интерактивная справочно-информационная система по диагностике и ремонту автомобилей. http://motordata.ru/ru

7. Программа обработки фото-видео изображений ImageJ, разработчик  Wayne Rasband (wayne@codon.nih.gov) , is at the Research Services Branch, National Institute of Mental Health, Bethesda, Maryland, USA. Свободная лицензия GPL;

8. ПЭВМ под управлением операционной системы Windows 10 с установленным набором офисных программ LebreOffice, лицензия: Mozilla Public License версия 2.0, \url{http://mozilla.org/MPL/2.0/.}

\subsubsection{Ограничения и пределы применения полученных результатов}

      Следующие допущения и условия, ограничивающие пределы применения полученных результатов, являются неотъемлемой частью данного экспертного заключения:
      
\noindent  - Результаты, полученные экспертом-техником, носят рекомендательный консультационный характер и не являются обязательными. Исполнитель высказывает своё субъективное суждение о наиболее вероятных будущих (абстрактных) расходах, их предполагаемом размере и даёт заключение в пределах своей компетенции.

\noindent - Под компетенцией эксперта-техника понимают его знания и опыт в области теории и методов экспертных исследований ТС, а также круг полномочий, представленных ему законом, и вопросов, которые он может решать на основе своих специальных познаний.
      В компетенцию эксперта-техника входит исследование технического состояния повреждённого ТС в целях установления характера повреждений ТС, установления причины возникновения технических повреждений технологии, методов, стоимости его ремонта.
      
\noindent - Исполнитель в рамках своих обязательств по заключённому договору об экспертном обслуживании признает свою ответственность перед заказчиком и настоящим утверждает, что экспертное заключение выполнено профессионально, тщательно и с должной заботливостью и внимаем, как это обычно принято для компетентного специалиста в области технической экспертизы ТС при ОСАГО, а полученная величина восстановительных расходов, разумна и реальна.

\noindent - Исполнитель считает, что поскольку, по общему правилу, оценка доказательств является прерогативой и компетенцией органа дознания, следствия или суда, а в досудебном порядке - страховщика, постольку после проверки результатов экспертизы последним, их признания и принятия решения о выплате страхового возмещения этап возможного оспаривания достоверности исследований между заказчиком и исполнителем завершён.
      Соответственно, обязанности Исполнителя по договору являются надлежаще исполненными в полном объёме и от исполнителя не требуется свидетельствовать по поводу произведённого исследования перед третьими лицами.
      
\noindent - Отдельные части настоящего экспертного исследования не могут трактоваться раздельно, а только в связи с полным текстом о проведённых расчётах.

\noindent - Исходные данные, использованные исполнителем при подготовке экспертного заключения, получены из надёжных источников и считаются достоверными. Тем не менее, исполнитель не может гарантировать абсолютную точность, поэтому там, где это, возможно, делаются ссылки на источники информации.

\noindent - В процессе экспертного исследования специальная юридическая экспертиза документов, касающихся прав собственности на ТС, не проводилась.

\noindent - Суждения, содержащиеся в экспертном заключении, основываются на текущей ситуации на дату аварии и, в будущем, могут быть подвержены изменениям.
Исполнитель не принимает на себя никакой ответственности за изменение экономических, юридических и иных факторов, которые могут возникнуть после даты исследования и повлиять на результаты технической экспертизы.

Данное заключение составлено на основании Правил Независимой Технической Экспертизы и может применяться только при решении вопроса о выплате страхового возмещения по ОСАГО.


\section{Исследование}

      Настоящее исследование проводится на основании материалов, предоставленных Заказчиком, а также на основании данных, самостоятельно полученных экспертом. Выводы, содержащиеся в настоящем Заключении, могут расцениваться как достоверные только в контексте того количества информации, на основании которого они были сделаны. При поступлении дополнительной или изменённой информации данные выводы могут быть
скорректированы. При анализе скрытых повреждений экспертом не принимается во внимание наличие или отсутствие записей о них в документах компетентных органов, в связи с отсутствием у сотрудников компетентных органов объективной возможности  идентификации таких повреждений на месте происшествия.

	\par Из предоставленных материалов   экспертом-техником установлена следующая общая информация об автомобиле, имеющая значение для дачи заключения:\\
 \parbox[]{10cm}{}
\begin{itemize}
	\item[ ] 
	\begin{description}
		\item[Марка, модель] -- \тс
		\item[VIN] -- \vin
		\item[Год выпуска] -- \год
		\item[Шасси] -- Отсутствует
		\item[Цвет ЛКП] -- \цвет
		\item[Пробег] --  \пробег\, км %, считан с одометра
		\item[Дата начала эксплуатации] -- \началоэкспл
		\item[Двигатель] -- \двигатель
		\item[Объем двигателя] -- 1328 $ \text{см}^3 $
		\item[Свидетельство о регистрации] -- \свид
		\item[ПТС] --\птс.	
	\end{description}
\end{itemize}

	\subparagraph*{} Идентификационный код автомобиля (VIN)  \vin \, содержит следующую информацию о транспортном средстве, имеющую значение для 	дачи заключения (Рис. \ref{fig:vin} ):\\[3mm]
%	
	\noindent\parbox[]{10cm}{
		\begin{itemize}
			\item[ ] 
			\begin{description}
		%		\item[Дата изготовления] \hfill \датаизготовления
				\item[Расположенние руля] \hfill Left
				\item[Двигатель] \hfill \двигатель
				\item[Объем двигателя] \hfill 1328 $ \text{см}^3 $
				\item[КПП] \hfill АКПП
				\item[Тип кузова] \hfill  \типкузова
				\item[Количество дверей] \hfill 3 
				%	\item[VDS] --
					
			\end{description}
	\end{itemize}}\\
	
\vspace{3mm}
	

	\begin{figure}[H]
		\centering
		\includegraphics[width=0.999\linewidth]{images/vin}
		\caption[]{Комплектация автомобиля VIN \vin\, по данным\textit{ \url{https://emex.ru/catalogs/original/?screen=units\&vin=}\вин} }
		\label{fig:vin}
	\end{figure}
	
	
	\subsubsection{Осмотр транспортного средства}
	
   \osm\, экспертом-техником проведён осмотр повреждённого транспортного средства \tc\, регистрационный знак \grz. Осмотр проводился в сухую, ясную погоду с 11-00  до 13-00 на открытой площадке и в гаражных условиях  по адресу: \местоосмотра.   При осмотре присутствовали владелец транспортного средства \тс\, и второй участник ДТП: \присутствовали. Соответствие маркировочных обозначений на кузове представленного ТС записям в регистрационных документах ТС экспертом-техником установлено. Видимые изменения конструкции ТС отсутствуют.  Представленный на исследование автомобиль \тс\, регистрационный знак \грз\, имеет кузов типа <<\типкузова». Кузов автомобиля окрашен двухслойной   %лессирующей (с металликовым эффектом) 
   эмалью (краской)  \colr \, цвета.
   
     
   Наличие, характер и объем технических повреждений, а так же  планируемые (предполагаемые) ремонтные воздействия для восстановления повреждённого  
   транспортного средства \tc\, регистрационный знак \grz, исследованы в присутствии заинтересованных лиц, зафиксированы в акте осмотра № \NomerDoc\,  (Приложение, <<Акт осмотра>> ),  и фотоматериалах (Приложение, <<Фототаблица>>) по принадлежности. 
          
          
\subsubsection{Причины возникновения технических повреждений}
          
     Причины возникновения технических повреждений и возможность их отнесения к
рассматриваемому ДТП исследованы при осмотре ТС. Для определения причины возникновения повреждений, указанных в Акте осмотра ТС  №  \NomerDoc\, (Приложение № 1) экспертом-техником изучены документы, представленные Заказчиком. По предоставленным документам экспертом установлена причина ДТП, установлены обстоятельства ДТП, выявлены повреждения ТС и установлены причины их образования. Проведено исследование характера выявленных повреждений, сопоставление повреждений ТС потерпевшего с повреждениями ТС иных участников ДТП в соответствии со сведениями, зафиксированными в документах о ДТП.  Проведена проверка взаимосвязанности повреждений на ТС с заявленными обстоятельствами ДТП, определён объем восстановительных работ (Приложение № 1, Акт осмотра ТС). 


\subsubsection{Определение восстановительных расходов}

 В соответствии с существующей экспертной методикой размер расходов на восстановительный ремонт определяется исходя из стоимости ремонтных работ (работ по восстановлению, в том числе окраске, контролю, диагностике и регулировке, сопутствующих работ), стоимости используемых в процессе восстановления транспортного средства деталей (узлов, агрегатов) и материалов взамен повреждённых. Расчёт размера расходов (в рублях) на восстановительный ремонт производится по формуле: 
      
\begin{equation}\label{eq:cr}
C_{\text{вр}}  =\sum{C_{ip}}= \sum\left({T_{ij}}\cdot {C_{i\text{нч}}}\right) + \sum{C_{ip^{\text{\,\,\,руб}}}} , \,\,\,\text{где:} 
\end{equation}
%\vspace{2mm}
\begin{itemize}
	\item[ ]$ C_{ip} $ -- стоимость работ i-го вида: $C_\text {зам} $, $ C_\text{восст} $, $ C_\text{рег} $, $C_\text{контр} $, $ C_\text{антикор} $, $ C_\text{зч} $, $ C_\text{ом} $,$ C_\text{соп} $, $ C_\text{вм} $, руб;
	\item[ ]$ T_{ij} $ -- трудоёмкость j-й операции(комплекса) по i-му виду работ, руб;
	\item[ ]$ C_{i\text{нч}} $ -- стоимость нормо-часа по i-му виду работ, руб;
	\item[ ]$ C_{ip^\text{\,\,руб}} $ -- стоимость работ $ C_{ip} $, принятая непосредственно в денежном выражении, руб.
\end{itemize}

\par При определении стоимости восстановительного ремонта АМТС с учётом износа под износом следует понимать количественную меру физического старения АМТС и его элементов, достигнутого в результате эксплуатации, т.е. эксплуатационный износ.
%
Расчёт износа производится в  соответствии с Положением Банка России от «19» сентября 2014 года № 432-П «О единой методике определения размера расходов на восстановительный ремонт в отношении повреждённого транспортного средства» [3].
Износ комплектующих изделий (деталей, узлов, агрегатов) рассчитывается по следующей формуле:
%
\begin{equation}\label{eq:I}
\text{И}_{\text{ки}} 
= 100\cdot\left( 1-e^ {-\left( \Delta_{T} \cdot T_{\text{КИ}} + \Delta_{L} \cdot L_{\text{КИ}} \right)}\right), \,\,\,\,\text{где:}   
\end{equation}
%
\begin{itemize}
	\item[ ]$ \text{И}_{\text{ки}} $ -- износ комплектующего изделия (детали, узла, агрегата) (процентов); 
	\item[ ]$ e $ -- основание натуральных логарифмов (e =  2,72);
	\item[ ]$ \Delta_{T}$ --  срок эксплуатации комплектующего изделия (детали, узла, агрегата) (лет);
	\item[ ]$ T_{\text{КИ}} $ -- стоимость работ $ C_{ip} $, принятая непосредственно в денежном выражении, руб;
	\item[ ]$ \Delta_{L} $ -- коэффициент, учитывающий влияние на износ комплектующего (детали, узла, агрегата) величины пробега транспортного средства с этим комплектующим изделием;
	\item[ ]$ L_{\text{КИ}} $ -- пробег транспортного средства на дату дорожно-транспортного происшествия (тысяч километров).  
\end{itemize}
\vspace{5mm}
\par Значения коэффициентов $ \Delta_{T}$  и $ \Delta_{L} $  для различных категорий и марок транспортных средств приведены в п. 5. сп. лит~[3]. При этом, на комплектующие изделия (детали, узлы, агрегаты), которые находятся в заведомо худшем состоянии, чем общее состояние транспортного средства в целом, и его основные части, вследствие влияния факторов, не учтённых при расчёте износа (например, проведение ремонта с нарушением технологии, не устранение значительных повреждений лакокрасочного покрытия), может быть начислен дополнительный индивидуальный износ. 
Износ шины транспортного средства рассчитывается по следующей формуле:
\begin{equation}\label{eq:sh}
\text{И}_{\text{ш}} = \frac{\text{Н}_{\text{н}}-\text{Н}_{\text{ф}}}{\text{Н}_{\text{н}}-\text{Н}_{\text{доп}}} \cdot{100}\%,  \,\,\,\,\text{где:} 
\end{equation}
%
\begin{itemize}
	\item[ ] $ \text{И}_{\text{ш}} $ -- износ шины, \%;
	\item[ ] $ \text{Н}_{\text{н}} $ -- высота рисунка протектора новой шины, мм;
	\item[ ] $\text{Н}_{\text{ф}} $ -- фактическая высота рисунка протектора шины, мм;
	\item[ ] $ \text{Н}_{\text{доп}} $ --минимально допустимая высота рисунка протектора шины в соответствии с требованиями законодательства Российской Федерации, мм.
\end{itemize}
%
\vspace{5mm}
\relax
%\renewcommand\baselinestretch{1}\small\normalsize
%
Износ шины дополнительно увеличивается для шин с возрастом от 3 до 5 лет - на 15 процентов, свыше 5 лет - на 25 процентов.

                                                 

\subsubsection{Данные для расчёта}

\noindent Объект экспертизы:  транспортное средство \tc\,
регистрационный знак \грз;\\ 
VIN: \вин;\\
Пробег:    \пробег км (установлен по показаниям одометра);\\
Год выпуска:     \год;\\ 
Дата ввода в эксплуатацию:  \началоэкспл;\\
Дата ДТП:  \датадтп;\\
Перечень повреждений ТС представлен в Приложении <<Акт осмотра>>;\\
Рыночная стоимость ТС: \tc\,
регистрационный знак \grz \, по данным открытых специализированных информационных источников составляет: $640 000$ (Шестьсот сорок тысяч) рублей.\\



% TODO: \usepackage{graphicx} required
\begin{figure}[H]
	\centering
	\includegraphics[width=0.65\linewidth]{images/1}
	\caption[]{Диаграмма рыночной стоимость автомобиля, аналогичного исследуемому \тс\, \textsl{источник:} \url{https://spec.drom.ru}, \url{https://automama.ru/ocenka-avto}}
	\label{fig:1}
\end{figure}

    
   \subsubsection{ Расчёт}
    
 Полный расчёт стоимости восстановительных расходов на ремонт ТС с учётом износа в соответствии с правилами обязательного страхования гражданской ответственности владельцев транспортных средств выполнен в  лицензированном для решения задач в рамках ОСАГО программном комплексе   Audatex ОСАГО Про и приведён в Калькуляции № \NomerDoc.
 Расчёт износа произведён программой  Audatex ОСАГО Про и представлен  в калькуляции расчёта затрат № \NomerDoc.

    \indent Результаты расчёта  стоимости восстановительных расходов ТС \тс\, \грз\, представлены ниже:
    %
    \begin{figure}[H]
    	\centering
    	\includegraphics[width=0.95\linewidth]{images/Screenshot_1}
    	%%	\caption{}
    	%%	\label{fig:screenshot001}
    \end{figure}
    \begin{figure}[H]
    	\centering
    	\includegraphics[width=0.95\linewidth]{images/Screenshot_2}
    	%%	\caption{}
    	%%	\label{fig:screenshot001}
    \end{figure}
    %\begin{figure}[H]
    %	\centering
    %	\includegraphics[width=0.9\linewidth]{images/screenshot002}
    %%%	\caption{}
    %	\label{aud}
    %\end{figure}
    \medskip
    \renewcommand\baselinestretch{1.2}\small\normalsize
    

\subparagraph{}Стоимость одного нормо-часа работ определена в соответствии с пунктом 3.8.1 Единой методики [3] путём применения электронных баз данных стоимостной информации.
      Трудоёмкость работ по разборке/сборке/замене  соответствует трудоёмкостям работ, рекомендованным заводом изготовителем ТС. Трудоёмкости окрасочных работ приняты согласно рекомендаций Единой методики, п.3.7.1. в соответствии с технологией  AZT (\url{http://www.schwacke.ru/down/azt _reparaturlackierung_ru.pdf}). Расчёт размера расходов на материалы произведён  согласно пункту 3.7.2 Приложения к Единой методике [3]. 
      Стоимость запасных частей определена в соответствии с пунктом 3.6.3 Единой методики путём применения электронных баз данных стоимостной информации (по утверждённому справочнику: \url{http://prices.autoins.ru/priceAutoParts/repair_parts.html} ).
  
\subparagraph{}Таким образом,  наиболее вероятная стоимость ремонта транспортного средства \tc\, регистрационный знак \грз, получившего повреждения в результате дорожно-транспортного происшествия  \датадтп\, составляет $609 422$ (Шестьсот девять тысяч четыреста двадцать два) рубля, с учётом округления составляет $ 609 000 $ (Шестьсот девять тысяч) рублей, размер затрат на восстановительный ремонт ТС с учётом износа составляет  $ 421 238 $ (Четыреста двадцать один рубль двести тридцать восемь) рублей, или с учётом округления составляет $ 420 000 $ (Четыреста двадцать тысяч) рублей.
      
% \input {sections/рынокОСАГО}
%\subsection{Расчёт утраты товарной стоимости ТС}

\par В целях обеспечения единства практики применения судами законодательства, регулирующего отношения в области обязательного страхования гражданской ответственности владельцев транспортных средств, Пленум Верховного Суда Российской Федерации, руководствуясь статьей 126 Конституции Российской Федерации, статьями 2 и 5 Федерального конституционного закона от 5 февраля 2014 года N 3-ФКЗ "О Верховном Суде Российской Федерации", постановляет дать следующие разъяснениz
Общие положения Постановление Пленума Верховного Суда Российской Федерации от 26 декабря 2017 г. N 58 г. Москва "О применении судами законодательства об обязательном страховании гражданской ответственности владельцев транспортных средств" 

п. 37. К реальному ущербу, возникшему в результате дорожно-транспортного происшествия, наряду со стоимостью ремонта и запасных частей относится также утрата товарной стоимости, которая представляет собой уменьшение стоимости транспортного средства, вызванное преждевременным ухудшением товарного (внешнего) вида транспортного средства и его эксплуатационных качеств в результате снижения прочности и долговечности отдельных деталей, узлов и агрегатов, соединений и защитных покрытий вследствие дорожно-транспортного происшествия и последующего ремонта.

 Утрата товарной стоимости подлежит возмещению и в случае, если страховое возмещение осуществляется в рамках договора обязательного страхования в форме организации и (или) оплаты восстановительного ремонта повреждённого транспортного средства на станции технического обслуживания, с которой у страховщика заключён договор о ремонте транспортного средства, в установленном законом пределе страховой суммы.

\par Расчёт утраты товарной стоимости в настоящем исследовании производится согласно \emph {Методическим рекомендациям по проведению судебных автотехнических экспертиз и исследований колёсных транспортных средств в целях определения размера ущерба, стоимости восстановительного ремонта и оценки}  [6].

\par Утрата товарной стоимости (УТС) обусловлена снижением товарной стоимости из-за ухудшения потребительских свойств вследствие наличия дефектов (повреждений), или следов их устранения либо наличия достоверной информации, что дефекты (повреждения) устранялись [6, п. 8].

	УТС может быть рассчитана для ТС, находящихся как в повреждённом, так и в отремонтированном состоянии (при возможности установить степень повреждения).

УТС может определяться при необходимости выполнения одного из нижеперечисленных видов ремонтных воздействий или если установлено их выполнение:

-	устранение перекоса кузова или рамы ТС;

-	замена несъемных элементов кузова ТС (полная или частичная); ремонт съёмных или несъемных элементов кузова (включая оперение) ТС (в том числе пластиковых капота, крыльев, дверей, крышки багажника);

-	полная или частичная окраска наружных (лицевых) поверхностей кузова (включая оперение) ТС, бамперов;

-	полная или частичная разборка салона ТС, вызывающая нарушение качества заводской сборки.

УТС не рассчитывается:

а)	если срок эксплуатации легковых автомобилей превышает 5 лет;

б)	если легковые автомобили эксплуатируются в интенсивном режиме, а срок эксплуатации превышает 2,5 года;


в)	в случае замены кузова до оцениваемых повреждений (за исключением кузова грузового КТС, установленного на раме за кабиной);

%)	если КТС ранее подвергалось восстановительному ремонту (в том числе окраске - полной, наружной, частичной; «пятном с переходом») или имело аварийные повреждения, кроме повреждений, указанных в [6, п. 8.4];

д)	если КТС имело коррозионные повреждения кузова или кабины на момент происшествия.



Нижеприведенные повреждения не требуют расчёта УТС вследствие исследуемого происшествия, а их наличие до исследуемого происшествия не обуславливает отказ от расчёта УТС при таких повреждениях:

а)	эксплуатационных повреждениях ЛКП в виде меления, трещин, а также повреждений, вызванных механическими воздействиями - незначительных по площади сколов, рисок, не нарушающих защитных функций ЛКП составных частей оперения;

б)	одиночного эксплуатационного повреждения оперения кузова (кабины) в виде простой деформации, не требующего окраски, площадью не более 0,25 дм2;

в)	повреждения, которые приводят к замене отдельных составных частей, которые не нуждаются в окрашивании и не ухудшают внешний вид КТС (стекло, фары, бампера неокрашиваемые, пневматические шины, колёсные диски, внешняя и внутренняя фурнитура и т. п.). Если, кроме указанных составных частей, повреждены составные части кузова, рамы, кабины или детали оперения - крылья съёмные, капот, двери, крышка багажника, - то расчёт величины УТС должен учитывать все повреждения составных частей в комплексе;

г)	в случае окраски молдингов, облицовок, накладок, ручек, корпусов зеркал и других мелких наружных элементов, колёсных дисков.

В случае исследуемого события для автомобиля \тс\, VIN \vin\,  условия,  при которых производится расчёт УТС выполняются.\\


\par Величина УТС зависит от вида, характера и объёма повреждений и ремонтных воздействий по их устранению.
\par Величина УТС ($ C_\text{YTC} $) определяется на дату оценки (исследования) по формуле: 

\begin{equation}\label{uts}
C_{YTC} = C_{TC} \cdot \dfrac{\sum\limits_{i=1}^n K_{YTCi}}{100\%},\hspace{5mm} \text{руб.},
\end{equation}

\noindent где:\\
\noindent $ C_{TC} $ -- стоимость ТС на дату оценки (исследования), руб;\\
$ K_{YTCi} $ -- коэффициент УТС по i-му элементу КТС, ремонтному воздействию, \%.
 
\par Рыночная стоимость транспортного средства ( $ C_{TC} $ ), согласно п.6.1. Единой методики [3], принимается равной средней стоимости аналога на указанную
дату по данным имеющихся инфор\-мационно-справочных материалов,
содержащих сведения о средней стоимости транспортного средства.

\par  При ремонте съёмной составной части сумма стоимости ремонта (включая стоимость разборки для ремонта и при необходимости снятия детали для ремонта) и величины УТС (без учёта УТС вследствие окраски) не должна превышать суммы стоимости этой составной части (с учётом коэффициента износа) и стоимости работ по ее замене.

\par   Значение коэффициента УТС $ K_{\text{утсокр}} $ при подетальной окраске наружных поверхностей кузова ТС рассчитывается с учётом количества окрашиваемых кузовных составных частей и бамперов по формуле:

\begin{equation}\label{f:yc}
K_{\text{утсокр}}=K_{\text{утсокр(1)}}+K_{\text{утсокр(N-1)}}\cdot(N-1), \hspace{5mm} \% 
\end{equation}
        
\noindent где:\\
\noindent $ \text{К}_{\text{утсокр(1)}} $ - коэффициент УТС по окраске первой кузовной составной части или бампера, \%;\\
$ \text{К}_{\text{утсокр(N-1)}} $ - коэффициент УТС по окраске второй и каждой следующей кузовной составной части или бампера, \%;\\
N - количество окрашиваемых составных частей, по которым рассчитывается УТС.\\
Значения коэффициентов УТС ($ K_{YTC} $) определены по результатам экспертной практики и приведены в приложении [6, Приложение 2.9].

\par Для исследуемого автомобиля \тс \, соответствующие ремонтным воздействиям  коэффициенты УТС приведены ниже в таблице:

\begin{table}[H]
		%\caption{}
	\begin{tabular}{|p{5mm}|p{80mm}|c|c|c|}
	\hline 
	\textbf{п/п} & \textbf{Наименование детали} &\textbf{ К-замена }& \textbf{К-ремонт }&\textbf{ К-окраска} \\ 
	\hline 
	1 & Дверь задняя левая & -- & -- & 0,5 \\ 
	\hline 
%	2 & Бампер задний & -- & -- & 0,35 \\ 
%	\hline 
	2 & Боковина левая & -- & 0,2 & 0,35 \\ 
	\hline 
		3 & Порог левой боковины  & -- & 0,2 & 0,35 \\ 
	\hline 

	
\end{tabular} 

\end{table}

\vspace{7mm}

$  \sum\limits_{i=1}^n K_{YTCi} = 0.5+0.2+0.35+0.2+0.35 = 1.6$\\
  
  
Рыночная стоимость транспортного средства \тс\, на момент повреждения по данным справочника \url { https://automama.ru/krasnodar/cars/ } составляет 660 000 (Шестьсот шестьдесят тысяч) рублей.
  
$   C_{TC} = C_{TC} \cdot \dfrac{\sum\limits_{i=1}^n K_{YTCi}}{100} = 660000 \cdot 1.6/100 = 10560 $%, или с учетом округления 372000 (Триста семьдесят две тысячи) рублей.\\
(Десять тысяч пятьсот шестьдесят) рублей.

\par Таким образом, величина УТС автомобиля \тс\, составляет (Десять тысяч пятьсот шестьдесят) рублей.
%\subsection{Расчёт стоимости годных остатков}



\par \indent  Согласно пп. <<a>> п. 18 ст. 12 Федерального закона  N 40-ФЗ  <<Об обязательном страховании гражданской ответственности владельцев транспортных средств>>  в случаях, при которых ремонт повреждённого имущества невозможен либо стоимость ремонта равна стоимости имущества на дату наступления страхового случая или превышает указанную стоимость размер подлежащих возмещению страховщиком убытков при причинении вреда имуществу потерпевшего определяется в размере действительной стоимости имущества на день наступления страхового случая за вычетом стоимости годных остатков. 

В нашем случае,  стоимость ремонта ТС \тс\, регистрационный знак \грз\, превышает его рыночную стоимость. Следовательно, в порядке, предусмотренном  Главой 5 Положения Банка России от 19 сентября 2014 г. № 432-П <<О единой методике определения размера расходов на восстановительный ремонт в отношении повреждённого транспортного средства>>  производится расчёт стоимости годных остатков. 

\par Под годными остатками автотранспортного средства понимаются работоспособные, имеющие остаточную стоимость детали (агрегаты, узлы) повреждённого автотранспортного средства, как правило, годные к дальнейшей эксплуатации, которые можно демонтировать с повреждённого автотранспортного средства и реализовать. 
Годные остатки должны отвечать следующим условиям:

1) деталь (агрегат, узел) не должна иметь повреждений, нарушающих ее целостность и товарный вид, а агрегат (узел), кроме того, должен находиться в работоспособном состоянии;

2) деталь (агрегат, узел) не должна иметь изменений конструкции, формы, целостности и геометрии, не предусмотренных изготовителем автотранспортного средства (например, дополнительные отверстия и вырезы для крепления несерийного оборудования);

3) деталь не должна иметь следов предыдущих ремонтных воздействий (следов правки, рихтовки, следов шпатлевки, следов частичного ремонта и т.д.).



 Стоимость годных остатков с учётом затрат на их демонтаж, дефектовку, хранение и продажу определяется по формуле:
 \begin{equation}\label{go}
C_{\text{ГО}}= C_{\text{Р}} \cdot K_{\text{В}}\cdot K_{\text{З}}\cdot K_{\text{ОП}} \cdot  \sum\limits_{i-1}^{n}\frac{C_i}{100}, \, \, \text{руб} 
\end{equation}
\noindent где: \,$ C_{\text{Р}} $ -- стоимость ТС в неповрежденном виде на момент происшествия;\\
$ K_{\text{З}} $-- коэффициент, учитывающий затраты на дефектовку, разборку, хранение, продажу;\\
$ K_{\text{В}} $ -- коэффициент, учитывающий срок эксплуатации АМТС на момент повреждения и спрос на его неповреждённые детали;\\
$ K_{\text{ОП}} $ -- коэффициент, учитывающий объем (степень) механических повреждений автомобиля;\\
$ C_i $ процентное соотношение (вес) стоимости неповреждённых элементов к стоимости автомобиля;\\
$ n  $- количество неповреждённых элементов (агрегатов, узлов).\\

Расчёт процентного соотношения (веса) стоимости неповреждённых элементов к стоимости ТС   \,\,
     % \begin{equation}\label{bb}
   $  \left( \sum\limits_{i-1}^{n}\frac{C_i}{100} \right)  $  
%   \end{equation}  
включает только установленные неповреждённые детали, узлы и агрегаты. Компоненты ТС, имеющие повреждения  вероятностного характера, и требующие диагностических работ для установления годности в расчёте не учитываются. 
 
  \begin{longtable}{|p{9cm}|p{4cm}|p{2cm}|}
 	\caption[]{\footnotesize {Таблица расчёта $ C_i $ }}
 	 \label{tab:7}\\
 	 \hline
 	 		Наименование агрегата, узла, детали & \%-ное соотношение (вес)  & Годные, \% \\
 	 		\hline \endhead
 		Кузовные детали, экстерьер, интерьер, в т.ч.: & 50 (45 \textless{}1\textgreater{}) & 0 \\
 		Передняя часть: & 14 &  \\
 		Капот & 1.9 & 1,9 \\
 		Крыло переднее (за 1 шт.) & 0.8 & 0,8 \\
 		Бампер передний (в сборе с усилителем, накладками и молдингами, спойлером) & 1.9 & 1,9 \\
 		Решетка (облицовка) радиатора & 0.8 & 0,8 \\
 		Лонжерон передний (за 1 шт.) & 0.8 & 0,8 \\
 		Брызговик крыла (за 1 шт.) & 1.4 & 1,4 \\
 		Стекло ветрового окна & 1.7 & 1,7 \\
 		Рамка радиатора & 1.4 & 1,4 \\
 		Щиток передка & 0.3 & 0,3 \\
 		Задняя часть: & 12 (14 \textless{}1\textgreater{}) & 0 \\
 		Бампер задний & 1.6 & 0 \\
 		Крыло заднее (боковина \textless{}1\textgreater{}) в сборе с арками (за 1 шт.) & 2.1 (3.1 \textless{}1\textgreater{}) & 0 \\
 		Стекло окна задка & 1.9 & 0 \\
 		Панель задка & 0.8 & 0 \\
 		Пол багажника & 0.8 & 0 \\
 		Облицовки багажника & 1.1 & 0 \\
 		Крышка багажника (дверь задка) & 1.6 & 0 \\
 		Средняя часть: & 24 (17 \textless{}1\textgreater{}) & 0 \\
 		Передняя стойка боковины (за 1 шт.) & 1.4 & 2,8 \\
 		Средняя стойка боковины с порогом и частью пола (за 1 шт.) & 1.4 (0 \textless{}1\textgreater{}) & 2,8 \\
 		Облицовки стоек боковины, порогов, уплотнители, центральная консоль, противосолнечные козырьки, плафоны освещения, коврики пола, зеркало заднего вида & 2.5 (2.1 \textless{}1\textgreater{}) & 2,5 \\
 		Двери в сборе с арматурой (за 1 шт.), & 1.9 & 5,7 \\
 		в т.ч. арматура дверей (за 1    дверной комплект) & 0.5 & 0 \\
 		Сиденья (все) & 1.1 & 1,1 \\
 		Панель крыши в сб. с обивкой, поперечинами и верх. частями стоек, & 3.5 & 2,7 \\
 		в т.ч. обивка панели крыши & 0.8 & 0 \\
 		Панель приборов в сборе с щитком приборов, решётками, вещевым ящиком, карманами и т.д. & 2.5 & 2,5 \\
 		Ремень безопасности передний (за 1 шт.) & 0.3 & 0,6 \\
 		Подушка безопасности пассажирская & 0.6 & 0,6 \\
 		Двигатель, навесное, охлаждение, впускная и выпускная система & 11 (13 \textless{}2\textgreater{}) & 13 \\
 		Двигатель в сборе без навесного оборудования & 4.9 & 0 \\
 		в т.ч. клапанная крышка & 0.5 & 0 \\
 		в т.ч. масляный поддон & 0.5 & 0 \\
 		в т.ч. блок цилиндров & 2.2 & 0 \\
 		Дроссельный узел в сборе с заслонкой, клапаном и датчиком & 1.4 & 0 \\
 		Генератор & 0.8 & 0 \\
 		Коллектор впускной & 0.5 & 0 \\
 		Коллектор выпускной & 0.5 & 0 \\
 		Радиатор охлаждения в сборе с кожухами, вентилятором & 0.8 & 0 \\
 		Стартер & 0.5 & 0 \\
 		Короб воздушного фильтра с патрубками & 0.5 & 0 \\
 		Выпускной тракт в сборе & 0.8 & 0 \\
 		Турбокомпрессор (турбонагнетатель) & 1.4 \textless{}2\textgreater{} & 0 \\
 		Интеркулер & 0.6 \textless{}2\textgreater{} & 0 \\
 		Топливная система & 2.5 & 2,5 \\
 		Бак топливный & 0.7 & 0 \\
 		Система подачи топлива & 1.8 & 0 \\
 		Трансмиссия & 4.5 & 4,5 \\
 		Усреднённый показатель с учётом всех возможных вариантов трансмиссии & 4.5 & 0 \\
 		Подвеска & 10 & 4 \\
 		Подвеска передняя в сборе с поперечиной & 5.5 (4.5 \textless{}4\textgreater{}) & 0 \\
 		Подвеска задняя в сборе с поперечиной & 4.5 (5.5 \textless{}4\textgreater{}) & 4,5 \\
 		Подвеска в сборе для полноприводных АМТС & 10 (5 \textless{}4\textgreater + 5 \textless{}4\textgreater{}) & 0 \\
 		Рулевое управление & 3 & 3 \\
 		Рулевая колонка в сборе с валом & 0.5 & 0 \\
 		Насос ГУР & 0.8 & 0 \\
 		Рулевой механизм & 1.2 & 0 \\
 		Рулевое колесо в сборе с подушкой безопасности & 0.5 & 0 \\
 		в т.ч.: подушка безопасности  водительская & 0.3 & 0 \\
 		Тормозная система & 3.5 & 3,5 \\
 		Главный тормозной цилиндр & 0.5 & 0 \\
 		Тормозной механизм колеса (за каждый колёсный узел) & 0.5 & 0 \\
 		Ручной (ножной) тормоз & 0.3 & 0 \\
 		Блок управления АБС & 0.7 & 0 \\
 		Электрооборудование & 12.5 & 0 \\
 		Провода свечные с катушками (комплект) & 0.5 & 0,5 \\
 		Монтажный блок & 0.5 & 0,5 \\
 		Блок управления двигателем & 1 & 1 \\
 		Фонари задние (за 1 шт.) & 0.5 & 1 \\
 		Зеркала заднего вида боковые (за 1 шт.) & 0.8 & 1,6 \\
 		Блок отопителя салона в сборе (корпус, двигатель, радиаторы) & 2.1 & 2,1 \\
 		Насос кондиционера & 0.5 & 0,5 \\
 		Конденсатор в сборе с осушителем, кожухом, вентилятором, трубками & 0.6 & 0,6 \\
 		Фары (за 1 шт.) & 1.1 & 1,1 \\
 		Жгут проводов ДВС & 0.9 & 0,9 \\
 		Жгут проводов панели приборов & 0.8 & 0,8 \\
 		Остальные жгуты проводов (все) & 0.3 & 0,3 \\
 		Фара противотуманная (за 1 шт.) & 0.8 & 1,6 \\ 
 		Прочее & 3/8 \textless{}1\textgreater{}/1 \textless{}2\textgreater /6 \textless{}3\textgreater{} & 4 \\
 		\hline
 		\textbf{ИТОГО,} \%: &  & \textbf{83,8}  \\
 		\hline	
 	\end{longtable}
 
\noindent  \begin{table}[H]
  	 \label{tab:KB}
	\caption{\footnotesize {Значения коэффициента Кв, учитывающего срок эксплуатации ТС}}
		 \begin{tabular}{|p{47mm} |p{53mm}| p{50mm}|}
	\hline
 		Срок эксплуатации автомобиля, лет & Значение Кв легковых автомобилей, малотоннажных грузовых на базе легковых и мототехники & Значение Кв грузовых автомобилей \\ \hline
 		0 - 5 (включительно)   &0.80                                                                                    & 0.80                             \\ \hline
 		6 - 10 (включительно)             & 0.65                                                                                    & 0.60                             \\ \hline
 		11 - 15 (включительно)            & 0.55                                                                                    & 0.50                             \\ \hline
 		16 - 20 (включительно)            & 0.40                                                                                    & 0.35                             \\ \hline
 		Более 20 лет                      & 0.35                                                                                    & 0.30                            \\ \hline
 	\end{tabular}
\end{table}


\noindent \begin{table}[H]
	\label{tab:KO}
	\caption{\footnotesize {Значение коэффициента $ K_{\text{оп}} $ , учитывающего объем (степень) механических повреждений автомобиля}} 
	\centering
\begin{tabular}{|p{47mm}| p{53mm}| p{50mm}|}
	\hline
	Объем механических повреждений & Соотношение стоимости неповреждённых элементов к стоимости автомобиля, Ci, \% & Значение коэффициента, учитывающего объем механических повреждений \\ \hline
	Незначительный     & 80 -    & 0.9 -  \\
	& 60 - 80      & 0.8 - 0.9      \\
	Средний    & 40 -     & 0.7 -    \\
	& 20 - 40     & 0.6 - 0.7         \\
	Значительный                   & 0 - 20                                                                        & 0.5 - 0.6                                                          \\ \hline
\end{tabular}
\end{table}

 \par
 
% \begin{equation}\label{k}
% C_{\text{ГО}}= C_{\text{Р}} \cdot K_{\text{В}}\cdot K_{\text{З}}\cdot K_{\text{ОП}} \cdot  \sum\limits_{i-1}^{n}\frac{C_i}{100} 
% \end{equation}
 
 Для исследуемого транспортного средства применимы следующие значения коэффициентов:\par
 $ K_{\text{В}} =0.8  $; 
 $ K_{\text{З}} =0.7 $; 
 $ K_{\text{ОП}} =0.8 $; $ \sum\limits_{i-1}^{n}\frac{C_i}{100} = 0.838 $ \par тогда:
 \par
$  C_{\text{ГО}} =  C_{\text{ГО}}= C_{\text{Р}} \cdot K_{\text{В}}\cdot K_{\text{З}}\cdot K_{\text{ОП}} \cdot  \sum\limits_{i-1}^{n}\frac{C_i}{100} =980000*0.8*0.7*0.8*0.838 = 367915 $ руб., или с учётом округления 370 000 (Триста семьдесят тысяч) рублей.
\par Таким образом, стоимость годных остатков ТС \тс \, \, составляет 370 000 (Триста семьдесят тысяч) рублей.



\section{В ы в о д ы}


    1) Наличие, характер и объем (степень) технических повреждений, причинённых ТС, определены при осмотре и зафиксированы в Акте осмотра № \NomerDoc\, и фототаблице повреждений, являющимися неотъемлемой частью настоящего экспертного заключения.\\[3mm]
    
    2) Направление, расположение и характер повреждений определены путём сопоставления полученных повреждений, изучения административных материалов по рассматриваемому событию, и  являются  следствиями рассматриваемого ДТП (события).\\[3mm]
    
    3) Технология и объем необходимых ремонтных воздействий зафиксированы в калькуляции № \NomerDoc\, по определению стоимости восстановительного ремонта транспортного средства \tc\, VIN  \vin. \\[3mm]
    
    4)  Стоимость восстановительного ремонта  транспортного средства \tc\, регистрационный знак \грз,\, \, получившего механические повреждения в результате дорожно-транспортного происшествия, имевшего место \датадтп\, с участием транспортных средств \тс\, регистрационный знак \грз\, и \tcb\, составляет $ 609 000 $ (Шестьсот девять тысяч) рублей рублей.\\[3mm]
    
    5) Размер затрат на проведение восстановительного ремонта с учётом износа (восстановительные расходы) транспортного средства \tc\, регистрационный знак \grz\, составляет $ 420 000 $ (Четыреста двадцать тысяч) рублей.\\[3mm]
    
    
%    6) Стоимость годных остатков ТС \тс\, регистрационный знак \грз\, оставляет  $ 10\,560$ (Десять тысяч пятьсот шестьдесят) рублей.
    
%    6) Величина утраты товарной стоимости транспортного средства \тс\,  регистрационный знак \грз\, составляет  
    
\vspace{15mm}

 \noindent Эксперт-техник   \hfill        Мраморнов А.В.
 
\vspace{3mm}
\noindent   \textit{  Государственный  реестровый номер эксперта-техника:   256}\\

\vspace{6mm}

\relax
\noindent Приложение к заключению:\\
\textit{
%	Приложение № 1. Расшифровка модельных опций ТС \тс \\
    Приложение № 1. Акт осмотра ТС \тс\\
    Приложение № 2. Фототаблица повреждений ТС\\
	Приложение № 3. Калькуляция стоимости восстановительных расходов ТС \тс\\
	Приложение № 4. Цифровые копии регистрационных документов ТС\\
	Приложение № 5. Цифровая копия постановления по делу об административном правонарушении дорожно-транспортном происшествии\\
	Приложение № 6. Правоустанавливающие документы эксперта-техника\\
}

%\includepdf[pages=-]{myfile.pdf}
%\includepdf[pages=-]{calc.pdf}
%
\DTLloaddb{stores}{remont.csv}

\begin{landscape}
\DTLdisplaydb{stores}
\end{landscape}



%\Ini[1]{yy} 
%\Ini{zz} 
%\begin{table}[!th] 
%	\begin{tabular}{|N{c}|X{xx}{c}|*{xx}{c}|S{xx}{zz}{c}|*{yy}{c}|} 
%		\hline 
%		\bf \NumIni \No{} п/п & A & B & A$\cdot$B\NoNum& C \\ 
%		\hline 
%		& -3.5 & 4.4 &&43.3 \\ 
%		& 31.31 &200.21 &&3 \\ 
%		& 1.23 &3.33 &&1.2 \\ 
%		\hline 
%		\NoNum&&&\NoNum $\sum(A\cdot B)=\,$\Show{zz}&$\prod C=\,$\Show{yy} \\ 
%		\hline 
%	\end{tabular} 
%\end{table} 
%


%
%
%\csvautotabular{test.csv}
%
%
%
%\begin{tabular}{|P{50mm}|P{50mm}|}\hline%
%	\bfseries Персоны  & \bfseries Индекс%
%	\csvreader[head to column names]{test.csv}{}%
%%	
%	{\\\givenname\  \name & \matriculation \,  \matriculation}%
%%	
%	\\\hline
%%	
%\end{tabular}
%
%Или
%
%\begin{tabular}{|r|l|c|}\hline%
%п/п	& Заголовок любой & Любой\\\hline\hline
%	%
%	\csvreader[late after line=\\\hline]%
%	%
%	{test.csv}{name=\name,givenname=\firstname,matriculation=\matnumber}%
%	%
%	{\thecsvrow & \firstname~\name & \matnumber}%
%\end{tabular}
%
%Еще более удобный и предпочтительный способ создания таблицы - это установка соответствующих клавиш параметров. Обратите внимание, что это дает вам возможность создать стиль pgfkeys, который содержит создание всей таблицы.
%
%\csvreader[tabular=|r|l|c|,
%table head=\hline & Таже таблица & но записана по другму\\\hline\hline,
%late after line=\\\hline]%
%%
%{test.csv}{name=\name,givenname=\firstname,matriculation=\matnumber}%
%%
%{\thecsvrow & \firstname~\name & \matnumber}%
%
%
%Еще один способ
%
%\csvstyle{myTableStyle}{tabular=|r|l|c|,
%	table head=\hline & Person & Matr.~No.\\\hline\hline,
%	late after line=\\\hline,
%	head to column names}
%
%\csvreader[myTableStyle]{test.csv}{}%
%{\thecsvrow & \givenname~\name & \matriculation}%
%
%
%
%
%Другой способ адресации столбцов - использовать их римские номера. Прямая адресация осуществляется с помощью \ csvcoli, \ csvcolii, \ csvcoliii,. , , :
%
%\csvreader[tabular=|r|l|c|,
%table head=\hline & Person & Matr.~No.\\\hline\hline,
%late after line=\\\hline]%
%{test.csv}{}%
%{\thecsvrow & \csvcolii~\csvcoli & \csvcoliii}%
%
%
%И еще один способ присвоения макросов столбцам - использовать арабские числа для назначения:
%
%\csvreader[tabular=|r|l|c|,
%table head=\hline & Person & Matr.~No.\\\hline\hline,
%late after line=\\\hline]%
%{test.csv}{1=\name,2=\firstname,3=\matnumber}%
%{\thecsvrow & \firstname~\name & \matnumber}%
%
%Для повторяющихся приложений синтаксис pgfkeys позволяет создавать собственные стили для согласованного и централизованного дизайна. Следующий пример легко модифицируется для получения более или менее настроек параметров.
%
%\csvset{myStudentList/.style={%
%		tabular=|r|l|c|,
%		table head=\hline & Person & #1\\\hline\hline,
%		late after line=\\\hline,
%		column names={name=\name,givenname=\firstname}
%}}
%\csvreader[myStudentList={Matr.~No.}]{test.csv}{matriculation=\matnumber}%
%{\thecsvrow & \firstname~\name & \matnumber}%
%\hfill%
%\csvreader[myStudentList={Grade}]{test.csv}{grade=\grade}%
%{\thecsvrow & \firstname~\name & \grade}%
%
%
%Кроме того, имена столбцов могут быть установлены с помощью \ csvnames → стр. 11, а определения стилей - с помощью \ csvstyle → стр. 11. При этом последний пример переписывается следующим образом:
%
%\csvnames{myNames}{1=\name,2=\firstname,3=\matnumber,5=\grade}
%\csvstyle{myStudentList}{tabular=|r|l|c|,
%	table head=\hline & Person & #1\\\hline\hline,
%	late after line=\\\hline, myNames}
%\csvreader[myStudentList={Matr.~No.}]{test.csv}{}%
%{\thecsvrow & \firstname~\name & \matnumber}%
%\hfill%
%\csvreader[myStudentList={Grade}]{test.csv}{}%
%{\thecsvrow & \firstname~\name & \grade}%
%
%Строки данных файла CSV также могут быть отфильтрованы. В следующем примере сертификат печатается только для учащихся с оценкой, не равной 5,0.
%
%\csvreader[filter not strcmp={\grade}{5.0}]%
%{test.csv}{1=\name,2=\firstname,3=\matnumber,4=\gender,5=\grade}%
%{\begin{center}\Large\bfseries Certificate in Mathematics\end{center}
%	\large\ifcsvstrcmp{\gender}{f}{Ms.}{Mr.}
%	\firstname~\name, matriculation number \matnumber, has passed the test
%	in mathematics with grade \grade.\par\ldots\par
%}%


%
\documentclass[a4paper,10pt]{report}
\usepackage{style/stylefile}


\setmainfont[Ligatures={TeX,Historic}]{Arial}


\begin{document}
    
%%%%%%%%%%%%%%%%%%%%%%%%%%%%%%%%%%%%%%%%%%%%%%%%
%
%    Титульные данные
%
%%%%%%%%%%%%%%%%%%%%%%%%%%%%%%%%%%%%%%%%%%%%%%%%
\newcommand{\NomerDoc}{27-03/2020}  % номер заключения
%
\newcommand{\dog}{25.02.2020} % Дата договора
%
\newcommand{\datastart}{25.02.2020} % Дата начала исследования
%	
%\newcommand{\datadtp}{\ldots}  % Дата ДТП        
%         
\newcommand{\osm}{25.02.2020} % ОСМОТР % Дата осмотра
%
\newcommand{\dataend}{25.03.2020} % Дата окончания
%
%\newcommand{\datastop}{\ldots}  % Дата приостановления
%\newcommand{\datarestart}{\ldots}
%
%%%%%%%%%%%   ДОКУМЕНТЫ  
%  
% Свидетельство о регистрации ТС
\newcommand{\свид}{ВК № 384784 }     
% Паспорт транспорного средства
\newcommand{\птс}{ТС № 001742 }     
\newcommand{\владелец}{\ldots}
\newcommand{\адресвладельца}{\ldots}
\newcommand{\заказчик}{ООО «Логистик-Юг», в лице генерального 
    директора Тарасова Андрея Ивановича}
\newcommand{\адресзаказчика}{г. Краснодар, ул. Стасова, д.178/1, офис 81}
% Страховой полис
\newcommand{\polis}{\ldots}  
%  Протокол
\newcommand{\prt}{\ldots}                        
 % Постановление
\newcommand{\постановление}{\ldots}
 % Определение
\newcommand{\определение}{\ldots}
% Повреждения
\newcommand{\pov}{\ldots} %Перечень повреждений  
% Присутствовали   
\newcommand{\присутствовали}{\ldots}
% Место осмотра
\newcommand{\местоосмотра}{г. Краснодар, хутор Ленина, МФТ-1, отделение 4, литер Г- 46}
%
%%%%%%%%%%%  ТРАНСПОРТНОЕ СРЕДСТВО
%
\newcommand{\tc}{KOMATSU FG15T-20}   % Транспортное сердство
\newcommand{\grz}{\ldots} % Регистрационный знак
\newcommand{\vin}{\ldots}  % VIN
%
\newcommand{\типкузова}{\ldots}
\newcommand{\двигатель}{К15-019305Х} 
%
\newcommand{\colr}{\ldots}  % Какого  цвета кузов
\newcommand{\цвет}{\ldots}  
\newcommand{\типлкп}{\ldots} 
%
\newcommand{\пробег}{\ldots}
\newcommand{\год}{2007}          % Год выпуска
\newcommand{\началоэкспл}{\ldots}
\newcommand{\датаизготовления}{\ldots}


%%%%%%%%% Другие транспортные средства

%
%\newcommand{\tca}{\ldots}
%\newcommand{\тса}{\ldots}
%%
\newcommand{\tcb}{\ldots}
%\newcommand{\тсв}{\cdod}



 
%
%%%%%%%%%%%%%%%%%%%%%                   % Если судебка 
%
\newcommand{\delosud}{гражданское дело 2-1033/19} % 

\newcommand{\delonum}{2-1033/19}  % номер дела

\newcommand{\opr}{дополнительной судебной автотехнической экспертизы}

\newcommand{\sud}{мирового судьи судебного участка № 11 г. Белореченска Краснодарского края Мозер Г.Л.}

\newcommand{\dataopr}{14.10.2019}

\newcommand{\isk}{Ильенко Р.В. к Шамояну Р.О. и САО "ВСК" о взыскании страхового возмещения}

%
\newcommand{\hod}{\ldots} % № ходатайство
%\newcommand{\hod2}{№ 3567 от 01.01.2008} % № ходатайство
%%%%%%%%%%%%%%%%%%%%%%%%%
% РЕЦЕНЗИЯ
\newcommand{\чел}{\ldots}
\newcommand{\назакл}{\ldots}

% example-image

\renewcommand{\chaptername}{Заключение эксперта}
\renewcommand{\refname}{Список}
\renewcommand{\bibname}{\large {Использованные нормативы и источники информации}}

\renewcommand{\epsilon}{\ensuremath{\varepsilon}}
\renewcommand{\phi}{\ensuremath{\varphi}}
\renewcommand{\kappa}{\ensuremath{\varkappa}}
\renewcommand{\le}{\ensuremath{\leqslant}}
\renewcommand{\leq}{\ensuremath{\leqslant}}
\renewcommand{\ge}{\ensuremath{\geqslant}}
\renewcommand{\geq}{\ensuremath{\geqslant}}
\renewcommand{\emptyset}{\varnothing}

%%%%%%%%%%%%%%%%%%%%%%%%%%%%%%%%%   ПРОИЗВОЛЬНЫЙ СЧЕТЧИК

\newcounter{@nnn}  % задаём имя счёчика 
\setcounter{@nnn}{0}  % устанавливаем его первое значение

\newcommand{\z}[2]{\par\addtocounter{@nnn}{1}  % формируем комманду 
	{\bf \arabic{@nnn}.   Работы по заказ-наряду  #1, произведеные на автомобиле #2:}}


%%%%%%%%%%%%%%%%%%%%%%%  Подсчет строк в таблице
\newcounter{rownum}
\setcounter{rownum}{0}
\newcommand{\Rownum}{\stepcounter{rownum}%
\arabic{rownum}}

%\def\contentsname{Содержание}
%Аннотация  \abstractname
%Часть       \partname
%Глава        \chaptername
%Список литературы  \refname
%Рис.                \figurename
%Таблица           \tablename
%Литература       \bibname

%Предметный указатель  \indexname
%Приложение                \appendixname
%Содержание          \contentsname
%Список иллюстраций \listfigurename
%Список таблиц        \listtablename
%\addto\captionsrussian{\def\refname{Список используемой литературы}}

%%%%%%%%%%%%%%   Размещение изображений
%\textfloatsep — расстояние между флоатс (в верхней или нижней части страницы) и текстом (по умолчанию, около 20pt)
%\floatsep — вертикальное расстояние между двумя флоатс (около 12pt)
%\intextsep — расстояние между флоатс вставленным "прямо здесь" (параметр h) и текстом (около 12pt)
%\abovecaptionskip и \belowcaptionskip — расстояние над и под подписью к флоат
\setcounter{totalnumber}{10}
\setcounter{topnumber}{10}
\renewcommand{\topfraction}{1}
\renewcommand{\textfraction}{0}
%%%%%%  Больше плавающих объектов на страницу
 \setlength{\textfloatsep}{10pt plus 1.0pt minus 2.0pt}
 \setlength{\floatsep}{5pt plus 1.0pt minus 1.0pt}
 \setlength{\intextsep}{5pt plus 1.0pt minus 1.0pt}

%%%%%%%%%%%%%%%%%%%%%%%%%%%%%%%%%%%%%%%%%%%%%%%%%%%%%%%%%%%%%%%
%
%   Заметка на полях  (ремарка)
%
%%%%%%%%%%%%%%%%%%%%%%%%%%%%%%%%%%%%%%%%%%%%%%%%%%%%%%%%%%%%%%%%
\newcommand{\rem}[1]
{
\marginpar{\scriptsize\textcolor{red}{#1}}
}
\newcommand{\рем}[1]
{
	\marginpar{\scriptsize\textcolor{red}{#1}}
}

%%%%%%%%%%%%%%%%%%%%%%%%%%%%%%%%%%%%%%%%%%%%%%%%%%%% ПЕРЕОПРЕДЕЛЕНИЕ ФОРМАТИРОВАНИЯ ЯЧЕЕК ТАБЛИЦЫ%%%%%%%%%%
%
\newcolumntype{P}[1]{>{\centering\arraybackslash}p{#1}}   %  \centering   \raggedleft  \raggedright
\newcolumntype{M}[1]{>{\raggedright\arraybackslash}m{#1}} %
\newcolumntype{G}[1]{>{\centering\arraybackslash}m{#1}} %

%%%%%%%%%%%% ВСАВКА с масштабированием ИЗОБРАЖЕНИЯ 2х3  В ТАБЛИЦУ
\newcommand{\imt}[1]
{\includegraphics[width=35mm, height=23mm, keepaspectratio=false]{#1}}

%%%% Переопределение команды для
%  Её вызов — \imgh{45.25mm}{zb}{Пример}
%  Первый параметр — ширина
%  Второй параметр — название файла
%  Третий параметр — название подписи к изображению
\newcommand{\imgh}[3]
{
	\begin{figure}[hpt!]
		\center{\includegraphics[width=#1]{foto/#2}}
		\caption{\small {#3}}
		\label{ris:#2}
	\end{figure}
}


\newcommand{\imgroot}[4]
{
	\begin{figure}[hpt!]
		\center{\includegraphics[angle=#4,width=#1]{foto/#2}}
		\caption{\small {#3}}
		\label{ris:#2}
	\end{figure}
}

%%Собственный  простейший список без нумерации и с обычными межстрочными интервалами
\newenvironment{compactlist}{
    \begin{list}{{$\bullet$}}{
            \setlength\partopsep{0pt}
            \setlength\parskip{0pt}
            \setlength\parsep{0pt}
            \setlength\topsep{0pt}
            \setlength\itemsep{0pt}
            \setlength{\itemindent}{\leftmargin}
            \setlength{\leftmargin}{0pt}
        }
    }{
    \end{list}
}
%%%%%%%%%%%%%%%%%%%%%%%%%%%%%%%%%%%%
%%
%% ПЕРЕОПРЕДЕЛЕНИЕ ДЛЯ ЗАПИСИ СТРОКИ АКТА ОСМОТРА
%%

\newcommand{\акт}[4]{\Rownum  & {\small #1}& #2  & #3 & #4\\  \toprule}

%%%%%%%%%%%%%%%
%% Переопределение для ЗАКЛЮЧЕНИЯ. Таблица ввода повреждений  с фото

\newcommand{\пов}[2]{\Rownum  & {\small #1 }&  \imt{foto/#2}\\ \hline \toprule}

%%%%%%%%%%%%%%%%
%%%  Переопределение длятаблицы ИСТОРИИ РЕМОНТА и сервисного обслуживания

\newcommand{\ист}[5]{#1 & #2  & #2 & #4   & #5 \\ \hline}

%%%%%%%%%%%%%%%%%%%%%%%%%%%%%%%%%%%
%%% ПЕРЕОПРЕДЕЛЕНИЕ ДЛЯ ТАБЛИЦЫ с Игдексом и Двумя Столбцами

\newcommand{\два}[2]{\small \Rownum  & {\small #1 }&  \small #2\\ \hline \toprule}

%%% ПЕРЕОПРЕДЕЛЕНИЕ ДЛЯ ТАБЛИЦЫ с Индексом и Пятью Столбцами

\newcommand{\пять}[5]{\small \Rownum  & \small #1 &  \small #2&\small #3&\small #4&\small #5\\ \hline \toprule}

%%%%%%%%%%%%%%%%%%%%%%%%%%%%%%%%%%%
\newcommand{\dee}{
	% вертикальные промежутки:
	\topsep=0pt % вокруг списка
	\parsep=0pt % между абзацами
	\itemsep=0pt % между пунктами % горизонтальные промежутки: \itemindent=0pt % абзацный выступ
	\labelsep=1ex % расстояние до метки
	\leftmargin=\parindent % отступ слева
	\rightmargin=0pt} % отступ справа
%%

%%%%%%%%%%%% Нумерованный список
\newcommand{\be}{\begin{enumerate}}
\newcommand{\en}{\end{enumerate}}

%%%% Вставить цитату
\newcommand{\цитата}[1]
{
	\begin{quote}
		\textcolor{gray}{#1}
	\end{quote}
}

\newcommand{\блеклый}[1]
{\textcolor{gray}{#1}[0.7]}

\newcommand{\сноска}[1]{\footnote{#1}}

\newcommand{\икс}{$x$}
\newcommand{\игрек}{$y$}
\newcommand{\зет}{$z$}
\newcommand{\audaОСАГО}{Audatex AudaWeb, в модуле ОСАГО ПРО}
\newcommand{\auda}{Audatex AudaWeb}

%%%%%%%%%%%%%%%%%%%%%%%%%%% ЧЕК БОКСЫ
\newcommand{\cmark}{\ding{51}}%$\checkmark $
\newcommand{\xmark}{\ding{55}}%
\newcommand{\done}{{$\square$}{\hspace{-6.5pt}\cmark}}
\newcommand{\wontfix}{{$\square$}{\hspace{-6.5pt}\xmark}}
%%%%%%%%%%%%%%%%%%%%%%%%%%%%%%%%%%%%%%%%%%

\newcommand{\г}{$\checkmark $}
\newcommand{\7}{$\checkmark $}
\newcommand{\галка}{\ding{51}}
\newcommand{\х}{\ding{55}}
\newcommand{\градус}{\circ}
\newcommand{\чек}{$\square$}

\newcommand{\чекг}{\done}
\newcommand{\чекх}{\wontfix}


% Площадь пореждений, М2
\newcommand{\s}[1]{$S_{\text{повр}} \approx#1\, m^2$}

\newcommand{\угол}[1]{$ #1^\circ $}
%%%% Стиль для колонтитулов

\newcommand{\грз}{\grz}
\newcommand{\вин}{\vin}
\newcommand{\датадтп}{\datadtp}
\newcommand{\датадоговора}{\dog}
\newcommand{\начато}{\datastart}
\newcommand{\датаосмотра}{\osm}
\newcommand{\датазаключения}{\zkl}
\newcommand{\страховойполис}{\polis}
\newcommand{\протокол}{\pr}
\newcommand{\повреждения}{\pov}
\newcommand{\иск}{\isk}
\newcommand{\тс}{\tc}
\newcommand{\окончено}{\dataend}
\newcommand{\прибл}{$ \approx $}
\newcommand{\тса}{\tca}
\newcommand{\тсб}{\tcb}
\newcommand{\ссылка}{\ref}
%\newcommand{}{}

%%%%%%%%%%%%%%%% ПЕРЕОПРЕДЕЛЕНИЕ  "По вопросу"     \повопросу{вопрос}
\newcommand{\повопросу}[1]{\,{\renewcommand\baselinestretch{0.86}\small\normalsize
\subsection{\underline{По  вопросу}\,\,\textbf{\small{<<#1>>}}}}
\renewcommand\baselinestretch{1.2}\small\normalsize}
%%%%%%%%%%%%%%%%%%%%%%%%%%%%%%%%%%%%%%%%%%%%%


\newcommand{\фото}[2]
{
    \begin{figure}[H]
        \center{\includegraphics[width=0.99\textwidth]{#1}}
        \caption{\small {#2}}
        \label{рис:#1}
    \end{figure}
}



\newcommand{\фотоб}[2]
{
	\begin{figure}[H]
		\center{\includegraphics[width=0.99\textwidth]{#1}}
		\caption*{\small {#2}}
		\label{рис:#1}
	\end{figure}
}



\newcommand{\фот}[2]
{
	\begin{figure}[H]
		\center{\includegraphics[width=0.99\textwidth]{#1}}
		\caption{\small {#2}}
	%	\label{рис:#1}
	\end{figure}
}

%%%%%%%%%%%%%%% ДВА РИСУНКА РЯДОМ            \дварядом{файл1}{подпись1}{файл2}{подпись2}
\newcommand{\дварядом}[4]{\begin{figure}[H]\centering
        \parbox[t]{0.49\textwidth}
        {\centering
            \includegraphics[width=.49\textwidth,  height=.32\textwidth]{foto/#1}
            \caption{\footnotesize {#2}}
            \label{рис:#1}}
        \hfil \hfil
        \parbox[t]{0.49\textwidth}
        {\centering
            \includegraphics[width=.49\textwidth, height=.32\textwidth]{foto/#3}
            \caption{\footnotesize {#4}}
            \label{рис:#3}}

\end{figure}}

%%%%%%%%%%%%%%%%%%%%%%%%%%%%%%%%%%%
% Два рядом с одной общей подписью
%%%%%%%%%%%%%%%%%%%%%%%%%%%%%%%%%%%

\newcommand{\дварисунка}[5]{\begin{figure}[H]
	\begin{minipage}{0.49\textwidth}
		\includegraphics[width=\linewidth,  height=.64\linewidth]{foto/#1}
		\subcaption{#2}
	\end{minipage}
	\hfill
	\begin{minipage}{0.49\textwidth}
		\includegraphics[width=\linewidth,  height=.64\linewidth]{foto/#3}
		\subcaption{#4}
	\end{minipage}

	\caption{#5}
	\label{рис:#1}
:\end{figure}}




%%%% СТС две стороны рядом
\newcommand{\стс}[4]{\begin{figure}[H]
        \centering
        \parbox[t]{0.49\textwidth}
        {\centering
            \includegraphics[width=.49\textwidth]{foto/#1}
            \caption{\footnotesize {#2}}
            \label{рис:#1}}
        \hfil \hfil
        \parbox[t]{0.49\textwidth}
        {\centering
            \includegraphics[width=.49\textwidth]{foto/#3}
            \caption{\footnotesize {#4}}
            \label{рис:#3}}

\end{figure}}


%%%%% ФОТО РЯДОМ С ТЕКСТОМ
%
%\newcommand{\фотосправа}[2]{
%    \begin{SCfigure}
%        \centering {\footnotesize \caption{#2}
%            \includegraphics[width = 0.6 \textwidth]{foto/#1}
%            \label{рис:#1}
%    \end{SCfigure}}



%%%% Переопределение команды для
%  Её вызов — \фотомасштаб{45.25mm}{название файла}{подпись рисунка}

%  Первый параметр — название файла
%  Второй параметр — название подписи к изображению
%  Третий параметр — ширина
\newcommand{\фотомасштаб}[3]
{
    \begin{figure}[H]
        \center{\includegraphics[width=#3]{foto/#1}}
        \caption{\small{#2}}
        \label{рис:#1}
    \end{figure}
}



%  Её вызов — \фотоповорот{45.25mm}{название файла}{подпись рисунка}{угол поворота}
%  Первый параметр — ширина
%  Второй параметр — название файла
%  Третий параметр — название подписи к изображению
\newcommand{\фотоповорот}[4]
{
    \begin{figure}[hpt!]
        \center{\includegraphics[angle=#4,width=#1]{foto/#2}}
        \caption{\small {#3}}
        \label{рис:#2}
    \end{figure}
}

%%% ИЗМЕРИТЬ ШИРИНУ СТРАНИЦЫ
\newcommand{\ширина}{\the\textwidth\\
    \printinunitsof{mm}\prntlen{\textwidth}}

%_____________________________________________________
%

\begin{center}
    {\Large \textbf{АКТ ОСМОТРА   № }}\\
    \vspace{2mm}
    {\normalsize Транспортного средства \тс\, VIN \vin}
\end{center}

\vspace{2mm}
\noindent Дата и время осмотра:\\
\noindent Место осмотра:\\
\vspace{2mm}

\noindent\parbox[l][5mm]{60mm}{ -------xxxxxxxxXXXXXXXXXXXXXXX----- }   \hspace{50mm} %\hfill
\parbox[l][5mm]{50mm}{  -----}

\begin{longtable}{G{3mm}|M{130mm}|G{5mm}|G{5mm}|G{5mm}}
     \hline %установленные при его осмотре и соответствующие им ремонтные воздействия}}
     %%\label{tab:5}\\
    \hline
    \toprule 
    \bf  {\footnotesize  n/n}  &\bf {\small Наименование  детали и описание повреждения} & \bf {\small E} & \bf {\small I}& \bf {\small L}\\\hline\hline   \toprule  \endhead 
%%%%______________________________________%%%%%%%%%%%%
%%%%%%%%%   ОПИСАНИЕ ПОВРЕЖДЕНИЙ   
%\\ps{ деталь }{E}{I}{L} 
   
\акт{ }{ }{ }{ }
\акт{ }{ }{ }{ }
\акт{ }{ }{ }{ }
%\акт{ }{ }{ }{ }
%\акт{ }{ }{ }{ }
%\акт{ }{ }{ }{ }
%\акт{ }{ }{ }{ }
%\акт{ }{ }{ }{ }
%\акт{ }{ }{ }{ }
%\акт{ }{ }{ }{ }
%\акт{ }{ }{ }{ }
%\акт{ }{ }{ }{ }
%\акт{ }{ }{ }{ }
%\акт{ }{ }{ }{ }
%\акт{ }{ }{ }{ }
%\акт{ }{ }{ }{ }
%\акт{ }{ }{ }{ }

  \end{longtable}
%%%%%%%%%%%%%%%%%%%%%%%%%%%%%%%



\end{document}

%%
%%\input{titul/shapАЭИс}
%
%
%\par Исследование произведено согласно экспертной методики [1] \emph{Методические рекомендации по проведению судебных автотехнических экспертиз и исследований колесных транспортных средств в целях определения размера ущерба, стоимости восстановительного ремонта и оценки}. Махнин\,Е.\,Л., Новоселецкий\, И.\,Н., Федотов\, С.\,В. // --М.: ФБУ РФЦСЭ при Минюсте России, 2018.-326 с.  ISBN 978-5-91133-185-6
%
%%\subsection{Вопросы экспертизы}
\subsection{Вопрос исследования}
\begin{enumerate}
%\item  <<Установить наличие, характер и объем (степень) технических повреждений транспортного средства  \tc?>>
%\item  <<Установить причины возникновения технических повреждений транспортного средства \tc \,и возможность их отнесения к рассматриваемому дорожно-транспортному происшествию (далее ДТП)?>>
%\item <<Установить технологию, объем восстановительного  ремонта \!транспортного средства \tc?>>
%\item <<Установить размер затрат на восстановительный ремонт (с учётом износа) транспортного средства \tc?>>
\item <<Определить размер ущерба, причиненного владельцу  транспортного средства \tc\,\грз\, \, в результате дорожно-транспортного происшествия, имевшего место \датадтп?>>
%\item <<Определить стоимость восстановительного ремонта  транспортного средства \tc\, регистрационный знак \грз,\, \, получившего механические повреждения в результате противоправных действий, имевших место \датадтп?>>
%\item <<Определить величину физического износа  транспортного средства \tc\,\грз\, \, получившего повреждения в результате дорожно-транспортного происшествия, имевшего место \датадтп?>>
%%	
\end{enumerate}
%\pagebreak

\subsection{Рсчет износа ТС \тс}

В общем случае, износ характеризует изменения определенного параметра (стоимости, эксплуатационных качеств и технических характеристик в виде посадок, зазоров, прочности, прозрачности и т.д.) или совокупности этих параметров относительно состояния, соответствующего новой составной части КТС в целом п. 7.4 [1]. Износ, используемый для определения стоимости восстановительного ремонта, характеризует изменения стоимости составной части в зависимости от ее оставшегося ресурса.


%%%%%%%%%%%%%%%%%%%%%%%  Алгоритм расчета  %%%%%%%%%%%%%%%%%%%%%%%%%%%%%%

В соответствии п. 7.5 [1] коэффициент износа составных частей (И) КТС (кроме автобусов и грузовых автомобилей) при определении стоимости восстановительного ремонта расчитывается по формуле:

\begin{equation}\label{eqsnos}
\text{И} =\text{И1}\cdot\text{П}+\text{И2}\cdot \text{Д}, \%  \,\,\,\, \text{где:}
\end{equation}

\begin{itemize}
	\item [] $ \text{И1} $ --усредненный показатель износа на 1000 км пробега, \%; 
	\item [] $ \text{П} $ -- общий пробег (фактический или расчетный) за срок эксплуатации КТС, тыс.км;
	\item [] $ \text{И2} $ -- усредненный показатель старения за 1 год эксплуатации, \%;
	\item [] $ \text{Д} $ -- срок эксплуатации КТС (от даты изготовления КТС до момента, на который определяется износ), лет. 
\end{itemize}

%\vspace{3mm}
%%%%%%%%%%%%%%%%%%%%%%%%  Нулевой и предельный износ %%%%%%%%%%%%%%%%%%%%%%%%%%%%%%
%\par $ ^*$Согласно п. 7.8.\, Методики [1]  для случаев, не регулируемых законодательством об ОСАГО, для составных частей КТС значение износа принимается равным нулю, срок эксплуатации которых не превышает 5 лет,  предельное значение износа комплектующих транспортного средства  не должно превышать 80\% стоимости запасных частей. Для составных частей, имеющих срок эксплуатации более 12 лет, при отсутствии факторов снижения износа (проведенный капитальный ремонт, замена составных частей  и т.д) рекомендуемое значение износа составляет 80\%.


\subsubsection{Данные для расчета}

\noindent Объект исследования: транспортное средство \tc\,
регистрационный знак: \грз\\ 
Идентификационный номер VIN: \вин\\
Пробег:    \пробег, км\\%(Установлен по показаниям одометра);\\
Год выпуска:     \год\\ 
Дата ввода в эксплуатацию:  \началоэкспл\\
Дата повреждения:  \датадтп\\
ПТС: \птс\\
%Рыночная стоимость ТС \tc\\ 
%регистрационный знак \grz составляет: $\cdot$ ($\cdot$) рублей;\\
%источник: \url{https://spec.drom.ru}
%\vspace{3mm}
%
%Из открытых банков данных полиции известно, что автомобиль с VIN  \вин\, как минимум дважды становился участником ДТП:\\
%15.06.2018  12:15, извещение о ДТП № 030043199\\  %, в котором автомобиль получил повреждения задней правой двери, заднего правого порога, заднего правого колеса, подушки SRS справа, Рис. \ref{ris:images/d1} 
%18.11.2019 12:30, извещение о ДТП № 790004991.\\%, в котором автомобиль получил повреждения деталей передней левой и задней частей кузова, Рис. \ref{ris:images/d2}.
%\noindent \textit{Источник:}  \url{https://xn--90adear.xn--p1ai/check/auto#\vin}
%\pagebreak


\subsubsection{Расчет}

Для исследуемого автомобиля \тс, параметры  расчета коэффициента износа приняты согласно справочным таблицам  ч. II, Приложение 2.4 [1], информация о пробеге предоставлена заказчиком, дата ввода в эксплуатацию принята согласно сведениям паспорта транспортного средства \птс:

\begin{itemize}
	\item [] $ \text{И1} = 0.3$ \,\% %--усредненный показатель износа на 1000 км пробега, \%; 
	\item [] $ \text{П} = 142.1 $ \, тыс. км %-- общий пробег (фактический или расчетный) за срок эксплуатации КТС, тыс.км;
	\item [] $ \text{И2} = 1.35 $ \, \% %-- усредненный показатель старения за 1 год эксплуатации, \%;
	\item [] $ \text{Д} = 14 $ \, лет %-- срок эксплуатации КТС (от даты изготовления КТС до момента, на который определяется износ), лет. 
\end{itemize}

\begin{equation}\label{eqsnosr}
\text{И} =\text{И1}\cdot\text{П}+\text{И2}\cdot \text{Д} = 0.3\cdot 142.1  + 1.35\cdot 14 = 61 \, \%
\end{equation}

Таким образом, согласно вышеприведенных расчетов, величина износа транспортного средства \тс\, регистрационный знак \грз\, VIN \, \vin\, на момент дорожно-транспортного происшествия \датадтп\, составляла 61\, \%. 

%\subsection{Выводы}
%
%\begin{enumerate}
%	\item \textbf{Величина  износа  транспортного средства \tc\, регистрационный знак \грз на момент дорожно-транспортного происшествия \датадтп\, составляла 61\, \%. 
%	}
%\end{enumerate}
%
%
%\vspace{15mm}
%\noindent{Эксперт-техник}      \hfill                        {Мраморнов А.В.}
%\vspace{25mm}
%\relax
%
%\noindent Приложения к заключению:\\
%\noindent \textit{\small 
%	%	Приложение № 1. Расшифровка модельных опций ТС \тс \\
%%	Приложение № 1. Акт осмотра ТС \тс\\
%%	Приложение № 2. Фототаблица повреждений ТС \тс\\
%%	Приложение № 3. Калькуляция стоимости восстановительного ремонта ТС \тс\\
%%	Приложение № 4. Цифровые копии регистрационных документов ТС\\
%%	Приложение № 1. Цифровая копия паспорта транспортного средства\\
%	Приложение. Правоустанавливающие документы\\}

%\begin{center}
	\textbf{{\LARGE Новая судебная практика взыскания с виновника ДТП ущерба сверх страхового возмещения без учёта износа деталей и запчастей}}
	\end{center}
\vspace{5mm}



Многие автомобилисты, которые понесли убытки в результате дорожно-транспортного происшествия, сталкиваются с невозможностью добиться возмещения причинённого ущерба в полном объёме даже в судебном порядке.


В случае, когда страховое возмещение недостаточно для того, чтобы полностью возместить причинённый вред, потерпевший вправе обратиться в суд с требованием к виновнику ДТП о возмещении разницы между страховым возмещением и фактическим размером ущерба. Однако, до недавнего времени такая схема зачастую работала только в теории, когда как на практике возмещение ущерба в полном объёме было практически невозможно. Разница между реальным ущербом и страховым возмещением может составлять десятки тысяч рублей. 
Суды при рассмотрении таких споров указывали, что расчёт стоимости восстановительного ремонта повреждённого транспортного средства для целей определения размера ущерба, возмещаемого причинителем вреда, осуществляется в соответствии с методикой, которая в свою очередь предусматривает расчёт причинённого ущерба с учётом износа деталей и запчастей. Указанную позицию судов закрепил и Обзор практики рассмотрения судами дел, связанных с обязательным страхованием гражданской ответственности владельцев транспортных средств, утверждённый Президиумом Верховного Суда РФ 22.06.2016 г. 
Вместе с тем, \textbf{указанный подход судов не отвечает требованиям действующего законодательства, вытекающим из деликтных обязательств причинителя вреда. }
Так в силу статьи 15 Гражданского кодекса Российской Федерации под убытками понимаются расходы, которые лицо, чьё право нарушено, произвело или должно будет произвести для восстановления нарушенного права, утрата или повреждение его имущества (реальный ущерб), а также неполученные доходы, которые это лицо получило бы при обычных условиях гражданского оборота, если бы его право не было нарушено (упущенная выгода). 
Исходя из этого, лицо, которое понесло убытки в результате повреждения имущества третьими лицами, может в силу закона рассчитывать на восстановление своих нарушенных прав. Применительно к случаю причинения вреда транспортному средству это означает, что в результате возмещения убытков в полном размере потерпевший должен быть поставлен в положение, в котором он находился бы, если бы его право собственности не было нарушено, то есть ему должны быть возмещены расходы на полное восстановление эксплуатационных и товарных характеристик повреждённого транспортного средства.


В марте 2017 года Конституционный суд РФ, решения которого обязательны для всех судов на территории Российской Федерации, выразил позицию, в корне изменившую положение дел в случае взыскания ущерба, причинённого потерпевшему в результате ДТП. 
Конституционный Суд РФ разъяснил, что \textbf{институт обязательного страхования гражданской ответственности владельцев транспортных средств, введённый в действующее законодательство с целью повышения уровня защиты прав потерпевших при причинении им вреда при использовании транспортных средств иными лицами, не может подменять собой институт деликтных обязательств, регламентируемый главой 59 ГК Российской Федерации, и не может приводить к снижению размера возмещения вреда, на которое вправе рассчитывать потерпевший на основании общих положений гражданского законодательства. }
При этом, <<Единая методика определения размера расходов на восстановительный ремонт в отношении повреждённого транспортного средства>> обязательства вследствие причинения вреда не регулирует. 
Следовательно, потерпевший при недостаточности страховой выплаты на покрытие причинённого ему фактического ущерба вправе рассчитывать на восполнение образовавшейся разницы за счёт лица, в результате противоправных действий которого образовался этот ущерб, путём предъявления к нему соответствующего требования. 
Соответственно, при исчислении размера расходов, необходимых для приведения транспортного средства в состояние, в котором оно находилось до повреждения, и подлежащих возмещению лицом, причинившим вред, должны приниматься во внимание реальные, т.е. необходимые, экономически обоснованные, отвечающие требованиям завода-изготовителя, учитывающие условия эксплуатации транспортного средства и достоверно подтверждённые расходы, в том числе расходы на новые комплектующие изделия (детали, узлы и агрегаты).

Как следует \textbf{из постановления Пленума Верховного Суда Российской Федерации от 23 июня 2015 года № 25 <<О применении судами некоторых положений раздела I части первой Гражданского кодекса Российской Федерации>>, если для устранения повреждений имущества истца использовались или будут использованы новые материалы, то за исключением случаев, установленных законом или договором, расходы на такое устранение включаются в состав реального ущерба истца полностью, несмотря на то что стоимость имущества увеличилась или может увеличиться по сравнению с его стоимостью до повреждения.} 
Поскольку полное возмещение вреда предполагает восстановление повреждённого имущества до состояния, в котором оно находилось до нарушения права, в таких случаях — притом что на потерпевшего не может быть возложено бремя самостоятельного поиска деталей, узлов и агрегатов с той же степенью износа, что и у подлежащих замене, — неосновательного обогащения собственника повреждённого имущества не происходит, даже если в результате замены повреждённых деталей, узлов и агрегатов его стоимость выросла.
Впоследствии \textbf{из Обзора практики рассмотрения судами дел, связанных с обязательным страхованием гражданской ответственности владельцев транспортных средств (утв. Президиумом Верховного Суда РФ 22.06.2016 г.), был исключён пункт 22, который закреплял противоречащую принципам действующего законодательства позицию судов}.



Как следует \textbf{из постановления Пленума Верховного Суда Российской Федерации от 23 июня 2015 года № 25 <<О применении судами некоторых положений раздела I части первой Гражданского кодекса Российской Федерации>>, если для устранения повреждений имущества истца использовались или будут использованы новые материалы, то за исключением случаев, установленных законом или договором, расходы на такое устранение включаются в состав реального ущерба истца полностью, несмотря на то что стоимость имущества увеличилась или может увеличиться по сравнению с его стоимостью до повреждения.} 
Поскольку полное возмещение вреда предполагает восстановление повреждённого имущества до состояния, в котором оно находилось до нарушения права, в таких случаях — притом что на потерпевшего не может быть возложено бремя самостоятельного поиска деталей, узлов и агрегатов с той же степенью износа, что и у подлежащих замене, — неосновательного обогащения собственника повреждённого имущества не происходит, даже если в результате замены повреждённых деталей, узлов и агрегатов его стоимость выросла.
Впоследствии \textbf{из Обзора практики рассмотрения судами дел, связанных с обязательным страхованием гражданской ответственности владельцев транспортных средств (утв. Президиумом Верховного Суда РФ 22.06.2016 г.), был исключён пункт 22, который закреплял противоречащую принципам действующего законодательства позицию судов}.
\end{document} 
